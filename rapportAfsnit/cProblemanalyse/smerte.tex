\section{Smerte}

Smerte er blevet defineret som: “en ubehagelig sensorisk og emotionel oplevelse forbundet med egentligt eller potentiel skade af væv eller beskrevet i vendinger tilsvarende en lignende skade.” af den Internationale Association for Studiet af Smerte (Pain) (IASP) [Seminars in arthritis and rheumatism, Classification of Chronic Pain]. 
Selvom smerte normalt er en følelse man forsøger at undgå, er det en nødvendig del af vores overlevelse. Det fortæller kroppen om farer eller skader som der skal reageres på, som for eksempel at sætte hånden på en varm kogeplade. For ikke at blive slemt forbrændt og gøre skade på hånden, registrere nerver i huden en høj temperatur, som hånden skal fjernes fra. Nervesignalet sendes til central nervesystemet (CNS), hvor det først når rygmarven og lidt senere hjernen. Her skelnes der mellem smerte sensation og perception. Smerte sensation er information om smerte, som nerverne i hånden der registrere den skadelige temperatur. Smerte perceptionen sker først når nervesignalet når op til hjernen og denne modtager signalet og opfatter det som smerte. Sensationen af smerte kan i rygsøjlen aktivere en refleks der får musklerne i armen til at trække hånden væk fra varmen, inden hjernen når at registrere og opfatte den egentlige smerte. [Fundamentals of Human Anatomy and Physiologi 9th edition] Denne form for smerte er kategoriseret som øjeblikkelig smerte og defineres som “god” eller “nødvendig” smerte, da det hjælper kroppen med at undgå skader.

\subsection{Smerte typer}
Modsat denne “gode” smerte findes “dårlig” eller “unødvendig” smerte. Denne smerte kaldes også kronisk smerte, da den oftest er længere varende, ved at have været mere eller mindre konstant i mindst tre måneder [Seminars in arthritis and rheumatism]. Personer som lider af kronisk smerte har sjældent en synlig grund til at skulle føle smerte, og smerten er af den grund enten oplevet som smerte i indre organer eller psykogene smerter, som er forestillingen om en smerte. 
Ved organisk smerte skelnes der mellem to typer af smerte: nociceptisk og neuropatisk. Nociceptisksmerte er skade på væv og skyldes aktivering af nociceptorer, specielle nerveceller som er følsomme overfor temperatur ændringer, mekanisk stimuli eller kemiske ændringer i eller omkring celler, ved hjælp af specielle porte og pumper på nervecellerne. Nociceptorer findes i huden på kroppens overflader og i og omkring indre organer, og nociceptisksmerte opdeles i somatisk og viseral sensation. Somatisk smerte sensation og den øjeblikkelige og let placerbare smerte som at sætte hånden på en kogeplade. Viseral smerte sensation er mere besværlig at placere. Smerten er typisk ikke øjeblikkelig, men mere trykkende og langvarig. At have ondt i maven er et eksempel på viseral smerte. Nociceptisksmerte er oftest ikke årsag til kroniske smerter, med mindre smerterne bliver ved. 
Neuropatisk smerte er modsat nociceptisksmerte ikke skade på væv, men på nervesystemet selv, herunder nerver, rygmarv, nerve plexus eller hjernen. Dette kan skyldes direkte skade, sygdom som iskæmi eller sclerose, traume, diabetes, infektioner eller kræft. Smerten kan opleves som konstant og langvarig, hvor er typisk eksempel er fantom smerter, men kan også være lejlighedsvis som ved hyperalgesi, hvor almindelig berøring opfattes som smertefuldt. [Seminars in arthritis and rheumatism, Classification of Chronic Pain] 
Psykogene smerter er en forestillet opfattelse af smerte, og den mest besværlige at præcisere, idet der ikke er og muligvis aldrig har været en fysisk grund til smerten. Hos en person med psykogene smerter er hjernen fuldt overbevidst om, at den oplever fysiske smerter og lider der af. Smerten er udelukkende psykisk hos personen, men er af den grund ikke mindre virkelig, grundet smertes subjektive natur. [Seminars in arthritis and rheumatism]

\subsection{Problemet ved kronisk smerte og operationer}
Der findes således flere forskellige former for smerte, hvor kun nogle få er beskrevet her [Classification of Chronic Pain]. Alle kan de lede til kronisk smerte. Man ved derfor godt hvad der kan give kronisk smerte, hvordan smerten opfattes og hvor i kroppen den kommer fra. Men man ved endnu ikke hvorfor kronisk smerte opstår. Hvorfor føler kroppen fantom smerter fra et legeme som ikke er der? Hvorfor registrere hjernen smerte fra indre organer, når der intet er i vejen med dem? Et større problem opstår når man forsøger at behandle smerterne, med et ønske om at dæmpe eller helt fjerne smerterne, men smerterne fortsætter eller forværres. Sådan ses det i 20\% af tilfælde efter total knæalloplastik (TKA) operationer [Chronic Postoperative Pain After Primary and Revision Total Knee Arthroplasty]. Her oplever 19\% af patienter efter den primære operation, og 47\% af patienter efter revision af operationen oplever svære til uudholdelige smerter. Dette sker på trods af at der i hele Danmark udføres knæoperationer som alle signifikant overholder indikationerne for behandlingskvaliteten [Dansk Knæalloplastikregister, Årsrapport 2016]. De udførte TKA operationer må siges at være nær perfekte, men patienter oplever alligevel fortsatte smerter. 

