\section{Klinisk udvælgelse af patienter}

Noter: 
Ifølge sundhedsstyrrelsen fra 2012 - ingen egentlige "bevidste" retningslinjer. Kun retningslinjer opstillet ud fra lægeligt konsensus. Hermed kan der opstå konfliker mellem læger og patienter samt læger imellem idet det er en vurderingssag ud fra den enkelte læges erfaring som bestemmer om patienten bliver tilbudt operationen eller ej. 
Sundhedsstyrrelsen har ligeledes opstillet retningslinjer for tilfælde som gør at en patient ikke bliver tilbudt en operation - f.eks. ingen smerter eller urealistiske forventninger til operationen. I forhold til den sidste er det ligeledes lægens opgave at forklare operationen og dennes udkom for patienten således patienten har realistiske forventninger. - Studie der viser at patienters forventninger har betydning for deres tilfredshed efter operationen. Ligeledes viser et andet studie at klinikerens mening har betydning for hvad patienten forventer. 
OARSI retningslinjer for knæalloplastik: (OARSI - eksperthold af 16 klinikere fra USA, Canada, Sverige, Holland og Frankrig)
Patienter skal kun have tilbudt alloplastik hvis både medicinske og ikke-medicinske indgreb ikke har haft tilstrækkelig virkning på patienten. 
Hvis total knæalloplastik ikke giver det forventede/ønskede resultat kan fiksering af knæleddet overvejest. 
PROBLEM: studier tyder på patienters køn og race kan have betydning for om de bliver tilbudt en operation eller ej - har denne opdeling en klinisk grund? Ikke nogen fundet klinisk grund - foreskellen afhænger (ud fra studiet) helt af patientens køn. I studiet undersøges patienter med "medium" artrose. I et studie med svær artrose bliver ingen signifikant forskel mellem mænd, kvinder og race fundet. Dette kan angive at køn og race spiller en mindre rolle i klinikerens udvælgelsesprocess ved svære tilfælde af knæartrose. 


Patienter som tilbydes en total knæ alloplastik udvælges på baggrund af en læge eller kirurgs observationer og erfaringer. Hermed afhænger udvælgelsen af patienter til en TKA operation alene af klinikeren, hvormed patienter kan opleve forskellige anbefalinger og behandlingsmuligheder ved forskellige klinikere. I et forsøg på at ensarte behandlingen af knæartrose for alle patienter i Danmark har Sundhedsstyrrelsen udarbejdet en rapport indeholdene nationale kliniske retningslinjer. Disse retningslinjer bygger hovedsageligt på lægeligt konsensus. Retningslinjerne omhandlende tilbud af TKA til patienter indeholder blandt andet at patienter kun tilbydes en TKA hvis ikke-kirurgiske behandlingsmetoder ikke har haft en tilstrækkelig virkning. En TKA kan dog tilbydes patienter som den første behandlingsmulighed hvis lægen/kirugen vurderer at patientens artrose er så svær ingen ikke-kirugriske behandlingsmuligheder vil have en tilstrækkelig virkning. Dette kan eksempelvis være hvis patienten har en svær fejlstilling af knæet eller svær instabilitet i leddet. \citep{sund2012}    
Internationale retningslinjer for behandling af patienter med artrose er opstillet af Osteoarthritis Research Society International (OARSI). Disse retningslinjer er udarbejdet af tværfagligt sundhedspersonale fra seks forskellige lande i Nordamerika og Europa. Retninglinjerne fra OARSI omhandlende knæalloplastik specificerer ligeledes at kun patienter som ikke har oplevet tilstækkelig virkning fra ikke-kirugiske behandlingsmetoder skal tilbydes en TKA. Desuden anses TKA af OARSI som en pålidelig og passende behandlingsmetode til at genoprette funktion og højne patientens livskvalitet. \citep{zhang2008} \\
Udfra disse retninglinjer skal alle knæartrose patienter som ikke opnår den ønskede virkning ved ikke-kirurgiske behandlingsmuligheder tilbydes en TKA. I praksis er det dog ikke alle patienter der opfylder de opstillede retningslinjer, som tilbydes en TKA. \citep{borkhoff2008} For en udefrakommende kan udvælgelsen af patienter som får tilbudt en operation og patienter som ikke gør, virke tilfældigt. Ligeledes kan der være forskel lægernes vurdering imellem, hvormed den ene læge ville tilbyde en patient operationen, mens en anden læge ikke ville. Sundhedsstyrrelsen har opstillet en række indikationer som kan få klinikeren til at fravælge patienten til en operation. Disse indikationer er eksempelvis hvis der er infektion i knoglen eller leddet, hvis patienten ingen smerter har i leddet eller hvis patienten har en kort forventet levetid. En anden indikator, som kan få en kliniker til at fravælge en patient, er hvis patienten har urealistiske forventninger til operationen. \citep{sund2012} I et studie har \cite{tejada2010} vist at patienter hvis forventninger til operationen bliver opfyldt oplever større tilfredshed efter operationen. Ligeledes er det vist at klinikeres forventninger til en operation påvirker patientens forventninger \citep{tejada2010}. Dette antyder at klinikere ved at forklare patienten hvad de kan forvente af operationen, kan være med til at mindske eller helt fjerne en af faktorerne som betyyder, at patienten ikke bliver tilbudt en TKA. \\
De nævnte indikatorer er alle nogle klinikeren bevidst skal tage med i sine overvejelser når det bedømmes om en patient skal tilbydes en TKA eller ej, men resultater fra nogle studier tyder på klinikere også påvirkes af ubevidste faktorer. Eksempelvis viser resultater fra et studie udarbejdet af \cite{borkhoff2008} at en patients køn har betydning for om klinikeren tilbyder patienten en TKA eller ej. I dette studie anvendte \cite{borkhoff2008} to standardiserede patienter med moderat knæartrose som besøgte 71 læger (38 alment praktiserende læger og 33 ortopædkiruger). Lægerne som deltog i studiet blev ikke informeret om hvem de to standardiserede patienter var. De to standardiserede patienter var ens på alle andre punkter end køn. \cite{borkhoff2008} fandt at 42\% af lægerne kun tilbød den mandlige patient en TKA, mens kun 8\% af lægerne kun tilbød den kvindelige patient en TKA. Hermed er sandsynligeheden for at en kvindelig patient med moderart knæartrose får tilbudt en TKA betydeligt mindre end sandsynligeheden for at en mandlig patient får tilbudt operationen. Dette er problematisk da størstedelen af knæartrose patienterne er kvinder, hvor flere af disse hermed først vil blive tilbudt en TKA når deres knæartrose er forværet. Idet resultatet af operationen bliver forværret når patientens artrose bliver sværrere, betyder dette at flere af de kvindelige artrosepatienter får et mindre godt resultat end hvad de ville have fået, hvis operationen var blevet tilbudt på samme tidspunkt som en kliniker ville tilbyde en mandlig patient operationen \citep{borkhoff2008}. \\
I en spørgeskemaundersøgelse hvor ortopædkiruger blev bedt om at vurdere betydningen af en patients køn i forhold til om de ville henstille patienten til en TKA eller ej svarede cirka 93\% af de adspurgte kiruger at køn ikke ville have nogen betydning for deres vurdering af patienten \citep{wright1995}. Dette antyder at den forskel \cite{borkhoff2008} fandt i deres forsøg, er som følge af en underbevidst bias hos klinikerne. Denne teori understøttes af resultaterne fra et studie udarbejdet af \cite{dy2014}, hvor det blev undersøgt om en patients race og køn ville have betydning for ortopædkirugers vurdering af patienten. I dette studie blev videoer med patienter med svær knæartrose anvendt. Ligesom i studiet af \cite{borkhoff2008} var patienterne kun forskellige i køn og race. \cite{dy2014} fandt ingen signifikant forskel på kirugernes vurdering af de fire standardiserede patienter. Denne forskel i resultaterne mellem \cite{borkhoff2008} og \cite{dy2014} kan skyldes forskellen i patienternes grad af artrose. Ved patienter med svær artrose ses ingen bias, mens der ved patienter med moderat artrose blev fundet et bias. Dette antyder at kirugernes bias kun har betydning for patienter, hvor der ikke er fuldstændig klare indikationer på at patienten skal opereres. I tilfælde hvor den patienten ikke har klare indikationer på en operation, ville en metode der vil kunne hjælpe klinikeren med at lave en vurdering uden bias være fordelagtig. Ligeledes vil en sådan metode kunne være med til at systematisere henstillingen af patienter til en TKA, således det sikres at patienten får tilbudt samme behandling uafhængigt af hvilken kliniker der vurderer patienten. 
\textbf{SKAL DER VÆRE NOGET MERE OM PRÆCIS HVILKEN METODE MAN VILLE KUNNE ANVENDE? ER DET NOK I FORHOLD TIL KIRUGERNS UDVÆLGELSE? - DET ER SVÆRT AT FINDE NOGET PRÆCIS OM HVILKE SPØRGSMÅL DE STILLER OG HVAD DE LÆGGER VÆGT PÅ.} 


   
