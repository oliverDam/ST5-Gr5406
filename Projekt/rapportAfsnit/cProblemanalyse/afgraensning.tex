\section{Problemafgrænsning}
Selvom TKA-operationerne bliver udført i henhold til de opstillede retningslinjer, oplever cirka 20\% af patienterne kroniske postoperative smerter.\\
Klinikere er ansvarlige for belsutningen om hvorvidt en patient er egnet til at modtage en TKA-operation og formår succesfuldt at udvælge 75 til 81\% af patienterne. Denne udvælgelse sker på baggrund af klinikernes erfaring, radiologiske fund, symptomvurdering samt patientens egne udtalelser. Den resterende patientgruppe har kroniske postoperative smerter, hvilket indikerer, at udvælgelsesmetoden ikke er fyldestgørende. Det kan derfor, for både klinikere og patienter være fordelagtigt, hvis den benyttede metode bliver optimeret. Optimeringen kan bestå i benyttelse af en teknologisk metodik, der kan supplere klinikerens beslutningstagen med faktiske resultater. Hvis en teknologisk metode skal kunne implementeres kræves det, at denne muliggør identificering af patientgruppen, hvis risiko for kroniske postoperative smerter, er størst. Den teknologiske metode bør ydermere være minimalt invasiv, omkostningseffektiv og let organisatorisk implementerbar. Disse kriterier opfyldes bedst af QST blandt de analyserede smertediagnosticeringsmetoder, hvormed det antydes, at QST vil være den teknologiske metode, som bedst vil kunne supplere klinikerens beslutningstagning. 

\subsection{Problemformulering}
\begin{center}
	\textit{Hvilke konsekvenser er forbundet med implementering og brug af QST, som supplement til
		klinikerens vurdering af en patientens henvisning til en TKA-operation?}
\end{center}