Selvom TKA-operationerne bliver udført i henhold til de opstillede retningslinjer, udvikler cirka 20~\% af patienterne kroniske postoperative smerter. Klinikere er ansvarlige for at vurdere hvorvidt en patient er egnet til at modtage en TKA-operation og formår succesfuldt at udvælge omkring 80~\% af patienterne. Denne udvælgelse sker på baggrund af klinikernes erfaring, radiologiske fund, symptomvurdering samt patientens egne udtalelser. Den resterende patientgruppe udvikler kroniske postoperative smerter, hvilket indikerer, at udvælgelsesmetoden ikke er god nok. Det kan derfor, for både klinikere og patienter være fordelagtigt, hvis den benyttede metode bliver forbedret. Forbedringen kan bestå i benyttelse af en teknologi, der kan supplere klinikerens beslutningstagen med kvantitative resultater. Hvis en teknologi skal kunne implementeres kræves det, at denne muliggør identificering af patientgruppen, hvis risiko for udviklingen af kroniske postoperative smerter, er forhøjet. Teknologien bør ydermere være ikke-invasiv og omkostningseffektiv. Disse kriterier opfyldes bedst af QST blandt de analyserede smertediagnosticeringsmetoder, hvormed det antydes, at QST vil være den teknologi, som bedst vil kunne supplere klinikerens beslutningstagning. \\
For at danne rammer for undersøgelsen af QST vil der blive taget udgangspunkt i de ortopædkirurgiske afdelinger i Region Nordjylland. 

\section{Problemformulering}
\begin{center}
	\textit{Hvilke konsekvenser er forbundet med implementering og brug af QST, som supplement til klinikerens vurdering af en patients egnethed til en TKA-operation på de ortopædkirurgiske afdelinger i Region Nordjylland?}
\end{center}