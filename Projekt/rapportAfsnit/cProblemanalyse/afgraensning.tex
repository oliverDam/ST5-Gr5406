\newpage \section{Problemafgrænsning}
Selvom operationerne bliver udført ‘perfekt’ så er 11 til 25\% af patienterne stadig utilfredse. Det tyder på at disse patienter er utilfredse pga. deres manglende resultater efter operationen, relateret til smerte og funktionsnedsættelse. Klinikerne er ansvarlig for belsutningstagen hvorvidt en patient er egnet til at modtage en TKA-operation. Klinikerne formår succesfuldt at udvælge 75 til 81\% af patienterne, på baggrund af deres erfaring, radiologiske fund, symptom vurdering, samt patientens egne udtalelser. Den resterende patientgruppe er utilfredse med resultatet, hvilket indikerer at udvælgelsesmetoden ikke er god nok, og at eventuelle bias kan have medvirket til dette resultat. Det kunne derfor, for klinikeren og patienten være fordelagtigt hvis den benyttede metode blev optimeret. Optimeringen kunne indebære afbenyttelse af en teknologisk metodik. Den teknologiske tilgang skulle medføre nogle faktiske resultater som skal supplere og bidrage til klinikerens beslutningstagen. Hvis en teknologisk metode skal kunne implementeres kræves det at denne muliggør identificering af patientgruppen, hvis risiko for kroniske komplikationer postoperativt, er størst. Den teknologiske tilgang bør ydermere være minimalt invasiv, omkostningseffektiv og let organisatorisk implementerbar. Disse kriterier opfyldes på bedste vis af QST, blandt de analyserede smertediagnosticeringsmetoder, hvormed det antydes at QST vil være den teknologiske tilgang som bedst vil kunne supplere klinikerens beslutningstagen. 

\subsection{Problemformulering}
\begin{center}
	\textit{Hvilke konsekvenser er forbundet med implementering og brug af QST, som supplement til klinikerens vurdering af en patientens henvisning til en TKA-operation}
\end{center}