\section{Teknologier til smerteklassificering} % Anden overskrift...
\textit{I denne sektion vil nuværende teknologier til undersøgelse af central sensibilisering blive beskrevet. De enkelte teknologiers egenskaber, vil overordnet blive analyseret. Afslutningsvis vil der, på baggrund af de enkelte teknologiers egenskaber, blive foretaget en sammenligning med henblik på at identificere fordele og ulemper ved de enkelte teknologier.}

%Til undersøgelse af aktivitet i encephalon og nervesystemets funktion generelt findes flere forskellige metoder.
\subsection{Functional Magnetic Resonance Imaging}
Functional Magnetic Resonance Imaging (fMRI) er en metode til at undersøge aktiviteten i neuronerne i hjernen.
Generelt er MRI en metode til at synliggøre protoner, hvormed det er muligt at afbilde kroppens væv, da disse primært er udgjort af hydrogen-atomer, som indeholder netop én proton. 
Ved en MRI-scanning udsættes objektet, eksempelvis en patient, for et eksternt magnetfelt, hvilket medfører, at der opstår et parallelt magnetfelt i objektet, hvormed det ved hjælp af radiobølger er muligt at detektere de tilstedeværende protoner. \citep{Wals2009} Udgifterne til fMRI omfatter blandt andet indkøb af en MR-scanner samt anvendelse og vedligehold. Prisen for en MR-scanner ligger mellem en og otte millioner kr. \citep{Glover2014}\\ 
Der findes forskellige teknikker til fMRI, hvoraf de fleste anvender blood oxygenation level-dependent (BOLD) kontrast. Ved anvendelse af denne kontrast, udnyttes det, at der ved aktivitet i hjernens forskellige områder, vil ske en ændring i mængden af ilt i blodet, hvilket påvirker magnetfeltets styrke. Dermed vil det, ved hjælp af kontrastvæsken, være muligt at følge blodgennemstrømningens styrke i hjernens forskellige områder, og dermed detektere områder med øget aktivitet. \citep{Wals2009} 

fMRI har i flere studier været anvendt til at undersøge neural aktivitet i forbindelse med knæsmerter.
I et studie af \citer{Parks2012} er det ved hjælp af fMRI blevet undersøgt, hvorvidt der er forskel på oplevelsen af kroniske smerter hos patienter med artrose og på smerter fremkaldt med tryk. Forsøgsdeltagerne var opdelt i en gruppe patienter med knæartrose og en gruppe raske personer. \citer{Parks2012} viste, at de to grupper oplevede den kunstigt fremkaldte knæsmerte ens. Hos artrosepatienterne forekom der forskel i oplevelsen af de kroniske og de kunstigt fremkaldte smerter. \citep{Parks2012}\\ 
I et andet studie af \citer{Hiramatsu2014} er det undersøgt hvilke forskelle, der forekom i cerebral respons hos en gruppe raske forsøgspersoner og en gruppe forsøgspersoner med kroniske knæsmerter forårsaget af artrose. Begge grupper blev udsat for akut smerte i knæet gennem invasive elektroder, imens der blev foretaget en fMRI-scanning. Studiet viste, at der hos gruppen af patienter med kroniske knæsmerter forekom en højere aktivitet i det dorsolaterale præfrontale cortex end hos gruppen af raske forsøgspersoner. \citep{Hiramatsu2014}
fMRI kan således påvise forskelle på hvordan smerte opleves af forskellige grupper af mennesker. \\
Resultater fra et studie af \citer{Bennet2011} viser dog, at anvendelse af fMRI kan være problematisk hvis der ikke i behandlingen af data tages højde for falske positiver på voxel-niveau. Disse falske positiver kan betyde, at der af scanneren opfatter et område med større blodgennemstrømning end ved hvile, selvom dette ikke er tilfældet. \citep{Bennet2011} 

\subsection{Quantitative sensory testing}
Quantitative sensory testing (QST) er en metode til undersøgelse af det sensoriske nervesystems funktion. Metoden kan anvendes til undersøgelse af forskellige egenskaber, herunder smertegrænser. QST eksponerer patienten for forskellige sensoriske stimuli, herunder varme og kulde, vibration og tryk. Dermed er det muligt at identificere grænserne for henholdsvis perception og smerte. \citep{Yarnitsky2006} Prisen for udstyr, som kan anvendes til QST, afhænger af hvilken QST-protokol, der anvendes. Priserne på QST-udstyr befinder sig i omegnen af 40.000 til 240.000 kr., og kræver, i modsætning til fMRI, ikke scanning eller anden billeddiagnostisk undersøgelse. \\
Der findes forskellige protokoller til udførelsen af QST, afhængigt af hvad formålet er med undersøgelsen. De forskellige protokoller indeholder forskellige tests, alt efter hvad fokusområdet for protokollen er. Et eksempel er en QST-protokol udviklet af German Research Network on Neuropathic Pain (DFNS) til undersøgelse af neuropatisk smerte. \citep{Rolke2006} I et studie af \citer{Petersen2016} blev en QST-protokol anvendt til undersøgelse af knæartrosepatienter før og efter en TKA-operation. Protokollen anvendt i dette studie bestod af tre tests der tilsammen undersøgte fire parametre. De undersøgte parametre var pressure detection threshold (PDT), pain tolerance threshold (PTT), temporal summation of pain (TSP) og conditioned pain modulation (CPM). \citep{Petersen2016} Hver af disse fire parametre er blevet vist at være abnormale ved patienter med knæartrose. \citep{Suokas2012} \citep{Petersen2016}

QST bliver i forskningsregi anvendt til undersøgelse af patienter, der får udført en TKA-operation. I et studie af \citer{Martinez2007} er en QST-protokol anvendt til at undersøge 20 patienter med artrose i knæet før og efter en TKA. Formålet hermed var at identificere faktorer, der har indvirkning på udviklingen af kroniske postoperative smerter. QST blev udført en enkelt gang før operationen og fire gange efter operationen. Perioden fra operationen og til QST undersøgelserne var på henholdsvis en dag, fire dage, en måned og fire måneder. Parametrene, der blev undersøgt ved QST, var tærskelværdier for temperatur, mekanisk smerte og hvordan de enkelte patienter responderede på udsættelse for temperaturer over tærskelværdierne. Studiet fandt, at der forekommer en sammenhæng mellem periodiske smerter efter operationen og de patienter, der oplever hyperalgesi under eksponering for varme. \citep{Martinez2007} Der skal imidlertid tages højde for, at der er flere fejlkilder forbundet med QST, da nøjagtigheden i høj grad afhænger af både patienten og klinikerens præcision under udførelsen af de enkelte tests. Det må således forventes, at der kan forekomme variationer mellem QST-målinger udført på den samme patient. \citep{Yarnitsky2006} I et review af \citer{Arendt-Nielsen2015}, blev det ved anvendelse af QST fundet, at perifer og central sensibilisering er et prominent fænomen ved knæartrose.

\subsection{Elektrofysiologiske undersøgelsesmetoder}
De elektrofysiologiske undersøgelsesmetoder dækker over tests, der detekterer elektrisk aktivitet i kroppens celler. Aktionspotentialet i neuroner kan måles ved anvendelse af invasive eller non-invasive elektroder. Til monitorering af neuronernes funktionalitet i hjernen anvendes elektroencephalografi (EEG) mens der til monitorering af neuronernes funktionalitet i resten af kroppen anvendes elektroneurografi (ENG). Prisen for et EEG system er cirka 550.000 kr. \citep{Biosemi2016}. For at en elektrofysiologisk undersøgelse kan anvendes til at stille en diagnose, skal resultaterne herfra understøttes af andre kliniske undersøgelser, herunder prøver og lægesamtaler. \citep{Robinson2008} 

I et studie af \citer{Brown2013}, er EEG blevet anvendt til at undersøge, hvordan to patientgrupper med forskellige sygdomme opfatter smerte. I studiet indgik en gruppe af patienter med artrose og en gruppe patienter med fibromyalgi. Fibromyalgi er en sygdom, der forårsager muskel- og ledsmerter \citep{Gigtforeningen2016}. Hos de to patientgrupper blev det undersøgt, hvordan hjernen genererer signal for henholdsvis en forventning om og en decideret udløst smerte. Disse data sammenlignes med en kontrolgruppe bestående af raske personer. Der blev foretaget to målinger for at undersøge aktiviteten under forventning om smerte samt en måling under påført smerte. For hver forsøgsperson blev det efter målingerne undersøgt, fra hvilken elektrode, der blev detekteret den højeste amplitude, hvormed denne og de otte nærmeste elektroder blev udvalgt til yderligere analyse og udregning af gennemsnitlig amplitude.

Ud fra resultaterne var det muligt at undersøge i hvilke områder, der forekom særlig aktivitet ved hjælp af blandt andet billeddannende programmer. Resultaterne for studiet antyder, at de to patientgrupper responderer ens på forventningen om smerte på trods af, at fibromyalgipatienternes respons var kraftigere. Disse resultater indikerer, at denne tendens kan være generelt gældende for patienter med kroniske smerter. Hermed kan elektrofysiologiske undersøgelsesmetoder benyttes til at identificerer knæartrosepatienter udfra raske personer. \citep{Brown2013} 

\subsection{Vurdering af teknologier til smerteklassificering}
En optimal teknologi som kan fungere som supplement til klinikerens beslutningstagen, skal opfylde en række kriterier. Et af disse kriterier er, at denne teknologi skal kunne anvendes til at klassificere smerte. Hvis ikke dette krav er opfyldt, vil teknologien ikke være anvendelig til at identificere patienterne i risikogruppen for at udvikle kroniske postoperative smerter, og vil hermed ikke kunne fungere som supplerende undersøgelsesmetode. Teknologien bør tilmed være non-invasiv i benyttelse. Ligeledes skal omkostningerne til anskaffelse af teknologien være mindst mulige.\\
De analyserede teknologier er vurderet ud fra ovenstående kriterier, og resultatet heraf ses i \tabref{tab:succeskriterier_metoder}.

\begin{table}[H]
	\centering
	\begin{tabular}{cccc}
		\hline
		\rowcolor[HTML]{C0C0C0} 
		Teknologi          & Smerteklassifikation & Non-invasiv & Anskaffelsespris  {[}kr.{]} \\ \hline
		fMRI               & x                   & -                & 1 til 8 mio.          \\
		QST                & x                   & x                & 40.000 til 240.000               \\
		Elektrofysiologisk & x                   & (x)              & 550.00                \\ \hline
	\end{tabular}
	\caption{I tabellen ses hvilke kriterier de forskellige teknologier opfylder. Ved angivelse af ‘x’, indikeres det, at kriteriet er opfyldt, mens ved ‘(x)’ er kriteriet delvist opfyldt. Ved angivelse af ‘-’ er kriteriet ikke opfyldt.}
	\label{tab:succeskriterier_metoder}
\end{table} \vspace{-.25cm}
Alle de analyserede teknologier opfylder første kriterie vedrørende klassificering af smerte. Teknologierne opfylder ikke alle kravet vedrørende at skulle være non-invasiv. Dette medfører, at fMRI bliver ekskluderet som et eventuelt supplement til klinikerens beslutningstagen. fMRI kategoriseres som en invasiv teknologi, da patienten får injiceret kontraststoffet før denne metode kan anvendes til klassificering af patienten. QST opfylder kriteriet ved ikke at benytte invasive metoder til klassificeringen af smerte. Elektrofysiologiske metoder opfylder kriteriet delvist, hvilket er et resultat af, at denne teknologi kan benytte både invasive og non-invasive elektroder til detektering af signalet. \\
I forhold til anskaffelsesprisen vil QST være billigst. Ved implementering af QST skal teknologien indkøbes fra ny, hvilket er et resultat af at teknologien på nuværende tidspunkt benyttes i forskningsregi. MR-scannere og elektrofysiologiske metoder er allerede implementeret og bliver benyttet til anden diagnosticering og behandling. Dermed kan det for disse teknologier antages, at nyt udstyr ikke nødvendigvis bør indkøbes. Dog betyder det, at disse teknologier anvendes til diagnosticering af andre sygdomme, at der skal tages højde for længere ventetider for patienten end ved benyttelsen af QST.\\
Ud fra kriterierne og ovenstående overvejelser vil QST være den bedst egnede teknologi som supplement til klinikerens udvælgelse af patienter til en TKA-operation.  



  
% 1 = fMRI techniques and protocols
% 2 = The dorsolateral prefrontal network is involved in pain perception in knee osteoarthritis patients
% 3 = Brain activity for chronic knee osteoarthritis: dissociating evoked pain from spontaneous pain
% 4 = Quantitative sensory testing
% 5 = The evolution of primary hyperalgesia in orthopedic surgery: Quantitative sensory testing and clinical evaluation before and after total knee arthroplasty
% 6 = Clinical electrophysiology
% 7 = When the brain expects pain: common neural responses to pain anticipation are related to clinical pain and distress in fibromyalgia and osteoarthritis 
%10 = Benefits, Shortcomings, and Costs of EEG Monitoring
%11 = http://time.com/money/2995166/why-does-mri-cost-so-much/ Glover2014
%12 = http://www.biosemi.com/faq/prices.htm Biosemi2016
%13 = http://nocitech.com/technology.html NociTech2016