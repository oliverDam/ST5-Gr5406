\section{Teknologier til undersøgelse} % Anden overskrift...
For at understøtte klinikerens vurdering af de enkelte patienter vedrørende henstilling til TKA, kan det overvejes, hvorvidt det vil være hensigtsmæssigt at tilføje en ekstra undersøgelse til patientens udredningsforløb forud for beslutningen herom. Relevante undersøgelser kan eksempelvis være objektive målinger på neural aktivitet i forskellige situationer, samt vurdering af patienternes individuelle respons på forskellige typer af stimuli. Til undersøgelse af aktivitet i encephalon og nervesystemets funktion generelt findes flere forskellige metoder.
  
\subsection{fMRI}
Functional Magnetic Resonance Imaging(fMRI) er en metode til at undersøge aktiviteten i neuronerne i hjernen.
Generelt er MRI en metode til at synliggøre protoner; dermed er det muligt at afbilde kroppens væv, da dette primært er udgjort af hydrogen-atomer, som indeholder netop én proton.     
Ved en MRI-scanning udsættes objektet, eksempelvis en patient, for et eksternt magnetfelt, hvilket medfører, at der opstår et parallelt magnetfelt i objektet. Dermed er det muligt at detektere de tilstedeværende protoner, da deres retning skaber det parallelle magnetfelt i kroppen.\citep{Wals2009}
Der findes forskellige teknikker til fMRI, hvoraf de fleste anvender Blood Oxygenation Level-Dependent(BOLD) kontrast. Ved anvendelse af denne kontrast, udnyttes det, at der ved aktivitet i hjernens forskellige områder, vil ske en ændring af mængden af ilt i blodet, hvilket påvirker magnetfeltets styrke. Dermed vil det, ved hjælp af kontrastvæsken, være muligt at følge blodgennemstrømningens styrke i hjernens forskellige områder. \citep{Wals2009}

\subsubsection{Anvendelse til detektion af smerte}
fMRI har i flere studier været anvendt til at undersøge neural aktivitet i forbindelse med knæsmerter.
I et studie af \citep{Parks2012} er det ved hjælp af fMRI blevet undersøgt, hvorvidt der er forskel på oplevelsen af kroniske smerter hos patienter med artrose og på smerter fremkaldt med tryk. Forsøgsdeltagerne var opdelt i en gruppe patienter med artrose i knæet og en gruppe raske individer. \textbf{xxxx} viste at de to grupper oplevede den kunstigt fremkaldte knæsmerte ens; der forekom stort set ingen variationer imellem de to grupper. Derimod viste der sig at være forskel på de kroniske smerter og de kunstigt fremkaldte smerter hos patienterne med artrose. \citep{Parks2012} 
I et andet studie af \citep{Hiramatsu2014} er det undersøgt, hvilke forskelle der forekom i cerebral respons hos en gruppe raske forsøgspersoner og en gruppe forsøgspersoner med kroniske knæsmerter forårsaget af artrose. Begge grupper blev udsat for akut smerte gennem invasive elektroder imens der blev foretaget en fMRI scanning. Studiet viste, at der hos gruppen af patienter med kroniske knæsmerter forekom en højere aktivitet i det dorsolaterale præfrontale cortex end hos gruppen af raske forsøgspersoner. \citep{Hiramatsu2014} 
fMRI kan således påvise forskelle på hvordan smerte opleves af forskellige grupper af mennesker. Nedenstående billeder viser et eksempel på aktiviteten i encephalon hos henholdsvis en person der oplever smerte og en person, der ikke oplever smerte. 

\subsection{Quantitative sensory testing}
Quantitative sensory testing(QST) er en metode til undersøgelse af det sensoriske nervesystems funktion. Metoden kan anvendes til undersøgelse af forskellige egenskaber, herunder smertegrænser. Ved en QST-undersøgelse eksponeres patienten for forskellig sensorisk sensation, herunder tests, der involverer varme og kulde, vibration og tryk. Dermed er det muligt at identificere grænserne for henholdsvis perception og smerte. \citep{Yarnitsky2006}
Der findes forskellige protokoller til QST-undersøgelse, der alle indeholder tests indenfor de forskellige områder, der ønskes undersøgt ved en QST. Et eksempel er en QST-protokol udviklet af German Network on Neuropathic Pain til undersøgelse af neuropatisk smerte. Ved brug af denne model anvendes syv forskellige tests til at vurdere 13 parametre, herunder seks temperaturtests til detektion af grænser for perception og smerte samt syv mekaniske tests til detektion af tilsvarende grænser, her med henholdsvis spidste og stumpe genstande. I forbindelse med udarbejdelse af modellen er der desuden lavet tests for at opstille referenceværdier, der tager højde for både køn og alder. citep{8}  

\subsubsection{Anvendelse til detektion af smerte}
QST bliver i forskningsregi anvendt til undersøgelse af patienter der får udført TKA. I et studie af \citep{Martinez2007} er en anden QST-protokol anvendt til at undersøge 20 patienter med artrose i knæet før og efter en TKA. Formålet hermed var at identificere faktorer, der har indvirkning på udviklingen af postoperative smerter efter operationen. QST-undersøgelserne blev udført henholdsvis før operationen og efterfølgende en og fire dage samt en og fire måneder efter operationen. Parametrene der blev undersøgt i QST-undersøgelsen var tærskelværdier for temperatur, mekanisk smerte og hvordan de enkelte patienter responderede på en eksponering for temperaturer over tærskelværdierne. Studiet fandt, at der forekommer en sammenhæng mellem periodiske smerter efter operationen og de patienter, der oplever hyperalgesi under eksponering for varme. \citep{Martinez2007} Der skal imidlertid tages højde for, at der er flere fejlkilder forbundet med QST-undersøgelser, da nøjagtigheden i høj grad afhænger af både patientens og undersøgerens præcision under udførelsen af de enkelte tests. Det må således forventes, at der kan forekomme variationer mellem enkelte QST-undersøgelser udført på den samme patient. \citep{Yarnitsky2006}

\subsection{Elektrofysiologiske undersøgelsesmetoder}
De elektrofysiologiske undersøgelsesmetoder dækker over tests, der kontrollerer elektrisk aktivitet i kroppens celler. I hviletilstand har et neuron en fast spænding over sin membran. Når dette membranpotentiale ændres som følge af ændringer i koncentrationen af natrium- og kaliumioner inden- og udenfor cellen, genereres et aktionspotentiale, som vandrer langs cellens akson og videre til de følgende neuroner. Dette kan detekteres ved anvendelse af invasive eller noninvasive elektroder. Til monitorering af neuronernes funktionalitet i encephalon anvendes elektroencephalografi (EEG) mens der til monitorering af neuronernes funktionalitet i resten af kroppen anvendes elektroneurografi (ENG). For at en elektrofysiologisk undersøgelse kan anvendes til at stille en diagnose, skal resultaterne herfra understøttes af andre kliniske undersøgelser, herunder prøver og lægesamtaler. \citep{Robinson2008}

\subsubsection{Anvendelse til detektion af smerte}
I et studie af \citep{Brown2013}, er EEG blevet anvendt til at undersøge, hvordan to patientgrupper med forskellige sygdomme opfatter smerte. I studiet indgik en gruppe af patienter med artrose og en gruppe patienter med sygdommen fibromyalgi, der forårsager muskel- og ledsmerter \citep{Brown2013}\citep{9}. Hos de to patientgrupper blev det undersøgt, hvordan encephalon genererer signal for henholdsvis en forventning om og en decideret udløst smerte. Disse data sammenlignes med en kontrolgruppe bestående af raske, smertefri personer. Der blev anvendt 61 elektroder til opsamling af EEG-signaler fra forsøgspersonerne og analyseret med henblik på at identificere aktivitet i de enkelte områder af hjernen samt . Resultaterne for studiet antyder, at de to patientgrupper responderer ens på forventningen om smerte, på trods af, at fibromyalgipatienternes respons var kraftigere. Disse resultater indikerer, at denne tendens kan være generelt gældende for patienter med kroniske smerter. \citep{Brown2013}
 % 1 = fMRI techniques and protocols
 % 2 = The dorsolateral prefrontal network is involved in pain perception in knee osteoarthritis patients
 % 3 = Brain activity for chronic knee osteoarthritis: dissociating evoked pain from spontaneous pain
 % 4 = Quantitative sensory testing
 % 5 = The evolution of primary hyperalgesia in orthopedic surgery: Quantitative sensory testing and clinical evaluation before and after total knee arthroplasty
 % 6 = Clinical electrophysiology
 % 7 = When the brain expects pain: common neural responses to pain anticipation are related to clinical pain and distress in fibromyalgia and osteoarthritis 