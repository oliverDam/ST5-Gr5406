\section{Klinisk udvælgelse af patienter}
Patienter som tilbydes en TKA udvælges på baggrund af en læge eller kirurgs observationer og erfaringer.
Hermed afhænger udvælgelsen af patienter til en TKA operation alene af klinikeren, hvormed patienter kan opleve forskellige anbefalinger og behandlingsmuligheder ved forskellige klinikere. I et forsøg på at standardisere behandlingen af knæartrose for alle patienter i Danmark har Sundhedsstyrrelsen udarbejdet en rapport indeholdene nationale kliniske retningslinjer. Disse retningslinjer bygger hovedsageligt på lægeligt konsensus. Retningslinjerne omhandlende tilbud af TKA til patienter indeholder blandt andet at patienter kun tilbydes en TKA hvis ikke-kirurgiske behandlingsmetoder ikke har haft en tilstrækkelig virkning. En TKA kan dog tilbydes patienter som den første behandlingsmulighed hvis lægen/kirugen vurderer at patientens artrose er så svær ingen ikke-kirugiske behandlingsmuligheder vil have en tilstrækkelig effekt. Dette kan eksempelvis være hvis patienten har en svær fejlstilling af knæet eller svær instabilitet i leddet. \citep{brostrom2012} \\
Sundhedsstyrrelsen har ligeledes opstillet en række indikationer som kan få klinikeren til at fravælge en patient til en operation. Disse indikationer er eksempelvis hvis der er infektion i knoglen eller leddet, hvis patienten ingen smerter har i leddet eller hvis patienten har en kort forventet levetid. En anden indikator, som kan få en kliniker til at fravælge at operere patienten, er hvis patienten har urealistiske forventninger til operationen. \citep{brostrom2012} I et studie har \cite{tejada2010} vist at patienter hvis forventninger til operationen bliver opfyldt, oplever større tilfredshed efter operationen. Ligeledes er det vist at klinikeres forventninger til en operation påvirker patientens forventninger \citep{tejada2010}. Dette antyder at klinikere ved at forklare patienten hvad de kan forvente af operationen, kan være med til at mindske eller helt fjerne faktoren omhandlende urealistiske forventninger, hvilket kan betyde, at patienten bliver tilbudt en TKA. \\
Ud fra et studie af \cite{skou2016} anses flere af de ovennævnte retningslinjer som nogle af de vigtigste overvejelser når en ortopædkirurg skal bestemme om en patient er egnet til at modtage en TKA. \cite{skou2016} fandt at ortopædkirurger anser radiografisk omfang, smerte i knæet ved udførelse af hverdagsaktiviteter, funktionelle begrænsninger samt utilfredsstillende virkning af ikke-kirurgiske behandlingsmetoder som de fire vigtigste faktorer for en patients egnethed til TKA. Det blev af \cite{skou2016} undersøgt om de faktorer kirurgerne mente var de vigtigste for en patients egnethed for en TKA, blev afspejlet i hvilke patienter ortopædkirugerne tilbød en TKA. Her blev det fundet at radiografisk omfang samt funktionelle begrænsninger var betydningsfulde i forhold til patientens egnethed til TKA. De to andre faktorer, smerte i knæet samt utilfredsstillende ikke-kirugriske behandlinger, var ikke drivene for ortopædkirurgens vurdering af patientens egnethed til TKA. Hermed blev der af \cite{skou2016} fundet en uoverenstemmelse mellem hvilke faktorer kirurgerne fandt vigtigst for patientens egnethed til TKA og hvilke faktorer de samme kirurger udvalgte patienter ud fra. Denne uoverensstemmelse viser hvor kompleks udvælgelsesprocessen for en TKA er, samt vanskelighederne ved at bestmme hvilke faktorer der har størst betydning for en klinikers beslutningsprocess. Udfra resultaterne fra studiet af \cite{skou2016} antydes det at klinikerne anvender både bevidste og ubevidste faktorer til bestemmelse af patienters egnethed til en TKA.    
    
Flere studier har undersøgt hvilke ubevidste faktorer der kan påvirke en klinikers beslutningstagning. Eksempelvis viser resultater fra et studie udarbejdet af \cite{borkhoff2008} at en patients køn har betydning for om klinikeren tilbyder patienten en TKA eller ej. I dette studie anvendte \cite{borkhoff2008} to standardiserede patienter med moderat knæartrose som besøgte 71 klinikere, bestående af 38 alment praktiserende læger og 33 ortopædkiruger. Klinikerne som deltog i studiet blev ikke informeret om hvem de to standardiserede patienter var. De to standardiserede patienter var ens på alle andre punkter end køn. \cite{borkhoff2008} fandt at 42\% af lægerne kun tilbød den mandlige patient en TKA, mens kun 8\% af lægerne kun tilbød den kvindelige patient en TKA. Hermed er sandsynligheden for at en kvindelig patient med moderart knæartrose får tilbudt en TKA betydeligt mindre end sandsynligheden for at en mandlig patient får tilbudt operationen. Dette er problematisk da en større andel af knæartrose patienterne er kvinder, hvor flere af disse hermed først vil blive tilbudt en TKA når deres knæartrose er forværret. Det indikeres at resultatet af operationen bliver forværret i takt med at kompleksiteten af patientens artrose stiger. Dette betyder at flere af de kvindelige artrosepatienter får et forværret resultat kontra hvis de var blevet tilbudt operationen ligestillet med de mandelige patienter. \citep{borkhoff2008} \\
I en spørgeskemaundersøgelse hvor ortopædkiruger blev bedt om at vurdere betydningen af en patients køn i forhold til om de ville henstille patienten til en TKA eller ej svarede cirka 93\% af de adspurgte kiruger at køn ikke ville have nogen betydning for deres vurdering af patienten \citep{wright1995}. Dette antyder at den forskel \cite{borkhoff2008} fandt i deres forsøg, er som følge af en underbevidst bias hos klinikerne. Denne teori understøttes af resultaterne fra et studie udarbejdet af \cite{dy2014}, hvor det blev undersøgt om en patients race og køn ville have betydning for ortopædkirurgers vurdering af patienter. I dette studie blev videoer med patienter med svær knæartrose anvendt. Ligesom i studiet af \cite{borkhoff2008} var patienterne kun forskellige i køn og race. \cite{dy2014} fandt ingen signifikant forskel på kirugernes vurdering af de fire standardiserede patienter. Denne forskel i resultaterne mellem \cite{borkhoff2008} og \cite{dy2014} kan skyldes forskellen i patienternes grad af artrose. Ved patienter med svær artrose ses ingen bias, mens der ved patienter med moderat artrose blev fundet bias. Dette antyder at kirurgernes bias kun har betydning for patienter, hvor der ikke er fuldstændig klare indikationer på at patienten skal opereres. \\
I tilfælde hvor patienten ikke har klare indikationer på en operation, ville en metode der vil kunne hjælpe klinikeren med at lave en vurdering uden bias være fordelagtig. Ved anvendelse af en sådan metode vil det ligeledes være muligt at inddrage andre faktorer i udvælgelsen af patienter til TKA. Dette vil eksempelvis kunne gøres ved at anvende en metode som kan hjælpe klinikeren med at vurdere patientens risiko for at udvikle kroniske smerter efter operationen. Hermed vil det teoretisk være muligt at finde alternative behandlingsmetoder til de cirka 20\% af TKA patienter som får kronisk smerte postoperativt. Ligeledes vil en sådan metode kunne være med til at systematisere henstillingen af patienter til en TKA, således det sikres at patienten får tilbudt samme behandling uafhængigt af hvilken kliniker der vurderer patienten. 


%Internationale retningslinjer for behandling af patienter med artrose er opstillet af Osteoarthritis Research Society International (OARSI). Disse retningslinjer er udarbejdet af tværfagligt sundhedspersonale fra seks forskellige lande i Nordamerika og Europa. Retninglinjerne fra OARSI omhandlende knæalloplastik specificerer ligeledes at kun patienter som ikke har oplevet tilstækkelig virkning fra ikke-kirugiske behandlingsmetoder skal tilbydes en TKA. Desuden anses TKA af OARSI som en pålidelig og passende behandlingsmetode til at genoprette funktion og højne patientens livskvalitet. \citep{zhang2008} \\
%Udfra disse retninglinjer skal alle knæartrose patienter som ikke opnår den ønskede virkning ved ikke-kirurgiske behandlingsmuligheder tilbydes en TKA. I praksis er det dog ikke alle patienter der opfylder de opstillede retningslinjer, som tilbydes en TKA. \citep{borkhoff2008} For en udefrakommende kan udvælgelsen af patienter som får tilbudt en operation og patienter som ikke gør, virke tilfældigt. \textbf{Læs Stens artikel og undersøg om der kan findes understøttende udtalelser til sætningen så denne sætning ikke virker så hård (påstands-agtig).} Ligeledes kan der være forskel lægernes vurdering imellem, hvormed den ene læge ville tilbyde en patient operationen, mens en anden læge ikke ville. Sundhedsstyrrelsen har opstillet en række indikationer som kan få klinikeren til at fravælge patienten til en operation. Disse indikationer er eksempelvis hvis der er infektion i knoglen eller leddet, hvis patienten ingen smerter har i leddet eller hvis patienten har en kort forventet levetid. En anden indikator, som kan få en kliniker til at fravælge at operere patienten, er hvis patienten har urealistiske forventninger til operationen. \citep{brostrom2012} I et studie har \cite{tejada2010} vist at patienter hvis forventninger til operationen bliver opfyldt oplever større tilfredshed efter operationen. Ligeledes er det vist at klinikeres forventninger til en operation påvirker patientens forventninger \citep{tejada2010}. Dette antyder at klinikere ved at forklare patienten hvad de kan forvente af operationen, kan være med til at mindske eller helt fjerne en af faktorerne som betyder, at patienten ikke bliver tilbudt en TKA. \\

   
