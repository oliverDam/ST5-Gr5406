\section{Klinisk udvælgelse af patienter}\label{kliniskudvaelgelse}
\textit{I dette afsnit undersøges det hvilke retningslinjer og faktorer, der har betydning for en klinikers henvisning af en patient til en TKA-operation. Dette gøres med henblik på en identifikation af eventuelle problematikker ved en udvælgelse af patienter, som hovedsageligt er baseret på en klinikers vurdering.}

Patienter, som tilbydes en TKA-operation, udvælges på baggrund af en klinikerens observationer og erfaringer. Hermed afhænger udvælgelsen af patienter til en TKA-operation af klinikere, hvormed patienter kan opleve varierende anbefalinger og behandlingsmuligheder afhængig af hvilken kliniker, der er ansvarlig for vurderingen. I et forsøg på at standardisere behandlingen af knæartrose for alle patienter i Danmark har Sundhedsstyrelsen udarbejdet en rapport indeholdende nationale kliniske retningslinjer. Disse retningslinjer bygger hovedsageligt på lægelig konsensus. Retningslinjerne vedrørende tilbud om en TKA-operation indeholder blandt andet, at patienter kun tilbydes en TKA-operation hvis non-invasive behandlingsmetoder ikke har en tilstrækkelig virkning. En TKA-operation kan tilbydes patienter som den første behandlingsmulighed, hvis klinikeren vurderer, at patientens artrose er så svær, at ingen non-invasive behandlingsmetoder vil have en tilstrækkelig effekt. Dette kan eksempelvis være, hvis patienten har en kraftig fejlstilling af knæet eller svær instabilitet i leddet. \citep{brostrom2012} \\
Sundhedsstyrelsen har ligeledes opstillet en række indikationer, som kan få klinikeren til at vurderer at en patient ikke er egnet til en TKA-operation. Disse indikationer er eksempelvis, hvis der er infektion i knoglen eller leddet, hvis patienten ingen smerter har i leddet eller hvis patienten har en kort forventet levetid. En anden indikator, som kan få en kliniker til at fravælge at operere patienten, er hvis patienten har urealistiske forventninger til operationen. \citep{brostrom2012} I et studie har \citer{tejada2010} vist, at patienter, hvis forventninger til operationen bliver opfyldt, oplever større tilfredshed efter operationen. Ligeledes er det vist, at klinikerens forventninger til en operation påvirker patientens forventninger \citep{tejada2010}. Dette antyder, at klinikere, ved at forklare patienten hvad de kan forvente af operationen, kan være med til at mindske eller helt fjerne faktoren vedrørende urealistiske forventninger. \\
Ud fra et studie af \citer{skou2016} anses flere af de ovennævnte retningslinjer som nogle af de vigtigste overvejelser, når en kliniker skal bestemme, om en patient er egnet til at modtage en TKA. \citer{skou2016} fandt, at klinikeren anser røntgenresultater, smerte i knæet ved udførelse af hverdagsaktiviteter, funktionelle begrænsninger samt utilfredsstillende virkning af non-invasive behandlingsmetoder som de fire vigtigste faktorer for en patients egnethed til en TKA-operation. Det blev af \citer{skou2016} undersøgt om de faktorer, klinikerne mente var de vigtigste for en patients egnethed for en TKA-operation, blev afspejlet i hvilke patienter klinikerne tilbød en TKA-operation. Her blev det fundet, at radiologiske resultater samt funktionelle begrænsninger var betydningsfulde i forhold til patientens egnethed til en TKA-operation. De to andre faktorer, smerte i knæet samt utilfredsstillende non-invasive behandlinger, var ikke drivende for klinikerens vurdering af patientens egnethed til en TKA-operation. Hermed blev der af \citer{skou2016} fundet en uoverensstemmelse mellem hvilke faktorer klinikerne fandt vigtigst i forhold til vurdering af behovet for en TKA-operation og hvilke faktorer de samme klinikere udvalgte patienter ud fra. Denne uoverensstemmelse viser, hvor kompleks udvælgelsesprocessen for en TKA er samt vanskelighederne ved at bestemme hvilke faktorer, der har størst betydning for en klinikers beslutningsprocess. Ud fra resultaterne fra studiet af \citer{skou2016} antydes det, at klinikerne anvender både bevidste og ubevidste faktorer til bestemmelse af patienters egnethed til en TKA-operation. Flere studier har undersøgt hvilke ubevidste faktorer, der kan påvirke en klinikers beslutningstagning. Eksempelvis antyder resultater fra et studie udarbejdet af \citer{borkhoff2008}, at en patients køn har betydning for, om klinikeren tilbyder patienten en TKA eller ej. I dette studie anvendte \citer{borkhoff2008} to standardiserede patienter med moderat knæartrose, som besøgte 71 klinikere, bestående af 38 alment praktiserende læger og 33 ortopædkiruger. Klinikerne, som deltog i studiet, blev ikke informeret om hvem de to standardiserede patienter var. De to standardiserede patienter var ens på alle andre punkter end køn. \citer{borkhoff2008} fandt, at 55\% af ortopædkirugerne kun tilbød den mandlige patient en TKA-operation, mens ingen af ortopædkirugerne kun tilbød den kvindelige patient en TKA-operation. Sandsynligheden hvormed en patient vil få tilbudt en TKA-operation afhænger af flere parametre. Eksempelvis kan patientens kommunikationsmetode have indflydelse på beslutningen vedrørende det efterfølgende behandlingsforløb. \citep{borkhoff2008}

 % Mulige grunde til den fundne bias i studiet af \cite{borkhoff2008} kan være, at størstedelen af klinikerne der deltog i forsøget var mænd. Ud af de 71 klinikere som deltog i forsøget var 59 af disse mænd. Resultaterne fra studiet viser ingen signifikant forskel i andelen af mandlige klinikere som kun tilbød den mandlige patient en TKA, og andelen af kvindelige klinikere som kun tilbød den mandlige patient en TKA. Det kunne dog undersøges om klinikere har tendens til oftere at henvise patienter af deres eget køn til en TKA end patienter af det modsatte køn. Dette kunne være relevant at undersøge da studier har vist forskelle i mænd og kvinders kommunikationsmetoder, hvormed klinikerens eget køn kan have en betydning for hvordan en patients beskrivelse af sygdommen opfattes og dermed vurderes. \citep{street2002}   
% Bemærk at det med den signifikante forskel handler om mænd vælger mænd etc. Der var en signikant forskel vedr. om det kun er mænd der bliver valgt
I en spørgeskemaundersøgelse, hvor ortopædkiruger blev bedt om at vurdere betydningen af en patients køn i forhold til, om de ville henstille patienten til en TKA-operation eller ej, svarede cirka 93~\% af de adspurgte klinikere, at køn ikke ville have nogen betydning for deres vurdering af patienten \citep{wright1995}. Dette kan antyde, at den forskel \citer{borkhoff2008} fandt i deres forsøg, er følge af en ubevidst bias hos klinikerne. Teorien understøttes også af resultaterne fra et studie udarbejdet af \citer{dy2014}, hvor det blev undersøgt, om en patients race og køn ville have betydning for klinikerens vurdering af patienter.\\
I dette studie blev videoer med patienter med svær knæartrose anvendt. Ligesom i studiet af \citer{borkhoff2008} var patienterne kun forskellige i køn og race. \citer{dy2014} fandt ingen forskel på klinikerens vurdering af de fire standardiserede patienter. Forskellen i resultaterne mellem \citer{borkhoff2008} og \citer{dy2014} kan skyldes forskellen i patienternes grad af artrose. Ved patienter med svær artrose forekom ingen bias, mens der ved patienter med moderat artrose blev fundet bias. Dette antyder, at klinikernes bias kun har betydning for patienter, hvor der ikke er klare indikationer på, at patienten bør opereres.\\
I tilfælde hvor der ikke er klare indikationer på, at patienten vil opnå bedring ved en TKA-operation, vil en teknologi, der kan hjælpe klinikeren med at udføre en vurdering uden bias være fordelagtig. Ved anvendelse af en sådan teknologi vil det være muligt at inddrage andre faktorer i udvælgelsen af patienter til TKA-operation. Dette vil eksempelvis kunne gøres ved at anvende en teknologi, som kan hjælpe klinikere med at vurdere patienters risiko for at udvikle kroniske smerter efter TKA-operationen. Hermed vil det være muligt at mindske fordelingen imellem den tekniske succes og den patientmæssig succes ved en TKA-operation.

Set i forhold til de opstillede succeskriterier for en TKA-operation beskrevet i \secref{tek_succes} er stort set alle TKA-operationer succesfulde, mens omkring 80\% af patienterne, der gennemgår en TKA-operation, ikke oplever kroniske smerter. \citep{Sakellariou2016} \citep{Petersen2015} En succesfuld operation kan dermed defineres forskelligt alt efter hvilken synsvinkel, der arbejdes ud fra. Ud fra en patientmæssig synsvinkel defineres en succesfuld TKA-operation som en operation, hvor patienten er tilfreds med resultatet af denne. Ud fra et teknisk perspektiv er en TKA-operation succesfuld, når denne opfylder kravene opstillet af DKA \citep{aarsrapport2016}. Problematikken opstår ved de TKA-operationer, som kun er succesfulde ud fra et teknisk perspektiv. For at mindske andelen af disse vil det være fordelagtigt at finde en teknologi, som hæver antallet af tilfredse patienter uden at sænke antallet af operationer, som opfylder de tekniske krav. Da kronisk smerte er en af de hyppigste årsager til utilfredshed blandt patienter efter en TKA-operation, vil en reduktion i antallet af patienter med kroniske postoperative smerter sænke antallet af utilfredse patienter. \citep{Bourne2010} 

Udfra disse kriterier vil en teknologi, som gør det muligt at identificere hvilke patienter, der er i risikogruppen for at få kroniske smerter efter en TKA-operation, være fordelagtig. Hermed vil en del af de TKA-operationer, som ikke gavner patienten, kunne undgås. De ovenstående kriterier vil potentielt kunne opfyldes af flere forskellige teknologier. En analyse af disse teknologier vil derfor kunne identificere den mest optimale teknologi til præoperativ undersøgelse af patienter, således patienter med risiko for udvikling af kroniske postoperative smerter kan identificeres.

