\section{Klinisk udvælgelse af patienter}
\textit{I dette afsnit undersøges det hvilke retningslinjer og faktorer der har betydning for en klinikers udvælgelse af en patient til TKA. Dette gøres med henblik på en bestemmelse af eventuelle problematikker ved en udvælgelse af patienter som kun er baseret på en klinikers vurdering. Hermed vil det være muligt senere at undersøge forskellige metoder som kan understøtte klinikerens vurdering.}

Patienter som tilbydes en TKA udvælges på baggrund af en læge eller kirurgs observationer og erfaringer.
\textbf{13} Hermed afhænger udvælgelsen af patienter til en TKA operation af klinikere, hvormed patienter kan opleve forskellige anbefalinger og behandlingsmuligheder ved forskellige klinikere. \textbf{13} I et forsøg på at standardisere behandlingen af knæartrose for alle patienter i Danmark har Sundhedsstyrrelsen udarbejdet en rapport indeholdene nationale kliniske retningslinjer. Disse retningslinjer bygger hovedsageligt på lægeligt konsensus. Retningslinjerne omhandlende tilbud af TKA til patienter indeholder blandt andet at patienter kun tilbydes en TKA hvis ikke-kirurgiske behandlingsmetoder ikke har haft en tilstrækkelig virkning. En TKA kan dog tilbydes patienter som den første behandlingsmulighed hvis lægen/kirugen vurderer at patientens artrose er så svær ingen ikke-kirugiske behandlingsmuligheder vil have en tilstrækkelig effekt. Dette kan eksempelvis være hvis patienten har en svær fejlstilling af knæet eller svær instabilitet i leddet. \citep{brostrom2012} \\
Sundhedsstyrrelsen har ligeledes opstillet en række indikationer som kan få klinikeren til at fravælge en patient til en operation. Disse indikationer er eksempelvis hvis der er infektion i knoglen eller leddet, hvis patienten ingen smerter har i leddet eller hvis patienten har en kort forventet levetid. En anden indikator, som kan få en kliniker til at fravælge at operere patienten, er hvis patienten har urealistiske forventninger til operationen. \citep{brostrom2012} I et studie har \cite{tejada2010} vist at patienter hvis forventninger til operationen bliver opfyldt, oplever større tilfredshed efter operationen. Ligeledes er det vist at klinikeres forventninger til en operation påvirker patientens forventninger \citep{tejada2010}. Dette antyder at klinikere ved at forklare patienten hvad de kan forvente af operationen, kan være med til at mindske eller helt fjerne faktoren omhandlende urealistiske forventninger, hvilket kan betyde, at patienten bliver tilbudt en TKA. \\
Ud fra et studie af \cite{skou2016} anses flere af de ovennævnte retningslinjer som nogle af de vigtigste overvejelser når en ortopædkirurg skal bestemme om en patient er egnet til at modtage en TKA. \cite{skou2016} fandt at ortopædkirurger anser radiografisk omfang, smerte i knæet ved udførelse af hverdagsaktiviteter, funktionelle begrænsninger samt utilfredsstillende virkning af ikke-kirurgiske behandlingsmetoder som de fire vigtigste faktorer for en patients egnethed til TKA. Det blev af \cite{skou2016} undersøgt om de faktorer kirurgerne mente var de vigtigste for en patients egnethed for en TKA, blev afspejlet i hvilke patienter ortopædkirugerne tilbød en TKA. Her blev det fundet at radiografisk omfang samt funktionelle begrænsninger var betydningsfulde i forhold til patientens egnethed til TKA. De to andre faktorer, smerte i knæet samt utilfredsstillende ikke-kirugriske behandlinger, var ikke drivene for ortopædkirurgens vurdering af patientens egnethed til TKA. Hermed blev der af \cite{skou2016} fundet en uoverenstemmelse mellem hvilke faktorer kirurgerne fandt vigtigst for patientens egnethed til TKA og hvilke faktorer de samme kirurger udvalgte patienter ud fra. Denne uoverensstemmelse viser hvor kompleks udvælgelsesprocessen for en TKA er, samt vanskelighederne ved at bestmme hvilke faktorer der har størst betydning for en klinikers beslutningsprocess. Udfra resultaterne fra studiet af \cite{skou2016} antydes det at klinikerne anvender både bevidste og ubevidste faktorer til bestemmelse af patienters egnethed til en TKA.    
    
Flere studier har undersøgt hvilke ubevidste faktorer der kan påvirke en klinikers beslutningstagning. Eksempelvis viser resultater fra et studie udarbejdet af \cite{borkhoff2008} at en patients køn har betydning for om klinikeren tilbyder patienten en TKA eller ej. I dette studie anvendte \cite{borkhoff2008} to standardiserede patienter med moderat knæartrose som besøgte 71 klinikere, bestående af 38 alment praktiserende læger og 33 ortopædkiruger. Klinikerne som deltog i studiet blev ikke informeret om hvem de to standardiserede patienter var. De to standardiserede patienter var ens på alle andre punkter end køn. \cite{borkhoff2008} fandt at 55\% af ortopædkirugerne og 32\% af de praktiserende læger kun tilbød den mandlige patient en TKA, mens ingen af ortopædkirugerne og 13\% af de praktiserende læger kun tilbød den kvindelige patient en TKA. Hermed er sandsynligheden for at en kvindelig patient med moderart knæartrose, ud fra resultaterne fra \cite{borkhoff2008}, får tilbudt en TKA betydeligt mindre end sandsynligheden for at en mandlig patient får tilbudt operationen. Dette er problematisk da en større andel af knæartrose patienterne er kvinder, hvor flere af disse hermed først vil blive tilbudt en TKA når deres knæartrose er forværret. Det indikeres at resultatet af operationen bliver forværret i takt med at kompleksiteten af patientens artrose stiger. \citep{fortin1999} Dette betyder at flere af de kvindelige artrosepatienter får et forværret resultat end hvis de var blevet tilbudt operationen ligestillet med de mandlige patienter. \citep{borkhoff2008} \\
\textbf{14} Mulige grunde til den fundne bias i studiet af \cite{borkhoff2008} kan være, at størstedelen af klinikerne der deltog i forsøget var mænd. Ud af de 71 klinikere som deltog i forsøget var 59 af disse mænd. Resultaterne fra studiet viser ingen signifikant forskel i andelen af mandlige klinikere som kun tilbød den mandlige patient en TKA, og andelen af kvindelige klinikere som kun tilbød den mandlige patient en TKA. Det kunne dog undersøges om klinikere har tendens til oftere at henvise patienter af deres eget køn til en TKA end patienter af det modsatte køn. Dette kunne være relevant at undersøge da studier har vist forskelle i mænd og kvinders kommunikationsmetoder, hvormed klinikerens eget køn kan have en betydning for hvordan en patients beskrivelse af sygdommen opfattes og dermed vurderes. \citep{street2002} \textbf{14}    

I en spørgeskemaundersøgelse hvor ortopædkiruger blev bedt om at vurdere betydningen af en patients køn i forhold til om de ville henstille patienten til en TKA eller ej svarede cirka 93\% af de adspurgte kiruger at køn ikke ville have nogen betydning for deres vurdering af patienten \citep{wright1995}. Dette antyder at den forskel \cite{borkhoff2008} fandt i deres forsøg, er som følge af en underbevidst bias hos klinikerne. Denne teori understøttes af resultaterne fra et studie udarbejdet af \cite{dy2014}, hvor det blev undersøgt om en patients race og køn ville have betydning for ortopædkirurgers vurdering af patienter. I dette studie blev videoer med patienter med svær knæartrose anvendt. Ligesom i studiet af \cite{borkhoff2008} var patienterne kun forskellige i køn og race. \cite{dy2014} fandt ingen signifikant forskel på kirugernes vurdering af de fire standardiserede patienter. Denne forskel i resultaterne mellem \cite{borkhoff2008} og \cite{dy2014} kan skyldes forskellen i patienternes grad af artrose. Ved patienter med svær artrose ses ingen bias, mens der ved patienter med moderat artrose blev fundet bias. Dette antyder at kirurgernes bias kun har betydning for patienter, hvor der ikke er fuldstændig klare indikationer på at patienten skal opereres. 

\textbf{15}
I tilfælde hvor der ikke er klare indikationer på at patienten vil have gavn af en TKA, ville en metode der vil kunne hjælpe klinikeren med at udføre en vurdering uden bias være fordelagtig. Ved anvendelse af en sådan metode vil det ligeledes være muligt at inddrage andre faktorer i udvælgelsen af patienter til TKA. Dette vil eksempelvis kunne gøres ved at anvende en metode som kan hjælpe klinikere med at vurdere patienters risiko for at udvikle kroniske smerter efter TKA operationen. Hermed vil det være muligt at mindske fordelingen imellem den naturvidenskabelige succes og den humane succes ved en TKA operation. Ud fra et naturvidenskabeligt synspunkt er stort set alle TKA operationer succesfulde, mens kun omkring 80\% af alle TKA operationer er succesfulde ud fra et humant synspunkt. \citep{aarsrapport2016} \citep{Bourne2010} En succesfuld operation defineres dermed forskelligt alt efter hvilken synsvinkel der arbejdes ud fra. Ud fra en human synsvinkel defineres en succesfuld TKA operation som en operation hvor patienten er tilfreds med resultatet af denne. Ud fra en naturvidenskabelig synsvinkel er en TKA operation succesfuld når denne opfylder kravene opstillet af \cite{aarsrapport2016}. Problematikken opstår ved de 20\% af TKA operationerne som kun er succesfulde ud fra en naturvidenskabelig vinkel. For at mindske denne procentdel vil det være fordelagtigt at finde en metode som hæver antallet af tilfredse patienter uden at sænke antallet af operationer som opfylder de naturvidenskabelige krav. Da kronisk smerte er en af de hyppigste årsager til utilfredshed blandt patienter efter en TKA operation, vil det at minimere antallet af patienter med postoperativ kronisk smerte kunne sænke antallet af utilfredse patienter. \citep{Bourne2010} For ikke at mindske andelen af naturvidenskabeligt succesfulde TKA operationer, vil en metode som ikke påvirker selve operationsteknikken være at foretrække. Ligeledes skal metoden kunne mindske eventuelle bias fra klinikere, og systematisere udvælgelsen af patienter til en TKA. Udfra disse kriterier vil en metode, som gør det muligt at forudsige patienter, som er i risikogruppen for at få kroniske smerter efter en TKA, være fordelagtig. Hermed vil en del af de TKA operationer som ikke gavner patienten kunne undgås, hvilket ville bidrage til at andelen af succesfulde operationer ud fra et humant synspunkt vil stige. Denne metode vil dog give en ny humanistisk problemstilling, idet de patienter som ikke tilbydes en TKA på grund af udvælgelsesmetoden, stadig vil have moderat til svær knæartrose. Disse patienter skal tilbydes en alternativ behandlingsmetode, for at løse den tilbageværende humanistiske problemstilling. Den alternative behandlingsmetode skal afhjælpe knæartrosen uden stor risiko for at patienten får kroniske smerter. Dette er tilfældet da problematikken blot flyttes til et andet område, hvis patienten ikke udredes. En sådan alternativ behandlingsmetode vil kunne undersøges nærmere efter en metode til forudsigelse af patienter med postoperative kroniske smerter er fundet. \\  
De ovenstående kriterier vil kunne opfyldes af flere forskellige metoder. En analyse af disse metoder vil derfor kunne antyde hvilken af disse metoder der vil være mest fordelagtig at anvende til præoperativ undersøgelse af patienter, således patienter med risiko for udvikling af postoperative kroniske smerter kan identificeres. \textbf{15}        

