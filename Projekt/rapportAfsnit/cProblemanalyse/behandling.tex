Hejhejhej
\section{Behandling}
Behandlingsforløbet af en patient med knæartrose indeholder flere komponenter, og kan både være kirurgiske og non-kirurgiske alt afhængt af graden af artrose.
\subsection*{Knæledet}
Knæet, \textit{articulatio genus}, er et synovialt, sammensat led med en bevægelsesgrad fra 0 til 135$\degree$ fleksion til 0 til 5$\degree$ ekstension. Knæleddet er legemets største led, hvormed det også er udsat for større mekaniske påvirkninger end noget andet led i kroppen. Hermed er knæleddet hyppigere end noget andet led sæde for patologiske forandringer. Knæleddet er sammensat af tre dele; \textit{femur}, \textit{tibia} og \textit{patella}. Disse er alle i slidfladerne beklædt med et tykt lag hyalinbrusk, op til syv millimeter på femur. Sammen med meniskerne, der fordeler trykket på en større overflade, er hyalinbrusken med til at mindske friktionen i leddet. [\textbf{(Bevægeapperatets anatomi) <- det skal være en rigtig kilde}]




\begin{figure}[H] 
\begin{center}
\includegraphics[width=0.8\textwidth]{figures/bProblemanalyse/Artose_knae}
\end{center}
\caption{Når det normale knæ undergår patologiskeforandringer ved knæartrose vil strukturne i knæet forandre sig. Brusken kan ved infektion, slid eller traume blive beskadiget, hvilket vil eksponerer knoglen og førere til smerte.\citep{schroder} \citep{adobe}} 
\label{fig:tka_implant} 
\end{figure}

\subsection{Ikke-kiurgisk behandling}

Artrose kan som tidligere nævnt ikke helbredes, og non-kirurgiske behandlingsmetoder vil derfor fortrinsvis søge at smertelindre samt forbedre funktionen af knæet \citep{brostrom2012}. En essentiel del af behandling af knæartrose, består af at informere og uddanne patienten, med henblik på at patienten opnår indsigt i sygdommen, samt at patienten aktiv inddrages i behandlingsforløbet. Herved ønskes det at patienten forstår vigtigheden af et vægttab, i så fald dette er nødvendigt. da dette kan være med til at reducere belastiningen på det afficerede led \citep{brostrom2012}.
Ved behandling af artrose benyttes også medcin af forskellig karakter til smertelindring samt forbedring af funktionen. De benyttede præperater er først og fremmest Parcetamol som  førstevalgspræperat, men også NSAID præparater kunne være gavnligt ved inflammation \citep{schroder}. Ved kraftige smertegener, hvor anden smertelindrende behandling ikke har haft den ønskede smertelindring, kan opioider også benyttes. Derudover findes andre medikamenter, som for eksempel steroidinjektioner og glucosamin præparater. Derudover benyttes også ganghjælpemidler og skinner. Disse benyttes dog i et mindre omfang \citep{brostrom2012}.\\

I behandlingsforløbet er træning en vigtig faktor, både før og efter en eventuel operationen. Dette afspejles i både nationale og internationale kliniske retningslinjer, hvor der er bred konsensus om, at træning er af væsentlig betydning ved behandling af knæartrose \citep{brostrom2012}. Et systematisk review med data fra over 4000 patienter, foretaget af \cite{Syssorenskou}, viste at hverken graden af den røntgenpåviste artrose eller smerteintensitet havde indvirkning på hvor stor effekt der kunne forventes af træningsforløbet. Det blev fundet at patienter med svær artrose oplevede samme smertereduktion som patienter med let til moderat artrose efter træningsforløbet. Et Cochranereview finder evidens for at den forventede smertelindring ved træning er lige så stor som ved brug af NSAID og endnu større end ved brug af parvetamol. Træning har ligeledes den fordel ikke at have bivirkninger som smertelindrende medicin har \citep{sorenskou}.
I et dansk studie, der strakte sig over 12 måneder blev 100 patienter  der var egnet til at modtage en TKA, randomiseret i to grupper. Den ene gruppe skulle modtage non-kirurgisk behandling som bestod af et træningsforløb, patientundervisning, indlægssåler og et eventuel vægtabsprogram. Den anden gruppe modtog kirurgisk behandling efterfulgt af et non-invasivt forløb. Gruppen der gennemgik kirurgisk behandling efterfulgt af et non-invasivt forløb, havde en større smertelindring en gruppen, som kun modtog ikke-kirugisk behandling. Dog havde gruppen der modtog kirurgisk behandling større risiko for at få alvorlige komplikationer. Forsøget viste desuden at i gruppen som skulle modtage non-kirurgisk behandling, fik over to tredjedele ikke foretaget en TKA inden de 12 måneder \citep{newEngland}. 

%Dette kan muligvis betyde at et non-invasivt forløb kan være være med til udskyde et operativt indgreb. (\textbf{vi skal nok passe på med at konkludere noget her})

%\subsection{Knæartrose}
%
%Knæartrose også kaldet slidgigt i knæene, har mange årsager. Hvor af nogle er overvægt, arv, traume eller tungt arbejde. Arterose er karakteriseret ved ødelæggelse af ledbrusken med dertil hørende reaktioner i de tillæggende knogler og slimhinder. Symptomerne på knæartrose er smerte, funktionstab og fejlstilling, hvilket besværliggøre hverdagen. 
%Ved knæartrose er sidste behandlings skridt kirurgi, afhængig af graden af traumet er forskellige kirurgiske indgreb en mulighed. [Nationale retningslinjer] 

\subsection{Kirurgisk behandling}

Når de non-kirurgiske behandlingsmuligheder ikke har kunne løse patienten fra smerter, er kirurgi det næste skridt i behandlingsforløbet. Der findes flere behandlingsmuligheder inden for kirurgiskbehandling af artrose. Valget af operation og typen af den afhænger af flere faktorer, blandt andet patients alder, aktivitetsniveau og hvor fremskreden artrosen er.

\subsubsection{Osteotomi}
Ved degenerative forandringer i knæleddet, grundet primær eller postttrumatisk knæledsarterose, kan patienter opleve belastningsreleaterede smerter, hvilket blandt andet kan skyldes fejlstilling. Osteotomi har tilformål at afhjælpe den mekaniske belastning i det berørte område, for derved at afhjælpe smerterne. Ved osteotomi fjernes der oftest en kile af tibia-knoglen og det resterende knogle sikres med skruer og metal plader. Proceduren ændre knæets mekaniske akse, hvilket vil ændre belastningen af de degenererede områder. \citep{Osteotomi_og_TKA} Ved yngre (<50år) og aktive patienter vil der være større sandsynelighed for at tilbyde osteotomi frem for den mere invasive TKA, derfor anbefales osteotomi af sundhedstyrelsen til behandling af mildere former for artrose med fejlstilling \citep{Osteotomi_og_TKA} \citep{brostrom2012}. Behandlingen ses som en midlertidig behandling der kan udskyde behovet for TKA. Ifølge et kohordestudie kan der forventes en smertelindring hos 80\%~ af patienterne der får udført osteotomi. \textbf{rigtige kilder tak!(79)(80)} Ifølge \cite{brostrom2012} må det forventes at 30 til 50\%~ af patienterne der får foretaget en osteotomi, senere vil få behov for en alloplastik operation. \citep{brostrom2012} Tilbagevendende smerter korreleres til tab af korrektionen, samt progression af artrosen. Får patienten således svære smerte igen, kan en TKA komme til overvejelse \citep{Osteotomi_og_TKA}. 
% \textbf{Der er data på mere specifikke tilfredsheds undersøgelser, men ser ikke nogen trund til at medtage dem. }

\subsubsection{Alloplastik}
Alloplastik er et operativt indgreb der har til formål helt eller delvist at udskifte knæleddet, med specielt designede metal- og plastkomponenter som varig erstatning for bruskfladerne i knæet. Operationen opdeles i TKA og UKA, hvilket henholdsvis er helt eller delvis udskiftning af knæleddet og afhænger af den specifikke diagnose. Der kan ved traume tilfælde eller svære beskadigelser af de anatomiske strukturer omkring knæet forekomme specialiserede udgaver af knæalloplastik.

\begin{figure}[H] 
\begin{center}
\includegraphics[width=0.7\textwidth]{figures/tka_implant}
\end{center}
\caption{Komponenterne til en total knæalloplastik, består af et femural og tibia implantat ofte bestående af en titaniumlegering. Patella- og tibiaindsatsen er lavet af polyethylen, hvilket er med til at mindske friktionen og efterligne knæledes naturlige bevægelse.\cite{1}} 
\label{fig:tka_implant} 
\end{figure}

Under selve operationen ligger patienten supineret på operationsbordet med knæet i en flekteret position. Et longitudinelt snit lægges over midten af patella. Patella og senerne eleveres og blotter knæleddet, hvilket giver kirurgen adgang til bruskfladerne på femur og tibia. Herefter fjerner kirurgen det ødelagte brusk, ved hjælp af en guideblok der skrues ind i femur og sikrer præcis fjernelse af den ønskede mængde væv. Dette gentages på tibia, hvorved der skabes plads til implantaterne. Midlertidige implantater indsættes for at sikre bevægelsesfriheden er bevaret og testes ved ekstension af knæet for at sikre at den rigtige mænge brusk og knogle materiale er fjernet. Når kirurgen er tilfreds med resultatet bores der guidehuller i henholdsvis femur, tibia og patella til fastemontering af de permanente implantater. Fastmontering sker ved at dække implantatet og monteringsstedet i bencement der limer proteserne fast til den eksisterende knogle struktur. Herefter sikres endnu engang at bevægelsesgraden er bibeholdt, førend indsnittet lukkes og operationen er fuldendt. En TKA operation varer typisk omkring én time, hvorefter patienten kan støtte på benet den følgende dag. Efter operationen følger et rehabiliteringsforløb for at støtte og styrke muskulaturen omkring knæet. \citep{Sanna2013} \citep{tka-technique}

Ifølge sundhedsstyrelsens vurdering er knæalloplastik, effektiv til at mindske smerte, øge funktion og derved bedre livskvalitet. Holdbarheden af knæimplantaterne vurderes ud fra antallet af implantater der er blevet udskiftet efter 10 år, hvor det findes at 90 til 95\%~ af implantaterne ikke er revideret. Dog skal det nævnes at det ikke er muligt at vurdere holdbarheden af den enkelte protese, da flertallet af patienter dør med en velfungerende implantat. \citep{brostrom2012}

\paragraph{Succeskriterier}

Succeskriterier for kirurgisk behandlingen af arterose er ifølge styrringsgruppen for Dansk knæalloplastikregister (DKA) \citep{aarsrapport2016} opdelt i fem kriterier.
%, som alle bygger revisionsraten. Altså patienter som har behov for en yderligere knæ operation.  

% Please add the following required packages to your document preamble:
% \usepackage{graphicx}
% \usepackage[table,xcdraw]{xcolor}
% If you use beamer only pass "xcolor=table" option, i.e. \documentclass[xcolor=table]{beamer}
\begin{table}[H]
\centering
\caption{Tabellen viser succeskriterierne for knæalloplastikoperationer med en standard acceptabel grænse sammenholdt med landsgennemsnittet. Spredning er indikeret i parentes. Tabellen er modificeret fra \citep{aarsrapport2016} }
\label{succeskriterier}
\resizebox{\textwidth}{!}{%
\begin{tabular}{lcc}
\hline
\rowcolor[HTML]{C0C0C0} 
Indikator                                                                                                                                                                                                                                                  & Standard    & \begin{tabular}[c]{@{}c@{}}Landsgennemsnit\\ (Spredning)\end{tabular} \\ \hline
\begin{tabular}[c]{@{}l@{}}\textbf{Genindlæggelse}\\ \\ Andel af alle patienter med primær knæalloplastik på baggrund af primær artrose, \\ der genindlægges uanset diagnose indenfor 30 dage efter udskrivning\end{tabular}                                        & Højest 10\% & 7,3 \% (5,8-9,5)                                                      \\ \\
\begin{tabular}[c]{@{}l@{}}\textbf{Revisionsrate det første postoperative år}\\ \\ Andel af alle patienter med primær knæalloplastik fra et givent operationsår, \\ der er revideret (dvs. implantat fjernes, udskiftes eller tilføjes) indenfor 1 år.\end{tabular} & Højest 3\%  & 1,8 \% (0,8-2,4)                                                      \\ \\
\begin{tabular}[c]{@{}l@{}}\textbf{Revisionsrate de første 2 postoperative år}\\ \\ Andel af alle patienter med primær knæalloplastik fra et givent operationsår, \\ der er revideret (dvs. implantat fjernes eller udskiftes) indenfor 2 år.\end{tabular}          & Højest 5\%  & 3.3\% (1,7-4,8)                                                       \\ \\
\begin{tabular}[c]{@{}l@{}}\textbf{Revisionsrate de første 5 postoperative år}\\ \\ Andel af alle patienter med primær knæalloplastik fra et givent operationsår, \\ der er revideret (dvs. implantat fjernes, udskiftes eller tilføjes) indenfor 5 år.\end{tabular}   & Højest 8\%  & 6,0\% (4,2-6,7)                                                       \\ \\
\begin{tabular}[c]{@{}l@{}}\textbf{Mortalitet indenfor 90 dage}\\ \\ Andel af patienter, der dør indenfor 90 dage efter primær knæalloplastik.\end{tabular}                                                                                                         & Højest 1\%  & 0,4\% (0,0-0,7)                                                      
\end{tabular}%
}
\end{table}

Som det ses af \tabref{succeskriterier} udføres der i hele Danmark knæoperationer som alle signifikant overholder indikationerne for behandlingskvaliteten \citep{aarsrapport2016}. 
På trods af at alle operationer overholder succeskriterier, ses det at patienter postoperativt har smerter og er utilfredse. Antallet af patienter på tværs af studier som er utilfredse er 11 til 25\%, se \tabref{patient_utilfreds} (n=31.114) hvoraf studiet med størst deltagelse (n=27.372) viste at 18\% af patienterne var utilfredse. \citep{Bourne2010}

Studiet af \cite{Sakellariou2016} rapporterer postoperative smerter hos patienter, hvor 39\% (n=272) oplevede moderat til alvorlige smerte\citep{Sakellariou2016}. Et andet studie viste at 19\% af patienterne havde svære til uudholdelige smerter efter den første operation. Ved revision var procentdelen af patienter med svære til uudholdelige smerter 47\%. \citep{Petersen2015}

Patienterne burde efter operationen være tilfredse samt smertefri, dette er imidlertid ikke tilfældet. Der er således et problem med resultatet af operationen, på trods af operationen overholder succeskriterierne \citep{aarsrapport2016}. Det må derfor antages problemet ikke ligger ved operationen, men da en del patienter oplever postoperative smerter, er det relevant at undersøge smertes indflydelse på resultatet af operationen. 

\subsection{Præoperatoriske risikofaktorer for vedvarende smerter efter TKA}

Ifølge et systematisk review og meta-analyse af \cite{Lewis2015} der har undersøgt hvilke predikatorer associeret med TKA operationer. I studiet fandt de at katastrofetænkning, mentalt helbred, smerte forud for operationen og smerte andre steder var de største årsager til postoperative smerter, mere end tre måneder, efter TKA operationer. Undersøgelsen havde en population på (n=30.000) patienter, og medtog studier mellem år 1980 og december 2012.\citep{Lewis2015} Der forligger ingen information om hvilke lande patienterne stammede fra, ej heler detaljer om operationen. Det vil dog kunne forventes en forbedring i både operations procedure og kvaliteten af alloplastikker over en periode på mere en tredive år. Studiet fandt at der var en signifikant sammenhæng mellem prædiktorer og kroniske postoperative smerter ved brug af uni- og multivariable analyse modeller.\citep{Lewis2015} Det fandtes at ved univariable modeller var der en signifikant effekt størrelse på >0.1 for flere prædiktorer, mens to prædiktorer, katastrofetænkning og preoperative smerter, havde en signifikant effektstørrelse ved multivariabel modellering.\citep{Lewis2015} 


%
%(1) http://www.robodoc.com/patient_about_faqs.html
%
%(2) https://www.youtube.com/watch?v=tKji04oFGdU

% (3) http://www.ortopaedi.dk/fileadmin/Guidelines/Referenceprogrammer/Osteotomi_og_TKA.pdf

	%(4) Surgical approaches in total knee arthroplasty.
	
%	(5) tka-technique
%
%(79) Dahl AW, Toksvig-Larsen S, Roos EM. A 2-year prospective study of patient- relevant outcomes in patients operated on for knee osteoarthritis with tibial osteot- omy. BMC Musculoskeletal Disorders 2005;6(1):18. Er lagt på mendlay
%(80) Hoell S, Suttmoeller J, Stoll V, Fuchs S, Gosheger G. The high tibial osteoto- my, open versus closed wedge, a comparison of methods in 108 patients. Arch Or- thop Trauma Surg 2005;125(9):638-643. Er lagt på mendelay