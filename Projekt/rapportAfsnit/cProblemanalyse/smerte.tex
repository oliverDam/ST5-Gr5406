\section{Smerte}
\textit{Dette afsnit omhandler smerte hvor de anatomiske og fysiologiske egenskaber vil beskrives. Herunder vil de forskellige kategorier af smerte forklares, og hvilken sammenhæng de har til kroniske smerter.}

\subsection{Smertens anatomi og fysiologi}
Smerte er defineret som: “\textit{An unpleasant sensory and emotional experience associated with actual or potential tissue damage, or described in terms of such damage.}” af The International Association for the Study of Pain (IASP) \citep{Giangregorio1997}, \citep{Carmon}.\\
Selvom smerte normalt er en følelse, der forsøges undgået, er det en nødvendig del af menneskets overlevelse. Det fortæller kroppen om farer eller skader som skal reageres på, så yderligere skade kan undgås.
Smerte er en oplevelse hjernen har når den modtager stimuli fra neuroner i kroppen. Oplevelsen kaldes perception af smerte og er forskellig fra sensationen af smerte. Sensationen sker idet neuroner stimuleres og genererer et aktionspotentiale. Perceptionen sker først når impulsen når til hjernen, og hjernen opfatter smerten. 

%smerte i martini: ca 500 og frem

%smerte perception i hjernen -> typer af smerte (hvor kommer smerten fra? er det skade på væv (er det således nociceptisk (somatisk/visceral(referredpain)) eller neuropatisk?) eller psykogen?)

%Smerter registreres af noiceptorer i kroppen. De fleste smertereceptorer er frie neuronender som forgrener sig ud i hele kroppen, og ligger frit i væv. De dækker således et stort område og kan reagere på forskellige typer stimuli som vævet bliver udsat for. Typen af stimuli bliver bestemt afhængig af hvilke porte som aktiveres på neuronerne. De besidder forskellige typer af natrium-/kaliumporte, som aktiveres ved temperaturændring, trykforandring og kemisk foranding, som ses på figur \ref{neuronport}.

%\begin{figure}[H] 
	%\begin{center}
%	\includegraphics[width=0.5\textwidth]{figures/neuronport}
%	\end{center}
%		\caption{a) Kemisk styret ionkanal der responderer på en kemisk binding b) Spændings styret ionkanal. Ionkanalen åbner ved ændring det transmembrane potentiale c) Mekanisk styret ionkanaler, som ved åbner ved mekanisk stimuli.   \citep{Martini}}
%		\label{neuronport}
%		\centering
%\end{figure}
%%Figuren er fra side 390 i Martini %% OBS, noget ift. frienerveender, og omkring transmembranpot.

%Termoreceptorer er ikke illustreret på figur \ref{neuronport}, men fungerer ved at proteinerne som portene består af, åbner sig ved bestemte temperaturer og derved lader natrium og kalium passere igennem \citep{kimball}. 
%Når en eller flere af disse typer porte åbner strømmer natrium ind i neuronet, mens kalium strømmer ud, og neuronets membranpotentiale stiger. Et gradientpotentiale opbygges og hvis dette er tilstrækkelig højt opnås neuronets tærskelværdi, og et aktionspotentiale sendes afsted. Dette potentiale når til en synapsespalte mellem to neuroner, hvor potentialet overføres ved frigivelse af kemiske transmitterstoffer, som acetylcholin (ACh). Modtagerneuronet har ACh recptorer som igangsætter opbygningen af et gradientpotentiale, så impulsen kan fortsætte. \\
%Impulsen fortsætter ad neuronerne til en dorsal ganglion, hvor flere neuroner samles inden de ledes ind i henholdsvis rygmarvens spinothalamiske tragt eller den posteriore søjle, afhængig af om de bringer information om upræcis berøring, tryk, temperatur og smerte, eller præcis berøring, vibration og proprioception. \citep{Martini}

%sendes til centralnervesystemet (CNS), hvor det først når rygmarven og lidt senere hjernen. Her skelnes der mellem smerte sensation og perception. Smertesensation er information om smerte, som nerverne i hånden der registrere den skadelige temperatur. Smerte perceptionen sker først når nervesignalet når op til hjernen og denne modtager signalet og opfatter det som smerte. Sensationen af smerte kan i rygsøjlen aktivere en refleks der får musklerne i armen til at trække hånden væk fra varmen, inden hjernen når at registrere og opfatte den egentlige smerte. \citep{Martini} Denne form for smerte er kategoriseret som akut-nødvendig smerte, da det hjælper kroppen med at undgå skader.

\subsection{Kroppens smertetransmission}
Smerter registreres af nociceptorer i kroppen. Disse smertereceptorer er frie neuronender som forgrener sig ud i hele kroppen, og ligger frit i vævet. De dækker således et stort område og kan reagere på de forskellige typer stimuli som vævet udsættes for. Disse stimuli, der aktiverer nociceptorene, kan skyldes tryk, stræk, termiske-, og kemiske påvirkninger. Receptorerne kan også være polymodale, hvilket vil sige at de kan respondere på flere typer stimuli. Aktiveringen af nociceptorerne sker oftest ikke direkte, men ved hjælp af transducerproteiner der oversætter forskellige typer stimuli, eksempelvis varmestimuli. Nociceptorerne kan dog også aktiveres direkte, for eksempel af bradykinin og prostaglandin, som er stoffer der er involverede i inflammationsreaktioner. Når nociceptorerne er aktiverede vil ionkanalerne lade Na2+ og Ca2+ strømme ind i det primære afferente neuron, hvorved et gradientpotentiale opbygges, og hvis dette er tilstrækkelig højt opnås neuronets tærskelværdi, hvorefter et aktionspotentiale sendes afsted. Fibrene der kan lede signalet, er enten A- $\delta$-fibre eller C-fibre, og disse adskiller sig fra hinanden rent anatomisk ved at A-fibrene er myeliserede, hvor C-fibrene ikke er. Dette betyder at A-fibre har en højere ledningshastighed end C-fibre, og derfor er A-fibre også årsåg til den første stikkende smerte efter en vævsskade, hvorimod C-fibres aktivering resulterer i en brændende smerte der fremkommer efterfølgende. \citep{smerter} 
Flertallet af primære afferente fibre føres til rygmarvens baghorn, hvor de danner synapse med ascenderende neuroner i rygmarven. Signaloverførslen i den første synapse sker overvejende ved hjælp af de excitatoriske transmitterstoffer Glutamat og Substans P. 
Neuronerne viderefører signalet op gennem tractus spinothalamicus, tractus spinomesencephalicusm og tractus spinoreticularis til hjernestammen og thalamus. Thalamus er involveret i koordinering af signaler til højere strukturer, herunder det primære og sekundære sensoriske cortex, den primære motoriske kortex og det frontal kortex. Disse hjerneområder varetager blandt andet information om lokation, intensitet, adfærdsforandringer og  kognitiv bearbejdning. 
Fra hjernestammen er der forbindelse til den periakvæduktale grå substans og til rostro-ventrale medulla, som er involveret i smerteregulering via descenderende baner som danner synapser i baghornet. I de descenderende baner er noradrenalin og serotonin de primære transmitterstoffer . \citep{smerter}

Til smertebehandling benyttes blandt andet serotonin-plus-noradrenalin-genoptagelseshæmmere(SNRI). Disse stoffer påvirker den præsynaptiske genoptagelse af serotin og noradrenalin, og dette resulterer i en øget mængde af disse i synapsespalten, som igen  resulterer i en øget aktivitet \citep{smerter}. Den øgede aktivitet, resulterer i en mindsket oplevelse af smerten. 

Derudover danner  interneuroner også synapse med det afferente første neuron og det afferente andet neuron. Disse interneuroner er ikke smerteførende A $\beta$ fibre, men er følsomme overfor berøring, og dette er grunden til at det virker smertelindrende eksempelvis at puste eller  berøre et område der gør ondt \citep{cindys268}.

\subsection{Smerte og knæartrose}
Ledkaplser, sener, periost og spongiøst er tæt innerveret af smertefibre, hvorimod ledbrusk ikke er inneveret af nociceptorer \citep{smerter}. Dette kan forklare hvorfor  røntgenfund og smerteintensitet korrelerer i så ringe grad som beskrevet i \ref{xxx}. \citep{Petersen2016}  \citep{smerter}
Ved Inflammation i ledkapslen ses en rekruttering af slumrende C og A $\delta$ fibre, hvilket kan føre til en sensibilisering af nervesystemet \citep{smerter}. En  forlænget inflammationstilstand, skade på den subkondrale knogle og en forlænget excitation af nociceptorer kan alle medføre en  perifær sensibilisering, som efterfølgende kan føre til en central sensibilisering \citep{Petersen2016} .
Ved perifær sensibilisering forståes, en reduktion i tærsklen og en øget respon af nociceptoren. Dette kan fremkomme som et resultat af en vedvarende påvirkninger imflammitoriske mediatorer som der er i beskadiget eller inflammatorisk væv. 
Central sensibilisering betyder at der sker en øgningen af exicitabiliten af neuronerne i CNS, så normale indput giver abnormale respons. Kan blive triggered af vedvarende aktiviet i nociceptorerne \citep{nature}


Perifær og central sensibilisering er blevet foreslået som to af de underliggende mekanismer ved smerter der er relateret til knæartrose og andre kroniske muskoloskeletale smertetilstande \citep{Arendt-Nielsen2010}, og op til 70\% af personer med knæartrose har somatosensoriske abnormaliteter der involverer sensibilisering af nervesystemet \citep{lars}. Dette kan videre føre til en  hypersensibilisering over for smerter (widespread pain sensitivity- generaliserende smerter)  \citep{Petersen2016}.  Kroniske muskuloskeletale smertepatienter, og dermed også knærtrosepatienter, har generelt et kraftigere respons på smertefuld stimulering, og hyperalgesi i dybt væv \citep{Arendt-Nielsen2010}. I kroniske muskoloskeletale smertetilstande, er det vist at eksempelvis temporal summation og repetitiv tryksmerte er faciliteret i forhold til raske personer \citep{widespread}. Ved knæartrose tyder det også på, at den descenderende smertefacilitation og hæmning har betydning for om knæartrosepatienter får kroniske smerter efter en TKA-operation \citep{Petersen2016}. Efter en eventuel TKA operation kan denne hypersensibilitet normaliseres \citep{Petersen2016} \citep{graven2012}. Dette er dog ikke tilfældet for en undergruppe af de patienter der får en TKA, navnligt dem med kroniske post-operative smerter \citep{Petersen2016}
Det er blevet foreslået at en kombination af tilstedeværelsen af disse biomarkører, muligvis kan være med til at determinere, hvem der udvikler kroniske smerter efter en TKA-operation \citep{Petersen2016}.


\subsection{Smertetyper}
%Modsat akut-nødvendig smerte findes unødvendig smerte. Denne smerte kaldes også kronisk smerte, da den oftest er længerevarende, ved at have været konstant i mindst tre måneder \citep{Giangregorio1997}. 

% kilde der siger at nociceptisk smerter både er akut og kronisk: http://www.slideshare.net/ssakpi/molecular-mechanisms-of-pain-part-1
% kilde der siger at akut og kronisk smerte er to helt forskellige ting, hvor nociceptisk og neuropatisk begger er under kronisk: http://www.denalihealthcaremi.com/tag/classification-of-pain/
% kilde som siger at smerte kan opdeles efter hvordan det bestemmes: udfra længde af smerten eller udfra smertens natur (oprindelse)
% hvis man skære en nerve over, burde smerten kunne kategoriseres som akut, men idet det er en skade på en nerve er det neuropatisk og derfor kronisk smerte?

Der findes flere måder at opdele og kategorisere smerte på, men generelt kan det opdeles i to overordnede kategorier; akut og kronisk. \\
\textbf{Her skal indsættes en figur (stamtræ) med de forskellige typer af smerter} 
Hver kategori har flere undergrupperinger, hvor det er omdiskuteret hvordan disse grupper skal placeres. \citep{Giangragorio1997} På figur \ref{smertediagram} er en oversigt over, hvordan smerten kategoriseres i dette projekt.

\begin{figure}[H]
	\caption{smertediagram}
	\label{smertediagram}
	\centering
	\includegraphics[scale=.8]{figures/smertediagram}
	\flushleft
\end{figure}

\subsubsection{Akut smerte}
Akut smerte er en nødvendig smerte, som fortæller hjernen om øjeblikkelig vævsskade, så kroppen hurtigt kan afværge dette. Akut smerte forårsages af traumer i eller på kroppen, og der skelnes herunder mellem to typer af smerte: nociceptisk og neuropatisk. Nociceptisk smerte udløses af skade på væv, herunder indre organer, overflader af kroppen og knogler. Nociceptisk smerte skyldes aktivering af nociceptorer, som oftest sidder som frie neuronender i væv. Som tidligere beskrevet er de følsomme overfor temperaturændringer, mekanisk stimuli eller kemiske ændringer, og kan derved aktiveres på mange måder. Da nociceptorer både er inde i og uden på kroppen, opdeles nociceptisk smerte i somatisk og visceral smertesensation. Somatisk smertesensation er information fra det \textit{yderste} af kroppen. Det er således sensation fra hud og muskulaturer i overfladen af torso, hoved, hals og lemmer. \citep{Martini} Smerten er øjeblikkelig og let placerbar. \\
Visceral smertesensation er information fra de indre organer i hals og abdomen. Sensation herfra opfattes sjældent, da de fleste indtryk er unødvendige at tage stilling til. Hvis der i midlertid opstår smerter i disse områder er denne besværlig at placere. Smerten er typisk ikke øjeblikkelig, men mere trykkende og langvarig; mavesmerter er et eksempel på visceral smerte. \\
Fordi viscerale og somatiske neuroner deles om samme rygmarvssegment, kan visceral information fra indre organer blive opfattet som somatisk information ved referret smerte. Smerte på grund af skade i indre organer vil derfor typisk opfattes som somatiske smerter. Ved referret smerte kan smerte i venstre arm og skulder således være forårsaget af smerte i hjertet.

\begin{figure}[H]
\begin{center}
\includegraphics[width=0.4\textwidth]{figures/forskudtsmerte}
	\caption{Forskudt smerte \citep{Martini}}
	\label{forskudtsmerte}
\end{center}
\end{figure}


Nociceptisk smerte er oftest ikke årsag til kroniske smerter, med mindre smerterne bliver ved.\\ 
Neuropatisk smerte opstår af skader på nervesystemet selv, herunder neuroner, rygmarv, neural plexus eller hjernen. 
Ved neuropatisk smerte registrer neuronerne et stimuli, som kan skyldes infektioner og sygdomme som iskæmi, sclerose, diabetes og kræft. Disse stimuli kan ligeledes være forårsaget af traumer fra smerter som startede med at være nociceptiske. Dette stimuli kan, afhængigt af hvor de påvirkede neuroner er lokaliseret, resultere i forskellige former for neuropatisk smerte. Smerten kan opleves som konstant og langvarig, hvor et typisk eksempel er fantomsmerter, men kan også være lejlighedsvis som ved hyperalgesi, hvor almindelig berøring opfattes som smertefuldt. \citep{Giangregorio1997}, \citep{Carmon}.

\subsubsection{Kronisk smerte}
Modsat akut smerte er kronisk smerte en længerevarende oplevelse af smerte. Det er af IASP defineret som at være smerteperception som varer længere end det generelt ville forventes \citep{Carmon}. Ofte sættes denne grænse ved tre måneder, men i nogle kliniske undersøgelser, eller hos nogle kræftpatienter kan smerten allerede efter en måned kategoriseres som værende kronisk. Som det kan ses af \figref{smertediagram} hænger akut og kronisk smerte sammen ved organisk smerte. Dette skyldes, at kronisk smerte ofte opstår på baggrund af en akut smerte, som, mod forventningen, ikke stopper. Kronisk smerte kan således opleves af en person som har oplevet enten nociceptisk eller neuropatisk smerte, hvis vedkommende efter heling stadig har smerter. Kronisk smerte kan ligeledes være af psykogen oprindelse.\\
Psykogene smerter er, til forskel fra nociceptisk og neuropatisk smerte, ikke en egentlig skade på kroppens væv. Det er en imaginær perception af smerte, og derved den mest besværlige at præcisere, idet der ikke er og muligvis aldrig har været en fysisk årsag til smerten. Hos en person med psykogene smerter er hjernen fuldt overbevidst om, at den oplever fysiske, nociceptiske eller neuropatiske, smerter og lider deraf. Smerten er udelukkende psykisk hos personen, men er af den grund ikke mindre virkelig, grundet smertes subjektive natur. \citep{Giangregorio1997}

%\subsection{Problemet ved kronisk smerte og operationer}
%Der findes således flere forskellige former for smerte, hvor kun nogle få er beskrevet her \cite{Carmon}. Alle kan de lede til kronisk smerte. Man ved derfor godt hvad der kan give kronisk smerte, hvordan smerten opfattes og hvor i kroppen den kommer fra. Men man ved endnu ikke hvorfor kronisk smerte opstår. Hvorfor føler kroppen fantom smerter fra et legeme som ikke er der? Hvorfor registrere hjernen smerte fra indre organer, når der intet er i vejen med dem? 



