\section{Smerte}
\textbf{Dette afsnit skal skrives sammen med det foregående}
Et større problem opstår når det forsøges at behandle smerterne, med et ønske om at dæmpe eller helt fjerne smerterne, men smerterne fortsætter eller forværres. Sådan ses det i 19\% af tilfælde efter TKA operationer \citep{Petersen2015}. Her oplever 19\% af patienter efter den primære operation, og 47\% af patienter efter revision af operationen oplever svære til uudholdelige smerter. Dette sker på trods af at der i hele Danmark udføres knæoperationer som alle signifikant overholder indikationerne for behandlingskvaliteten \citep{aarsrapport2016}. Hermed tyder det på at patienternes smerter ikke skyldes fejl ved operationen.


\subsection{Et eller andet}

Smerte er defineret som: “\textit{en ubehagelig sensorisk og emotionel oplevelse forbundet med egentligt eller potentiel skade af væv eller beskrevet i vendinger tilsvarende en lignende skade.}” af The International Association for the Study of Pain (IASP) \citep{Giangregorio1997}, \citep{Carmon}.\\
Selvom smerte normalt er en følelse der forsøges undgået, er det en nødvendig del af menneskets overlevelse. Det fortæller kroppen om farer eller skader som skal reageres på, som for eksempel at sætte hånden på en varm kogeplade. \textbf{Tilføj hvilke aktiveringsveje smerten har, hvilke stoffer der benyttes etc.} For ikke at blive slemt forbrændt og gøre skade på hånden, registrerer nerver i huden en høj temperatur, som hånden skal fjernes fra. Nervesignalet sendes til centralnervesystemet (CNS), hvor det først når rygmarven og lidt senere hjernen. Her skelnes der mellem smerte sensation og perception. Smertesensation er information om smerte, som nerverne i hånden der registrere den skadelige temperatur. Smerte perceptionen sker først når nervesignalet når op til hjernen og denne modtager signalet og opfatter det som smerte. Sensationen af smerte kan i rygsøjlen aktivere en refleks der får musklerne i armen til at trække hånden væk fra varmen, inden hjernen når at registrere og opfatte den egentlige smerte. \citep{Martini} Denne form for smerte er kategoriseret som akut-nødvendig smerte, da det hjælper kroppen med at undgå skader.

\subsection{Smertetyper}
Modsat akut-nødvendig smerte findes unødvendig smerte. Denne smerte kaldes også kronisk smerte, da den oftest er længerevarende, ved at have været konstant i mindst tre måneder \citep{Giangregorio1997}. 

Smerte opdeles i to overordnede kategorier; skader på væv og psykogene.
\textbf{Her skal indsættes en figur (stamtræ) med de forskellige typer af smerter} 

Ved skader på væv skelnes der mellem to typer af smerte: nociceptisk og neuropatisk. Nociceptisk smerte er skade på væv og skyldes aktivering af nociceptorer, ved hjælp af specielle porte og pumper på nervecellerne. Nociceptorer er specielle nerveceller som er følsomme overfor temperaturændringer, mekanisk stimuli eller kemiske ændringer i eller omkring celler. Disse findes i huden på kroppens overflader og i og omkring indre organer, og nociceptisk smerte opdeles i somatisk og visceral sensation. Somatisk smertesensation og den øjeblikkelige og let placerbare smerte som at sætte hånden på en kogeplade. Visceral smertesensation er mere besværlig at placere. Smerten er typisk ikke øjeblikkelig, men mere trykkende og langvarig. At have ondt i maven er et eksempel på viseral smerte. Nociceptisk smerte er oftest ikke årsag til kroniske smerter, med mindre smerterne bliver ved.\\ 
Neuropatisk smerte opstår af skeder på nervesystemet selv, herunder neuroner, rygmarv, neural plexus eller hjernen. Dette kan skyldes infektioner, traume eller sygdomme som iskæmi, sclerose, diabetes og kræft. Smerten kan opleves som konstant og langvarig, hvor et typisk eksempel er fantomsmerter, men kan også være lejlighedsvis som ved hyperalgesi, hvor almindelig berøring opfattes som smertefuldt. \citep{Giangregorio1997}, \citep{Carmon}.\\
Psykogene smerter er en forestillet perception af smerte, og den mest besværlige at præcisere, idet der ikke er og muligvis aldrig har været en fysisk grund til smerten. Hos en person med psykogene smerter er hjernen fuldt overbevidst om, at den oplever fysiske smerter og lider deraf. Smerten er udelukkende psykisk hos personen, men er af den grund ikke mindre virkelig, grundet smertes subjektive natur. \citep{Giangregorio1997}

\subsection{Problemet ved kronisk smerte og operationer}
Der findes således flere forskellige former for smerte, hvor kun nogle få er beskrevet her \citep{Carmon}. Alle kan de lede til kronisk smerte. Man ved derfor godt hvad der kan give kronisk smerte, hvordan smerten opfattes og hvor i kroppen den kommer fra. Men man ved endnu ikke hvorfor kronisk smerte opstår. Hvorfor føler kroppen fantom smerter fra et legeme som ikke er der? Hvorfor registrere hjernen smerte fra indre organer, når der intet er i vejen med dem? 



