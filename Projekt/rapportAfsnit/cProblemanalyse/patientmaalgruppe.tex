\section{Patientgruppe}
\textit{Følgende afsnit omhandler omfanget af lidelsen, knæartrose. Afsnittet redegør for patientomfanget, samt de forskellige disponeringsfaktorer, sammenkoblet med lidelsen. Ydermere vil patienternes patientforløb blive redegjort, hvoraf den sidste fase vil blive analyseret. Ovenstående vil danne grundlag for at klassificere en disponeret patientgruppe til knæalloplastik.}

Knæartrose er en lidelse hvor primært knæets ledbrusk gradvist bliver nedbrudt, og der sekundært sker forandringer i leddets knogler. Disse deformationer er irreversible, og kan dermed kan knæartrose kun afhjælpes og ikke kurreres. Lidelsen kan opdeles i en primær- og sekundær artrose. Dette adskilles per definition ved at den sekundære artrose indebærer tidligere skader, sygdom, inflammation, overvægt samt traume. Knæartrose er en tilstand hvis hyppigste symptomer indebærer smerter samt nedsat mobilitet hos den udsatte. Smerterne udtrykkes i forskellig grad, fra led igangsættende smerte til kronisk tilstedeværende smerte. Generelt for knæartrose, så forværres symptomerne i takt med graden af lidelsen øges. [lægehåndbog, knæartrose]

Forekomsten af knæartrose er sammenholdt med en længere række faktorer, hvoraf man er i risiko for at være disponeret. Dette omhandler eksempelvis, overbelastning igennem arbejde og fritid, tidligere knæskader, arv, overvægt samt køn(kvinder)[Knæartrose – nationale kliniske retningslinjer, Sundhedsstyrrelsen]. Knæartrose er tilstede blandt 45\% af alle 80-årige blandt befolkningen. Dette tilfælde kan formodes at stige, som resultat af at der ses en tendens i samfundet, at levealderen i Danmark stiger. Dette er ikke det eneste tilfælde hvorfor prævalensen vil stige. En af de disponerende faktorer for knæartrose er overvægt, hvilket 47\% af den danske befolkning kan kategoriseres under. Ydermere stiger forekomsten af overvægt med alderen, hvilket forud for overvægten ligeledes er tilfældet for knæartrose. Overvægtige er disponeret for knæartrose med en relativ risiko på tre, hvoraf en kombination af ovenstående faktorer øger risikoen for lidelsen. [patienthåndbogen, overvægt og fedme][lægehåndbog, overvægt][patienthåndbog, slidgigt i knæ][lægehåndbog, knæartrose]

Resultatet af en patients komplikationer kan medføre igangsættelse af et behandlingsforløb. Et behandlingsforløb for en patient med knæartrose består af flere faser, hvis mål er smertelindrende, mobilitetsforøgelse samt forebyggende. Generelt kan faserne opdeles i ikke-invasive og invasive metoder. Hvilken metode som afhjælper patienten afhænger heraf af graden af knæartrose. Første fase, hvis nødvendig, består af i en livsstilsændring, hvor en vægtreduktion samt øget fysiskaktivitet uden belastning, vil være fordelagtigt. Hvis dette ikke er tilstrækkeligt kan medicinsk behandling i form af smertelindrende medikamenter benyttes, enten som enkeltstående behandling eller sideløbende med fysioterapi. Hvis ikke, de ikke-invasive behandlingsmetoder afhjælper lidelsen i en grad hvor patienten er tilfreds, så bliver de invasive behandlingsmetoder taget til overvejelse. Overvejelsen heraf indebærer af den diagnosticerede grad af artrose, hvilket består af en sammenkobling af den kliniske vurdering, verificeret med forandringer i knæet opnået gennem røntgenbilleder. Baggrunden for at den kliniske vurdering skal verificeres forud for kirurgi, er at smerte fra hofte og ryg, kan projiceres til knæet.\\
Resultatet heraf er at patienten skal besidde svær slidgigt før patienten kvalificeres til kirurgi, og hermed en total knæalloplastik (TKA).  [lægehåndbog, knæartrose][Knæartrose – nationale kliniske retningslinjer, Sundhedsstyrrelsen]

TKA er det sidst mulige behandlingsmetode for at udrede patientens komplikationer vedrørende knæartrose. Dette resulterede i at der i 2014 blev udført omtrent 9,8 tusinde TKA operationer, fordelt på førstegangs- og revisions operationer. [Dansk knæalloplastikregister, 2015] I takt med at TKA er den sidst mulige behandlingsmulighed, er operationstilfredshed en betydningsfuld problematik. I 2012 var 81-85\% af patienter der havde modtaget en TKA operation tilfredse, 8-11\%var decideret utilfredse, og resten var i tvivl eller til dels utilfreds. Dette er altså ensbetydende med at der potentielt er 19\% af alle operationer fra et patientøjemed som er ikke succesfulde. Resultatet heraf er at op mod 19\% ikke kan udredes fra deres smerter samt eventuel nedsatte mobilitet, trods alle behandlingsmetoder har været benyttet. [Knæartrose – nationale kliniske retningslinjer, Sundhedsstyrrelsen] Studiet af xxxxx har lavet en risikovurdering vedrørende kroniske smerter postoperativ TKA. Resultaterne betød at op mod 39\% af studiets patienter oplevede moderat til alvorlig smerte, 1 år postoperativt TKA. Ifølge International Association of Pain (IASP) er der tale om kroniske smerte, da dette er tilfældet ved vedvarende smerter tre måneder postoperativt. [Risk Assessment for Chronic Pain…,link] 

\textit{Afrunding af afsnittet: 
Afgrænsning for hvem der er med i målgruppen - f.eks. er det alle som har ondt der er målgruppe eller er det alle der har ondt og er utilfredse? (Skal man inddrage det med skuffelsen – man forventede det ville bliver perfekt, men kan stadig ikke løbe 20 km?) (Er man utilfreds med smerte eller forventning og eller er problemet at man ikke har den mobilitet man forventede.) }

Knæartrose er på bekostning af samfundets udvikling, en lidelse i vækst da den umiddelbare disponerede målgruppe er voksende. Resultatet heraf medfører at antallet af registrerede tilfælde vedrørende komplikationer sandsynligvis ligeledes vil stige, og der vil forekommer flere patienter med kroniske smerter postoperativt TKA, uden mulighed for yderligere alternativ behandling. 


Afsnittet mangler:
\begin{itemize}
	\item Hvad indebærer sådan en fordeling?(procent tilfredse)
	\item Er det alle som får operationen foresaget af slidgigt?
	\item Afgrænsning for hvem der er med i målgruppen - f.eks. er det alle som har ondt der er målgruppe eller er det alle der har ondt og er utilfredse? (Skal man inddrage det med skuffelsen – man forventede det ville bliver perfekt, men kan stadig ikke løbe 20 km?) 

\end{itemize}


