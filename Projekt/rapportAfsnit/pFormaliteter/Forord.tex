\section*{Forord}
Dette projekt er udarbejdet af gruppe 5406, som består af sundhedsteknologistuderende på 5. semester, i perioden fra den 2. september til den 19. december 2016. Projektet er udarbejdet på baggrund af det overordnede tema: Klinisk teknologi. Projektet omfatter en vurdering af Quantitative Sensory Testing (QST) som supplement til en klinikers vurdering af knæartrosepatienters risiko for at udvikle kroniske smerter efter en total knæalloplastik (TKA). Gruppen har gennem projektperioden modtaget vejledning fra tre forskellige vejledere. I den forbindelse skal Kristian Kjær Petersen have tak for vejledning i forhold til teknologiske aspekter i projektet. Ligeledes skal metodevejleder Pia Britt Elberg have tak for vejledning i forhold til metodiske aspekter, herunder analysemodeller og litteratursøgning. Klinisk vejleder Sten Rasmussen skal have tak for information og vejledning i forhold til det kliniske aspekt ved teknologiens anvendelse, patienter og økonomi. Igennem projektperioden er der taget kontakt til producenter og forhandlere af udstyr relevant for analysen. I den forbindelse vil gruppen gerne takke Cephalon A/S og NociTech for informationer vedrørende specifikke produkters egenskaber, priser og oplæringstider. For forståelse af den nuværende kliniske procedure har gruppen besøgt ortopædkirurgisk ambulatorium på Aalborg Universitetshospital, Farsø, hvormed personalet her også skal have tak for præsentationen af arbejdsgang og undersøgelsesmetoder i forbindelse med knæartrose.


\section*{Læsevejledning}
Projektrapporten er inddelt i XX kapitler, der yderligere er inddelt i afsnit og tilhørende underafsnit. Kapitler, afsnit og underafsnit er nummereret, og angivet i rapportens indholdsfortegnelse. Kilder refereres til ved anvendelse af Vancouver metoden, hvor hver kilde angives med et tal, [x]. Ved anvendelse af kilder som en del af en sætning angives kilden med efternavn på første forfatter, årstal og kildens tal, således \textit{Efternavn, årstal [x]}. I litteraturlisten er alle kilder anvendt i projektet samlet. Appendiks X-X findes efter rapportens litteraturliste.
Figurer og tabeller samt disses placering i forhold til hinanden er ligeledes indikeret med tal. Hermed er eksempelvis den første figur i kapitel 1 angivet som figur 1.1. Hvert domæne er tilknyttet et kapitel. Jævnfør projektets generelle metode \ref{metode}, indgår der forskellige analyser i projektet. Domænerne kan læses i den opstillede rækkefølge i rapporten, men kan også læses enkeltvis for at opnå information indenfor et specifikt domæne.\\
Ord, der anvendes i forkortet form gennem projektrapporten, skrives ud første gang de nævnes, hvorefter der følger en parentes med forkortelsen for ordet.