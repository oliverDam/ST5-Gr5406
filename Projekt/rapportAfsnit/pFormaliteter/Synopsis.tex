Chronic postoperative pain in patients, who have undergone a total knee arthroplasty (TKA), is a common problem. Approximately 20\% of patients with knee osteoarthritis develop chronic postoperative pain after a TKA operation. It is believed that central sensitization can be a factor in the development of chronic postoperative pain. Quantitative sensory testing (QST) is a battery of tests, which are used to detect patient response to stimuli. Three of these tests: pressure-pain threshold (PPT), temporal summation of pain (TSP) and conditioned pain modulation (CPM), can be used to detect central sensitization in patients.\\
In this study, a systematic search of current literature is performed to assess, how QST can be implemented as a supplement to the decision of whether or not the patient is eligible for a TKA operation is made. This assessment is analyzed within nine different perspectives such as QST, the patient and the organization. \\
% Konklusion! skal skrives til/eller om når konklusionen er skrevet
It is concluded that PPT, TSP and CPM could work as a supplementary diagnostic tool to reinforce the decision, but it is necessary to compile normative data before the clinician can categorize the patients based on the tests.        
