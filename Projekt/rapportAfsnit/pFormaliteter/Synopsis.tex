Chronic postoperative pain in patients, who have undergone a Total Knee Alloplastic (TKA), is a common problem. Approximately 20\% of patients with knee osteoarthritis develop chronic postoperative pain after a TKA operation. It is believed that central sensitization can be a factor in the development of chronic postoperative pain. Quantitative Sensory Testing (QST) is a battery of tests, which are used to detect patients’ response to stimuli. Three of these tests, pressure-pain threshold (PPT), temporal summation of pain (TSP) and conditioned pain modulation (CPM), can be used to detect central sensitization in patients. In this study, a systematic review is preformed to assess, how QST can be implemented as a tool for elevating the foundation on which the decision of whether or not the patient is eligible for a TKA operation is made. This assessment is made from nine/eight different perspectives regarding QST, the patient and the organization. \\
% Konklusion! skal skrives til/eller om når konklusionen er skrevet
It is concluded that the three QST parameters could work as a supplement for a surgeon and elevate the foundation of the decision, but it is necessary to compile normative data before the surgeon can categorize the patients based on the tests.        
