\documentclass[a4paper,11pt,fleqn,dvipsnames,twoside,openright,oldfontcommands]{memoir} 	% Openright aabner kapitler paa hoejresider (openany begge)

\usepackage{xfrac}
\usepackage{booktabs}
\usepackage[table,xcdraw]{xcolor}

%%%%%%%%% Indsat random
%makes it possible to refer to the name of a chapter rather than just the number.
\usepackage{nameref}
\usepackage{pdfpages}
\usepackage{marvosym}
\usepackage{setspace}
\usepackage{graphicx} % For at sætte 2 billeder ved siden af hinanden

%package for writing program code in latex
\usepackage{listings}
%%%%%%%%%%%%%%%%%%%%%%

% ¤¤ Oversaettelse og tegnsaetning ¤¤ %
\usepackage[T1]{fontenc}					% Output-indkodning af tegnsaet (T1)
\usepackage[danish]{babel}					% Dokumentets sprog
\usepackage[utf8]{inputenc}					% Input-indkodning af tegnsaet (UTF8)
\usepackage{ragged2e,anyfontsize}			% Justering af elementer
\usepackage{fixltx2e}						% Retter forskellige fejl i LaTeX-kernen			
\usepackage{gensymb}						% Nu kan der skrices procent tegn med \degree				
				
																							
% ¤¤ Figurer og tabeller (floats) ¤¤ %
\usepackage{graphicx} 						% Haandtering af eksterne billeder (JPG, PNG, EPS, PDF)
%\usepackage{eso-pic}						% Tilfoej billedekommandoer paa hver side
\usepackage{wrapfig}						% Indsaettelse af figurer omsvoebt af tekst. \begin{wrapfigure}{Placering}{Stoerrelse}
\usepackage{multirow}                		% Fletning af raekker og kolonner (\multicolumn og \multirow)
\usepackage{multicol}         	        	% Muliggoer output i spalter
\usepackage{rotating}						% Rotation af tekst med \begin{sideways}...\end{sideways}
\usepackage{colortbl} 						% Farver i tabeller (fx \columncolor og \rowcolor)
\usepackage{xcolor}							% Definer farver med \definecolor. Se mere: http://en.wikibooks.org/wiki/LaTeX/Colors
\usepackage{flafter}						% Soerger for at floats ikke optraeder i teksten foer deres reference
\let\newfloat\relax 						% Justering mellem float-pakken og memoir
\usepackage{float}							% Muliggoer eksakt placering af floats, f.eks. \begin{figure}[H]
\usepackage{array,booktabs,xcolor,longtable} % kan lave \hdashline i tabellertabe
\usepackage{arydshln}
\usepackage{tabu}
\usepackage{epstopdf} 						% Muligoer brugen af eps filer i latex

	
	
% ¤¤ Matematik mm. ¤¤
\usepackage{amsmath , amsthm , amsfonts , amssymb, float, stmaryrd} 		% Avancerede matematik-udvidelser
%\usepackage{mathtools}						% Andre matematik- og tegnudvidelser
\usepackage{textcomp}                 		% Symbol-udvidelser (f.eks. promille-tegn med \textperthousand )
\usepackage{rsphrase}						% Kemi-pakke til RS-saetninger, f.eks. \rsphrase{R1}
\usepackage[version=3]{mhchem} 				% Kemi-pakke til flot og let notation af formler, f.eks. \ce{Fe2O3}
\usepackage{siunitx}						% Flot og konsistent praesentation af tal og enheder med \si{enhed} og \SI{tal}{enhed}
\sisetup{output-decimal-marker = {,}}		% Opsaetning af \SI (DE for komma som decimalseparator) 

% ¤¤ Referencer og kilder ¤¤ %
\usepackage[danish]{varioref}				% Muliggoer bl.a. krydshenvisninger med sidetal (\vref)
\usepackage[numbers]{natbib}				% Udvidelse med naturvidenskabelige citationsmodeller
%\usepackage{xr}							% Referencer til eksternt dokument med \externaldocument{<NAVN>}
%\usepackage{glossaries}					% Terminologi- eller symbolliste (se mere i Daleifs Latex-bog)
\usepackage{lastpage}					% Gør det mulig at refere til sidste side 

% ¤¤ Misc. ¤¤ %
\usepackage{listings}						% Placer kildekode i dokumentet med \begin{lstlisting}...\end{lstlisting}
\usepackage{lipsum}							% Dummy text \lipsum[..]
\usepackage[shortlabels]{enumitem}			% Muliggoer enkelt konfiguration af lister
\usepackage{pdfpages}						% Goer det muligt at inkludere pdf-dokumenter med kommandoen \includepdf[pages={x-y}]{fil.pdf}	
\pdfoptionpdfminorversion=6					% Muliggoer inkludering af pdf dokumenter, af version 1.6 og hoejere
\pretolerance=2500 							% Justering af afstand mellem ord (hoejt tal, mindre orddeling og mere luft mellem ord)


% Kommentarer og rettelser med \fxnote. Med 'final' i stedet for 'draft' udloeser hver note en error i den faerdige rapport.
\usepackage[footnote,final,danish,silent,nomargin]{fixme}		


%%%% CUSTOM SETTINGS %%%%

% ¤¤ Marginer ¤¤ %
\setlrmarginsandblock{3.0cm}{2.5cm}{*}		% \setlrmarginsandblock{Indbinding}{Kant}{Ratio}
\setulmarginsandblock{2.5cm}{3.0cm}{*}		% \setulmarginsandblock{Top}{Bund}{Ratio}
\checkandfixthelayout 						% Oversaetter vaerdier til brug for andre pakker

%	¤¤ Afsnitsformatering ¤¤ %
\setlength{\parindent}{0mm}           		% Stoerrelse af indryk
\setlength{\parskip}{3mm}          			% Afstand mellem afsnit ved brug af double Enter
\linespread{1,1}							% Linie afstand



% ¤¤ Indholdsfortegnelse ¤¤ %
\setsecnumdepth{subsection}		 			% Dybden af nummerede overkrifter (part/chapter/section/subsection)
\maxsecnumdepth{subsection}					% Dokumentklassens graense for nummereringsdybde
\settocdepth{section} 					% Dybden af indholdsfortegnelsen

% ¤¤ Lister ¤¤ %
\setlist{
  topsep=0pt,								% Vertikal afstand mellem tekst og listen
  itemsep=-1ex,								% Vertikal afstand mellem items
} 

%hyperlinks in the tabel of contents - comment this out before the report is printed.
\usepackage{hyperref}
\hypersetup{
	bookmarks = true,  % Show 'bookmark'-frame in pdf.
	colorlinks = true, % True = colored links, False = framed links.
	citecolor = black,  % Link color for references.
	linkcolor = black,  % Link color in table of contents.
	urlcolor = black,   % Link color for extern URLs.
}

% ¤¤ Opsaetning af figur- og tabeltekst ¤¤ %
\usepackage{caption}
\usepackage{subcaption}
\captionnamefont{\small\bfseries\itshape}	% Opsaetning af tekstdelen ('Figur' eller 'Tabel')
\captiontitlefont{\small}					% Opsaetning af nummerering
\captiondelim{. }							% Seperator mellem nummerering og figurtekst
\hangcaption								% Venstrejusterer flere-liniers figurtekst under hinanden
%\captionwidth{0.9\textwidth}					% Bredden af figurteksten
\setlength{\belowcaptionskip}{0pt}			% Afstand under figurteksten
\captionsetup[figure]{labelfont={bf,it},font={it}} % sætter nummer til fed og kursis. Resten til fed + skriften er mindre end resten
\captionsetup[table]{labelfont={bf,it},font={it}} 


% ¤¤ Opsaetning af listings ¤¤ %

\definecolor{commentGreen}{RGB}{34,139,24}
\definecolor{stringPurple}{RGB}{208,76,239}

\lstset{language=Matlab,					% Sprog
	basicstyle=\ttfamily\scriptsize,		% Opsaetning af teksten
	keywords={for,if,while,else,elseif,		% Noegleord at fremhaeve
			  end,break,return,case,
			  switch,function},
	keywordstyle=\color{blue},				% Opsaetning af noegleord
	commentstyle=\color{commentGreen},		% Opsaetning af kommentarer
	stringstyle=\color{stringPurple},		% Opsaetning af strenge
	showstringspaces=false,					% Mellemrum i strenge enten vist eller blanke
	numbers=left, numberstyle=\tiny,		% Linjenumre
	extendedchars=true, 					% Tillader specielle karakterer
	columns=flexible,						% Kolonnejustering
	breaklines, breakatwhitespace=true,		% Bryd lange linjer
}

% ¤¤ Navngivning ¤¤ %
\addto\captionsdanish{
	\renewcommand\appendixname{Appendiks}
	\renewcommand\contentsname{Indholdsfortegnelse}	
	\renewcommand\appendixpagename{Appendiks}
	\renewcommand\appendixtocname{Appendiks}
	\renewcommand\cftchaptername{\chaptername~}				% Skriver "Kapitel" foran kapitlerne i indholdsfortegnelsen
	\renewcommand\cftappendixname{\appendixname~}			% Skriver "Appendiks" foran appendiks i indholdsfortegnelsen
}

% ¤¤ Kapiteludssende ¤¤ %
\definecolor{numbercolor}{gray}{0.7}		% Definerer en farve til brug til kapiteludseende
\newif\ifchapternonum

\makechapterstyle{jenor}{					% Definerer kapiteludseende frem til ...
	\renewcommand\beforechapskip{0pt}
	\renewcommand\printchaptername{}
	\renewcommand\printchapternum{}
	\renewcommand\printchapternonum{\chapternonumtrue}
	\renewcommand\chaptitlefont{\fontfamily{pbk}\fontseries{db}\fontshape{n}\fontsize{20}{35}\selectfont\raggedright}
	\renewcommand\chapnumfont{\fontfamily{pbk}\fontseries{m}\fontshape{n}\fontsize{0.35in}{0in}\selectfont\color{black}}
	\renewcommand\printchaptertitle[1]{%
		\noindent
		\ifchapternonum
		\begin{tabularx}{\textwidth}{X}
			{\let\\\newline\chaptitlefont ##1\par} 
		\end{tabularx}
		\par\vskip-2.5mm\hrule
		\else
		\begin{tabularx}{\textwidth}{Xl}
			{\parbox[b]{\linewidth}{\chaptitlefont ##1}} & \raisebox{-5pt}{\chapnumfont \thechapter}
		\end{tabularx}
		\par\vskip2mm\hrule
		\fi
	}
}											% ... her

\chapterstyle{jenor}						% Valg af kapiteludseende - Google 'memoir chapter styles' for alternativer

% ¤¤ Sidehoved ¤¤ %

\makepagestyle{AAU}							% Definerer sidehoved og sidefod udseende frem til ...
\makepsmarks{AAU}{%
	\createmark{chapter}{left}{shownumber}{}{. \ }
	\createmark{section}{right}{shownumber}{}{. \ }
	\createplainmark{toc}{both}{\contentsname}
	\createplainmark{lof}{both}{\listfigurename}
	\createplainmark{lot}{both}{\listtablename}
	\createplainmark{bib}{both}{\bibname}
	\createplainmark{index}{both}{\indexname}
	\createplainmark{glossary}{both}{\glossaryname}
}
\nouppercaseheads											% Ingen Caps oenskes

\makeevenhead{AAU}{Gruppe 5406}{}{\leftmark}				% Definerer lige siders sidehoved (\makeevenhead{Navn}{Venstre}{Center}{Hoejre})
\makeoddhead{AAU}{\rightmark}{}{Aalborg Universitet}		% Definerer ulige siders sidehoved (\makeoddhead{Navn}{Venstre}{Center}{Hoejre})
\makeevenfoot{AAU}{\thepage}{}{}							% Definerer lige siders sidefod (\makeevenfoot{Navn}{Venstre}{Center}{Hoejre})
\makeoddfoot{AAU}{}{}{\thepage}								% Definerer ulige siders sidefod (\makeoddfoot{Navn}{Venstre}{Center}{Hoejre})
\makeheadrule{AAU}{\textwidth}{0.5pt}						% Tilfoejer en streg under sidehovedets indhold
\makefootrule{AAU}{\textwidth}{0.5pt}{1mm}					% Tilfoejer en streg under sidefodens indhold

\copypagestyle{AAUchap}{AAU}								% Sidehoved for kapitelsider defineres som standardsider, men med blank sidehoved
\makeoddhead{AAUchap}{}{}{}
\makeevenhead{AAUchap}{}{}{}
\makeheadrule{AAUchap}{\textwidth}{0pt}
\aliaspagestyle{chapter}{AAUchap}							% Den ny style vaelges til at gaelde for chapters
% ... her

\pagestyle{AAU}												% Valg af sidehoved og sidefod


%%%% CUSTOM COMMANDS %%%%

% ¤¤ Billede hack ¤¤ %
\newcommand{\figur}[4]{
		\begin{figure}[H] \centering
			\includegraphics[width=#1\textwidth]{billeder/#2}
			\caption{#3}\label{#4}
		\end{figure} 
}

% ¤¤ Specielle tegn ¤¤ %
\newcommand{\decC}{^{\circ}\text{C}}
\newcommand{\dec}{^{\circ}}
\newcommand{\m}{\cdot}


%%%% ORDDELING %%%%

\hyphenation{}

%%%%Fra engelsk til dansk i \autoref{•} %%%%
\renewcommand{\figureautorefname}{Figur}
\renewcommand{\sectionautorefname}{Afsnit}
\renewcommand{\subsectionautorefname}{Afsnit}
\renewcommand{\subsubsectionautorefname}{Afsnit}
\renewcommand{\tableautorefname}{Tabel}
\renewcommand{\appendixautorefname}{Appendiks}
\renewcommand{\equationautorefname}{Ligning}
\renewcommand{\itemautorefname}{Punkt}
\renewcommand{\chapterautorefname}{Kapitel}
\raggedbottom
%Figure references:
\newcommand{\figref}[1]{figur~\ref{#1}}

%Figure references after full stop/period:
\newcommand{\Figref}[1]{Figur~\ref{#1}}

%Table references:
\newcommand{\tabref}[1]{tabel~\ref{#1}}

%Table references after full stop/period:
\newcommand{\Tabref}[1]{Tabel~\ref{#1}}

%Appendix references:
\newcommand{\appref}[1]{appendiks~\ref{#1}}

%Appendix references after full stop/period:
\newcommand{\Appref}[1]{Appendiks~\ref{#1}}

%Section references:
\newcommand{\secref}[1]{afsnit~\ref{#1}}

%Section references:
\newcommand{\Secref}[1]{Afsnit~\ref{#1}}

%chapter references: 
\newcommand{\chapref}[1]{kapitel~\ref{#1}}

%chapter references: 
\newcommand{\Chapref}[1]{Kapitel~\ref{#1}}


%Units:
%inserting '\omit' before '{\put' prior ot final compile will fix allignment (and generate errors)
\newcommand{\unit}[1]{{\put(300,0){$\hfill\left[\: #1 \:\right]$}}}

%Text:
\newcommand{\tx}[1]{\text{#1}}

%Equation references:
%1 equation:
\renewcommand{\eqref}[1]{ligning~(\ref{#1})}
%2 equations:
\newcommand{\eqrefTwo}[2]{ligning~(\ref{#1}) og (\ref{#2})}
%3 equations:
\newcommand{\eqrefThree}[3]{ligning~(\ref{#1}), (\ref{#2}) og (\ref{#3})}
%4 equations:
\newcommand{\eqrefFour}[4]{ligning (\ref{#1}), (\ref{#2}), (\ref{#3}) og (\ref{#4})}
%5 equations:
\newcommand{\eqrefFive}[5]{ligning (\ref{#1}), (\ref{#2}), (\ref{#3}), (\ref{#4}) og (\ref{#5})}


%Equation references after full stop/period:
%1 equation:
\newcommand{\Eqref}[1]{Ligning~(\ref{#1})}
%2 equations:
\newcommand{\EqrefTwo}[2]{Ligning~(\ref{#1}) og (\ref{#2})}
%3 equations:
\newcommand{\EqrefThree}[3]{Ligning~(\ref{#1}), (\ref{#2}) og (\ref{#3})}
%4 equations:
\newcommand{\EqrefFour}[4]{Ligning (\ref{#1}), (\ref{#2}), (\ref{#3}) og (\ref{#4})}
%5 equations:
\newcommand{\EqrefFive}[5]{Ligning (\ref{#1}), (\ref{#2}), (\ref{#3}), (\ref{#4}) og (\ref{#5})}
\begin{document}

\clearpage
\thispagestyle{empty}

\begin{center}
	\vspace*{\baselineskip}
	\rule{\textwidth}{1.6pt}\vspace*{-\baselineskip}\vspace*{2pt} % Thick horizontal line
	\rule{\textwidth}{0.4pt}\\[\baselineskip] % Thin horizontal line
	
	{\huge Teknologivurdering af QST \\\hspace*{2ex} som et prædiktivt diagnostisk supplement for udviklingen af kroniske postoperative smerter efterfulgt af en TKA-operation.\\[0.5\baselineskip] \large Projektrapport 4. semester}\\[0.2\baselineskip] % Title
	
	\rule{\textwidth}{0.4pt}\vspace*{-\baselineskip}\vspace{3.2pt} % Thin horizontal line
	\rule{\textwidth}{1.6pt}\\[\baselineskip] % Thick horizontal line
	\vspace*{5\baselineskip}
	\begin{figure}[H]
		\centering
		\begin{minipage}[c]{1\textwidth}
			\includegraphics[width=.65\textwidth]{figures/forside2.PNG}
		\end{minipage}
		\hfill
	\end{figure}
	\vspace*{\fill}
	\scshape % Small caps
	{\Large Gruppe 5406:\par}
	Christian Ulrich, Kristine S. Sørensen, Martin S. Jensen \\
	Morten S. Larsen, Nikoloine S. Kristensen, Oliver Damsgaard
	
	\vspace*{.2\baselineskip} % Whitespace between location/year and editors
	Aalborg Universitet,  01/09/2016 - 19/12/2016 \par % Location and year
\end{center} % Center all text
%{\color{white}X \\ X \\ X \\}
\begin{center}
	\rule{\textwidth}{0.4pt}\vspace*{-\baselineskip}\vspace{3.2pt} % Thin horizontal line
	\rule{\textwidth}{1.6pt}\\[\baselineskip] % Thick horizontal line
\end{center} \clearpage


\chapter{Statusseminar}\vspace{-0.75cm}
\section{Indledning}
% baggrund for projektet
Artrose er den mest udbredte gigtsygdom og en af de mest udbredte kroniske sygdomme i Danmark. 
%(\textbf{Note: tilføj hvad en kronisk sygdom er}) \citep{sygdom}. 
Artrose er en kronisk ledsygdom der kan ramme alle ledstrukturer, men primært rammes ledfladernes bruskdele \citep{schroder}. Prævalensen for artrose i Danmark er omkring 800.000 personer, og der sker årligt over 20.000 indlæggelser med artrose som aktionsdiagnose. \citep{sygdom}
Forekomsten af artrose stiger med alderen, hvor personer over 55 år repræsentere den største gruppe af artrose patienter. Over de seneste 15 år er der sket en stigning i artrose operationer fra 2500 i år 2000, til over 9000 i år 2015. \citep{aarsrapport2016} \\

Af artrose lidelser er knæartose en af de hyppigst forekommende. Denne form for artrose er den førende årsag til funktionsnedsættelse i de nedre ekstremiteter \citep{bezwick2012}. 
Knæartrose medfører at ledbrusken nedbrydes, samtidig med at der forløber en række reaktioner i knoglen under brusken, samt i synovialmembranen \citep{brostrom2012}. Som følge af den tiltagende bruskmangel kan der opstå ledskurren og fejlsstilling, hvilket kan medføre belastningssmerter og i sidste ende funktionstab \citep{ugeskrift2011}.
Der er stor variation i hvordan personer der lever med knæartrose påvirkes, og nogle vil derfor kunne leve relativt upåvirkede med sygdommen, mens andre vil opleve at sygdommen svækker både arbejdsevne og livskvalitet \citep{sygdom}.
Generalt for knæartrose oplever patienter smerter i knæleddet. Smerterne varierer meget, og kan gå fra at være let irritable til uudholdelige. Knæartose patienter kommer igennem et længere bahandlingsforløb, men da knæartrose er irreversibel, kan artrose kun afhjælpes og ikke kurreres, og de fleste ender med at få en total knæalloplastik (TKA).\\ 
% intro til projektet
Ifølge sundhedsstyrelsens vurdering er knæalloplastik effektiv til at mindske smerte, øge funktion og derved bedre livskvalitet, og idet TKA-operationen er den sidste behandlingsmulighed, er operationstilfredshed en betydningsfuld problematik. %det vigtigt at opnå dette mål. Det ses dog hos TKA-patienter at 19\% efter den primære operation og 47\% efter revision fortsat oplever svære smerter \citep{Petersen2015}. 
Et studie af \citer{Bourne2010} viser imidlertid, at 11 til 25\% af TKA-patienter generelt er utilfredse efter operationen, hvorved behandlingen fra et patientøjemed ikke vurderes som succesfuldt.
Modsat ses det af sundhedsstyrelsens årsrapport for totale knæalloplastikoperationer at alle succeskriterier overholdes for alle operationer. \citep{aarsrapport2016} 
Der opstår således et dilemma i at behandlingen af knæartrose, set fra sygehusvæsenets perspektiv, er succesfuldt, men fra patientens synspunkt er mislykket.
Der opstilles derfor en initierende problemstilling:

\subsection*{Initierende problemstilling}
\begin{center}
	\textit{Hvorfor får nogle knæartrose patienter kroniske postoperative smerter efter en total knæalloplastik?}
\end{center}


\section{Problemanalyse}
\textit{måske der skal være noget tekst her inden det hele starter?} \\

\subsection{Patientmålgruppe}

En længere række faktorer har betydning for udviklingen af artrose. Hvis en eller flere af disse faktorer er tilstede, er den påvirkede mere disponeret for knæartrose. Dette sker ved overbelastning igennem arbejde og fritid, tidligere knæskader, genetisk arv, overvægt samt køn \citep{brostrom2012}. Knæartrose er til stede blandt 45\% af alle 80-årige i befolkningen. Antallet af personer med knæartose kan formodes at stige da levealderen i Danmark stiger. %Dette er ikke den eneste faktor, hvorfor prævalensen kan antages at stige. 
En anden risikofaktorer for udviklingen af knæartrose er overvægt, hvilket 47\% af den danske befolkning kan kategoriseres som. Ydermere stiger andelen af overvægtige med alderen, hvilket ligeledes er tilfældet for knæartrose. \citep{Vestergaard2014} \citep{Vestergaard2016} \citep{Lind2016} \citep{Lind2016b} 
%Overvægtige med en høj body-mass-index (BMI>30\citep{definitionfedme1999}) er disponeret for knæartrose med en relativ risiko på en faktor tre, hvoraf en kombination af ovenstående faktorer øger risikoen for lidelsen. Dog kan der opstå problematikker vedrørende benyttelsen af BMI, da metoden ikke skelner mellem fedt og muskler. \textbf{(8)} \citep{brostrom2012} \citep{Vestergaard2014} \citep{Vestergaard2016} \citep{Lind2016} \citep{Lind2016b}

En patients symptomer kan medføre igangsættelsen af et behandlingsforløb. Et behandlingsforløb for en patient med knæartrose består af flere faser, hvis mål er smertelindring, mobilitetsforøgelse samt forebyggelse. Generelt kan faserne opdeles i non-invasive og invasive metoder. Hvilken metode som hjælper patienten afhænger af graden af knæartrose.

\begin{figure}[H]
	\centering
	\includegraphics[width=0.9\textwidth]{../figures/bProblemanalyse/flowchart_behandlingsforloeb.png}
	\caption{På figuren ses et flowchart indeholdende de forskellige behandlingsmetoder der forekommer igennem et patientforløb med knæartrose.}
	\label{fig:flow_behandlingsfaser}
\end{figure}\vspace{-.25cm}

Patientgruppen som postoperativt er utilfredse er svært definerbar. Problematikken opstår i og med klassificeringen bag de potentielt 11 til 25~\% utilfredse patienter er vedrørende postoperative smerte samt funktion. Det kan forestilles at der blandt nogle af patienterne findes en forventningsfaktor, hvilket gør de postoperativt kategoriserer sig selv som værende utilfreds, omend de rent faktisk har opnået en forbedring af både smerte og eller mobilitet.

%Det kan tænkes at forventningsfaktoren kan være medvirkende til at kategorisere dem som værende utilfredse, som resultat af skuffelsen af ikke at fungere som et individ med et fuldt funktionsdygtigt knæ. Denne antagelse understøttes af \citer{Bourne2010}, som beskriver de største prædiktorer omhandlende utilfredshed efterfulgt af en TKA-operation. Den faktor som besidder den største score er, patientens forventninger til operationen ikke er mødt, hvilket medfører 10,7 gange større risiko for utilfredshed. \citep{Bourne2010} 
%I et andet studie af \cite{Keudell2013} bliver patienternes alder sammenkoblet med deres forventninger. Det tyder på at den ældre patientgruppe (>65 år) har generelt har lavere forventninger til operationsresultatet, end den yngre patientgruppe (<55 år). I dette studie indikeres det at den ældre patientgruppe generelt har større tilfredshed, end den yngre. Dette kan antages at have en sammenhæng med påstanden fra \cite{Bourne2010}, vedrørende prædiktoren til utilfredshed, omhandlende forventninger til operationen. \textbf{NOTE:(7 (mangler analyse?))}

\subsection{Kirurgisk behandling}

Kirurgiske behandlingsmetoder indebære osteotomi og knæalloplastik. Osteotomi har tilformål at afhjælpe den mekaniske belastning i det berørte område, for derved at afhjælpe smerterne. Ved osteotomi fjernes der oftest en kile af tibia-knoglen og det resterende knogle sikres med skruer og metal plader. Proceduren ændre knæets mekaniske akse, hvilket vil ændre belastningen af de degenererede områder. \citep{Osteotomi_og_TKA}
Alloplastik er et operativt indgreb der har til formål helt eller delvist at udskifte knæleddet, med specielt designede metal- og plastkomponenter som varig erstatning for bruskfladerne i knæet. Operationen opdeles i TKA og UKA, hvilket henholdsvis er helt eller delvis udskiftning af knæleddet og afhænger af den specifikke diagnose. Der kan ved traume tilfælde eller svære beskadigelser af de anatomiske strukturer omkring knæet forekomme specialiserede udgaver af knæalloplastik.

\begin{figure}[H] 
	\begin{center}
		\includegraphics[width=0.5\textwidth]{../figures/tka_implant}
	\end{center}
	\caption{Komponenterne til en total knæalloplastik, består af et femural og tibia implantat ofte bestående af en titaniumlegering. Patella- og tibiaindsatsen er lavet af polyethylen, hvilket er med til at mindske friktionen og efterligne knæledes naturlige bevægelse.\citep{1}} 
	\label{fig:tka_implant} 
\end{figure}

Efter en vellykket operation burde patienter ikke oplever smerter, dette er imidlertid ikke tilfældet. Der er således et problem med resultatet af operationen, på trods af operationen overholder sundhedsstyrelsens succeskriterier \citep{aarsrapport2016}. Det må derfor antages problemet ikke ligger ved operationen, men da en del patienter oplever postoperative smerter, er det relevant at undersøge smertes indflydelse på resultatet af operationen. 

\subsection{Smerte}

Smerte er defineret som: “\textit{An unpleasant sensory and emotional experience associated with actual or potential tissue damage, or described in terms of such damage.}” af The International Association for the Study of Pain (IASP) \citep{Giangregorio1997} \citep{Carmon}.
%Smerter registreres af noiceptorer i kroppen. De fleste smertereceptorer er frie neuronender som forgrener sig ud i hele kroppen, og ligger frit i væv. De dækker således et stort område og kan reagere på forskellige typer stimuli som vævet bliver udsat for. Typen af stimuli bliver bestemt afhængig af hvilke nociceptorer som aktiveres. Forskellige nociceptorer besidder forskellige typer af natrium-/kaliumporte, som aktiveres ved temperaturændring, trykforandring og kemisk foranding, som ses på figur \ref{neuronport}.
Der findes flere måder at opdele og kategorisere smerte på, men generelt kan det opdeles i to overordnede kategorier; akut og kronisk. Efter TKA-operationer oplever patienter akut smerte, men denne fortager sig hurtigt hvis alt forløber som forventet. Problemet opstår når patienterne får kroniske postoperative smerter. Kronisk smerte er af IASP defineret som at være smerteperception som varer længere end det generelt ville forventes, oftest sættes denne grænse ved tre måneder \citep{Carmon}. Kronisk smerte kan skyldes fejl i nervesystemet på baggrund af akutte nociceptiske eller neuropatiske skader. Efter den akutte smerte har fortages sig, fortsætter smerteperceptionen, tilsyneladende uden grund. Patienten kan så opleve konstante smerte, periodiske smerte, smerte ved bevægelse eller opbyggende smerter som kulminere i det uudholdelige. \citep{Giangregorio1997}
%Kronisk smerte kan også være af psykogen oprindelse. Psykogene smerter er en imaginær perception af smerte, og derved besværlig at præcisere, idet der ikke er og muligvis aldrig har været en fysisk årsag til smerten. Hos en person med psykogene smerter er hjernen fuldt overbevidst om, at den oplever fysiske, nociceptiske eller neuropatiske, smerter og lider deraf. Smerten er udelukkende psykisk hos personen, men er af den grund ikke mindre virkelig, grundet smertes subjektive natur. \citep{Giangregorio1997}

\subsection{Klinisk udvælgelse af patienter til TKA-operation}

%Patienter som tilbydes en TKA udvælges på baggrund af en læge eller kirurgs observationer og erfaringer.
Sundhedsstyrrelsen har udarbejdet nationale retningslinjer for hvordan udvælgelsen til TKA-operation skal ske, men udvælgelsen udføres af en læge eller kirurg, ved hjælp af retningslinjerne og dennes personlige observationer og erfaringer. Patienter kan hermed opleve forskellige anbefalinger og behandlingsmuligheder ved forskellige klinikere. 
%Sundhedsstyrrelsen har udarbejdet nationale retningslinjer samt en række indikationer som kan få klinikeren til at fravælge en patient til en operation. Disse indikationer er eksempelvis infektioner i knogler eller leddet, hvis patienten ingen smerter har i leddet eller hvis patienten har en kort forventet levetid. 
%En anden indikator, som kan få en kliniker til at fravælge at operere patienten, er hvis patienten har urealistiske forventninger til operationen.
Et studie af \citer{skou2016} har fundet en uoverenstemmelse mellem hvilke faktorer kirurgere fandt vigtigst for patientens egnethed til TKA og hvilke faktorer de samme kirurger udvalgte patienter ud fra. Denne uoverensstemmelse viser hvor kompleks udvælgelsesprocessen for en TKA er, samt vanskelighederne ved at bestemme hvilke faktorer der har størst betydning for en klinikers beslutningsprocess. Udfra resultaterne fra studiet af \citer{skou2016} antydes det at klinikerne anvender både bevidste og ubevidste faktorer til bestemmelse af patienters egnethed til en TKA. Desuden er det fundet at klinikerne har svært ved at vurdere patienter, hvis ikke der er klare indikationer som gør patienten enten egnet eller uegnet til en TKA-operation. 
I de tilfælde hvor der ikke er klare indikationer på at patienten vil have gavn af en TKA, ville en teknologisk metode der kunne hjælpe klinikeren med at vurdere patienten. Ved anvendelse af en sådan teknologisk metode vil det ligeledes være muligt at inddrage andre faktorer i udvælgelsen af patienter til TKA, samt undgå mulige ubevidste bias hos kilnikeren. 

\subsection{Teknologier til vurdering af smerte}
For at understøtte klinikerens vurdering af de enkelte patienter vedrørende henstilling til TKA, kan det overvejes, hvorvidt det vil være hensigtsmæssigt at tilføje en ekstra undersøgelse til patientens udredningsforløb forud for beslutningstagen. %Relevante undersøgelser indebære objektive målinger af neural aktivitet i forskellige situationer, samt vurdering af patienternes individuelle respons på forskellige typer af stimuli. 
Der findes flere teknologier hvilke objektivt kunne undersøge patienters perception af smerte og reaktion på stimuli, men der fokuseres her udelukkende på quantitative sensory testing (QST) for korthedens skyld. \\
QST er en metode til objektiv undersøgelse af det sensoriske nervesystems funktion. %Metoden kan anvendes til undersøgelse af forskellige egenskaber, herunder smertegrænser. 
Metoden eksponerer patienten for forskellig sensorisk sensation, som varme, kulde, vibration og tryk. Patienten placere sin opfattelse af smerten på en VAS (Visual Analog Scale) hvor smerten rangeres mellem 1 (mindst smerte) til 10 (værst tænkelige smerte). Dermed er det muligt at identificere patientens perception af smerte.
QST bliver i forskningsregi anvendt til undersøgelse af patienter der får udført TKA. I et studie af \citer{Martinez2007} undersøges 20 patienter med artrose i knæet før og efter en TKA. Formålet hermed var at identificere faktorer, der har indvirkning på udviklingen af postoperative smerter. 
%%QST blev udført henholdsvis før operationen og efterfølgende en og fire dage samt en og fire måneder efter operationen. Parametrene der blev undersøgt i QST'en var tærskelværdier for temperatur, mekanisk smerte og hvordan de enkelte patienter responderede på en eksponering for temperaturer over tærskelværdierne. 
Studiet fandt, at der forekommer en sammenhæng mellem periodiske smerter efter operationen og de patienter, der oplever hyperalgesi under eksponering for varme. \citep{Martinez2007} 
%Der skal imidlertid tages højde for, at der er flere fejlkilder forbundet med QST, da nøjagtigheden i høj grad afhænger af både patientens og undersøgerens præcision under udførelsen af de enkelte tests. Det må således forventes, at der kan forekomme variationer mellem enkelte QST udført på den samme patient. \citep{Yarnitsky2006}

\subsection{Problemafgrænsning}

\textit{tekst} \\
\textit{tekst} \\
\textit{tekst} \\

\subsection*{Problemformulering}

\begin{center}
	\textit{Hvilke konsekvenser er der forbundet med en implementering af QST som supplement til klinikerens vurdering af patienten til indstilling til TKA, på de ortopædkirurgiske afdelinger i Region Nordjylland?}
\end{center}




\begingroup
\label{litteraturliste}
\raggedright
\bibliographystyle{unsrtnat}
\bibliography{../kilder}
\endgroup


\end{document}




