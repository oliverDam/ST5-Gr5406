\subsection{Repeterbarhed}
I et studie af \citer{Nielsen2015} er repeterbarheden for manuelt udførte PPT-undersøgelser blevet undersøgt. De 136 forsøgspersoner, der indgik i studiet, blev undersøgt ved brug af et trykalgometer fra Somedic.  Forsøget blev udført på benet (quadriceps) og på armen (biceps brachii) i forsøgspersonernes dominerende side. Studiet var struktureret i to sessioner med minimum en uges mellemrum. Forsøgets resultater viste, at ICC mellem de målte data for de to sessioner var 0,89 for ben og 0,87 arme. Begge disse værdier er i studiet defineret som værende høje, da de overskrider den fastsatte tærskelværdi herfor på 0,75. Analysen af resultaterne viste endvidere en bias både for ben og arme i form af en forskel mellem de beregnede middelværdier for de to sessioner. Det samme studie undersøgte repeterbarheden for PPT-undersøgelser udført med computerstyret cuff-algometri. Til denne undersøgelse blev et cuff-algometer fra Nocitech placeret henholdsvis på den nedre del af benet og den øvre del af armen i den nondominante side. Forsøgspersonerne skulle under forsøget kvantificere deres smerte ved hjælp af en elektronisk VAS-tabel. Når smerten blev uudholdelig skulle forsøgspersonerne selv trykke på en knap, der stoppede trykøgningen i cuff-algometeret, hvorefter forsøget var afsluttet. Resultaterne viste en korrelationskoefficient på 0,79 for benet og 0,85 for armen, hvilket igen i dette studie er defineret som høje korrelationer. Ligesom for den manuelle PPT-undersøgelse viste dette forsøg en systemisk fejl mellem de to sessioner, men kun for målingerne på armen.\\
I et litteraturstudie af \citer{Kennedy2016} blev repeterbarheden for CPM-undersøgelser på forskellige områder af kroppen analyseret. Analysen var baseret på 10 studier, hvoraf fem studier anvendte PPT som teststimuli. Typen af konditionerede stimuli varierede i mellem de fem studier. I et studie blev den opfølgende test lavet under samme session som den primære, imens der i tre andre studier blev udført en opfølgende test i løbet af to til 10 dage. I det sidste studie blev der både foretaget en opfølgende test i samme session som den primære og yderligere en opfølgning tre dage senere. \citer{Kennedy2016} opdelte studiernes resultater i henholdsvis repeterbarhed for teststimuli og for konditioneret stimuli. Det fremgår, at ICC for forsøgene, hvor den opfølgende test er udført i samme session som den primære, ligger på 0,82-0,87 for teststimuli (PPT), hvilket overskrider grænsen for høj korrelation på 0,75. For studierne, der har afholdt to uafhængige sessioner ligger ICC i intervallet 0,65-0,79 og defineres derfor som værende god til høj.For de konditionerede stimuli, hvor den opfølgende test blev lavet i samme session som den primære, lå ICC på 0,60-0,94. For studierne, der udførte den opfølgende test i en separat session var ICC i intervallet 0,61-0,82.\\ 
Et studie af \citer{Imai2016} undersøgte ligeledes, hvordan CPM-undersøgelsers repeterbarhed påvirkes ved ændring af test- og konditionerede stimuli. I studiet indgik 26 raske mænd. Hver forsøgsperson gennemgik to identiske sessioner med højst 3 ugers mellemrum. Der blev anvendt fire forskellige typer teststimuli, hvoraf de to var baseret på tryk (manuel- og cuff-algometri). Trykket blev påført med et algometer fra Somedic og en cuff fra Nocitech. Både det håndholdte og cuff-algometeret var placeret på underbenet ved forsøgene. Den konditionerede stimuli blev påført kontralateralt for teststimuli og var udgjort af CPT og cuff-algometeret. I studiet er ICC-værdierne for forsøgene med trykbaseret teststimuli henholdsvis 0,49 og 0,44 for CPT mens de for cuff-algometri er 0,04 og 0,53.\\ 
I studiet af \citer{Nielsen2015} blev TSP undersøgt i forlængelse af de cuffbaserede tests af PPT og smertetolerance (PTT). Undersøgelsen foregik ved at cuff-algometeret blev pustet op til et tryk svarende til den fundne smertetolerance 10 gange af to sekunders varighed. Resultaterne for studiet viste en ICC på 0,60 for benet og 0,43 for armen.  

Flere studier har undersøgt repeterbarheden for udførelsen af PPT- CPM- og TSP-undersøgelser. På baggrund af resultaterne kan det ses, at der generelt forekommer en høj ICC for PPT-undersøgelser baseret på de definerede tærskelværdier i de respektive studier. For CPM-undersøgelser forekommer der generelt en større variation i resultaterne. ICC-værdierne for TSP fundet af \citer{Nielsen2015} er i intervallet middel til god, men dette resultat kan man fordel underbygges af flere studier. 

