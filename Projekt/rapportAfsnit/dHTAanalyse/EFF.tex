%\chapter{Klinisk effektivitet (EFF)}
\section{Formål}
I dette domæne analyseres teknologiens virkningsgrad ud fra et klinisk synspunkt. For at kunne danne et vurderingsgrundlag for, hvorvidt QST skal implementeres som supplement til klinikerens vurdering, er det nødvendigt at undersøge teknologiens effektivitet og egentlige effekt.\\
For at kunne vurdere virkningsgraden af QST er det nødvendigt at undersøge de gavnlige og ikke-gavnlige effekter ved brug af teknologien. Denne effektvurdering skal danne grundlag for forståelsen af, hvordan QST kan fungere som supplement til klinikerens beslutning. Ligeledes vil effekten af teknologien sammenholdes med de eventuelle sikkerhedsmæssige risici. Dette skal samlet bidrage til beslutningen om hvorvidt udbyttet af QST er tilstrækkelig i forhold til de sikkerhedsmæssige risici, og i hvilken grad QST kan benyttes som supplement. \\
Ifølge analysen omhandlende klinikerens vurdering, er denne mangelfuld i forhold til identificering af patienter med kroniske postoperative smerter. Beslutningen bygger ikke på tilstrækkelig viden, hvoraf det er relevant at vurdere nøjagtigheden af QST. Nøjagtigheden af resultaterne fra QST skal undersøges, med henblik på at belyse hvorvidt patienter som udvikler kroniske postoperative smerter kan identificeres mere præcist ved anvendelse af QST end uden QST. Derudover vil det også være relevant at undersøge, hvor stor en andel af patienter uden kroniske postoperative smerter teknologien kan identificere. Da patienterne med kroniske postoperative smerter bør kunne identificeres præoperativt med QST som supplement, er det relevant at undersøge hvordan QST-resultaterne for patienter uden kroniske postoperative smerter og en patient med kroniske postoperative smerter adskiller sig fra hinanden. Repeterbarheden for QST undersøges ydermere for at kunne vurdere dens egnethed som et pålideligt supplement til klinikerens beslutning.

\section{HTA spørgsmål}
\textit{Effekt og skade:}
\begin{itemize}
	\item Hvilke gavnlige effekter er der ved QST-undrsøgelserne? %D0001, 
	\item Hvad er net-benefit af QST?\fxnote{Det her spørgsmål er ikke besvaret!?} %D0020, D0029 
\end{itemize}
\textit{Nøjagtighed:}
\begin{itemize}
	\item Hvordan adskiller de præoperative QST-resultater sig fra patienter med kroniske postoperative smerter og patienter uden kroniske postoperative smerter? %B0018, B0008, B0009, B0010d
	\item Hvorvidt er undersøgelser med QST repeterbare? 
\end{itemize}
%\subsection{Metode \citep{HTAcore}}
%For at besvare analysespørgsmålene, foretages en systematisk litteratursøgning i relevante databaser som PubMed, Medline, Cochrane library med mere. Grundlæggende vil vurderingen af relevante artikler vurderes i forhold til evidenshirakiret (se afsnit ref{xx} ). Dette betyder at som udgangspunkt vil meta-studier og randomiserede studier være at foretrække. I forhold til at opnå indsigt i sundhedsfordele og ulemper der er relateret til QST vil studier hvori der indgår information om mortalitet, morbiditet og quality of life blive taget i betragtning. For at opnå viden om nøjagtigheden af QST, vil studier hvori teknologiens sensitivitet og specificitet dokumenteres blive taget i betragtning. Ved vurdering af virkningsgradens normale forhold, kan det blive nødvendigt at undersøge om virkningsgraden kan extrapoleres fra virkning under optimale forhold til normale forhold.

\section{Metode}
For at opnå viden om nøjagtigheden af QST, vil studier hvori teknologiens effektivitet dokumenteres blive taget i betragtning. Ved vurdering af virkningsgradens normale forhold, kan det blive nødvendigt at undersøge om virkningsgraden kan extrapoleres fra virkning under optimale forhold til normale forhold. Litteratursøgningen er inspireret af PICO (Patient/Problem, Intervention, Comparison and Outcome).\fxnote{Kilde på det her?} Dette betyder, at søgningen er afgrænset til kun at omhandle prædiktering af kroniske postoperative smerter efter TKA ved hjælp af de tre QST-parametre; PPT, TSP og CPM. Det blever foreslået, at disse parametre kan afvige fra normalen ved en række sygdomstilstande \citep{Petersen2016}. Studier omhandlende andre sygdomstilstande en artrose inkluderes ikke, da projektgruppen ikke har fundet evidens for, at disse resultater uden videre kan generaliseres til TKA-patienter. 

\section{Effekt og skade}
\textit{I følgende afsnit undersøges, den samlede effekt af behandlingen, hvori QST er tiltænkt at indgå. QST er kun et delelement i patientens behandling og derfor er en vurdering af isolereret vurdering af QST-teknologien ikke tilstrækkelig.\fxnote{Den her argumentation kan jeg slet ikke følge - det giver da mening at se på effekten af QST alene}}

\subsection{Gavnlige effekter ved QST} \fxnote{Jeg kan ikke se sammenhængen mellem det her og spørgsmålet - og slet ikke argumentationen for kun at fokusere på medicin og ikke selve QST-undersøgelserne}
Ved at kunne prædiktere om patienten får kroniske smerter, kan patienterne informeres om den mulige forhøjede risiko for at opleve kroniske smerter, som der kan være forbundet med en TKA-operation. Patienten kan således informeres om, at medicinsk behandling kan reducere risikofaktorer for udvikling af kroniske postoperative smerte.%, herunder TSP og CPM. 
For eksempel er NMDA antagonisten, Pregabalin, blevet foreslået til at reducere kronisk smerte som følge af TKA. Der er dog ikke et entydigt resultat af, hvor stor denne effekt er. \citer{Lunn2016} har i et studie med 300 patienter undersøgt effekten af Gabapentin, hvor der ikke blev fundet en signifikant forskel mellem placebogruppen og gruppen der fik medicin. Dette står dog i kontrast til et andet studie af \citer{Buvanendran2010}, som med 120 forsøgspersoner viser at der var en signifikant effekt ved medicnsk behandling med Pregabalin. 
Det optimale udbytte af den samlede diagnosticering og behandling, er at patienten ikke oplever kronisk smerte efter en TKA-operation. %\citer{Kristensen2015} har i et studie med 215 primære TKA-patienter vist at kronisk smerte 3 år efter operationen har en sammenhhæng med Knee Society Score (KSS) som er en funktionsscore der afspejler patientens knæfunktionalitet.

\fxnote{Der mangler et afsnit om net-benefit!}

\section{Nøjagtighed}
\textit{I følgende afsnit vil QST-undersøgelsernes nøjagtighed blive beskrevet, både i forhold til den kliniske effekt, men også i forhold til teknologiens relibilitet. Dette er for at kunne vurdere egentheden af QST-undersøgelserne, som et brugbart supplement.}

\subsection{Klinisk effekt}
For at vurdere hvor nøjagtig QST er, benyttes data fra studier der har undersøgt kronisk smerte efter en TKA. Kronisk smerte defineres som smerte der rækker ud over den normale vævshelingsperiode. I forhold til TKA-patienter kan denne periode vare i op til 12 måneder \citep{Wylde2016review}. Derfor vil de inkluderede studier også benytte denne minimumsperiode som referencepunkt. \\
%Under litteratursøgningen blev 6 studier fundet relevante. Disse er i det følgende beskrevet i kronologisk rækkefølge.

\citer{Wylde2013} undersøgte forholdet mellem præoperative smertetærskler hos TKA patienter og disse patienters oplevelse af smerte 12 måneder efter TKA-operationen. I studiet deltog 51 patienter der var indstillet til TKA. Disse fik målt deres smerteintensitet ved hjælp af WOMAC (Western Ontario and McMaster Universities Osteoarthritis Pain scale). Denne smertetest tager gangsmerter, smerte ved brug af trapper, smerter i siddende eller liggende position og smerter ved stående position, i betragtning. \fxnote{Kilde?} I studiet blev blandt andet PPT målt. Disse målinger blev sammenlignet med målinger på 50 personer uden knæsmerter. I studiet blev der fundet en signifikant korrelation mellem præoperativ lav PPT, målt på underarmen og øget smerte 12 måneder efter TKA-operationen.

Wylde et al, 2014 har i et dobbeltblindet, randomiseret single-center studie med 239 patienter, undersøgt om PPT målt på underarmen kunne prædiktere om patienter ville komme til at opleve smerte 12 måneder efter operationen. Smerten blev vurderet ud fra WOMAC smertescoren. Det blev fundet at lav PPT korrelerede i forhold til patientens oplevelse af præoperativ smerteintensitet. Derudover blev det fundet at PPT ikke kunne prædiktere niveauet af smertelindring efter TKA-operation, uafhængigt af præoperativ smerteintensitet.

Wright et al, 2014 har i et studie undersøgt om der eksisterede en sammenhæng i forhold til målt PPT hos 53 TKA-patienter som rapporterede henholdsvis moderat til svær smerte eller ingen smerte, 12 måneder efter TKA-operation. Til at måle PPT blev et algometer med $1 cm^{2}$ probe anvendt (Somedic,Sverige). I studiet blev det fundet at patienter med moderat til svær smerte havde reduceret PPT ved knæet.

<<<<<<< HEAD
\citer{Peteersen2015} har i et studie undersøgt korrelation mellem PPT, TSP og CPM i forhold til udvikling af kronisk smerte efter TKA-operation. Smerteintensiteten blev målt ved hjælp af VAS, og de tre parametre blev målt henholdsvis før operationen, 2 måneder efter operationen og 12 måneder efter operationen. Patienterne blev opdelt i en gruppe med lav smerte (VAS < 3) (N=61) og en gruppe med høj smerte (VAS>=3) (N=17). Inddelingen var baseret på smerten hos patienten 12 måneder efter TKA-operation. Petersen fandt at præoperativt PPT ikke kunne prædiktere smerte 12 måneder efter operationen. \\ % En von Frey stimulator blev benyttet til at inducere TSP. Smerten blev induceret tæt på det afficerede led. Stimulationen blev påført 10 gange efter hinanden. TSP blev defineret som forskellen i intensiteten mellem første og sidste stimulation. CPM blev defineret som forskellen i PPT før og efter den betingede stimulation. Til at inducere den toniske stimulation blev den kontralaterale hånd i forhold til det afficerede knæ nedsænket i isvand.\\ 
Studiet fandt at TSP med 25.6 gram mekanisk stimulus på det afficerede knæ var relateret til smerte 12 månender efter operationen. I studiet havde gruppen med høj smerte efter 12 måneder, også en højere smerteintensitet før operationen sammenlignet med gruppen med lav-intensitet smerte (P=0.009) 12 måneder efter operationen (P < 0.001). Studiet viste også at PPT-målingerne for lavsmerte gruppen blev normaliserede efter operation (P<0.05) og den kontralaterale arm (P = 0.059). For gruppen med høj smerte efter 12 måneder var præoperativ TSP korreleret til smerte 12 måneder efter operationen (R=0.240, P =0.037).
Studiet viste at der var en positiv korrelation mellem 12 måneders postoperativ smerte intensitet, preoperativt smerteintensitet (R=0.229, P=0.045) og præoperativ TSP (R=0.240, P=0.037). Ingen korrelation blev fundet mellem PPT og postoperativ smerte 12 måneder efter operationen. Derudover var der ikke en korrelation mellem CPM og post-operativ smerte 12 måneder efter operationen. \citep{Petersen2015}


%Petersen et al. 2016 har i et single center studie undersøgt post-operativ smertelindring 12 måneder efter TKA-operation. I studiet deltog 103 patienter. Patienterne blev profilerede og derefter opdelt i 4 undergrupper før TKA-operationen. Disse var henholdsvis:
%\begin{itemize}
%	\item a) Faciliteret TSP/nedsat CPM (N=16)
%	\item b) Faciliteret TSP/normal CPM (N=15)
%	\item c) Normal TSP/nedsat CPM (N=44)
%	\item d) Normal TSP/normal CPM (N=28)
%\end{itemize} 
%Der blev benyttet et computerstyret cuff-algometer og en elektronisk VAS-måleenhed. I studiet blev en række parametre målt; TSP, CPM, PDT, PTT og PPT. \\
=======
\citer{Petersen2015} har i et studie undersøgt korrelation mellem PPTs, TS og CPM i forhold til udvikling af kronisk smerte efter TKA. Smerteintensiteten blev målt ved hjælp af VAS, og de tre parametre blev målt henholdsvis før operationen, 2 måneder efter operationen og 12 måneder efter operationen. Patienterne blev opdelt i en gruppe med lav smerte ( VAS < 3) (N=61) og en  gruppe med høj smerte ( VAS>=3) (N=17). Inddelingen var baseret på smerten hos patienten 12 måneder efter TKA. Petersen fandt at præoperativt PPT ikke kunne prædiktere smerte 12 måneder efter operationen. \\ %En von Frey stimulator blev benyttet til at inducere TS. Smerten blev induceret tæt på det afficerede led. Stimulationen blev påført 10 gange efter hinanden. Temporal summation blev defineret som forskellen i intensiteten mellem første og sidste stimulation. CPM blev defineret som forskellen i PPT før og efter den betingede stimulation. Til at inducere den toniske stimulation blev den kontralaterale hånd i forhold til det afficerede knæ nedsænket i isvand.\\
Studiet fandt dog at temporal summation med 25.6 gram mekanisk stimulus på det afficerede knæ var relateret til smerte 12 måender efter operationen. I studiet havde gruppen med høj smerte efter 12 måneder også en højere smerteintensitet før operationen sammenlignet med gruppen med lav-intensitet smerte (P=0.009) 12 måneder efter operationen (P < 0.001). Studiet viste også at PPT-målingerne for lavsmerte gruppen blev normaliserede efter operation (P<0.05) og den kontralaterale arm: (P = 0.059). For gruppen med høj smerte efter 12 måneder var præoperativ TS korreleret til smerte 12 måneder efter operationen ( R=0.240, P =0.037).
Studiet viste at der var en positiv korrelation mellem 12 måneders postoperativ smerte intensitet, preoperativt smerteintensitet ( R=0.229, P=0.045) og præoperativ TS ( R=0.240, P=0.037). Ingen korrelation blev fundet mellem PPT og postoperativ smerte 12 måneder efter operationen. Derudover var der heller ikke en korrelation mellem CPM og post-operativ smerte 12 måneder efter operationen. \fxnote{Det her studie er jo beskrevet en gang ovenover - hvorfor står det her to gange?}

\citer{Petersen2016} har i et single center  studie undersøgt post-operativ smertelindring 12 måneder efter TKA-operation. I studiet deltog 103 patienter. Patienterne blev profilerede og derefter opdelt i 4 undergrupper før TKA-operationen. Disse var henholdsvis a)  Faciliteret TSP/nedsat CPM (N=16), b) Faciliteret TSP/normal CPM (N=15), c)  Normal TSP/nedsat CPM ( N=44), d) Normal TSP/normal CPM. Der blev benyttet et computerstyret cuff-algometer og en elektronisk VAS-måleenhed. I studiet blev en række parametre målt; TSP, CPM, PDT, PTT og PPT. \\
>>>>>>> origin/master
\citer{Petersen2016} fandt, at en lav PDT var associeret med en mindre postoperativ smertelindring (R=-0.222, P=0.034). Derudover blev der fundet en sammenhæng mellem præoperativ smerte og postoperativ smertelindring (R=0.263, P=0.080).
Hverken faciliteret TSP eller nedsat CPM alene, kunne påvise en signifikant sammenhæng i forhold til smerte 12 måneder efter TKA-operation. Studiet viste derudover, at gruppen med faciliteret TSP sammen med nedsat CPM (gruppe a) var den gruppe der oplevede mindst smertelindring.
Disse resultater kan indikere, at gruppe a har mindre sandsynlighed for at opleve en smertelindring.\\
Det er vigtigt at påpege at dette studie er en eksplorativ undersøgelse, og dette indebærer også at der ikke blev udarbejdet en plan på forhånd, for hvorledes de statistiske udregninger skulle udføres.
Derudover nævner forfatteren at studiet ikke var designet til at prædiktere, hvilke patienter der ville udvikle kroniske smerter. \citep{Petersen2016}

Wylde, 2016 har i et studie undersøgt sammenhængen mellem præ-operativ udbredt hyperalgesi og graden af radiografisk artrose, i forhold til smertegrad før en TKA og 12 måneder efter. I studiet deltog 241 TKA-patienter. Smerten blev vurderet ud fra WOMAC-skalaen. Den radiografiske artrose blev vurderet ud fra Kellgren og Lawrence skalaen, som vurderer sværhedsgraden fra 0-4, hvor 4 er sværest. Den udbredte hyperalgesi blev målt ved hjælp af algometer med $1 cm^{2}$ probe (Somedic, sverige) ved måling af PPT på underarmen.
I studiet blev der fundet en signifikant korrelation mellem radiografisk artrosesværhedsgrad og postoperativ smerteintensitet. Derudover var højere PPT signifikant associeret med mindre præoperativ smerte.	
\fxnote{Hvor er sammenligningen mellem studierne - analysedelen af det her afsnit mangler. Hvorfor er de her studier relevante, hvordan hænger de sammen, og hvad siger de om forskellen mellem kronisk smerte og ikke-kronisk smerte patienter? Der bliver heller ikke her svaret på spørgsmålet}

\subsection{Repeterbarhed for QST}
I et studie af \citer{Nielsen2015} er repeterbarheden for manuelt udførte PPT-undersøgelser blevet undersøgt. De 136 forsøgspersoner, der indgik i studiet, blev undersøgt ved brug af et trykalgometer fra Somedic. Forsøget blev udført på låret (quadriceps) og på overarmen (biceps brachii) i forsøgspersonernes dominante side. Studiet var struktureret i to sessioner med minimum en uges mellemrum. Forsøgets resultater viste, at inter-korrelations koefficienten (ICC) mellem de målte data for de to sessioner var 0,89 for ben og 0,87 arme. Begge disse værdier er i studiet defineret som værende høje, da de overskrider den fastsatte tærskelværdi på 0,75. Analysen af resultaterne viste endvidere en bias både for ben og arme i form af en forskel mellem de beregnede middelværdier for de to sessioner. Det samme studie undersøgte repeterbarheden for PPT-undersøgelser udført med computerstyret cuff-algometri. Til denne undersøgelse blev et cuff-algometer fra Nocitech placeret henholdsvis omkring læggen og overarmen i den non-dominante side. Forsøgspersonerne skulle under forsøget kvantificere deres smerte ved hjælp af en elektronisk VAS-tabel. Når smerten blev uudholdelig skulle forsøgspersonerne selv trykke på en knap, der stoppede trykpåvirkningen i cuff-algometeret, hvorefter forsøget var afsluttet. Resultaterne viste en interkorrelationskoefficient på 0,79 for benet og 0,85 for armen, hvilket igen i dette studie er defineret som høje korrelationer. Ligesom for den manuelle PPT-undersøgelse viste dette forsøg en systemisk fejl mellem de to sessioner, men kun for målingerne på armen.\\

I et litteraturstudie af \citer{Kennedy2016} blev repeterbarheden for CPM-undersøgelser på forskellige områder af kroppen analyseret. Analysen var baseret på 10 studier, hvoraf fem studier anvendte PPT som teststimuli. Typen af konditionerede stimuli varierede imellem de fem studier. I et studie blev den opfølgende test lavet under samme session som den primære, imens der i tre andre studier blev udført en opfølgende test i løbet af to til ti dage. I det sidste studie blev der både foretaget en opfølgende test i samme session som den primære og yderligere en opfølgning tre dage senere. \citer{Kennedy2016} opdelte studiernes resultater i henholdsvis repeterbarhed for teststimuli og for konditioneret stimuli. Det fremgår, for forsøgene, hvor den opfølgende test er udført i samme session som den primære, at ICC ligger på 0,82-0,87 for teststimuli (PPT), hvilket overskrider grænsen for høj korrelation på 0,75. For studierne, der har afholdt to uafhængige sessioner ligger ICC i intervallet 0,65-0,79 og defineres derfor som værende god til høj. For de konditionerede stimuli, hvor den opfølgende test blev lavet i samme session som den primære, lå ICC på 0,60-0,94. For studierne, der udførte den opfølgende test i en separat session var ICC i intervallet 0,61-0,82.\\ 
Et studie af \citer{Imai2016} undersøgte ligeledes, hvordan CPM-undersøgelsers repeterbarhed påvirkes ved ændring af test- og konditionerede stimuli. I studiet indgik 26 raske mænd. Hver forsøgsperson gennemgik to identiske sessioner med højst 3 ugers mellemrum. Der blev anvendt fire forskellige typer teststimuli, hvoraf de to var baseret på tryk (manuel- og cuff-algometri). Trykket blev påført med et algometer fra Somedic og en cuff fra Nocitech. Både det håndholdte og cuff-algometeret var placeret på underbenet ved forsøgene. Den konditionerede stimuli blev påført kontralateralt for teststimuli og var udgjort af cold pressor threshold (CPT) og cuff-algometeret. I studiet er ICC-værdierne for forsøgene med trykbaseret teststimuli henholdsvis 0,49 og 0,44 for CPT, mens de for cuff-algometri er 0,04 og 0,53.\\ 
I studiet af \citer{Nielsen2015} blev TSP undersøgt i forlængelse af de cuffbaserede tests af PPT og smertetolerance (PTT). Undersøgelsen foregik ved at cuff-algometeret blev pustet op til et tryk svarende til den fundne smertetolerance 10 gange af to sekunders varighed. Resultaterne for studiet viste en ICC på 0,60 for benet og 0,43 for armen. 

Flere studier har undersøgt repeterbarheden for udførelsen af PPT- TSP- og CPM-undersøgelser. På baggrund af resultaterne kan det ses, at der generelt forekommer en høj ICC for PPT-undersøgelser baseret på de definerede tærskelværdier i de respektive studier. For CPM-undersøgelser forekommer der generelt en større variation i resultaterne. ICC-værdierne for TSP fundet af \citer{Nielsen2015} er i intervallet middel til god, men dette resultat kan med fordel underbygges af flere studier. 

%% Skal vi nævne at Cuffen er udviklet af AAU
Det er ligeledes væsentligt for vurderingen af den samlede behandling, hvilke tiltag der er mulige, og hvor stor effekten er af disse, hvis QST kan identificere patienter som har en forhøjet risiko for at udvikle kronisk smerte efter en TKA-operation.

\section{Delkonklusion}
Det er af væsentlig betydning at tage i betragtning, at de fundne studier har undersøgt virkningsgraden for QST under forskningsmæssige forhold. Flere af de undersøgte studier har vist, at der statistisk set er en sammenhæng mellem de undersøgte prædiktorer, og kroniske smerte efter TKA-operation. Det er dog karakteristisk for studierne, at der foreligger divergerende resultater, og at der generelt er en ringe korrelation mellem de undersøgte parametre og forekomst af kronisk smerte efter TKA-operation. Det har heller ikke være muligt, at opspore konkrete tal i forhold til specificitet og sensitivitet, i forhold til at kunne vurdere den kliniske effekt af QST, og dermed kunne vurdere det diagnostiske potentiale for teknologien. Dette skal betragtes i lyset af manglen på egentlige tærskelværdier for hvornår man definerer en patient som en kronisk smerte patient, men også for hvorledes man definerer en patient til at have faciliteret TSP eller svækket CPM. Det giver også anledning til bekymring, i forhold til i hvor høj grad resultaterne kan overføres fra forskningsregi til daglig praksis.
I denne analyse blev der generelt fundet en god repeterbarhed for måling af PPT. Resultaterne for TSP- og CPM-parametrene havde dog en højere grad af spredning i resultaterne.
Det er dog essentielt at understrege, at QST i forhold til prædiktion af kronisk postoperativ smerte, er en teknologi der stadig er under afprøvning og udvikling, og de lovende indikationer som er vist i studierne skal replikeres for at kunne cementere den kliniske effektivitet af QST.


