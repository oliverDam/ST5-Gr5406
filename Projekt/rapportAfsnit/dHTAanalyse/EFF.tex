\section{Klinisk effektivitet (EFF)}
\subsection{Formål}
I dette domæne analyseres teknologiens virkningsgrad ud fra et klinisk synspunkt. For at kunne danne et vurderingsgrundlag for, hvorvidt QST skal implementeres som supplement til klinikerens vurdering, er det nødvendigt at undersøge teknologiens effektivitet og egentlige effekt.\\
For at kunne vurdere virkningsgraden af QST er det nødvendigt at undersøge de gavnlige og ikke-gavnlige effekter ved brug af teknologien. Denne effektvurdering skal danne grundlag for forståelsen af,  hvordan QST kan fungere som supplement til klinikerens belsutning. Ligeledes vil effekten af teknologien sammenholdes med de eventuelle sikkerhedsmæssige risici. Dette skal samlet bidrage til beslutningen om hvorvidt udbyttet af QST er tilstrækkelig i forhold til sikkerhedsmæssige risici, og i hvilken grad QST kan benyttes som supplement. \\
Ifølge analysen omhandlende klinikerens vurdering, er denne ikke god nok. Beslutningen bygger ikke på tilstrækkelig viden, hvoraf det er relevant at vurdere nøjagtigheden af QST. Nøjagtigheden af resultaterne fra QST skal undersøges, med henblik på at belyse hvorvidt patienter som udvikler kroniske postoperative smerter kan identificeres mere præcist ved anvendelse af QST end uden QST. Derudover vil det også være relevant at undersøge, hvor stor en andel af patienter uden kroniske postoperative  smerter teknologien kan identificere. Da patienterne med kroniske postoperative smerter bør kunne identificeres præoperativt med QST som supplement, er det relevant at undersøge hvordan resultaterne for patienter uden kroniske postoperative smerter og en patient med kroniske postoperative smerter adskiller sig fra hinanden. Repeterbarheden for QST undersøges ydermere for at kunne vurdere dens egnethed som et pålideligt supplement til klinikerens beslutning.
\subsection{HTA spørgsmål}
\textit{Effekt og skade:}
\begin{itemize}
	\item Hvilke gavnlige effekter er der ved teknologien? %D0001, 
	\item Hvad er net-benefit af teknologien? %D0020, D0029
\end{itemize}
\textit{Nøjagtighed:}
\begin{itemize}
	\item Hvordan adskiller de præoperative QST-resultater sig fra patienter med kroniske postoperative smerter og patienter uden kroniske postoperative smerter? %B0018, B0008, B0009, B0010d
	\item Hvorvidt er undersøgelser med QST repeterbare? 
\end{itemize}
%\subsection{Metode \citep{HTAcore}}
%For at besvare analysespørgsmålene, foretages en systematisk litteratursøgning i relevante databaser som PubMed, Medline, Cochrane library med mere. Grundlæggende vil vurderingen af relevante artikler vurderes i forhold til evidenshirakiret (se afsnit ref{xx} ). Dette betyder at som udgangspunkt vil meta-studier og randomiserede studier være at foretrække. I forhold til at opnå indsigt i sundhedsfordele og ulemper der er relateret til QST vil studier hvori der indgår information om mortalitet, morbiditet og quality of life blive taget i betragtning. For at opnå viden om nøjagtigheden af QST, vil studier hvori teknologiens sensitivitet og specificitet dokumenteres blive taget i betragtning. Ved vurdering af virkningsgradens normale forhold, kan det blive nødvendigt at undersøge om virkningsgraden kan extrapoleres fra virkning under optimale forhold til normale forhold.

