\section{Klinisk effektivitet (EFF)}
\subsection{Formål}
I dette område analyseres teknologiens virkningsgrad ud fra et klinisk synspunkt. Formålet med dette er at undersøge både hvor godt QST virker under forskningsmæssige forhold, og i praksis på klinikken. Som basis for vurdering af den samlede virkning af QST, vil både sundhedsfordele og ulemper i forhold til prædiktion af postoperative smerter naturligvis betragtes fra et klinisk perspektiv. Dette er vigtig information for klinikeren i forhold til at træffe en beslutning om det videre behandlingsforløb for patienten, og dermed kunne vælge en behandling der gør mest mulig gavn.\\
For at kunne vurdere virkningsgraden af QST er det nødvendigt at undersøge de gavnlige og ikke-gavnlige effekter ved brug af teknologien. Herunder medregnes også en undersøgelse af hvorvidt de gavnlige effekter kan medføre negative helbredsmæssige konsekvenser, som ikke direkte er relateret til teknologien.\\
Derudover vil en vurdering af hvorvidt teknologiens testresultater har indflydelse på patientens livskvalitet, og i så fald i hvilket omfang, også være nødvendig for at vurdere den overordnede virkningsgrad af teknologien. \\
Endvidere vil en undersøgelse af muligheder og potentielle konsekvenser efter et givent testresultat også være nødvendig for at belyse alle aspekter af den videre behandling.\\
For at vurdere det diagnostiske potentiale af QST er det også nødvendigt at have information om hvor stor en andel af de af patienter der oplever postoperative smerter, som teknologien er i stand til identificere. Ligeledes er det nødvendigt at vide hvor stor en andel af de patienter der ikke oplever kroniske smerter, QST er i stand til at identificere.
\subsection{HTA spørgsmål} %(hvad)

\textit{Effekt og skade:}
\begin{itemize}
	\item Hvilke gavnlige effekter er der ved teknologien? %D0001, 
	\item Hvad er net-benefit af teknologien? %D0020, D0029
\end{itemize}

\textit{Patient effekt:}
\begin{itemize}
	\item Hvordan påvirker teknologiens resultat patientens ikke-helbredsmæssige livskvalitet? %D0030
	\item Hvilken effektiv behandlingsmulighed understøtter teknologiens resultater? (hvis ja/TKA, hvis nej/drugs) %D0024
\end{itemize}
\textit{Nøjagtighed:}
\begin{itemize}
\item Hvorvidt kræves det nøjagtig præcision for benyttelsen af teknologien for korrekt at kunne understøtte klinikerens beslutning? %D1003, D1004, D0021
\end{itemize}

\subsection{Metode \citep{HTAcore}}
For at besvare analysespørgsmålene, foretages en systematisk litteratursøgning i relevante databaser som PubMed, Medline, Cochrane library med mere. Grundlæggende vil vurderingen af relevante artikler vurderes i forhold til evidenshirakiret (se afsnit ref{xx} ). Dette betyder at som udgangspunkt vil meta-studier og randomiserede studier være at foretrække. I forhold til at opnå indsigt i sundhedsfordele og ulemper der er relateret til QST vil studier hvori der indgår information om mortalitet, morbiditet og quality of life blive taget i betragtning. For at opnå viden om nøjagtigheden af QST, vil studier hvori teknologiens sensitivitet og specificitet dokumenteres blive taget i betragtning. Ved vurdering af virkningsgradens normale forhold, kan det blive nødvendigt at undersøge om virkningsgraden kan extrapoleres fra virkning under optimale forhold til normale forhold.
