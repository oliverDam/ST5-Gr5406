\section{Beskrivelse og tekniske karakteristika for teknologien (TEC)}


\subsection{Formål}
I teknologidomænet analyseres QST som et teknologisk supplement til klinikerens vurdering af en patients indstilling til en TKA-operation. \\
For at kunne tage en beslutning om netop dette forventes et indgående kendskab til teknologiens egenskaber og dennes virkemåde. Teknologien benyttes andetsteds, men hvordan denne konkret er tilpasset problemstillingen vedrørende kroniske postoperative smerter efter TKA, er relevant. I sammenhæng med hvordan teknologien fungerer, undersøges hvad det kræves for at benytte teknologien og heraf vedligeholdelse. Relevansen herfor opstår i at dette er essentielt til senere analyseafsnit, samt det er nødvendig viden netop at forstå helheden af teknologien.\\
Enhver teknologi har begrænsninger, og det er essentielt for vurderingsgrundlaget netop at kende disse. Heraf bør det undersøges hvordan eksterne faktorer kan påvirke QST-resultaterne. \\
For endeligt at kunne klassificere den undersøgte teknologi og hvordan denne kan fungere i samspil med den nuværende, er en sammenligning af disse nødvendig. I sammenligningen undersøges det hvordan disse adskiller sig fra hinanden, samt interagerer sammen. %
\subsection{HTA spørgsmål}
\textit{Teknologiens egenskaber:}

\begin{itemize}
\item Hvordan virker teknologien?  %B0001
\item Hvad er teknologiens oprindelige formål og hvordan er den tilpasset knæartrose på nuværende tidspunkt? %B0003, A0022
\item Hvad kræves det at benytte teknologien, og i hvilken grad kræves der vedligeholdelse? %B0012, B0013
\item Hvilke forskelle er der på præoperative QST-resultater mellem en patient med kroniske postoperative smerter og en patient uden kroniske postoperative smerter? %B0018, B0008, B0009, B0010
\end{itemize}

\textit{Teknologiens begrænsninger:}
\begin{itemize}
\item Hvilke eksterne faktorer kan påvirke QST-resultaterne?
\end{itemize}

\textit{Sammenligning med nuværende metoder:}
\begin{itemize}
\item Hvordan adskiller QST-teknologien og den nuværende medicinske teknologi sig fra hinanden? %B0001, B0002
\end{itemize}

\subsection{Metode \citep{HTAcore}}
Vidensindsamlingen til besvarelse af det teknologiske område vil foregå ved at søge efter materiale igennem HTA/MTV agenturer, i relevante sundhedssystemer og sundhedsudbydere. Databasen der anvendes afhænger af hvilket spørgsmål det søges at besvare. Til indsamling af viden omhandlende de tekniske fakta vedrørende teknologien vil der blive benyttet relevante datablade fra producenter, beskrivende reviews samt ekspertudtalelser fra personel som benytter teknologien, hvis nødvendig. Til besvarelse af hvordan teknologien er udviklet, samt benyttet vil der hovedsageligt blive benyttet reviews fra anerkendte databaser (PubMed, Medline, the Cochrane Library) samt bøger.  \\
Til besvarelse af TEC-analysen bestræbes det at undgå grå litteratur, men i tilfælde, hvor der foreligger begrænsede mængder videnskabelig litteratur, kan der supplementeres med non-peer reviewed-, ikke-publiceret materiale, fortroligt kommercielt materiale samt generelle internetsøgninger.

\section{Sikkerhed (SAF)}
\subsection{Formål}  %(hvorfor)
Sikkerhed er et vigtigt domæne ved analyse af en sundhedsmæssig teknologi, da det ikke ønskes at skade patienter eller brugere af teknologien. Det skal herunder bemærkes at visse teknologier vil påføre patienten smerte. I sådanne tilfælde skal det overvejes hvordan teknologien gør skade og det er væsentligt, at benefit-health-balancen er uligevægtig, således at der opnås større effekt end skade. QST er en teknologi som fungerer ved at påføre patienten smerte, og det skal derfor undersøges hvordan QST medfører helbredsmæssige risici for patienter, og hvorvidt disse er acceptable i forhold til effekt af undersøgelsen og sikkerhed. 
Konsekvensen af eventuel fejldiagnosticering undersøges med henblik på at tydeliggøre begge scenarier. Dette indebærer blandt andet undersøgelse af udførelsen samt præcisionen af databehandlingen. For at imødegå de undersøgte sikkerhedsrisici, bør der tages højde for eventuelle sikkerhedsforanstaltninger, og heraf om dette er nødvendigt. 
\subsection{HTA spørgsmål} %(hvad)

\textit{Patientsikkerhed:}
\begin{itemize}
	\item Hvilke sikkerhedsmæssige risici kan forekomme ved benyttelsen af teknologien, og hvordan forårsages disse? %C0008
	\item Hvilke konsekvenser vil type 1-2 fejl medføre for patientens sikkerhed? %C0006
\end{itemize}
\textit{Sikkerhedsforanstaltninger:}
\begin{itemize}
	\item Hvilke sikkerhedsforanstaltninger skal foretages før teknologien bør anvendes?  %C0062
\end{itemize}

\subsection{Metode \citep{HTAcore}} %(hvordan)
Til analyse af sikkerhed skal det generelt bestemmes hvilke negative effekter teknologien medfører, samt kvantiteten i form af frekvens, incidens og alvorlighed. Ved benyttelse af videnskabelig litteratur til besvarelse af sikkerhedsmæssige HTA-spørgsmål, er det relevant at vurdere hvordan disse data er behandlet og hvilke metoder er anvendt til undersøgelse i studierne. Der vil generelt tages udgangspunkt i studier som har haft fokus på at dokumentere smerter hos patienter, og ikke studier hvor smerte ikke har været hovedfokus. Specielt randomiserede kliniske studier og meta-analyser heraf, skal være baggrund for analysen. Studierne skal ligeledes have overholdt retningslinjerne for patientsikkerhed opsat af WHO (World Health Organization), samt Declaration of Helsinki af WMA (World Medical Association). 
Til sammenligning af data fra studier, vil der vurderes udfra benefit-health-balance og QALY, da disse faktorer er sammenlignelige på tværs af studier. 

\section{Klinisk effektivitet (EFF)}
\subsection{Formål}
I dette område analyseres teknologiens virkningsgrad ud fra et klinisk synspunkt. Formålet med dette er at undersøge både hvor godt QST virker under forskningsmæssige forhold, og i praksis på klinikken. Som basis for vurdering af den samlede virkning af QST, vil både sundhedsfordele og ulemper i forhold til prædiktion af postoperative smerter naturligvis betragtes fra et klinisk perspektiv. Dette er vigtig information for klinikeren i forhold til at træffe en beslutning om det videre behandlingsforløb for patienten, og dermed kunne vælge en behandling der gør mest mulig gavn.\\
For at kunne vurdere virkningsgraden af QST er det nødvendigt at undersøge de gavnlige og ikke-gavnlige effekter ved brug af teknologien. Herunder medregnes også en undersøgelse af hvorvidt de gavnlige effekter kan medføre negative helbredsmæssige konsekvenser, som ikke direkte er relateret til teknologien.\\
Derudover vil en vurdering af hvorvidt teknologiens testresultater har indflydelse på patientens livskvalitet, og i så fald i hvilket omfang, også være nødvendig for at vurdere den overordnede virkningsgrad af teknologien. \\
Endvidere vil en undersøgelse af muligheder og potentielle konsekvenser efter et givent testresultat også være nødvendig for at belyse alle aspekter af den videre behandling.\\
For at vurdere det diagnostiske potentiale af QST er det også nødvendigt at have information om hvor stor en andel af de af patienter der oplever postoperative smerter, som teknologien er i stand til identificere. Ligeledes er det nødvendigt at vide hvor stor en andel af de patienter der ikke oplever kroniske smerter, QST er i stand til at identificere.
\subsection{HTA spørgsmål} %(hvad)

\textit{Effekt og skade:}
\begin{itemize}
	\item Hvilke gavnlige effekter er der ved teknologien? %D0001, 
	\item Hvad er net-benefit af teknologien? %D0020, D0029
\end{itemize}

\textit{Patient effekt:}
\begin{itemize}
	\item Hvordan påvirker teknologiens resultat patientens ikke-helbredsmæssige livskvalitet? %D0030
	\item Hvilken effektiv behandlingsmulighed understøtter teknologiens resultater? (hvis ja/TKA, hvis nej/drugs) %D0024
\end{itemize}
\textit{Nøjagtighed:}
\begin{itemize}
\item Hvorvidt kræves det nøjagtig præcision for benyttelsen af teknologien for korrekt at kunne understøtte klinikerens beslutning? %D1003, D1004, D0021
\end{itemize}

\subsection{Metode \citep{HTAcore}}
For at besvare analysespørgsmålene, foretages en systematisk litteratursøgning i relevante databaser som PubMed, Medline, Cochrane library med mere. Grundlæggende vil vurderingen af relevante artikler vurderes i forhold til evidenshirakiret (se afsnit ref{xx} ). Dette betyder at som udgangspunkt vil meta-studier og randomiserede studier være at foretrække. I forhold til at opnå indsigt i sundhedsfordele og ulemper der er relateret til QST vil studier hvori der indgår information om mortalitet, morbiditet og quality of life blive taget i betragtning. For at opnå viden om nøjagtigheden af QST, vil studier hvori teknologiens sensitivitet og specificitet dokumenteres blive taget i betragtning. Ved vurdering af virkningsgradens normale forhold, kan det blive nødvendigt at undersøge om virkningsgraden kan extrapoleres fra virkning under optimale forhold til normale forhold.

\section{Omkostninger og økonomisk evaluering (ECO)}
\subsection{Formål}
Formålet med ECO er at informere om værdi for pengene i relation til implementeringen af QST som et supplement til klinikerens beslutningsgrundlang til indstilling af patienter til en TKA-operation. Evidens fra SAF og EFF domænerne anvendes til at danne grundlag for analyserne. Teknologiens effekt skal vurderes i forhold til omkostningerne for at kunne danne grundlaget for, om QST skal implementeres eller ej. Hvis en implementering skal kunne finde sted, er det vigtigt at tage højde for hvor ressourcerne skal komme fra, herunder om yderligere ressourcer skal tilføjes, eller oml finansieringen skal ske gennem allokering af nuværende ressourcer. Det er vigtigt da stigende omkostninger til sundhedssektoren sætter større krav til informationer der præsenteres for beslutningstagerne.
\subsection{HTA spørgsmål}
\textit{Teknologiens effekt ift omkostninger:}
\begin{itemize}
	\item Hvad indebærer omkostningerne ved indførelsen af teknologien som supplement til klinikerens beslutning? (Cost-benefit) %E0001, E0002, E0009, 
	\item Hvad er effekten ved at implementere teknologien? (Cost–effektiveness=smerte+funktion) %E0006
\end{itemize}

\textit{Ressourceudnyttelse:}
\begin{itemize}
	\item Hvordan ændrer implementeringen af teknologien benyttelsen af anden teknologi og brugen af ressourcer?% D0023, G0006
	\item Hvordan vil implementeringen af teknologien påvirke budgettet krediteret til regionen? %G0007, F0012
\end{itemize}

\subsection{Metode \citep{HTAcore}}
Vidensindsamlingen til besvarelse af det økonomiske domæne vil foregå ved at søge efter materiale igennem HTA/MTV agenturer, i relevante sundhedssystemer og sundhedsudbydere. Herunder databaser såsom Statistikbanken og Sundhedsdatastyrelsen. Hvilket HTA-spørgsmål der ønskes besvares er afhængig af, hvilke databaser der undersøges. Ved en analyse af ECO benyttes der tre typer af litteratur, review af publiceret økonomisk evidens, reviews af eksisterende økonomiske evalueringer og \textit{de Novo} økonomiske evalueringer. For at udvælge de relevante analyser til ECO må det tages højde for tre faktorer; Meningen med den økonomiske evaluering, tilgængeligheden af brugbar data samt guidelines for de enkelte ECO-analyser. For en evaluering af QST er det fundet relevant at benytte; (CEA),(CUA),(CBA) og (BIA) analyser\textbf{Dette bliver diskuteret på mødet}. Disse giver et billede af QST omkostninger målt i både monetære enheder og QALY, samt at redegøre for tilbagebetalingstiden for hele implementeringen og dennes indvirkning på hospitalets budget. Når resultater fra de forskellige analyser forelægges må det siden hen vurderes transferabiliteten mellem afdelinger/regioner med videre. Det vil ved en analyse af ECO domænet være nødvendigt at foretage antagelser og simplificeringer, såfremt der ikke foreligger præcise tal, der er nødvendige for den pågældende analyse. Antagelser lavet i forbindelse med analyserne vil blive udført på en sådan måde at de fremstår transparente for ikke at blive misvisende. 


