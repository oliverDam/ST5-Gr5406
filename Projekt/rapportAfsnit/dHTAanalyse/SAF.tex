\section{Sikkerhed (SAF)}
\subsection{Formål}
I sikkerhedsdomænet analyseres hvilke sikkerhedsmæssige konsekvenser der kan forekomme ved implementering og brug af QST. For at teknologien kan implementeres og bruges, bør denne være sikker for både patienten og brugeren. Derfor er det nødvendigt at undersøge eventuelle sikkerhedsmæssige risici. Denne undersøgelse danner grundlag for en vurdering af hvorvidt QST er sikker at benytte, og deraf eventuelle konsekvenser ved brugen. \\
Patientsikkerhed undersøges da patienterne eksponeres for QST, hvorfor det er nødvendigt at kende eventuelle sikkerhedsrisici ved brugen af QST. Ligeledes undersøges brugersikkerhed i forhold til brugeren som benytter teknologien. \\
Såfremt  implementeringen og brugen af QST er forbundet med sikkerhedsmæssige konsekvenser, skal det undersøges hvilke sikkerhedsmæssige foranstaltninger der bør tages. Dette gøres for, at sikre, at patient og sundhedspersonale ikke udsættes for unødige farer ved brug af QST. Ved identificering af sikkerhedsrisici og eventuelle sikkerhedsforanstaltninger, er det muligt at imødegå nogle af konsekvenserne ved implementering af QST, og hermed give et bedre grundlag for vurdering om hvorvidt QST skal implementeres.
\subsection{HTA spørgsmål}
\textit{Patient- og brugersikkerhed:}
\begin{itemize}
	\item Hvilke sikkerhedsmæssige risici kan forekomme ved benyttelsen af teknologien, og hvordan forårsages disse? %C0008
\end{itemize}
\textit{Sikkerhedsforanstaltninger:}
\begin{itemize}
	\item Hvilke sikkerhedsforanstaltninger skal foretages før teknologien bør anvendes?  %C0062
\end{itemize}
%\subsection{Metode}
%Til analyse af sikkerhed skal det generelt bestemmes hvilke negative effekter teknologien medfører, samt kvantiteten i form af frekvens, incidens og alvorlighed. Ved benyttelse af videnskabelig litteratur til besvarelse af sikkerhedsmæssige HTA-spørgsmål, er det relevant at vurdere hvordan disse data er behandlet og hvilke metoder er anvendt til undersøgelse i studierne. Der vil generelt tages udgangspunkt i studier som har haft fokus på at dokumentere smerter hos patienter, og ikke studier hvor smerte ikke har været hovedfokus. Specielt randomiserede kliniske studier og meta-analyser heraf, skal være baggrund for analysen. Studierne skal ligeledes have overholdt retningslinjerne for patientsikkerhed opsat af WHO (World Health Organization), samt Declaration of Helsinki af WMA (World Medical Association). 
%Til sammenligning af data fra studier, vil der vurderes udfra benefit-health-balance og QALY, da disse faktorer er sammenlignelige på tværs af studier.
