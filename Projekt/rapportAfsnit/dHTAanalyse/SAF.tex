\textit{I følgende afsnit vil sikkerhedsdomænet analyseres for de sikkerhedsmæssige risici, der kan være i sammenhæng med implementering af QST som supplement til klinikerens beslutningstagen omkring knæartrosepatienter. Det vil blive undersøgt om der eksisterer farer for patient eller bruger ved anvendelse af QST, samt om implementeringen vil kræve indførelse af bestemte sikkerhedsforanstaltninger.}

\section{Formål} \label{SAF_chap}
I sikkerhedsdomænet analyseres hvilke sikkerhedsmæssige konsekvenser, der kan forekomme ved implementering og brug af QST. Teknologien bør være sikker for både patienten og brugeren. Derfor er det nødvendigt at undersøge eventuelle sikkerhedsmæssige risici. Denne undersøgelse danner grundlag for en vurdering af hvorvidt QST er sikker at benytte, og deraf eventuelle konsekvenser ved brugen. \\
Patientsikkerhed undersøges, da patienterne eksponeres for QST, hvorfor det er nødvendigt at kende eventuelle sikkerhedsrisici ved brugen af denne metode. Ligeledes undersøges brugersikkerhed i forhold til klinikeren. \\
Såfremt implementeringen og brugen af QST er forbundet med sikkerhedsmæssige konsekvenser, skal det undersøges hvilke sikkerhedsmæssige foranstaltninger, der bør tages. Dette gøres for at sikre, at patient og sundhedspersonale ikke udsættes for unødige farer ved brug af QST. Ved identificering af sikkerhedsrisici og eventuelle sikkerhedsforanstaltninger er det muligt at imødegå nogle af konsekvenserne ved implementering af QST og hermed give et bedre grundlag for vurdering om, hvorvidt QST skal implementeres.
\section{HTA spørgsmål}
\textit{Patient- og brugersikkerhed:}
\begin{itemize}
\item Hvilke sikkerhedsmæssige risici kan forekomme ved benyttelsen af QST-undersøgelserne, og hvordan forårsages disse? %C0008
\end{itemize}
\textit{Sikkerhedsforanstaltninger:}
\begin{itemize}
\item Hvilke sikkerhedsforanstaltninger skal foretages ved benyttelsen af QST?  %C0062
\end{itemize}

\section{Metode}
Til sikkerhedsafsnittet skal der hovedsageligt bruges kilder omhandlende QST-undersøgelsens effekt og eventuelle skader på forsøgspersoner. Studierne skal have fokus på rapporteringen af, hvordan teknologien påvirker forsøgspersonen, frem for hvordan teknologien fungerer, da dette er undersøgt i \chapref{TEC_chap}. Studier, der ikke søger at dokumentere effekten af påvirkning af forsøgspersoner, vil derfor ikke blevet medtaget. Det blev i \chapref{TEC_chap} bestemt at fokusere på tre QST-undersøgelser: PPT, TSP og CMP, og undersøgelser ved mekanisk trykpåvirkning. For at vurdere hvorvidt disse undersøgelser er sikre for forsøgspersonen og klinikeren, er det nødvendigt at undersøge grænser for, hvad hud og muskelvæv kan modstå af mekanisk trykpåvirkning. Den fundne grænse skal sammenlignes med hvad en forsøgsperson kan blive udsat for ved en QST-undersøgelse for at kunne vurdere, om denne påvirkning kan påføre forsøgspersonen skade. Sikkerhedsforanstaltninger, som skal indføres for at opnå en sikker og pålidelig test, skal ligeledes undersøges. Dette gøres ved undersøgelse af forskellige typer af QST. Derfor vil der blive søgt efter studier, der sammenholder forskellige QST-systemer og protokoller for derudfra at kunne undersøge hvor i undersøgelsen, der kan opstå risiko for skader og hvordan disse kan forebygges. Ligeledes vil dette kunne give viden om, hvordan undersøgelsen kan sikres at være præcis og pålidelig. \citep{HTAcore}

\section{Patientsikkerhed}
\textit{Det vil i følgende afsnit blive undersøgt, hvilke fysiske grænser hud og muskler har overfor trykpåvirkning, samt hvilken trykpåvirkning en patient bliver udsat for ved en QST-undersøgelse. Dette gøres for at kunne vurdere om trykpåvirkningen ved QST-undersøgelser kan udsætte patienter for en sikkerhedsmæssig risiko.}

\subsection{Sikkerhedsmæssige risici for patienter}
Det er ifølge internationale retningslinjer for patientsikkerhed bestemt, at fagpersoner i bedste evne og hensigt skal agere med patienters helbred som førsteprioritet \cite{helsinki2013}. Dette er en af mange retningslinjer, som har til formål at udbrede et internationalt dækkende budskab om, at der til enhver tid skal overvejes, hvorvidt en patient kan bringes i fare eller påføres skade. Disse retningslinjer er således bestemt for at sikre overholdelse af etiske principper i forhold til behandling af patienter eller ved forskning, der inddrager forsøgspersoner. Det fremgår desuden af internationale retningslinjer, at nationale retningslinjer for det pågældende land hvori behandlingen eller undersøgelsen udføres ligeledes skal overholdes. Da det her undersøges, hvorvidt QST kan implementeres på ortopædkirurgiske afdelinger i Region Nordjylland, bør det overvejes, om QST overholder internationale såvel som de danske retningslinjer for patientsikkerhed. \cite{helsinki2013} \\
Da QST bevidst påfører patienter smerte, skal det undersøges, hvorvidt denne smerte er acceptabel i forhold til, om det pådrager patienten skader. Internationale retningslinjer proklamerer, at en given undersøgelse altid skal underlægges klinikerens viden og bevidsthed om at sikre patientens helbred, velfærd og rettigheder. Det pointeres ligeledes, at selvom forsknings primære mål er at finde ny viden, må forskningens formål aldrig overskygge patientsikkerheden. Forskningens udbytte skal derfor altid være af større betydning end den risiko, det kan udsætte patienten for. \cite{helsinki2013} Dette betyder generelt, at det kan accepteres, at en patient udsættes for radioaktiv stråling, hvis det indgår som led i diagnosticering af en livstruende tumor. I dette tilfælde overskygger risikoen ved undersøgelsen den risiko, som truer patientens helbred. Danske nationale kliniske retningslinjer (NKR) for patientsikkerhed er opsat af Sundhedsstyrelsen og omhandler primært sundhedsprofessionelle indenfor den danske sundhedssektor. Her udgør retningslinjerne et sundhedsfagligt beslutningsværktøj, som skal sikre, at udredning, behandling, pleje og rehabiliteringen i sektoren er ensartet og af høj kvalitet. \citep{nkr2016} \citep{kommissorium2012} De danske retningslinjer skal således sikre, at en patient vil få behandling af samme kvalitet ligegyldigt hvor i landet, denne bliver behandlet. Disse retningslinjer sikrer ligeledes, at viden deles mellem regioner sådan en patient altid vil modtage den bedst tilgængelige behandling. \cite{nkr2016} \\
På baggrund af teknologianalysen i \chapref{TEC_chap} er der udvalgt tre tests fra QST, der undersøger PPT, TS og CPM. Disse anvender alle at påvirke patienten med mekanisk tryk. Det undersøges derfor, hvilke fysiske grænser hud og muskelvæv har overfor trykpåvirkning og hvornår et givent tryk kan udøve skade på kroppen. \\

\subsubsection{Mekanisk trykpåvirkning}
Ved mekanisk påvirkning af tryk vil materialer deformeres, hvis kraften overstiger deres flydegrænse, hvorefter materialet vil brydes, hvis kraftpåvirkningen fortsætter og overstiger brudstyrken. Materialer som hud og blødt væv har en vis form for elasticitet, men vil ligeledes deformeres ved en kraftpåvirkning og destrueres, når cellerne i kroppen ødelægges. Dette vil ses som blodansamlinger og sår på hud og væv. Skader som følge af tryk afhænger af størrelsen på kraften og størrelsen af området som påvirkes. Trykket følger forholdet $P = \frac{F}{A}$, hvor P er trykket, F er kraften og A er arealet som påvirkes. Et studie af \citer{aisling2012} har undersøgt mekaniske egenskaber ved menneskehud. Undersøgelsen har anvendt samples af menneskehud fra ryggen som er blevet testet i et apparatur der kan strække huden. Elasticiteten af huden er blevet målt løbende og indtil huden nåede sin brudstyrke. Brudstyrken blev bestemt til $21,6 \pm 8,4 MPa$. Dette stemmer overens med en studie af \citer{jussila2005}, der undersøgte mulighederne for at finde en simulator for menneskehud til tests af skydevåben. For at kunne fungere som et brugbart alternativ skulle simulanten overholde nogle grænser, heriblandt en brudstyrke på $18 \pm 2 N/mm^{2} = 18 MPa$. Andre studier har undersøgt effekten af påvirkning af tryk over en længere periode. \citer{sanders1995} har sammenholdt forskellige studier og fundet, at skader på hud og muskelvæv ligeledes opstår ved en svagere kraftpåvirkning, hvis påvirkningen sker over længere tid. I litteraturen findes det, at der sker skader på hud og underliggende muskelstrukturer ved tryk på $13 kPa$ ved påvirkning i to timer. Dette skyldes at et tryk på $13 kPa$ er over det systoliske tryk, hvorved der lukkes af for blodgennemstrømningen i det trykpåvirkede område. Som følge deraf opstår der ved påvirkning i seks timer således komplet muskelnekrose. \citep{sanders1995} Da nogle QST-protokoller idag anvender cuff til PPT-tests, bliver der påvirket et større område end $1 cm^{2}$ som ved det håndholdte trykalgometer. Et studie af \citer{denheltnye2016} anvender et cuff-algometer der dækker et op til $13cm$ bredt område, hvor der er fastsat en maksimal trykgrænse på $100kPa$.

\subsubsection{Overholdelse af skadegrænser}
Der kan opstilles bestemte grænser for, hvad hud og muskelvæv kan modstå i forhold til påvirkning ved mekanisk tryk. Disse grænser kan sammenlignes med hvad en patient påvirkes med ved en QST-undersøgelse. \\
Ifølge \citer{Rolke2006b}, som en del af QST-protokollen fra German Research Network on Neuropathic Pain (DFNS), bliver en patient ved PPT-testen i en QST-undersøgelse, maksimalt udsat for et mekanisk tryk på $2000 kPa$, på et område af $1 cm^{2}$ ved test med et håndholdt algometer. Dette er mindre end $21,6 \pm 8,4 MPa$, som det er fundet af \citer{aisling2012} og \citer{jussila2005}. Det er dog over $13 kPa$, som ifølge \citer{sanders1995} kan påføre muskelskader. Dette er kun ved trykpåvirkning i seks timer, hvor trykpåvirkning ved QST-undersøgelsen kun tager omkring 40 sekunder \cite{Rolke2006b}. Ved anvendelse af et cuff-algometer kan der påvirkes et større område af patienten, end ved et håndholdt trykalgometer. Dette er dog ikke et sikkerhedsmæssigt problem da trykpåvirkning stadig ikke overskrider den fastsatte grænse på $2000 kPa$ som opsat af DFNS. Den største forskel kan ske i patientens opfattelse af påvirkningen, da denne ikke påvirkes på et lille område, som $1 cm^{2}$ ved håndholdt algometer, som ved et cuff-algometer som dækker hele vejen omkring et legeme. % overflade areal af en cylinder: 2*pi*r*H, hvor højden ville være 13cm -> 81,68cm*r = overfladen. Dvs det påvirkede areal vil være afhængig af hvor tyk patienten arm, ben, lår er.

%Det tryk som opleves af patienten er således meget anderledes. Ved det håndholdte algometer vil en patient, ved et tryk på $13kPa$ opleve et tryk på $13kPa = \frac{F}{0,01m^{2} \Rightarrow F=13kPa*0,01m^{2} \Rightarrow F=130N$. Et tryk på $130N$ svare til at placere $13,25kg$ på patienten. Ved samme trykpåvirkning med en cuff på patientens ben, som dækker et område af 


\section{Sikkerhedsforanstaltninger}
\textit{I følgende afsnit vil det blive undersøgt hvorvidt anvendelse af QST kan udsætte personale for sikkerhedsmæssige risici.}

\subsection{Sikkerhedsforanstaltninger ved anvendelse af QST}
Selvom der generelt ikke er farer forbundet med udførelsen af en QST-undersøgelse, bør der stadig tages sikkerhedsforanstaltninger i forhold til at sikre reproducerbarhed af de enkelte undersøgelser. I et studie af \citer{shy2003} foretages en vurdering af QST i forhold til parametrene effektivitet, anvendelighed og sikkerhed. I studiet blev artikler anvendt som omhandlede forskellige QST-systemer, der blev sammenlignet i forhold til de ovennævnte parametre. I studiet konkluderes det, at det er væsentligt, at klinikeren modtager vejledning fra producenterne af QST-udstyr i forhold til anvendelsen heraf. Undersøgelserne bør desuden foretages i et stille lokale, der er indrettet til formålet, således resultaterne af undersøgelsen ikke påvirkes af udefrakommende faktorer. Derudover bør den samme kliniker udføre både den primære og de eventuelt opfølgende undersøgelser, da dette sikrer, at disse bliver foretaget på samme måde hver gang, hvormed reproducerbarheden øges. \citep{shy2003}
Ved anvendelse af computerstyrede metoder til QST-undersøgelse, er udførelsen af undersøgelserne i højere grad automatiseret og dermed mere uafhængige af klinikeren \citep{Nielsen2015}. Det kan dermed antages, at der skal tages færre sikkerhedsforanstaltninger i forhold til opnåelse af korrekte resultater.  

\section{Delkonklusion}
Af QST-protokollen udarbejdet af DFNS fremgår det, at det maksimale mekaniske tryk, en patient bliver udsat for, er $2000 kPa$, hvilket overskrider grænsen for muskelskade fundet i studiet af \citer{sanders1995}. Da denne grænseværdi er fundet efter trykpåvirkning i seks timer og QST-undersøgelsen udføres på cirka 40 sekunder vurderes det imidlertid, at der ikke vil opstå skader på patienter, og i så fald vil skaderne forbundet hermed vil være minimale. Dermed overholdes de nationale og internationale retningslinjer for patientsikkerhed, da det resultat, der opnås ved QST-undersøgelsen, bidrager til vurderingen af, hvorvidt patienten skal have foretaget en TKA-operation. Dette udbytte er af større betydning end de sikkerhedsmæssige risici, patienten udsættes for.
For at sikre, at balancen mellem udbytte og risici opretholdes, er det væsentligt at der tages visse sikkerhedsforanstaltninger for at sikre høj reproducerbarhed af undersøgelserne. Sikkerhedsforanstaltningerne omfatter blandt andet vejledning fra producenter, korrekt indretning af lokale og at patienten konsulterer den samme kliniker gennem alle QST-undersøgelser. Computerstyrede metoder er mere automatiserede, hvorved det vurderes, at disse vil kræve minimal indførelse af sikkerhedsforanstaltninger.


