\section{Sikkerhed (SAF)}
\subsection{Formål}  %(hvorfor)
Sikkerhed er et vigtigt domæne ved analyse af en sundhedsmæssig teknologi, da det ikke ønskes at skade patienter eller brugere af teknologien. Det skal herunder bemærkes at visse teknologier vil påføre patienten smerte. I sådanne tilfælde skal det overvejes hvordan teknologien gør skade og det er væsentligt, at benefit-health-balancen er uligevægtig, således at der opnås større effekt end skade. QST er en teknologi som fungerer ved at påføre patienten smerte, og det skal derfor undersøges hvordan QST medfører helbredsmæssige risici for patienter, og hvorvidt disse er acceptable i forhold til effekt af undersøgelsen og sikkerhed. 
Konsekvensen af eventuel fejldiagnosticering undersøges med henblik på at tydeliggøre begge scenarier. Dette indebærer blandt andet undersøgelse af udførelsen samt præcisionen af databehandlingen. For at imødegå de undersøgte sikkerhedsrisici, bør der tages højde for eventuelle sikkerhedsforanstaltninger, og heraf om dette er nødvendigt. 
\subsection{HTA spørgsmål} %(hvad)

\textit{Patientsikkerhed:}
\begin{itemize}
	\item Hvilke sikkerhedsmæssige risici kan forekomme ved benyttelsen af teknologien, og hvordan forårsages disse? %C0008
	\item Hvilke konsekvenser vil type 1-2 fejl medføre for patientens sikkerhed? %C0006
\end{itemize}
\textit{Sikkerhedsforanstaltninger:}
\begin{itemize}
	\item Hvilke sikkerhedsforanstaltninger skal foretages før teknologien bør anvendes?  %C0062
\end{itemize}

\subsection{Metode \citep{HTAcore}} %(hvordan)
Til analyse af sikkerhed skal det generelt bestemmes hvilke negative effekter teknologien medfører, samt kvantiteten i form af frekvens, incidens og alvorlighed. Ved benyttelse af videnskabelig litteratur til besvarelse af sikkerhedsmæssige HTA-spørgsmål, er det relevant at vurdere hvordan disse data er behandlet og hvilke metoder er anvendt til undersøgelse i studierne. Der vil generelt tages udgangspunkt i studier som har haft fokus på at dokumentere smerter hos patienter, og ikke studier hvor smerte ikke har været hovedfokus. Specielt randomiserede kliniske studier og meta-analyser heraf, skal være baggrund for analysen. Studierne skal ligeledes have overholdt retningslinjerne for patientsikkerhed opsat af WHO (World Health Organization), samt Declaration of Helsinki af WMA (World Medical Association). 
Til sammenligning af data fra studier, vil der vurderes udfra benefit-health-balance og QALY, da disse faktorer er sammenlignelige på tværs af studier. 