\section{Formål} 
I dette domæne analyseres hvilke sikkerhedsmæssige konsekvenser, der kan forekomme ved implementering og brug af QST-protokollen. Teknologien bør være sikker for både patienten og brugeren. Derfor er det nødvendigt at undersøge eventuelle sikkerhedsmæssige risici. Denne undersøgelse danner grundlag for en vurdering af hvorvidt QST er sikker at benytte, og deraf eventuelle konsekvenser ved brugen. \\
Patientsikkerhed undersøges, da patienterne eksponeres for tryk ved QST-protokollen, hvorfor det er nødvendigt at kende til eventuelle sikkerhedsrisici ved brugen af denne metode. \\
Såfremt implementeringen og brugen af QST-protokollen er forbundet med sikkerhedsmæssige konsekvenser, skal det undersøges hvilke sikkerhedsmæssige foranstaltninger, der bør tages. Dette gøres for at sikre, at patienten ikke udsættes for unødige farer ved brug af protokollen. Ved identificering af sikkerhedsrisici og eventuelle sikkerhedsforanstaltninger er det muligt at imødegå nogle af konsekvenserne ved implementering af QST og hermed danne et bedre grundlag for vurdering om, hvorvidt QST-protokollen skal implementeres.

\section{Analyse-spørgsmål}
Patientsikkerhed:
\begin{enumerate}
\item \textit{Hvilke sikkerhedsmæssige patientrisici kan forekomme ved benyttelsen af QST-protokollen?} %C0008
\end{enumerate}
Sikkerhedsforanstaltninger:
\begin{enumerate}[resume]
\item \textit{Hvilke sikkerhedsforanstaltninger skal foretages ved anvendelse af QST-protokollen?}  %C0062
\end{enumerate}

\section{Metode}
Til besvarelse af dette domæne tager litteratursøgningen udgangspunkt i den generelle skitserede metode (jævnfør \secref{litteratursogning}). I SAF-domænet vil der hovedsageligt blive søgt efter kilder omhandlende QST-undersøgelsernes påvirkning samt eventuelle skader på forsøgspersoner. Studierne skal have fokus på rapporteringen af, hvordan teknologien påvirker forsøgspersonen. Studier, der ikke dokumenterer påvirkningen af QST på forsøgspersoner, vil derfor ikke blevet medtaget. Søgeprotokollen for TEC-domønet kan findes i bilag (\secref{SAF_sog}). \\
Det blev i \chapref{TEC_chap}, bestemt at fokusere på QST-parametrene, PPT, TSP og CMP, udført med mekanisk trykpåvirkning. For at besvare analyse-spørgsmål (1) vurderes det hvorvidt disse undersøgelser er sikre for forsøgspersonen. Dette gøres ved at undersøge grænser for, hvad hud og væv kan modstå af mekanisk trykpåvirkning. Den fundne grænse sammenlignes med hvad en forsøgsperson kan blive udsat for ved en QST-undersøgelse for at kunne vurdere, om denne påvirkning kan påføre forsøgspersonen skade. Analyse-spørgsmål (2) undersøges ved at søge efter studier, der sammenholder forskellige QST-systemer og protokoller for derudfra at kunne undersøge hvor i undersøgelsen, der kan opstå risiko for skader og hvordan disse kan forebygges. Ligeledes vil dette kunne give viden om, hvordan der kan sikres korrekt anvendelse af teknologien.

\section{Patientsikkerhed}
\textit{Det vil i følgende afsnit blive undersøgt, hvilke fysiske grænser hud og væv har overfor trykpåvirkning, samt hvilken trykpåvirkning en patient bliver udsat for ved en QST-undersøgelse. Dette gøres for at kunne vurdere om trykpåvirkningen ved QST-protokollen kan udsætte patienter for en sikkerhedsmæssig risiko.}

\subsection{Sikkerhedsmæssige risici for patienter}
Det er ifølge internationale retningslinjer for patientsikkerhed bestemt, at fagpersoner i bedste evne og hensigt skal behandle med patienters helbred som førsteprioritet \citep{helsinki2013}. Dette er en af mange retningslinjer, som har til formål at udbrede et internationalt budskab om, at der til enhver tid skal overvejes, hvorvidt en patient kan bringes i fare eller påføres skade, som resultat af behandling. Disse retningslinjer er således bestemt for, at sikre overholdelse af etiske principper i forhold til behandling af patienter eller ved forskning, der inddrager forsøgspersoner. Da det undersøges, hvorvidt QST-protokollen kan implementeres på ortopædkirurgiske afdelinger i Region Nordjylland, skal det overvejes, om protokollen overholder internationale såvel som de danske retningslinjer for patientsikkerhed. \citep{helsinki2013} \\
QST-protokollen påfører bevidst patienter smerte, hvoraf det undersøges, hvorvidt denne smerte er acceptabel i forhold til, om det pådrager patienten skader. Internationale retningslinjer proklamerer, at en given undersøgelse altid skal underlægges klinikerens viden og bevidsthed om at sikre patientens helbred, velfærd og rettigheder. Det pointeres ligeledes, at selvom forsknings primære mål er at finde ny viden, må forskningens formål aldrig overskygge patientsikkerheden. Udbytte heraf skal derfor altid være af større betydning end den sikkerhedsrisiko, det kan udsætte patienten for. \citep{helsinki2013} De danske nationale kliniske retningslinjer (NKR) for patientsikkerhed er opsat af Sundhedsstyrelsen, og henvender sig primært til det sundhedsfaglige personale indenfor den danske sundhedssektor. Her udgør retningslinjerne et sundhedsfagligt beslutningsværktøj, som skal sikre, at udredning, behandling, pleje og rehabiliteringen i sektoren er ensartet og af høj kvalitet. \citep{nkr2016} \citep{kommissorium2012} \\
På baggrund af teknologianalysen i \chapref{TEC_chap} er der udvalgt tre undersøgelsesparametre vedrørende QST-protokollen, der undersøger PPT, TS og CPM. Undersøgelsen af QST-parametrene udsætter patienten for mekanisk tryk. Det undersøges derfor, hvilke fysiske grænser hud og væv har overfor trykpåvirkning og hvornår et givent tryk kan medføre skade på kroppen.

\subsubsection{Mekanisk trykpåvirkning}
Ved mekanisk påvirkning af tryk vil materialer deformeres, hvis kraften overstiger deres flydegrænse. Materialet vil tilmed brydes, hvis kraftpåvirkningen fortsætter og overstiger brudstyrken. Materialer som hud og blødt væv har en vis form for elasticitet, men vil ligeledes deformeres ved en kraftpåvirkning og destrueres, når cellerne i kroppen ødelægges. Dette vil ses som blodansamlinger og sår på hud og væv. Skader som følge af tryk afhænger af størrelsen på kraften og arealet af området som påvirkes. Trykket følger forholdet $P~=~\frac{F}{A}$, hvor P er trykket, F er kraften og A er arealet som påvirkes. Arealet som påvirkes har derfor stor betydning for trykket, da trykket vil falde i takt med et større areal påvirkes af samme kraft.Et studie af \citer{aisling2012} har undersøgt mekaniske egenskaber ved menneskehud. Undersøgelsen har anvendt samples af menneskehud fra ryggen som er blevet testet i et apparatur der kan strække huden. Elasticiteten af huden er blevet målt løbende og indtil huden nåede sin brudstyrke. Brudstyrken blev bestemt til $21,6 \pm 8,4~MPa$. Dette stemmer overens med et studie af \citer{jussila2005}, der finder en brudstyrke på $18~MPa$. Ydermere har studier undersøgt effekten af påvirkning af tryk over en længere periode. \citer{sanders1995} har sammenholdt forskellige studier og fundet, at skader på hud og muskelvæv ligeledes opstår ved en svagere kraftpåvirkning, hvis påvirkningen sker over længere tid. I studiet findes det, at der sker skader på hud og underliggende muskelstrukturer ved tryk på $13~kPa$ ved påvirkning i to timer. Dette skyldes at et tryk på $13~kPa$ er over det systoliske tryk, hvorved der lukkes af for blodgennemstrømningen i det trykpåvirkede område. Som følge deraf opstår der ved påvirkning i seks timer således komplet muskelnekrose. \citep{sanders1995} Nogle QST-protokoller anvender cuff-algometri til PPT-tests, hvor det påvirkede areal omhandler et større område end $1~cm^{2}$ som ved det håndholdte trykalgometre. 

\subsubsection{Overholdelse af skadegrænser}
Der kan opstilles bestemte grænser for, hvad hud og muskelvæv kan modstå i forhold til påvirkning ved mekanisk tryk. Disse grænser kan sammenlignes med hvad en patient påvirkes med ved en QST-undersøgelse. \\
Ifølge \citer{Rolke2006b}, som en del af QST-protokollen fra German Research Network on Neuropathic Pain (DFNS), bliver en patient ved PPT-testen i en QST-undersøgelse, maksimalt udsat for et mekanisk tryk på $2000~kPa$ af et håndholdt algometer, på et område af $1~cm^{2}$. Dette er mindre end den påvirkningen på $21,6 \pm 8,4~MPa$, som vævet kunne modstå, som det blev fundet af \citer{aisling2012} og \citer{jussila2005}. Det er dog over $13~kPa$, som ifølge \citer{sanders1995} kan påføre muskel- og vævsskader. Dette er kun ved trykpåvirkning i seks timer, hvor trykpåvirkning ved QST-undersøgelsen kun tager omkring 40 sekunder \cite{Rolke2006b}. Ved anvendelse af et cuff-algometer kan der påvirkes et større område af patienten, end ved et håndholdt trykalgometer. Et studie af \citer{denheltnye2016} anvender et cuff-algometer der dækker et op til $13~cm$ bredt område, hvor der er fastsat en maksimal trykgrænse på $100~kPa$, hvilket vil resultere i en lavere trykpåvirkning da manchettens areal er større end trykalgometerets areal. Dette er dog ikke et sikkerhedsmæssigt problem da trykpåvirkning stadig ikke overskrider den fastsatte grænse på $2000~kPa$ som opsat af DFNS. Den største forskel kan ske i patientens opfattelse af påvirkningen, da denne ikke påvirkes på et lille område, som $1~cm^{2}$ ved håndholdt algometer, som ved et cuff-algometer som dækker hele vejen omkring et legeme. % overflade areal af en cylinder: 2*pi*r*H, hvor højden ville være 313~cm -> 81,68~cm*r = overfladen. Dvs det påvirkede areal vil være afhængig af hvor tyk patienten arm, ben, lår er.
%Det tryk som opleves af patienten er således meget anderledes. Ved det håndholdte algometer vil en patient, ved et tryk på $13kPa$ opleve et tryk på $13~kPa = \frac{F}{0,01~m^{2} \Rightarrow F=~3~kPa*0,01m^{2} \Rightarrow F=130~N$. Et tryk på $130~N$ svare til at placere $13,25~kg$ på patienten. Ved samme trykpåvirkning med en cuff på patientens ben, som dækker et område af 

\section{Sikkerhedsforanstaltninger}
\textit{Da det i ovenstående afsnit ikke blev fundet sikkerhedsrisici ved anvendelsen af QST-protokollen, bør der umiddelbart ikke tages nogle sikkerhedsforanstaltninger.} 

\section{Delkonklusion}
Af QST-protokollen udarbejdet af DFNS fremgår det, at det maksimale mekaniske tryk, en patient bliver udsat for, er 2000~kPa, hvilket overskrider grænsen for muskelskade fundet i et studie. Da denne grænseværdi er fundet efter trykpåvirkning i seks timer og da QST-undersøgelsen for parametrene udføres på cirka 40 sekunder vurderes det imidlertid, at der ikke vil opstå skader på patienter, og i så fald vil skaderne forbundet hermed vil være minimale. Dermed overholdes de nationale og internationale retningslinjer for patientsikkerhed, hvilket medfører at QST-protokollen ikke pådrager patienter skader. Udbyttet af QST-protokollen er større end de sikkerhedsmæssige risici, patienten udsættes for. Dette sikrer, at protokollen på en sikker måde bidrager til klinikerens vurderingen af, hvorvidt patienten skal have foretaget en TKA-operation.
