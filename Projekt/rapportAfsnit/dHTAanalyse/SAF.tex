\section{Formål}
I sikkerhedsdomænet analyseres hvilke sikkerhedsmæssige konsekvenser, der kan forekomme ved implementering og brug af QST. Teknologien bør være sikker for både patienten og brugeren. Derfor er det nødvendigt at undersøge eventuelle sikkerhedsmæssige risici. Denne undersøgelse danner grundlag for en vurdering af hvorvidt QST er sikker at benytte, og deraf eventuelle konsekvenser ved brugen. \\
Patientsikkerhed undersøges, da patienterne eksponeres for QST, hvorfor det er nødvendigt at kende eventuelle sikkerhedsrisici ved brugen af denne metode. Ligeledes undersøges brugersikkerhed i forhold til klinikeren. \\
Såfremt implementeringen og brugen af QST er forbundet med sikkerhedsmæssige konsekvenser, skal det undersøges hvilke sikkerhedsmæssige foranstaltninger, der bør tages. Dette gøres for at sikre, at patient og sundhedspersonale ikke udsættes for unødige farer ved brug af QST. Ved identificering af sikkerhedsrisici og eventuelle sikkerhedsforanstaltninger er det muligt at imødegå nogle af konsekvenserne ved implementering af QST og hermed give et bedre grundlag for vurdering om, hvorvidt QST skal implementeres.
\section{HTA spørgsmål}
\textit{Patient- og brugersikkerhed:}
\begin{itemize}
\item Hvilke sikkerhedsmæssige risici kan forekomme ved benyttelsen af teknologien, og hvordan forårsages disse? %C0008
\end{itemize}
\textit{Sikkerhedsforanstaltninger:}
\begin{itemize}
\item Hvilke sikkerhedsforanstaltninger skal foretages ved benyttelsen af teknologien?  %C0062
\end{itemize}

\section{Metode \citep{HTAcore}}
Til sikkerhedsafsnittet skal der hovedsageligt bruges kilder omhandlende QST-undersøgelsens effekt og eventuelle skader på forsøgspersoner. Studierne skal have fokus på rapporteringen af, hvordan teknologien påvirker forsøgspersonen, frem for hvordan teknologien fungerer, da dette er undersøgt i teknologianalysen \ref{?tek?}. Studier, der ikke søger at dokumentere effekten af påvirkning af forsøgspersoner, vil derfor ikke blevet medtaget. Det blev i teknologianalysen \ref{?tek?} bestemt at fokusere på tre QST-undersøgelser: PPT, TS og CMP, og undersøgelser ved mekanisk trykpåvirkning. For at vurdere hvorvidt disse undersøgelser er sikre for forsøgspersonen og klinikeren, er det nødvendigt at undersøge grænser for, hvad hud og muskelvæv kan modstå af mekanisk trykpåvirkning. Den fundne grænse skal sammenlignes med hvad en forsøgsperson kan blive udsat for ved en QST-undersøgelse for at kunne vurdere, om denne påvirkning kan påføre forsøgspersonen skade. Sikkerhedsforanstaltninger, som skal indføres for at opnå en sikker og pålidelig undersøgelse, skal ligeledes undersøges. Dette gøres ved undersøgelse af forskelliger typer af QST-undersøgelser. Derfor vil der blive søgt efter studier, der sammenholder forskellige QST-systemer og protokoller for derudfra at kunne undersøge hvor i undersøgelsen, der kan opstå risiko for skader og hvordan disse kan forebygges. Ligeledes vil dette kunne give viden om, hvordan undersøgelsen kan sikres at være præcis og pålidelig.

\section{Patientsikkerhed}
\textit{Her kommer et kursivt indledende afsnit}
\subsection{Sikkerhedsmæssige risici for patienter}
Det er ifølge internationale retningslinjer for patientsikkerhed bestemt, at fagpersoner i bedste evne og hensigt skal agere med patienters helbred som førsteprioritet \cite{helsinki2013}. Dette er en af mange retningslinjer, som skal sørge for et internationalt dækkende budskab om, at der til enhver tid skal overvejes, hvorvidt en patient kan bringes i fare eller påføres skade. Det er således bestemt for at sikre overholdelse af etiske principper for behandling af patienter eller ved forskning, der inddrager forsøgspersoner. Det fremgår desuden af internationale retningslinjer, at nationale retningslinjer ligeledes altid skal overholdes for det pågældende land hvori behandling eller undersøgelse udføres. Da det her undersøges, hvorvidt QST kan implementeres på ortopædkirurgiske afdelinger i Region Nordjylland, bør det overvejes, om QST overholder internationale såvel som de danske retningslinjer for patientsikkerhed. \cite{helsinki2013} \\
Da QST bevidst påfører patienter smerte, skal det undersøges, hvorvidt denne smerte er acceptabel i forhold til, om det pådrager patienten skader. Internationale retningslinjer proklamerer, at en given undersøgelse altid skal underlægges klinikerens viden og bevidsthed om at sikre patientens helbred, velfærd og rettigheder. Det fremsiges ligeledes, at selv om forsknings primære mål er at finde ny viden, må forskningens formål aldrig overskygge patientsikkerheden. Forskningens udbytte skal derfor altid være af større betydning end den risiko, det kan udsætte patienten for. \cite{helsinki2013} Dette betyder generelt, at det kan accepteres, at en patient udsættes for radioaktiv stråling, hvis det indgår som led i diagnosticering af en livstruende tumor. I dette tilfælde overskygger risikoen ved undersøgelsen den risiko, som truer patientens helbred. Danske nationale kliniske retningslinjer (NKR) for patientsikkerhed er opsat af Sundhedsstyrelsen og omhandler primært sundhedsprofessionelle indenfor den danske sundhedssektor. Her udgør retningslinjerne et sundhedsfagligt beslutningsværktøj, som skal sikre, at udredning, behandling, pleje og rehabiliteringen i sektoren er ensartet og af høj kvalitet. \cite{nkr2016, kommissorium2012} De danske retningslinjer skal således sikre, at en patient vil få samme behandling af samme kvalitet ligegyldigt hvor i landet, denne bliver behandlet. Disse retningslinjer sikrer ligeledes, at viden deles mellem regioner sådan en patient altid vil modtage den bedst tilgængelige behandling. \cite{nkr2016} \\
På baggrund af teknologianalysen i afsnit \ref{?tek?} er der udvalgt tre tests fra QST, der undersøger PPT, TS og CPM . Disse anvender alle at påvirke patienten med mekanisk tryk. Det undersøges derfor, hvilke fysiske grænser hud og muskelvæv har overfor trykpåvirkning og hvornår et givent tryk kan udøve skader på kroppen. \\
\subsubsection{Mekanisk trykpåvirkning}
Ved mekanisk påvirkning af tryk vil materialer deformeres, hvis kraften overstiger deres flydegrænse, hvorefter materialet vil brydes, hvis kraftpåvirkningen fortsætter og overstiger brudstyrken. Materialer som hud og blødt væv har en vis form for elasticitet, men vil ligeledes deformeres ved en kraftpåvirkning og destrueres, når cellerne i kroppen ødelægges. Dette vil ses som blodansamlinger og sår på hud og væv. Skader som følge af tryk afhænger af størrelsen på kraften og størrelsen af området som påvirkes. Det følger forholdet $P = \frac{F}{A}$, hvor F er kraften og A er arealet som påvirkes. Et studie af \citer{aisling2012} har undersøgt mekaniske egenskaber ved menneskehud. Undersøgelsen har anvendt samples af menneskehud fra ryggen som er blevet testet i et apparatur som kan strække huden. Elasticiteten af huden er blevet målt løbende og indtil huden nåede sin brudstyrke. Brudstyrken blev bestemt til $21,6 \pm 8,4 MPa$. Dette stemmer overens med en studie af \citer{jussila2005}, der undersøgte mulighederne for at finde en simulator for menneskehud til tests af skydevåben. For at kunne fungere som et brugbart alternativ skulle simulanten overholde nogle grænser, her i blandt en brudstyrke på $18 \pm 2 \frac{N}{mm^{2}} = 18 MPa$. Andre studier har undersøgt effekten af påvirkning af tryk over en længere periode. \citer{sanders1995} har sammenholdt forskellige studier og fundet, at skader på hud og muskelvæv ligeledes opstår ved en svagere kraftpåvirkning, hvis påvirkningen sker over længere tid. I litteraturen findes det, at der sker skader på hud og underliggende muskelstrukturer ved tryk på $13 kPa$ ved påvirkning i to timer. Ved samme påvirkning i seks timer sker komplet muskelnekrose. \cite{sanders1995} \\
\subsubsection{Overholdelse af skadegrænser}
Der kan således opstilles bestemte grænser for, hvad hud og muskelvæv kan modstå i forhold til påvirkning ved mekanisk tryk. Disse grænser kan sammenlignes med hvad en patient påvirkes med ved en QST-undersøgelse. \\
Ifølge \citer{rolke2006}, som en del af QST-protokollen fra German Research Network on Neuropathic Pain (DFNS), bliver en patient ved PPT-testen i en QST-undersøgelse, maksimalt udsat for et mekanisk tryk på $2000 kPa$, på et område af $1 cm^{2}$. Dette er mindre end $21,6 \pm 8,4 MPa$, som det er fundet af \citer{aisling2012} og \citer{jussila2005}. Det er dog over $13 kPa$, som ifølge \citer{sanders1995} kan påføre muskelskader. Dette er dog kun ved trykpåvirkning i seks timer, hvor trykpåvirkning ved QST-undersøgelsen kun tager omkring 40 sekunder \cite{rolke2006}. 
\section{Sikkerhedsforanstaltninger}
\textit{Her kommer et kursivt indledende afsnit.}
\subsection{Sikkerhedsforanstaltninger ved anvendelse af QST}
Selvom der generelt ikke er farer forbundet med udførelsen af en QST-undersøgelse, bør der stadig tages sikkerhedsforanstaltninger i forhold til at sikre reproducerbarhed af de enkelte undersøgelser. I et studie af \citer{shy2003} foretages en vurdering af QST i forhold til parametrene effektivitet, anvendelighed og sikkerhed. I studiet blev anvendt artikler om forskellige QST-systemer, der blev sammenlignet i forhold til de ovennævnte parametre. I studiet konkluderes, at det er væsentligt, at klinikeren modtager vejledning fra producenterne af QST-udstyr i forhold til anvendelsen heraf. Undersøgelserne bør desuden foretages i et stille lokale, der er indrettet til formålet, således resultaterne af undersøgelsen ikke påvirkes af udefrakommende faktorer. Derudover bør den samme kliniker udføre både den primære og de eventuelt opfølgende undersøgelser, da dette sikrer, at disse bliver foretaget på samme måde hver gang, hvormed reproducerbarheden øges. \citep{shy2003}
Ved anvendelse af computerstyrede metoder til QST-undersøgelse, er udførelsen af undersøgelserne i højere grad automatiseret og dermed mere uafhængige af klinikeren \citep{Nielsen2015}. Det kan dermed antages, at der skal tages færre sikkerhedsforanstaltninger i forhold til opnåelse af korrekte resultater.  

\section{Delkonklusion}
Af QST-protokollen udarbejdet af DFNS fremgår det, at det maksimale mekaniske tryk, en patient bliver udsat for, er $2000 kPa$, hvilket overskrider grænsen for muskelskade fundet i studiet af \citer{sanders1995}. Da denne grænseværdi er fundet efter trykpåvirkning i seks timer og QST-undersøgelsen udføres på cirka 40 sekunder vurderes det imidlertid, at skaderne forbundet hermed vil være minimal. Dermed overholdes de internationale retningslinjer for patientsikkerhed, da det resultat, der opnås ved QST-undersøgelsen, bidrager til vurderingen af, hvorvidt patienten skal have foretaget en TKA-operation. Dette udbytte er af større betydning end de sikkerhedsmæssige risici, patienten udsættes for.
For at sikre, at balancen mellem udbytte og risici opretholdes, er det væsentligt at der tages visse sikkerhedsforanstaltninger for at sikre høj reproducerbarhed af undersøgelserne. Sikkerhedsforanstaltningerne omfatter blandt andet vejledning fra producenter, korrekt indretning af lokale og at patienten konsulterer den samme kliniker gennem alle QST-undersøgelser. Computerstyrede metoder er mere automatiserede, hvorved det vurderes at det vil kræve minimal indførelse af sikkerhedsforanstaltninger.


