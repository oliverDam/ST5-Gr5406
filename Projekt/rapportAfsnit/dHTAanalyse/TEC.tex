\section{Problemformulering}
\begin{center}
	\textit{Hvilke konsekvenser er forbundet med implementering og brug af QST, som supplement til klinikerens vurdering af en patientens henvisning til en TKA-operation}
\end{center}

\section{Beskrivelse og tekniske karakteristika for teknologien (TEC)}
\subsection{Formål}
I teknologidomænet analyseres QST som et teknologisk supplement til klinikerens vurdering af en patients henvisning til en TKA-operation.  For at muliggøre dette kræves et kendskab til både klinikerens vurdering, og QST. Gennem problemanalysen er klinikernes vurdering undersøgt, mens teknologidomænet danner grundlag for kendskab til QST.\\
For at opnå et kendskab til QST forventes en forståelse for teknologiens egenskaber, virkemåde samt begrænsninger. Heraf undersøges det oprindelige formål med QST, med henblik på at bestemme, hvordan denne er tilpasset og brugt i sammenhæng med TKA-operationer. Dette bidrager til at vurderingsgrundlaget for hvorvidt QST kan fungere i samspil med klinikeren. Ydermere er det nødvendigt at vide hvad QST kan, hvordan dette er muligt, samt hvordan teknologien begrænses. Hermed er det muligt at vurdere hvordan og i hvilket omfang QST vil kunne fungere som supplement til klinikeren.\\
Da der er blevet tilegnet et kendskab til QST, er der grundlag for at undersøge og sammenligne QST med klinikerens vurdering. En sammenligning af QST med klinikerens metode er nødvendig for at kunne vurdere hvorvidt QST har evnen til at fungere som supplement hertil. 
\subsection{HTA spørgsmål}
\textit{Teknologiens egenskaber:}
\begin{itemize}
	\item Hvad er teknologiens oprindelige formål og hvordan er den tilpasset knæartrose på nuværende tidspunkt? %B0003, A0022
	\item Hvordan virker teknologien?  %B0001
\end{itemize}

\textit{Teknologiens begrænsninger:}
\begin{itemize}
	\item Hvilke eksterne faktorer kan påvirke QST-resultaterne?
\end{itemize}

\textit{Sammenligning med nuværende metoder:}
\begin{itemize}
	\item Hvordan adskiller QST-teknologien og den nuværende medicinske teknologi sig fra hinanden? %B0001, B0002
\end{itemize}
%
%\subsection{Metode \citep{HTAcore}}
%Vidensindsamlingen til besvarelse af det teknologiske område vil foregå ved at søge efter materiale igennem HTA/MTV agenturer, i relevante sundhedssystemer og sundhedsudbydere. Databasen der anvendes afhænger af hvilket spørgsmål det søges at besvare. Til indsamling af viden omhandlende de tekniske fakta vedrørende teknologien vil der blive benyttet relevante datablade fra producenter, beskrivende reviews samt ekspertudtalelser fra personel som benytter teknologien, hvis nødvendig. Til besvarelse af hvordan teknologien er udviklet, samt benyttet vil der hovedsageligt blive benyttet reviews fra anerkendte databaser (PubMed, Medline, the Cochrane Library) samt bøger.  \\
%Til besvarelse af TEC-analysen bestræbes det at undgå grå litteratur, men i tilfælde, hvor der foreligger begrænsede mængder videnskabelig litteratur, kan der supplementeres med non-peer reviewed-, ikke-publiceret materiale, fortroligt kommercielt materiale samt generelle internetsøgninger.
