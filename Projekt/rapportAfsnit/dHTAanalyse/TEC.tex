\section{Formål}
I teknologidomænet analyseres QST som et teknologisk supplement til klinikerens vurdering af en patients henvisning til en TKA-operation.  For at muliggøre dette kræves et kendskab til både klinikerens vurdering, og QST. Gennem problemanalysen er klinikernes vurdering undersøgt, mens teknologidomænet danner grundlag for kendskab til QST.\\
For at opnå et kendskab til QST forventes en forståelse for teknologiens egenskaber, virkemåde samt begrænsninger. Heraf undersøges det oprindelige formål med QST, med henblik på at bestemme, hvordan denne er tilpasset og brugt i sammenhæng med TKA-operationer. Dette bidrager til at vurderingsgrundlaget for hvorvidt QST kan fungere i samspil med klinikeren. Ydermere er det nødvendigt at vide hvad QST kan, hvordan dette er muligt, samt hvordan teknologien begrænses. Hermed er det muligt at vurdere hvordan og i hvilket omfang QST vil kunne fungere som supplement til klinikeren.\\
Da der er blevet tilegnet et kendskab til QST, er der grundlag for at undersøge og sammenligne QST med klinikerens vurdering. En sammenligning af QST med klinikerens metode er nødvendig for at kunne vurdere hvorvidt QST har evnen til at fungere som supplement hertil. 
\section{HTA spørgsmål}
\textit{Teknologiens egenskaber:}
\begin{itemize}
	\item Hvad er teknologiens oprindelige fokusområde og hvordan er den tilpasset knæartrose på nuværende tidspunkt? %B0003, A0022
	\item Hvordan virker teknologien?  %B0001
\end{itemize}

\textit{Teknologiens begrænsninger:}
\begin{itemize}
	\item Hvilke eksterne faktorer kan påvirke QST-resultaterne?
\end{itemize}

\textit{Sammenligning med nuværende metoder:}
\begin{itemize}
	\item Hvordan adskiller QST-teknologien og den nuværende medicinske teknologi sig fra hinanden? %B0001, B0002
\end{itemize}

\section{Metode \citep{HTAcore}}
Vidensindsamlingen til besvarelse af det teknologiske område vil foregå ved at søge efter materiale igennem HTA/MTV agenturer, i relevante sundhedssystemer og sundhedsudbydere. Databasen der anvendes afhænger af hvilket spørgsmål det søges at besvare. Til indsamling af viden omhandlende de tekniske fakta vedrørende teknologien vil der blive benyttet relevante datablade fra producenter, beskrivende reviews samt ekspertudtalelser fra personel som benytter teknologien, hvis nødvendig. Til besvarelse af hvordan teknologien er udviklet, samt benyttet vil der hovedsageligt blive benyttet reviews fra anerkendte databaser (PubMed, Medline, the Cochrane Library) samt bøger.  \\
Til besvarelse af TEC-analysen bestræbes det at undgå grå litteratur, men i tilfælde, hvor der foreligger begrænsede mængder videnskabelig litteratur, kan der supplementeres med non-peer reviewed-, ikke-publiceret materiale, fortroligt kommercielt materiale samt generelle internetsøgninger.

% Generelt er litteratur søgt således at antal hits per søgning, er af en overskuelig karakter. Dette betyder at alle overskrifter for den pågældende søgning er blevet gennemgået for relevans. Ved de udvalgte artikler, er abstract efterfølgende blevet gennemlæst for at sortere materialet. De artiker som er blevet udvalgt relevante til besvarelse af HTA-spørgsmålet, er blevet kritisk gennemgået. 
% Ved en enkelt artikel er der blevet benyttet snowballing vedrørende puplicerede artikler tilhørende den skrevne forfatter. 

\section{Fokusområde for QST}
I perioden omkring det 19-århundrede blev der udviklet flere medicinske værktøjer til kvantitativ vurdering af sensation. Vurderingselementerne omhandlede klassificering af tærskelværdier, tolerancer og stimuli-respons forhold. Nødvendigheden af at kunne kvantificere en tilstand er en essentiel faktor for ethvert videnskabeligt resultat. Validiteten af et videnskabeligt resultat bør altså delvist dannes på grundlag af en kvantificering af relevante parametre. \citep{Yarnitsky2006} Videnskabsmændene i det sene 19-århundrede beskæftigede i større grad tilstande som ikke forvoldte smerter, og heraf blev de mest populære fund relateret til de termiske sanser samt vibrationssensation. \citep{Yarnitsky1997} QST muliggør at undersøge tilstande ved benyttelsen af andre stimuli. QST omfatter termisk, mekanisk og elektrisk, iskemisk og kemisk påvirkning. De forskellige stimuli bidrager til at kunne undersøge forskellige parametre. Parametrenes reaktion på de forskellige stimuli kan klassificeres som værende af forringet effekt eller forøget effekt, og visualiseres ofte igennem en VAS-skala. \citep{Yarnitsky2006}

Til klassificering af tærskelværdier samt tolerancer bliver der generelt benyttet to basisgrupperinger, med- og uden reaktionstid. Der er blevet udviklet flere metoder til begge af disse basisgrupperinger. Grundlæggende for metoderne omhandlende reaktionstid er transmissionstiden fra et påført stimuli til reaktion. Ved metoder uden benyttelse af reaktionstid indebærer det oftest 'ja/nej' reaktioner til intensitetsforudbestemte stimuli, eller benyttelsen af VAS, hvorefter intensiteten ændres.\fxnote{Method of levels, Method of limits} \citep{Yarnitsky1997} \citep{Yarnitsky2006} 

QST betegnes som en subjektiv vurderingsmetode, da det omfatter en subjektiv respons indenfor et psykofysisk parameter til et kontrolleret stimuli. Den subjektive respons er på baggrund af patienten deltager under frivillig kontrol. \citep{Mucke2016} I og med vurderingsmetoden er subjektiv, kan resultatet påvirkes af distraktioner, kedsommelighed, mental træthed og forvirring. Ydermere kan en subjektiv respons ydermere bidrage til at patienten bevidst fejlrapporterer på baggrund af en interessekonflikt i resultatet. \citep{Yarnitsky2006} Betydningen af at QST er en subjektiv vurderingsmetode, bør behandlingen omfatte kvalitetskontrol. Det nævnes af \citer{Yarnitsky2006} at resultater opnået med 'Method of Limits' imellem sessioner kan variere op mod 150\%, mens ved benyttelsen af en anden algoritme, opstod en individuel variation på 5\%. For at sammenligne resultater imellem sessioner er korrelationsteknikker blevet kritiseret. Kritikken opstår på baggrund af at korrelationsteknikken måler styrken af forholdet, og ikke hvor ens disse er. En tilgang som blev fremstillet mere pålidelig er at udregne en repeterbarhedsfaktor (r), da med et konfidensinterval på 95~\%, vil to resultater fra samme patient vil afvige med mindre end r-faktoreren. Dog er skal det tages højde for at det kan være problematisk at udføre statistiske beregninger mellem grupper. Dette er tilfældet da resultaterne stammer fra subjektive vurderinger. \citep{Zaslansky1998} I relation til afbenyttelse af normativ data vedrørende tærskelværdier og toleranceværdier skal den eksakte metode benyttelse. Hvis ikke der tages højde for den eksakte metode vil resultaterne sandsynligvis ikke være repræsentative og heraf afvige fra normalen der tages udgangspunkt i. \citep{Yarnitsky1997}

\section{Klinisk anvendelse af QST}
QST bliver benyttet i flere kliniske sammenhænge. I mange tilfælde bliver QST benyttet som et diagnostisk værktøj, men kan også bruges til at vurdere omfanget af en sensorisk tilstand. Generelt bliver QST benyttet til at kunne klassificere sygdomme relateret til CNS, heraf flere omhandlende sensationstab. Dette er også tilfældet ved benyttelsen relateret til diabetes. 50\% af diabetikere har perifer neuropati, og ved benyttelse af en termisk parameter af QST, kan man tidligt i patologien identificere dette. Neuropati er ligeledes fundet ved mange med nyresvigt, hvilket kan klassificeres igennem undersøgelse af vibrationssensation og termisksensation. \citep{Yarnitsky1997} \citep{Yarnitsky2006} \\
QST bliver også benyttet til at klassificere smerter, men til dette er der relateret en bred problemstilling. Problematikken opstår i og med det psykofysike og psykologiske som QST er bestående af, samt forskellig smerteopfattelse og smertereaktion. \citep{Yarnitsky1997} Det er anerkendt at mennesker har forskellige smertemønstre i relation til smertestimuli. Det kan imellem patienter varierer fra at være yderst følsom til at være upåvirket. Forståelsen af disse smertemønstre, kan med den korrekte QST tilgang, bidrage til at kunne forudsige responsen af en given interaktion. \citep{Yarnitsky2006} Ydermere kan QST i smerteregi, klinisk blive benyttet til at lave kvalitative sammenligninger mellem forskellige grupperinger \citep{Arendt-Nielsen2009}. \\
Dog menes det generelt, at QST-resultater ikke, som det eneste resultat, kan benyttes til at stille en diagnose. Dette er på baggrund af tidligere nævnte årsager relateret til det subjektive aspekt i metoden. \citep{Yarnitsky2006}

\section{Tilpasning af QST til knæartrose}
QST bliver i et knæartroseregi forsøgt benyttet til at skabe en association imellem præoperativ smertesensation og udviklingen af kronisk postoperativ smerte. Dette ses også ved den øgede interesse omhandlende kirurgisk afbenyttelse af teknologien. Der ses et potentiale i afbenyttelsen af præoperative QST-undersøgelser, som prædiktion vedrørende omfanget af kroniske postoperative smerter. \citep{Wylde2013} \citep{Lunn2013} QST kan undersøge denne smerte sensation igennem flere forskellige parametre eksempelvis, tryk-smerte tærskel (PPT), kulde- og varme-smerte tærskel og tolerance, kulde-smerte rating, temporal summation (TS), betinget smertemodulation (CPM) og dybt vævs respons på mekanisk smerte. \citep{Cornelius2015} Dette er blot et uddrag af parametre der kan testes for skal skabe en smerte sensationsprofil. Ved benyttelse af mange parametre vil QST være omfattende og tidskrævende, hvilket kan antages at være problematisk i et klinisk regi. Dette skaber en begrænsning i benyttelse af QST, og dermed bør kun diagnostisk relevante måleparametre blive benyttet. \citep{Lunn2013}\textbf{ Studier antyder, at central sensibilisering har betydning for udviklingen af kroniske postoperativesmerter efter en TKA-operation \citep{Suokas2012}. Central sensibilisering opstår som følge af de degenererende forandringer, der sker i et led på grund af knæartose. \citep{Arendt-Nielsen2015}  Efter en succesfuld og smertefri TKA operation forbedres den centrale sensibilisering i nogle tilfælde, mens ved patienter med kroniske postoperative smerter forbedres den centrale sensibilisering ikke i samme grad som for andre patienter. \citep{Arendt-Nielsen2015} Central sensibilisering kan undersøges ud fra forskellige QST parametre. Parametrene er PPT, TS og CPM. \citep{Arendt-Nielsen2015}} Flere studier har ligeledes undersøgt hvilke parametre som er diagnostisk realteret til udviklingen af kroniske postoperative smerter. Her er den største konsensus ligeledes, at QST parametrene med størst diagnostisk relevans, nedsat PPT, faciliteret TS og nedsat CPM. Alle af disse QST parametre kan profileres igennem mekanisk QST. \citep{Petersen2015} \citep{Petersen2016} \citep{Wylde2015} \\
I et studie af \citer{Petersen2016}, blev knæartrose patienter inddelt i grupperinger omhandlende deres QST-resultater vedrørende TS og CPM. Disse inddelinger resulterede i at hverken TS eller CPM, som enkeltstående måleparameter, statistisk kan benyttes som indikerende faktor for en patients postoperative resultat. Studiet indikerer heraf at patienter både med faciliteret TS og nedsat CPM, er i øget risiko for at udvikle kroniske postoperative smerter. Dette resultat understreger at multible QST parametre bør benyttes for at QST kan opfylde dets kliniske formål. \citep{Petersen2016} Flere studier indikerer at udbredt hyperalgesi kan være et prædiktivt QST parameter, omhandlende en patients kroniske postoperative smerter. \citep{Petersen2016} \citep{Wylde2013} Modsat indikerer studiet af \citer{Wylde2015}, at udbredt hyperalgesi ikke er en prædiktiv fakor for en patientens resultat af TKA, men at dette blot er et parameter som kunne give en ekstra værdi til klassificeringen af patienten. Dette er tilfældet da forhøjet udbredt hyperalgesi, vurderet igennem PPT, var associeret med alvorlige præoperative smerter. Den forhøjede udbredte hyperalgesi, var ydermere også associeret med kroniske postoperative smerter ved total hofte udskiftning, hvoraf det endnu ikke er påvist omhandlende TKA. \citep{Wylde2015} \\
Til videre analyse vil QST parameterene, PPT, TS og CPM udelukkende blive benyttet. 