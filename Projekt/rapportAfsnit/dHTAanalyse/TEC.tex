\section{Formål} 
I dette domæne analyseres QST som et teknologisk supplement til klinikerens vurdering af en patients henvisning til en TKA-operation. For at muliggøre dette kræves viden om både klinikerens vurdering og QST. Gennem problemanalysen er klinikerens vurdering undersøgt, mens TEC-domænet danner grundlag for viden om QST.\\
For at opnå viden om QST forventes en forståelse for teknologiens egenskaber, virkemåde samt begrænsninger. Heraf undersøges det oprindelige formål med QST, med henblik på at bestemme, hvordan denne er tilpasset til brug i forhold til indstilling til TKA-operationer. Ydermere er det nødvendigt at vide hvordan QST fungerer, samt hvilke begrænsninger der forekommer for QST. Hermed er det muligt at vurdere hvordan og i hvilket omfang QST vil kunne fungere som supplement til klinikerens vurdering af en patients egnethed til en TKA-operation.\\
Med et kendskab til QST, er der grundlag for at undersøge og sammenligne QST med klinikerens vurdering. 

\section{Analyse-spørgsmål}
Egenskaber ved QST:
\begin{enumerate}
	\item \textit{Hvad er det oprindelige anvendelsesområde for QST og hvordan er QST tilpasset knæartrose?} %B0003, A0022}
	\item \textit{Hvordan virker QST?}  %B0001}
\end{enumerate}
Begrænsninger ved QST:
\begin{enumerate}[resume]
	\item \textit{Hvilke faktorer kan påvirke QST-resultaterne?}
\end{enumerate}
Sammenligning med nuværende metode:
\begin{enumerate}[resume]
	\item \textit{Hvordan adskiller QST og klinikeres nuværende metode sig fra hinanden?} %B0001, B0002}
\end{enumerate}

\section{Metode}
Til besvarelse af dette domæne tager litteratursøgningen  udgangspunkt i den generelle metode (jævnfør \chapref{metode}). Ud fra det valgte analyse-spørgsmål er der udarbejdet inklusion- og eksklusionskriterier for at specificere og afgrænse søgningen til relevant litteratur. Dette har bidraget til at præcisere søgningen, inden den generelle gennemgang af materiale. Igennem søgningerne er der til hvert analyse-spørgsmål blevet benyttet forskellige kombinationer af søgeord, som det fremgår af søgeprotokollen (jævnfør \appref{TEC_sog}). Kombinationerne af forskellige søgeord har bidraget til en bredere afsøgning af litteraturen indenfor det specifikke analyse-spørgsmål. Dette har bidraget til at litteratursøgningen til det specifikke analyse-spørgsmål indeholder tilstrækkelig viden til netop at kunne besvare dette. Der er i TEC-domænet kun benyttet videnskabelig litteratur i form af bøger og peer-reviewed materiale. \\
Til besvarelse af analyse-spørgsmål (1) er der indsamlet litteratur omkring hvordan QST har udviklet sig. Denne litteratur danner grundlag for viden omkring det oprindelige formål for QST samt hvordan QST har udviklet sig til den form som kan anvendes til patienter med knæartrose. Herigennem vil det være muligt at bestemme præcis hvilke QST-parametre der skal undersøges for at besvare analyse-spørgsmål (2). Ved dette spørgsmål søges efter litteratur hvor de fundne QST-parametre anvendes, og metoderne for undersøgelse af disse parametre beskrevet i de enkelte studier sammenlignes. \\
Analyse-spørgsmål (3) besvares gennem viden fra de to foregående analyse-spørgsmål omkring hvordan undersøgelserne af de udvalgte QST-parametre udføres, hvormed det er muligt at uddrage begrænsninger ved disse. Det sidste analyse-spørgsmål, (4), besvares gennem en sammenfatning af viden fra problemanalysen samt de foregående analyse-spørgsmål.   
 
\section{Teknologiens egenskaber}
\textit{I følgende afsnit undersøges egenskaberne for QST. Dette gøres for at skabe et kendskab til det oprindelige formål for QST samt hvilke QST-parametre der kan anvendes til undersøgelse af patienter med knæartrose. Ligeledes beskrives undersøgelsesmetoden for de fundne QST-parametre.}

\subsection{Anvendelsesområde for QST}
I midten af det 19. århundrede blev der udviklet flere medicinske værktøjer til kvantitativ vurdering af perception af stimuli. Denne vurdering byggede blandt andet på klassificering af tærskelværdier, tolerancer og stimuli-respons forhold. \citep{Yarnitsky1997} Forskerne i det sene 19. århundrede beskæftigede sig i større grad med tilstande, som ikke forvoldte smerter, og heraf blev de fleste fund relateret til termisk- samt vibrationssensation. \citep{Yarnitsky1997}\\
QST er en fællesbetegnelse for tests der undersøger perception af stimuli hvormed der findes forskellige QST-protokoller der omfatter termisk, mekanisk, elektrisk, iskæmisk og kemisk stimuli. \citep{Yarnitsky2006} De forskellige typer af stimuli bidrager til at kunne undersøge forskellige typer af nervefibre. Forsøgspersonernes reaktion på de forskellige stimuli kan klassificeres som værende af forringet eller forøget effekt, og visualiseres ofte igennem en visual analogue scale (VAS). \citep{Yarnitsky2006} VAS benyttes til subjektiv bedømmelse af smerte, hvor smerteintensiteten angives som et tal mellem 0 og 10, hvor 0 er er ingen smerte og 10 er den værst tænkelige smerte \citep{smerter}. 

QST betegnes som en subjektiv vurderingsmetode, da det omfatter en subjektiv respons til et kontrolleret stimuli. \citep{Mucke2016} Da vurderingsmetoden er subjektiv, kan resultatet blandt andet blive påvirket af distraktioner, kedsommelighed, mental træthed og forvirring. Ydermere kan en subjektiv respons bidrage til, at patienten bevidst fejlrapporterer på baggrund af en interesse i et bestemt resultat. \citep{Yarnitsky2006} Da QST er en subjektiv vurderingsmetode, bør behandlingen omfatte kvalitetskontrol, eksempelvis beståede af reproducerbarhed og statistiske sammenhænge.

Generelt bliver QST anvendt til at klassificere sygdomme relateret til både til CNS og det perifere nervesystem (PNS), heraf flere omhandlende sensationstab. Fastsættelsen af sensationstærskler er relateret til PNS, hvorimod smertekontrol er relateret til CNS. \\ 
En af de sygdomme QST eksempelvis anvendes til at diabetes, hvor det ses at 50~\% af diabetikere har perifer neuropati, og ved anvendelse af en termisk QST-parameter, kan dette identificeres tidligt i patogenesen. Neuropati er ligeledes fundet ved flere patienter med nyresvigt, hvilket kan klassificeres igennem undersøgelse af vibration- og termisk sensation. \citep{Yarnitsky2006} \citep{Yarnitsky1997} \\
QST benyttes også til at klassificere smerter samt smertemekanismer, hvortil der er relateret en problemstilling. Da QST bygger på psykofysiske parametre og subjektive resultater opstår et problem i, at der patienter imellem er forskel i smerteopfattelse og smertereaktion. \citep{Yarnitsky1997} Forståelsen af disse smertemønstre kan med den korrekte QST-tilgang bidrage til at kunne forudsige responsen på en given interaktion. \citep{Yarnitsky2006} Ydermere kan QST i smerteregi, klinisk blive benyttet til at lave kvantitative sammenligninger mellem forskellige grupper \citep{Arendt-Nielsen2009}. \\
Generelt benyttes QST-resultater ikke, som det eneste resultat til at stille en diagnose. Dette er på baggrund af tidligere nævnte årsager relateret til det subjektive aspekt i teknologiens metode. \citep{Yarnitsky2006}

\subsubsection{Tilpasning til knæartrose}
QST bliver i knæartroseregi forsøgt benyttet til at skabe en association imellem præoperativ smertesensation og udviklingen af kronisk postoperativ smerte. Der ses et potentiale i anvendelsen af præoperative QST-undersøgelser som prædiktor for risikoen for udviklingen af kroniske postoperative smerter. \citep{Wylde2013} Ved anvedelse af QST kan klinikeren undersøge smertesensation igennem flere forskellige parametre eksempelvis, pressure pain threshold (PPT), thermal threshold and tolerance, cold pain rating, TSP og CPM. \citep{Cornelius2015} Dette er kun et uddrag af parametre, der kan undersøges for at skabe en smertesensationsprofil. Ved benyttelse af mange parametre vil QST være omfattende og tidskrævende, hvilket kan antages at være problematisk i et klinisk regi. Dette skaber en begrænsning i benyttelse af QST, og dermed bør kun diagnostisk relevante parametre blive benyttet. \citep{Nielsen2009} En sammenfatning af flere studier antyder, at central sensibilisering har betydning for udviklingen af kroniske smerter efter en TKA-operation \citep{Suokas2012}.\\
Central sensibilisering kan undersøges ud fra følgende QST-parametre; PPT, TSP og CPM. \citep{Arendt-Nielsen2015b} Flere studier har ligeledes undersøgt hvilke parametre, som er diagnostisk relateret til udviklingen af kroniske postoperative smerter. Disse QST-parametre kan undersøges ved mekanisk stimuli. \citep{Petersen2015} \citep{Petersen2016} \citep{Wylde2015b} \\
I et studie af \citer{Petersen2016} blev knæartrosepatienter grupperet efter deres QST-resultater vedrørende TSP og CPM. Disse inddelinger resulterede i at hverken TSP eller CPM, som enkeltstående måleparameter, statistisk kan benyttes som indikerende faktorer for en patients postoperative resultat. Det blev i stedet fundet at patienter med faciliteret TSP og nedsat CPM har mindre smertelindring end andre patienter. Dette resultat understreger at multible QST-parametre bør benyttes for at QST kan opfylde dets kliniske formål. \citep{Petersen2016} Flere studier indikerer at udbredt hyperalgesi kan være en prædiktiv faktor, af en patients risiko for udviklingen af kroniske postoperative smerter. \citep{Petersen2016} \citep{Wylde2013} Dette understøttes også i studiet af \citer{Wylde2016c} hvis resultater indikerer, at patienter med en sværere grad af knæartrose eller mere udbredt hyperalgesi får et mindre godt udbytte af TKA, end patienter med mindre udbredt hyperalgesi. Udbredt hyperalgesi kan undersøges med anvendelse af PPT. Dette resulterer i, at udbredt hyperalgesi kan være en prædiktiv faktor, men resultaterne er af svag statistisk evidens. \citep{Wylde2016c}\\
På baggrund af ovenstående analyse ses en indikation på, at QST-parametrene PPT, TSP og CPM kan anvendes til identificering af patienter i risiko for at udvikle kroniske postoperative smerter. Heraf vil den videre analyse udelukkende omhandle QST-parametrene PPT, TSP og CPM. Disse parametre betegnes herefter sammen som QST-protokollen. 

\subsection{Undersøgelse ved QST-protokollen}
De tre QST-parametre PPT, TSP og CPM kan testes ved forskellige typer stimuli. PPT testes mekanisk, mens TSP og CPM kan testes mekanisk, kemisk, elektrisk eller termisk. Oftest anvendes mekanisk stimuli i form af tryk. \citep{Yarnitsky2006} \citep{Suokas2012} Det er ifølge \citer{Imai2016} blevet påvist, at ved udførsel af CPM er pålideligheden størst ved anvendelse af mekanisk trykstimuli sammenlignet med benyttelse af kulde og varme. Heraf vil omtalte QST parametre fremtidigt tage udgangspunkt i mekanisk stimuli i form af tryk. Til trykstimuli kan udstyr fra Somedic eller NociTech anvendes \citep{Petersen2016} \citep{Wylde2015b}. 

Ved test af PPT undersøges det, om patienten har en forstærket reaktion på tryk. Dette kan gøres både på områder i umiddelbar nærhed af det påvirkede knæ og på områder, som er længere væk fra knæet. En lav PPT-værdi, og dermed højere sensitivitet for stimuli, i området omkring det påvirkede knæ antyder perifer sensibilisering, mens en lav PPT-værdi i områder væk fra det påvirkede knæ antyder central sensibilisering. \citep{Suokas2012} Det påførte tryk stiger, indtil patienten begynder at opfatte trykket som smertefuldt, og angiver dette ved eksempelvis tryk på en knap. PPT angives i kPa og defineres som styrken af det påførte tryk, når patienten angiver, at trykket bliver smertefuldt. Oftest gentages målingen tre gange, hvorefter gennemsnittet af de tre målinger anvendes som patientens PPT-værdi. \citep{Petersen2015} \citep{Wylde2015b} 

En forhøjet TSP kan antyde central sensibilisering, da reguleringen af TSP i neuroner er formindsket ved central sensibilisering. \citep{Arendt-Nielsen2015b} Hermed reagerer personer med central sensibilisering stærkere på gentagende stimuli end personer som ikke har central sensibilisering, hvilket illustreres på \figref{fig:TSP_rask_syg}. \citep{Arendt-Nielsen2015b} 

\begin{figure}[H] 
	\begin{center}
		\includegraphics[width=0.6\textwidth]{figures/dHTAanalyse/TSP_rask_syg.jpg}
	\end{center}
	\caption{Figuren viser, hvordan patienter med faciliteret TSP responderer anderledes på gentagende stimuli, end patienter uden faciliteret TSP. \citep{Reynolds2016}} 
	\label{fig:TSP_rask_syg} 
\end{figure} \vspace{-.25cm}

For at undersøge en patients TSP påføres patienten gentagne tryk med samme intensitet, med tilsvarende intervaller. I studiet af \citer{Petersen2016} blev patienten tilført tryk på et sekunds varighed efterfulgt af en pause på et sekund. Patienten blev i alt tilført 10 tryk. For hvert tryk angav patienten smerten ud fra VAS, og TSP blev udregnet som gennemsnittet af VAS for de første fire tryk trukket fra gennemsnittet af VAS for de sidste tre tryk. \citep{Petersen2016}

Descenderende smerteregulering er en betydende faktor for udviklingen af central sensibilisering. Den descenderende smerteregulering regulerer neuronernes reaktion på stimuli, og består af en balance mellem inhiberende og exciterende signaler. For personer med normal descenderende smerteregulering er denne hovedsageligt inhiberende. Ved central sensibilisering forskubbes balancen i den descenderende smerteregulering, hvormed neuronernes reaktion på stimuli ikke inhiberes på samme niveau som tidligere. Denne forskydning i balancen kan ske enten ved, at færre inhiberende signaler sendes, eller at flere exciterende signaler sendes til neuronerne. \citep{Arendt-Nielsen2015b} Den descenderende smerteregulering undersøges ved CPM. Ved test af CPM udsættes patienten for smertestimuli et sted på kroppen, mens PPT måles et andet sted på kroppen, som benævnes teststedet. Før den smertefulde stimuli påføres patienten, bliver PPT målt på teststedet. \citep{Petersen2016} CPM defineres som forskellen i PPT på teststedet før og efter den smertefulde stimuli er tilført et andet sted på kroppen. \citep{Petersen2015} 

\section{Teknologiens begrænsninger}
\textit{Efter beskrivelse af de tre QST-parametre, PPT, TSP og CPM undersøges begrænsningerne for undersøgelsesmetoderne. Hermed bestemmes eventuelle svage punkter ved hver af parametrene.}

\subsection{Begrænsende faktorer ved benyttelse af QST}
Ved benyttelse af QST-protokollen som et supplement til klinikerens beslutning bør teknologiens begrænsende faktorer vurderes. \\
Ved benyttelsen af QST-protokollen bør de forskellige parametres metode undersøges. Det kan forestilles, at målingerne til at danne parametrenes resultater, udføres med for kort et interval, kan der opstå forstyrrelser i form af en carry-over effekt. En carry-over effekt er når der opstår fejlresultater på baggrund af videreførsel af tidligere stimuli-respons, til fremtidige forsøg. \citep{Porta2008} For PPT-målinger vil denne carry-over effekt ske, hvis transmitterstoffer som blev udskilt fra neuroner ved første stimuli, ikke er blevet reabsorberet, før et nyt stimuli tilføres \citep{Martini2012}. Hvis PPT-undersøgelserne udføres med for kort interval, kan det tænkes, at der vil opstå en carry-over effekt, som kan resultere i en falsk PPT-værdi. Herfor er det vigtigt at overveje, hvor lang tid der går mellem hver måling, således carry-over effekten så vidt muligt undgås for alle QST-parametre. \citep{Porta2008} 

For at kunne benytte QST-protokollen som et led i den diagnostiske proces kræves det, at klinikeren har adgang til et normativt datasæt til klassificering af abnormale tilstande. Det normative datasæt skal bestå af normale tærskler og tolerancer, samt abnormale tærskler og tolerancer, før klinikeren kan adskille patientgrupper fra hinanden. Udviklingen af sådanne normative datasæt er nødvendig før en mulig implementering. Ved fremadrettet benyttelse af normative data kræves det, at den samme metode benyttes. Hvis ikke den samme metode benyttes kan det forestilles, at resultaterne vil afvige fra det normative datasæt, og heraf give falsk negative og positive resultater. Heraf vil standardiserede normative datasæt bidrage til at skabe pålidelige resultater og dermed gøre det muligt at kunne bestemme QST-protokollens sensitivitet og specificitet. \citep{Yarnitsky1997} Behovet for et normativt datasæt ses ligeledes i studiet af \citer{Petersen2016}, hvor forsøgspopulationen inddeles i grupperinger vurderet på baggrund af arbitrære valg. Forfatterene pointerer tilmed i studiet, at en normativ inddeling af patienter er nødvendig, og bør optimeres og gøres generaliserbar før implementering af QST-undersøgelserne. \citep{Petersen2016} 

\section{Sammenligning med nuværende metoder}
\textit{I følgende afsnit analyseres det hvordan QST-protokollen ud fra de foregående fund, kan bidrage til vurderingsgrundlaget. Dette gøres ved at undersøge samspillet mellem QST-protokollen og den nuværende medicinske teknologi i form af klinikerens vurdering.}

\subsection{Sammenspil mellem QST og klinisk vurdering} 
Den nuværende metode til udvælgelse af knæartrosepatienter til en TKA-operation bygger på klinikerens observationer og vurdering af patientens symptomer samt de objektive fund. Klinikeren samtaler ligeledes med patienten omkring sygdommen. Dermed opnås en kvalitativ vurdering af patientens tilstand. \citep{skou2016} \citep{Troelsen2012} Ved anvendelse af QST-protokollen tilføjes en kvantitativ målemetode, som er en mulig prædiktor for udvikling af kroniske postoperative smerter. Såfremt QST-protokollen har den ønskede effekt, vil tilføjelsen styrke vurderingsgrundlaget. Dette antages, da der gives et mere udførligt helhedsbillede end en beslutningsmetode som kun er bygget på den ene af de to slags observationer. \citep{Gronmo2012} \\
Herudfra kan QST-protokollen fungere som et supplement til klinikerens udvælgelse på baggrund af dens kvantitative karakteristika, og mulighed for tilføjelse af ny viden til klinikerens beslutningsgrundlag. Dette kræver dog, at QST nøjagtigt kan identificere patienterne med forhøjet risiko for udvikling af kroniske postoperative smerter.   

\section{Delkonklusion}
QST var oprindeligt udviklet til undersøgelse af forsøgspersoners sensation af stimuli. Siden er der blevet udviklet en række forskellige protokoller, der har hver sit formål. Den QST-protokol som er anvendelig til undersøgelse af muskuloskeletale smerter, og hermed kroniske postoperative smerter, indeholder parametre, som kan antyde central sensibilisering. Disse tre parametre er PPT, TSP og CPM. Det er ikke påvist, at PPT, TSP og CPM som enkeltstående parametre kan anvendes til identificering af patienter, som vil udvikle kroniske postoperative smerter, men resultater fra et studie antyder, at patienter med forhøjet TSP og inhiberet CPM, har flere postoperative smerter end andre patienter. Undersøgelserne der omhandler de tre QST-parametre har en række begrænsninger. En af disse begrænsninger er, at der, på nuværende tidspunkt, ikke er fundet normative værdier for de tre parametre. Uden normative værdier er det ikke muligt for klinikeren at vurdere, om patienten har abnormale værdier eller ej. Ligeledes er et problem med QST at undersøgelsernes resultater afhænger af patientens subjektive svar og reaktioner. \\
På trods af begrænsningerne ved anvendelse af QST-protokollen indikerer flere studier, at denne har potentialet til at styrke klinikerens vurderingsgrundlag. For at dette er muligt kræves det, at begrænsningerne ved undersøgelserne nedbringes. Hvis dette udbedres kan QST-protokollen styrke vurderingsgrundlaget, idet der ved anvendelse af protokollen tilføjes en kvantitativ metode til klinikerens overvejende kvalitative metode. Samlet set bør QST-protokollen, når udviklingen og optimeringen er udarbejdet kunne fungere som supplement til klinikeren, og heraf øge kvaliteten af den nuværende behandling.



