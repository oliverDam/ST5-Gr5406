\section{Beskrivelse og tekniske karakteristika for teknologien (TEC)}

\subsection{Formål}
I teknologidomænet analyseres QST som et teknologisk supplement til klinikerens vurdering af en patients indstilling til en TKA-operation. \\
For at kunne tage en beslutning om netop dette forventes et indgående kendskab til teknologiens egenskaber og dennes virkemåde. Teknologien benyttes andetsteds, men hvordan denne konkret er tilpasset problemstillingen vedrørende kroniske postoperative smerter efter TKA, er relevant. I sammenhæng med hvordan teknologien fungerer, undersøges hvad det kræves for at benytte teknologien og heraf vedligeholdelse. Relevansen herfor opstår i at dette er essentielt til senere analyseafsnit, samt det er nødvendig viden netop at forstå helheden af teknologien.\\
Enhver teknologi har begrænsninger, og det er essentielt for vurderingsgrundlaget netop at kende disse. Heraf bør det undersøges hvordan eksterne faktorer kan påvirke QST-resultaterne. \\
For endeligt at kunne klassificere den undersøgte teknologi og hvordan denne kan fungere i samspil med den nuværende, er en sammenligning af disse nødvendig. I sammenligningen undersøges det hvordan disse adskiller sig fra hinanden, samt interagerer sammen. %
\subsection{HTA spørgsmål}
\textit{Teknologiens egenskaber:}

\begin{itemize}
\item Hvad er teknologiens oprindelige formål og hvordan er den tilpasset knæartrose på nuværende tidspunkt? %B0003, A0022
\item Hvordan virker teknologien?  %B0001
\item Hvad kræves det at benytte teknologien, og i hvilken grad kræves der vedligeholdelse? %B0012, B0013
\item Hvilke forskelle er der på præoperative QST-resultater mellem en patient med kroniske postoperative smerter og en patient uden kroniske postoperative smerter? %B0018, B0008, B0009, B0010
\end{itemize}

\textit{Teknologiens begrænsninger:}
\begin{itemize}
\item Hvilke eksterne faktorer kan påvirke QST-resultaterne?
\end{itemize}

\textit{Sammenligning med nuværende metoder:}
\begin{itemize}
\item Hvordan adskiller QST-teknologien og den nuværende medicinske teknologi sig fra hinanden? %B0001, B0002
\end{itemize}

\subsection{Metode \citep{HTAcore}}
Vidensindsamlingen til besvarelse af det teknologiske område vil foregå ved at søge efter materiale igennem HTA/MTV agenturer, i relevante sundhedssystemer og sundhedsudbydere. Databasen der anvendes afhænger af hvilket spørgsmål det søges at besvare. Til indsamling af viden omhandlende de tekniske fakta vedrørende teknologien vil der blive benyttet relevante datablade fra producenter, beskrivende reviews samt ekspertudtalelser fra personel som benytter teknologien, hvis nødvendig. Til besvarelse af hvordan teknologien er udviklet, samt benyttet vil der hovedsageligt blive benyttet reviews fra anerkendte databaser (PubMed, Medline, the Cochrane Library) samt bøger.  \\
Til besvarelse af TEC-analysen bestræbes det at undgå grå litteratur, men i tilfælde, hvor der foreligger begrænsede mængder videnskabelig litteratur, kan der supplementeres med non-peer reviewed-, ikke-publiceret materiale, fortroligt kommercielt materiale samt generelle internetsøgninger.