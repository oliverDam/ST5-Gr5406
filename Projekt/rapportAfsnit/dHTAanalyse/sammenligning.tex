\section{QST og klinisk vurdering}

Spørgsmål:
Hvordan kan QST-teknologien supplere den nuværende medicinske teknologi?

QST:
kvantitativ vurdering - håndgribelige resultater 
viden om central sensibilisering

Klinisk:
viden om patient, forventninger, objektiv vurdering mm.
kvalitativ vurdering - klinikerens egen vurdering/holdninger
radiologiske fund


Den nuværende metode til udvælgelse af knæartrosepatienter til en TKA-operation bygger på kirurgers observationer og samtaler med patienten, og er dermed en kvalitativ vurdering. \citep{Troelsen2012} \citep{skou2016} Den eneste kvantitative parameter som anvendes ved den nuværende metode er radiologiske fund, der ikke altid er indikative for patientens oplevelse af smerte \citep{Leary2016}. Ved anvendelse af QST fåes hermed en kvantitativ målemetode som kan antyde udviklingen af postoperative kroniske smerter. Tilføjelsen af QST vil styrke vurderingsgrundlaget idet en beslutningsmetode, som bygger på både kvalitative og kvantitative observationer giver et mere udførligt helhedsbillede end en beslutningsmetode som kun er bygget på den ene af de to slags observationer. \citep{Gronmo2012} \\
Herudfra kan QST fungere som et suplement til klinikerens udvælgelse på baggrund af dens kvantitative karakteristika, og mulighed for tilføjelse af nu viden til beslutningsgrundlaget. Dette kræver, at QST nøjagtigt kan identificere patienterne med forhøjet risiko for udvikling af kroniske postoperative smerter.   




 


 

Denne vurdering er bedst da ... der er endnu ingen gode kvalitative metoder til vurdering af kronisk smerte ... QST kan være en mulighed hvis man kan overkomme begrænsningerne ... giver evaluerigen en kvantitativ del ... ting der bygger på begge typer data giver et stærkere vurderingsgrundlag ... udfra hvad der er fundet om teknologien kan den stadig godt være et supplement til klinikeren . 
