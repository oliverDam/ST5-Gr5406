\section{Formål}
I dette domæne undersøges, hvordan implementering af QST-protokollen på ortopædkirurgiske afdelinger vil påvirke personalets arbejdsgang og daglige opgaver. Implementering og brug af QST-protokollen kan potentielt medføre ændringer i intern kommunikation på afdelinger og personale imellem, hvorved det skal undersøges, hvilke ændringer dette vil medføre. Ændringer i kommunikation og arbejdsgange påvirker, hvordan sygehusledelse og personale skal prioritere tid og arbejdskraft. Dette kan medføre en ændring i antallet af daglige patientkonsultationer, ændring i arbejdsopgaver samt ændring i allokering af personale. \\
Ligeledes skal det undersøges, hvilke tiltag der bør tages forud for implementeringen af QST-protokollen, da korrekt anvendelse af en ny teknologi kan være vanskeligt at sikre, hvis personalet ikke introduceres korrekt hertil. Det undersøges derfor, i hvilket omfang oplæring vil være nødvendigt for at sikre korrekt og sikker brug af QST-protokollen. Ved implementering af protokollen vil der tilføjes en ny undersøgelsesmetode til afdelingen, som muligvis vil kræve regelmæssig vedligeholdelse for at sikre, at udstyret kan anvendes korrekt og ikke udsætter patienter for sikkerhedsmæssige risici. Dermed er det væsentligt at undersøge behovet for vedligeholdelse af QST-udstyret.

\section{Analyse-spørgsmål}
Organisatoriske ændringer:
\begin{enumerate}
	\item \textit{Hvordan påvirker QST-protokollen arbejdsgangene og intern kommunikation på ortopædkirurgisk afdeling?} %G0001, G0008
	\item \textit{Hvilke overvejelser skal der tages forbehold for, før QST-protokollen bliver anvendt korrekt?} %G0010
\end{enumerate}
Oplæring og vedligeholdelse:
\begin{enumerate}[resume]
	\item \textit{Hvorvidt kræves der efteruddannelse af personalet i forbindelse med anvendelse af QST-protokollen?} %G0003
	\item \textit{I hvilken grad er vedligeholdelse af QST-udstyr påkrævet?} %B0012, B0013
\end{enumerate}

\section{Metode}
Til besvarelse af dette domæne tager litteratursøgningen udgangspunkt i den generelle metode (jævnfør \secref{litteratursogning}). I domænet er litteratursøgningen også udbredt til ikke udelukkende at søge materiale i videnskabelige databaser. Dette er nødvendigt, da analysen omhandler den nuværende organisationsstruktur samt behandlingsforløbet for TKA-operationer. Viden herom er tilgængelig gennem Region Nords hjemmeside, som har specifikt materiale om interne strukturer og Aalborg Universitetshospitals (AAUH) behandlingsforløb for TKA-operationer. Der er derudover inddraget grå litteratur i form af mailkorrespondancer med producenter af QST-udstyr. Søgeprotokollen for SAF-domænet kan findes i \appref{SAF_sog}. Ydermere er den ortopædkirurgiske afdeling på Aalborg Universitetshospital Farsø besøgt, hvor patientkonsultationer i et TKA-behandlingsforløb blev observeret. \\
Analyse-spørgsmål (1) besvares på baggrund af en redegørelse for behandlingsforløbet, hvor det nuværende forløb sammenholdes med det forløb, hvor QST-protokollen er implementeret. Dermed er det muligt at undersøge forskelle i personalets arbejdsgange og opgaver. \\
Det undersøges for analyse-spørgsmål (2) hvilke forholdsregler, der ved implementering af QST-protokollen kan sikre, at personalet opnår kendskab til korrekt anvendelse af teknologien.
Implementering af QST-protokollen vil ligeledes kræve personale, som korrekt kan anvende teknologien, og vedligeholde udstyret. For analyse-spørgsmål (3) undersøges det derfor i hvilken grad implementering af QST-protokollen vil kræve efteruddannelse af personale, så korrekt anvendelse sikres. Vedligeholdelse af udstyr antages ligeledes at have en potentiel påvirkning på arbejdsgangen. \\
Til besvarelse af analyse-spørgsmål (4) er det væsentligt, at personalet, der skal udføre QST-undersøgelserne, modtager undervisning af producenterne af udstyret, for at sikre korrekt anvendelse heraf. På baggrund heraf tages kontakt til producenter af udstyr fra NociTech og Somedic. Der tages ligeledes kontakt til Cephalon A/S, som forhandler medicinsk udstyr fra Medoc. Dette gøres med henblik på indsamling af information om oplæring og varighed heraf samt de enkelte produkters krav i forhold til vedligeholdelse.

\section{Organisatoriske ændringer}
\textit{I følgende afsnit beskrives det nuværende behandlingsforløb for patienter med knæartrose. Dette sammenlignes med forløbet, hvor QST er implementeret, hvormed ændringer i arbejdsgange og kommunikation kan specificeres. Ligeledes undersøges det hvilke tiltag, der kræves ved implementering af en ny teknologi på en hospitalsafdeling, så det sikres, at teknologien bliver anvendt korrekt.}

\subsection{Nuværende behandlingsforløb for TKA-patienter}
For at kunne vurdere hvilke ændringer implementeringen af QST-protokollen på ortopædkirurgiske afdelinger vil medføre, er det nødvendigt først at kende det nuværende forløb for TKA-patienter. Et overblik over det nuværende patientforløb er illustreret i \figref{nuTKAforlob}, hvor det kan ses, at forløbet overordnet er inddelt i fire trin: Forundersøgelse, informationsdag, indlæggelse og et efterambulant forløb. 

\begin{figure}[H] 
	\begin{center}
		\includegraphics[width=0.7\textwidth]{figures/ORG/nuTKAforlob}
	\end{center}
	\caption{Figuren illustrerer det nuværende TKA-forløb for patienter med knæartrose, fra forundersøgelse til efterambulant forløb efter TKA-operationen. \citep{pritka2015}} 
	\label{nuTKAforlob} 
\end{figure} \vspace{-.25cm}
Som det fremgår af \figref{nuTKAforlob} starter hele forløbet i første trin kaldet forundersøgelse med en henvisning til en ortopædkirurgisk afdeling. Denne henvisning vil oftest komme fra egen læge. Patienten vil komme til en forundersøgelse ved en kliniker fra ortopædkirurgisk afdeling. Klinikeren konsulterer patienten, og de fastlægger i samarbejde en plan for behandlingen på baggrund af klinikerens vurdering af stadiet for knæartrosen. I denne forbindelse besluttes det, om patienten ønsker og er egnet til en TKA-operation. \citep{pritka2015}. \\
Under konsultationen samtaler klinikeren med patienten, mens det afficerede knæ palperes. Patienten fortæller klinikeren, hvordan og hvor på knæet der opleves smerter. Patient og kliniker gennemgår ligeledes røntgenbilleder for at vurdere stadiet af artrosen \citep{pritka2015}.
Blandt andet ud fra patientens røntgenbillede, reaktion på tryk og sin egen indstilling til forløbet (jævnfør klinisk udvælgelse \secref{kliniskudvaelgelse}), vurderer klinikeren, om patienten skal indstilles til TKA-operation. Patienter som indstilles til en TKA-operation præsenteres for protesen og teknikken ved operationen, så vedkommende er indforstået med indgrebet. Derefter informeres patienten om indlæggelsesforløbet. Patienter som ikke indstilles til TKA-operation, ledes videre til et andet forløb, hvilket ligger udenfor dette projekts fokusområde.
Efterfølgende får patienten praktisk information om indlæggelsen af personale fra ambulatoriet. Afsluttende aftales dato for informationsdag og operation i samråd med patienten. \citep{pritka2015} \\
Efterfølgende følger en informationsdag, hvor patienten modtager information om det samlede behandlingsforløb. Operationen sker på den planlagte dag, og forløber som beskrevet i \secref{kirurgiskbehandling} om kirurgisk behandling. Efter operationen er patienten indlagt i to dage inden udskrivelse. Patienten har opfølgning ved fysioterapeut to måneder efter indgrebet og en telefonisk opfølgning med en sygeplejerske et år efter indgrebet. \citep{pritka2015}
%På informationsdagen gennemgås indlæggelsesforløbet fra indlæggelse til udskrivning i detaljer, herunder træningsprogram, operationens forløb, anæstesi og opvågning. Patienten har ligeledes mulighed for at få besvaret egne spørgsmål til hele forløbet. 
%På operationsdagen modtages patienten om morgenen og instrueres i at påklæde sig operationstøj, komme på toilettet og muligvis tage bad, hvis dette ikke er gjort hjemmefra. Patienten præmedicineres med vanlig smertebehandling og smertestillende medicin ordineret af speciallægen. Patienten bedøves herefter med sedativa og får spinalnaæstesi. Patienten klargøres til operation ved at operatøren kontrollerer hvilket knæ, der skal opereres på ved at tjekke røntgenbilleder. Benet, som skal opereres, fastspændes i en sidestøtte, så benet er i korrekt position. En blodtomhedsmanchet spændes om låret over knæet som skal opereres. Operation og indsættelse af protese forløber som beskrevet i \secref{kirurgiskbehandling}. \citep{pritka2015} \\
%Patienten lægges til opvågning og derefter til sengeafdelingen, hvor personale kontrollerer patientens generelle tilstand. Resten af dagen for operationen observeres og vejledes patienten af personale. Når patienten har opnået følelse og bevægelse af det opererede ben, kan patienten mobiliseres med hjælp af et gangstativ, og dermed komme op at gå.\fxnote{Fast-track-surgury} Første dag efter operation gennemgår patienten fysioterapi og et træningsprogram to gange i løbet af døgnet. Resten af dagen går med hvile samt kontrol og dokumentation af patientens tilstand. På andendagen gennemgår patienten igen fysioterapi to gange i løbet af døgnet. Ved samtale vurderes, om patienten har opnået udskrivningskriterierne og udskrives hvis disse er opfyldt. \citep{pritka2015}
%Opfølgende kommer patienten to måneder efter operationsdagen til kontrol ved en fysioterapeut. Her vurderes patientens tilstand og selvtræningsprogram tilpasses herefter. Næste kontrol sker ét år efter operationsdagen ved telefonisk kontakt med en sygeplejerske fra ortopædkirurgisk ambulatorium. \citep{pritka2015}

\subsection{Behandlingsforløb ved implementering af QST-protokollen}
I forhold til implementering af QST-protokollen som supplement til klinikerens vurdering, vil det essentielle i patientforløbet være vurderingen, klinikeren laver i samarbejde med patienten, da denne vurdering er afgørende for, om patienten får en TKA-operation eller ej. Som det fremgår af introduktionen i \chapref{indledning} får 20~\% af patienterne kroniske smerter efter en TKA-operation. Det vurderes derfor, at QST-protokollen skal indgå som et led i forundersøgelsen, hvormed de 20@\% potentielt kan identificeres, inden de indstilles til en TKA-operation. På \figref{fig:QSTKAforlob} er det illustreret, hvor det vil være hensigtsmæssigt, at implementere QST-protokollen i forløbet. 

\begin{figure}[H]
\begin{center}
	\includegraphics[width=.8\textwidth]{figures/ORG/QSTTKAforlob}
\end{center}
	\caption{Figuren illustrerer på hvilket trin QST-protokollen skal implementeres i det nuværende TKA-forløb. Den tilføjede QST-undersøgelse er angivet med en grøn kasse.}
	\label{fig:QSTKAforlob}
\end{figure}\vspace{-.25cm}

Som det fremgår af \figref{fig:QSTKAforlob} indgår QST-protokollen ikke direkte sammen med forundersøgelsen med klinikeren, men et trin senere. Dette skyldes, at det ikke ville være tidsmæssigt eller økonomisk hensigtsmæssigt, at undersøge patienter med QST-protokollen, hvis klinikeren allerede ved forundersøgelsen kan ekskludere patienten fra en TKA-operation. Efter en QST-undersøgelse vil det kræve en klinikers vurdering af resultaterne og en sammenholdning med vurderingen fra samtale med patienten. QST-protokollen vil således kunne bidrage på samme niveau som røntgenbilleder gør på nuværende tidspunkt. Implementering af QST-protokollen vil dermed indføre et ekstra led i forløbet efter patientens forundersøgelse med klinikeren, hvor det er blevet besluttet, at patienten muligvis er egnet til en TKA-operationen. 

Ifølge informationer fra NociTech og Cephalon A/S, som henholdsvis er producent og leverandør af QST-udstyr, tager en QST-undersøgelse, med PPT, TSP og CPM, omkring 20 minutter at udføre. Denne undersøgelse vil foregå efter patientens samtale med klinikeren. Dermed vil der ske en forlængelse af undersøgelsestiden, der ligger forud for indstillingen og bookingen af en TKA-operation. Det vil senere undersøges nærmere hvilke færdigheder, det kræver for sundhedspersonalet at udføre undersøgelsen. Såfremt en sygeplejerske skal varetage undersøgelsen, vil klinikeren kunne starte forundersøgelse og samtale med næste patient imens sygeplejersken foretager QST-undersøgelse af den første patient. Det vil resultere i, at den ekstra tid den samlede forundersøgelse vil tage, er den tid det vil tage klinikeren at vurdere QST-resultaterne og sammenholde disse med vurderingen fra samtale med patienten. Da samtalen med patienten typisk tager omkring 15-20 minutter, \fxnote{observation fra Farsø} kan dette tilpasses med, at klinikeren kan videresende en patient til QST-undersøgelse og tage en ny patient ind til samtale. Når samtale med den anden patient er endt, vil QST-resultaterne for den første patient være klar og klinikeren ville kunne vurdere disse før konsultationen af tredje patient.

Implementeringen af QST-protokollen vil således skabe en ny arbejdsgang for klinikeren og dennes vurdering af patienter med hensyn til en ekstra vurdering af resultater. Det antages, at dette vil svare til vurdering af røntgenbilleder i forbindelse med konsultationen. Hvis QST-protokollen skal varetages af en sygeplejerske, vil dette skabe en ny kommunikationsvej ved overlevering af QST-resultaterne. Da denne overlevering antages at være på niveau med røntgenbilleder, anses dette ligeledes ikke som at have en større effekt på kommunikationen imellem sygeplejersken og klinikeren, da resultaterne udelukkende skal tilføjes til journalsystemet. Sygeplejerskens arbejdsgange vil påvirkes idet der tilføjes en ny arbejdsopgave til den nuværende arbejdsbyrde. Dette vil medføre en omallokering af sygeplejerskerne eller ansættelse af yderligere arbejdskraft.

\subsection{Korrekt anvendelse af QST-protokollen}
For at sikre, at en teknologi vil blive brugt og anvendt korrekt efter implementering er det vigtigt, at teknologien introduceres til personalet, som skal anvende denne. Hvis ikke personalet har forståelse for teknologien, dens anvendelse og resultater, er det muligt, at teknologien ikke vil blive anvendt korrekt. 

For at sikre korrekt anvendelse af QST-protokollen vil dette kræve introduktion til teknologien. Producenterne Medoc og NociTech anbefaler oplæring af personale til udførelse af QST-undersøgelser, ved køb af QST-udstyr. Dette sikrer, at personalet kan anvende udstyret korrekt og derved overholdes patientsikkerhed. Derudover sikres mere korrekt udførelse af undersøgelser, hvormed disse vil have større præcision. Ligeledes sikres det for producenterne, at deres udstyr anvendes korrekt og at disse fungerer efter hensigten.\\
Det er derfor væsentligt at få oplært personalet i korrekt anvendelse af QST-udstyr. Det vil i denne forbindelse være relevant at undersøge omfanget af oplæringen i forhold til brug og vedligeholdelse af udstyret. Dette er medvirkende til at sikre korrekt udførelse af QST-protokollen og korrekt vedligeholdelse, så udstyret fungerer.

\section{Oplæring og vedligeholdelse}
\textit{I følgende afsnit vil det på baggrund af forrige analyse-spørgsmål blive undersøgt i hvilken grad der kræves oplæring af personale for korrekt anvendelse af QST-udstyr. Herunder redegøres der for omfanget af vedligeholdelsen af QST-udstyret, så det kan vurderes, om personale på afdelingen kan varetage dette, eller om det vil kræve specialiseret personale.}

\subsection{Efteruddannelse}
For at personalet på de ortopædkirurgiske afdelinger kan udføre en QST-undersøgelse med udstyret, er det nødvendigt at gennemgå oplæring i anvendelsen heraf. Dette er ligeledes for at forbedre præcisionen af undersøgelsen som beskrevet i EFF-domænet, \chapref{EFF_chap}. Køb af trykalgometeret Algomed fra Medoc omfatter både algometeret og tilhørende software til udførelse og aflæsning af målinger \citep{AlgomedData}. Det oplyses af Medoc, at der medfølger cirka to timers undervisning i korrekt brug af udstyret og anvendelsen af den tilhørende software. Da Algomed er et håndholdt trykalgometer antages det, at undervisningen ligeledes omfatter korrekt placering og brug af dette. \\
%Det antages, at oplæringstiden for trykalgometeret fra Somedic ligeledes vil være omkring to timer, da dette system indbefatter måleudstyr lignende Algomed trykalgometeret samt software \citep{SomedicSenselab2016}. Trykalgometeret fra Somedic kan dog fungere uden tilkobling til computersystem, da resultatet kan aflæses direkte på displayet på algometeret \citep{SomedicSenselab2016}. For trykalgometeret fra Somedic antages det, at fokusområderne for undervisningen vil være tilsvarende undervisningen i Algomed fra Medoc, da begge algometre er håndholdte.
Det oplyses af NociTech, at der ved oplæring i brug af cuff-algometeret skal beregnes en halv dag per person. Fokusområdet for undervisningen vil således antages at være korrekt placering af manchetter og brugen af den tilhørende software.\\
Det kan forventes, at klinikeren som skal fortolke QST-resultater skal oplæres i at tolke disse. På baggrund af en eventuel udvikling af normative datasæt kræves det, at klinikeren forstår hvad de enkelte parametre i datasættet betyder, hvordan de bagvedliggende processer fungerer og hvordan patienter adskilles. 

\subsection{Vedligeholdelse af udstyr}
Det er af Cephalon A/S oplyst, at Algomed fra Medoc kræver udskiftning af det genopladelige 9V batteri, når dette ikke længere fungerer. Derudover kræves ingen vedligeholdelse af trykalgometeret.Det er fra NociTech oplyst, at manchetterne, der anvendes til QST-undersøgelse, kan anvendes til 200 målinger, hvorefter disse skal udskiftes.\\
Heraf vil der ved anvendelse af begge typer QST-udstyr være minimal vedligeholdelse, og vil hermed kunne varetages af personale på hospitalet.

\section{Delkonklusion}
Da implementeringen af QST-protokollen vil medføre et nyt trin i behandlingsforløbet for knæartrosepatienter, vil det lede til nye arbejdsopgaver og derved nye arbejdsgange for personalet. Det vurderes, at klinikeren, som udfører forundersøgelse og konsultation med patienter, vil få mere arbejde, da vedkommende skal samtale med en patient to gange i stedet for én. Dette sker, da de patienter, som klinikeren tidligere ville have indstillet direkte til en TKA-operation nu først skal have foretaget en QST-undersøgelse, inden den endelige vurdering kan laves. Resultatet fra den ekstra undersøgelse skal vurderes på samme niveau som røntgen, og der kræves derfor en yderligere samtale med klinikeren. \\
Dette betyder, at der ved implementering af QST-protokollen må forventes, at der kan konsulteres færre patienter om dagen end uden protokollen. Dette skal opvejes imod, at QST-protokollen potentielt kan sikre, at 20\% flere patienter vil få en korrekt behandling. Det foreslåede forløb vil ligeledes kræve, at eksempelvis en sygeplejerske skal ansættes til at foretage QST-undersøgelserne. Hvis der per dag eksempelvis skal foretages fire QST-undersøgelser, vil dette minimum beskæftige en sygeplejerske i 80 minutter. Denne arbejdstid er på baggrund af en ny arbejdsopgave, hvoraf det kan forventes, at en allerede ansat sygeplejerske ikke kan påtage den nytilkomne arbejdsbyrde. Der vil opstå en ny kommunikationsvej mellem undersøgeren og klinikeren, men det vurderes ikke som et problem, da det vil ske på niveau med kommunikationen mellem undersøger og kliniker ved røntgenundersøgelser. Det vurderes som værende hensigtsmæssigt at uddanne flere sygeplejersker til at foretage QST-undersøgelser. Dette er tilfældet, da det ikke er ønskværdigt, at udførelsen af QST-undersøgelser skal være afhængig af en enkelt sygeplejerske. Ydermere vil disse sygeplejersker, per person, kunne oplæres på cirka en halv arbejdsdag med undervisning fra leverandører. Det vurderes derfor, at det ikke vil kræve ansættelse af specialuddannet personale, men i stedet eventuel ansættelse af en ny sygeplejerske eller oplæring og allokering af nuværende personale. Vedligeholdelse af QST-udstyret vil ligeledes ikke kræve tilsyn fra teknikere eller andre specialister, da vedligeholdelsen udelukkende omfatter udskiftning af enkelte komponenter. \\
Det konkluderes herved, at implementering af QST-protokollen vil have en indvirkning på arbejdsgangen på ortopædkirurgiske afdelinger af samme omfang som eksempelvis en røntgenundersøgelse. Konsekvenserne er, at kvantiteten af gennemførte behandlingsforløb kan falde, men der vil samtidig ske en stigning i kvaliteten at udførte behandlinger.
