\section{Formål}
I organisationsanalysen undersøges det, hvordan implementering af QST-undersøgelserne på ortopædkirurgiske afdelinger vil påvirke personalets arbejdsgang og daglige opgaver. Implementering af en ny teknologi kan generelt medføre ændringer i intern kommunikation afdelinger og personale imellem, hvorved det skal undersøges, hvilke ændringer dette vil medføre. Dette indeholder et organisatorisk aspekt, da ændringer i kommunikation og arbejdsgange påvirker, hvordan sygehusledelse og personale skal prioritere tid og arbejdskraft. Dette kan medføre en ændring i antallet af daglige patientkonsultationer, ændring i arbejdsopgaver samt allokering af personale. \\
Ligeledes skal det undersøges, hvilke tiltag der bør tages forud for implementeringen af QST, da korrekt anvendelse af en ny teknologi kan være vanskeligt at sikre, hvis personalet ikke introduceres grundigt hertil. Det undersøges derfor, i hvilket omfang oplæring vil være nødvendigt for at sikre korrekt og sikker brug af QST-undersøgelserne. Ved implementering af QST vil der tilføjes en ny undersøgelsesmetode til afdelingen, som muligvis vil kræve regulær vedligeholdelse for at sikre, at udstyret kan anvendes korrekt og ikke udsætter patienter for sikkerhedsmæssige risici. Dermed er det væsentligt at undersøge behovet for vedligeholdelse for QST-systemerne.

\section{HTA spørgsmål}
Organisatoriske ændringer:
\begin{enumerate}
	\item \textit{Hvordan påvirker QST arbejdsgangene og intern kommunikation på ortopædkirurgisk afdeling?} %G0001, G0008
	\item \textit{Hvilke overvejelser skal der tages forbehold for, førend en teknologi bliver anvendt korrekt?} %G0010
\end{enumerate}
Oplæring og vedligeholdelse:
\begin{enumerate}[resume]
	\item \textit{Hvorvidt kræves der efteruddannelse af personalet i forbindelse med anvendelse af QST?} %G0003
	\item \textit{I hvilken grad er vedligeholdelse af QST-systemer påkrævet?} %B0012, B0013
\end{enumerate}


\section{Metode}
For ORG-domænet udvides litteratursøgningen til ikke udelukkende at søge materiale i videnskabelige databaser. Dette er nødvendigt, da analysen omhandler den nuværende organisationsstruktur samt behandlingsforløb for TKA. Viden herom er tilgængelig gennem RN’s hjemmeside, som har specifikt materiale om interne strukturer og Aalborg Universitetshospitals (AAUH) behandlingsforløb ved TKA. Der er desuden inddraget grå litteratur i form af en mailkorrespondance med producenter af QST-udstyr. Gruppen har ligeledes besøgt ortopædkirurgisk afdeling på Aalborg Universitetshospital Farsø, og observeret en patientkonsultation i et TKA-behandlingsforløb. \\
Analysen redegøre for HTA-spørgsmål (1) det nuværende behandlingsforløb for TKA og vurderer, hvor implementering af QST vil være mest hensigtsmæssig. Dette gøres på baggrund af en redegørelse for behandlingsforløbet, hvor det nuværende forløb sammenholdes med det forløbet, hvor QST er implementeret. Dermed er det muligt at undersøge forskelle i personalets arbejdsgange og opgaver. \\
Det undersøges for HTA-spørgsmål (2) hvilke forholdsregler, der ved implementering af QST kan sikre, at personalet opnår kendskab til korrekt anvendelse af teknologien.
Implementering af QST vil ligeledes kræve personale, som korrekt kan anvende teknologien samt en form for vedligeholdelse af udstyret. For HTA-spørgsmål (3) undersøges det derfor i hvilken grad implementering af QST vil kræve efteruddannelse af personale, så korrekt anvendelse kan sikres. Vedligeholdelse af udstyr antages ligeledes at have en potentiel effekt på arbejdsgangen. \\
Ud fra afsnit \chapref{EFF_chap} og for besvarelse af HTA-spørgsmål (4) er det væsentligt, at klinikpersonalet, der skal udføre QST-undersøgelser, modtager undervisning af producenterne af udstyret, for at sikre korrekt anvendelse heraf. På baggrund heraf tages kontakt til producenter af udstyr fra NociTech og Somedic. Der tages ligeledes kontakt til Cephalon A/S, som forhandler medicinsk udstyr fra Medoc. Dette gøres med henblik på indsamling af information om oplæring og varighed heraf samt de enkelte produkters krav i forhold til vedligeholdelse.


\section{Organisatoriske ændringer}
\textit{I følgende afsnit beskrives det nuværende behandlingsforløb for patienter med knæartrose. Dette sammenlignes med forløbet, hvor QST er implementeret, hvormed ændringer i arbejdsgange og kommunikation kan specificeres. Ligeledes undersøges det hvilke tiltag, der kræves ved implementering af en ny teknologi på en hospitalsafdeling, førend det sikres, at teknologien vil blive accepteret efter hensigten, så den vil blive modtaget og anvendt korrekt.}


\subsection{Nuværende behandlingsforløb for TKA-patienter}
For at kunne vurdere hvilke ændringer implementeringen af QST-undersøgelserne på ortopædkirurgiske afdelinger vil medføre, er det nødvendigt først at kende det nuværende forløb for TKA-patienter. Et overblik over det nuværende patientforløb er illustreret i \figref{nuTKAforlob}, hvor det kan ses, at forløbet overordnet er inddelt i fire trin: Forundersøgelse, informationsdag, indlæggelse og et efterambulant forløb. 


\begin{figure}[H] 
	\begin{center}
		\includegraphics[width=0.7\textwidth]{figures/ORG/nuTKAforlob}
	\end{center}
	\caption{Figuren illustrerer det nuværende TKA-forløb for patienter med knæartrose.} 
	\label{nuTKAforlob} 
\end{figure} \vspace{-.25cm}


Som det fremgår af \figref{nuTKAforlob} starter hele forløbet i området forundersøgelse med henvisning til en ortopædkirurgisk afdeling. Denne henvisning vil oftest komme fra egen læge. Patienten vil komme til en forundersøgelse ved en kirurg fra ortopædkirurgisk afdeling. Kirurgen konsulterer patienten, og de fastlægger i samarbejde en plan for behandlingen på baggrund af kirurgens vurdering af stadiet for knæartrosen. I denne forbindelse besluttes det, om patienten ønsker og er egnet til en TKA-operation. \\
Under konsultationen samtaler kirurgen med patienten, mens der føles og trykkes på det afficerede knæ. Patienten fortæller kirurgen, hvordan og hvor på knæet der opleves smerter. Patient og kirurg gennemgår ligeledes røntgenbilleder for at vurdere stadiet af artrosen \citep{pritka2015}.
Ud fra patientens reaktion på tryk, røntgenbilledet og patientens egen indstilling til forløbet, vurderer kirurgen, om patienten skal indstilles til TKA-operation. Patienter som indstillet til TKA-operation præsenteres for protesen og teknikken ved operationen, så vedkommende er indforstået med indgrebet. Derefter informeres patienten om indlæggelsesforløbet. Patienter som ikke indstilles til TKA-operation, udelukkes af TKA-behandlingsforløbet og ledes videre til et andet forløb, hvilket ligger udenfor denne rapports fokus.
Efterfølgende får patienten praktisk information om indlæggelsen af personale fra ambulatoriet. Afsluttende aftales dato for informationsdag og operation i samråd med patienten. \citep{pritka2015} \\
Efterfølgende følger en informationsdag, hvor patienten modtager information om det samlede behandlingsforløb. Operationen sker på den planlagte dag, og forløber som beskrevet i \chapref{problemanalysen}. Efter operationen er patienten indlagt i to dage inden udskrivelse. Patienten har opfølgning ved fysioterapeut to måneder og et år efter indgrebet. \citep{pritka2015} 


På informationsdagen gennemgås indlæggelsesforløbet fra indlæggelse til udskrivning i detaljer, herunder træningsprogram,  operationens forløb, anæstesi og opvågning. Patienten har ligeledes mulighed for at få besvaret egne spørgsmål til hele forløbet. 
På operationsdagen modtages patienten om morgenen og instrueres i at påklæde sig operationstøj, komme på toilettet og muligvis tage bad, hvis dette ikke er gjort hjemmefra. Patienten præmedicineres med vanlig smertebehandling og smertestillende medicin ordineret af speciallægen. Patienten bedøves herefter med sedative og indsover. Patienten klargøres til operation ved at operatøren kontrollerer hvilket knæ, der skal opereres på ved at tjekke røntgenbilleder. Benet, som skal opereres, fastspændes i en sidestøtte, så benet er i korrekt position. En blodtomhedsmanchet spændes om låret over knæet som skal opereres. Operation og indsættelse af protese forløber som beskrevet i problemanalysen jf. \chapref{problemanalysen}. \citep{pritka2015} \\
Patienten lægges til opvågning og derefter til sengeafdelingen, hvor personale kontrollerer patientens generelle tilstand. Resten af dagen for operationen observeres og vejledes patienten af personale. Første dag efter operation gennemgår patienten fysioterapi og et træningsprogram to gange i løbet af døgnet. Resten af dagen går med hvile samt kontrol og dokumentation af patientens tilstand. På andendagen gennemgår patienten igen fysioterapi to gange i løbet af døgnet. Ved samtale vurderes, om patienten har opnået udskrivningskriterier og denne udskrives og hjemsendes hvis disse er opfyldt. \citep{pritka2015}
Opfølgende kommer patienten to måneder efter operationsdagen til kontrol ved en fysioterapeut. Her vurderes patientens tilstand og selvtræningsprogram tilpasses herefter. Næste kontrol sker ét år efter operationsdagen ved telefonisk kontakt med en sygeplejerske fra ortopædkirurgisk ambulatorium. \citep{pritka2015}




\subsection{Behandlingsforløb ved implementering af QST}
I forhold til implementering af QST som supplement til klinikeren, vil det essentielle i patientforløbet være vurderingen, kirurgen laver i samarbejde med patienten, da denne vurdering er afgørende for, om patienten får en TKA-operation eller ej. Som det fremgår af \chapref{EFF_chap} bliver op imod 20\% af patienterne fejlvurderet og modtager TKA, hvor det viser sig ikke at have lindrende effekt på deres smerter. Det vurderes derfor, at QST skal indgå som et led i forundersøgelsen, hvormed de 20\% kan identificeres, inden de indstilles til en TKA-operation. På \figref{fig:QSTKAforlob} er det illustreret, hvor det vil være hensigtsmæssigt, at implementere QST i forløbet. 

\begin{figure}[H]
\begin{center}
	\includegraphics[width=.8\textwidth]{figures/ORG/QSTTKAforlob}
\end{center}
	\caption{TKA-forløb med QST som supplement}
	\label{fig:QSTKAforlob}
\end{figure}

Som det fremgår at \figref{fig:QSTKAforlob} indgår QST-undersøgelserne ikke direkte sammen med forundersøgelsen med kirurgen, men et trin senere. Dette skyldes, at det ikke ville være tidsmæssigt eller økonomisk hensigtsmæssigt, at undersøge patienter med QST, hvis kirurgen allerede ved forundersøgelsen kan ekskludere patienten fra en TKA-operation. Efter en QST-undersøgelse vil det kræve en kirurgs vurdering af resultaterne og en sammenholdning med vurderingen fra samtale med patienten. QST vil således kunne bidrage på samme niveau som røntgenbilleder gør på nuværende tidspunkt. Implementering af QST vil dermed indføre et ekstra led i forløbet efter patientens samtale med kirurgen, hvor det er blevet besluttet, at patienten kan gennemføre TKA-operationen. 

Ifølge informationer fra NociTech og Cephalon A/S, som henholdsvis er producent og leverandør af QST-udstyr, tager en QST-undersøgelse, med PPT, TSP og CPM, omkring 20 minutter at udføre. Denne undersøgelse vil foregå efter patientens samtale med kirurgen. Dermed vil der ske en forlængelse af undersøgelsestiden, der ligger forud for indstillingen og bookingen af en TKA-operation.  Det vil senere undersøges nærmere hvilke færdigheder, det kræver for sundhedspersonalet at udføre undersøgelsen. Såfremt en sygeplejerske skal varetage undersøgelsen, vil kirurgen kunne starte forundersøgelse og samtale med næste patient imens sygeplejersken foretager QST-undersøgelse af den første patient. Det vil resultere i, at den ekstra tid den samlede forundersøgelse vil tage, er den tid det vil tage kirurgen at vurdere QST-resultaterne og sammenholde disse med vurderingen fra samtale med patienten. Da samtalen med patienten typisk tager omkring 15 og 20 minutter, \fxnote{observation fra Farsø} kan dette tilpasses med, at kirurgen kan videresende en patient til QST-undersøgelse og tage ny patient ind til samtale. Når samtale med den anden patient er endt, vil QST-resultaterne for den første patient være klar og kirurgen ville kunne vurdere disse mellem konsultationen af anden og tredje patient.

Implementeringen af QST vil således skabe en ny arbejdsgang for lægen og dennes vurdering af patienter med hensyn til en ekstra vurdering af resultater. Det antages, at dette vil svare til vurdering af røntgenbilleder i forbindelse med konsultationen.
Hvis QST-undersøgelsen skal varetages af en sygeplejerske, vil dette skabe en ny kommunikationsvej ved overlevering af QST-resultaterne. Da denne overlevering antages at være på niveau med røntgenbilleder, anses dette ligeledes ikke som at have en større effekt på kommunikationen. 


\subsection{Korrekt anvendelse af QST}
Hvilke overvejelser skal der tages forbehold for, førend en teknologi bliver accepteret og implementeret? %G0010
For at sikre, at en teknologi vil blive brugt og anvendt korrekt efter implementering er det vigtigt at teknologien introduceres til personalet, som skal anvende denne. Hvis ikke personalet har forståelse for teknologien, dens anvendelse og resultater, er det muligt, at teknologien ikke vil blive anvendt korrekt eller i værste fald slet ikke blive brugt. Dette vil ved implementering af QST dog ikke være en relevant problematik, da QST vil implementeres som en obligatorisk del af behandlingsforløbet for TKA. Personalet kan således ikke vælge at undlade udførelse af undersøgelsen, hvormed det ikke vil være et problem, hvorvidt teknologien anvendes eller ej, men nærmere om anvendelsen sker korrekt. 

For at sikre korrekt anvendelse af QST vil dette kræve introduktion til teknologien. Vigtigheden heraf er en kendt faktor blandt mange producenter og leverandører af medicinsk udstyr i dag, herunder også producenter og leverandører af QST-udstyr. Producenterne Medoc og Nocitech anbefaler derfor oplæring af personale til udførelse af QST-undersøgelse, ved køb af QST-udstyr. Dette sikrer, at personalet kan anvende udstyret korrekt og derved overholdes patientsikkerhed. Derudover sikres mere korrekt udførelse af undersøgelser, hvormed disse vil have større præcision. Ligeledes sikres det for producenterne, at deres udstyr anvendes korrekt og de holder muligheden åben for senere at kunne levere udstyr til sundhedssektoren. \\
Det er derfor væsentligt at få oplært personalet i korrekt anvendelse af QST-udstyr. Det vil i denne forbindelse være relevant at undersøge omfanget af oplæringen i forhold til brug og vedligeholdelse af udstyret. Dette er medvirkende til at sikre korrekt udførelse af at QST-undersøgelser og korrekt vedligeholdelse så udstyret fungerer.
\section{Oplæring og vedligeholdelse}
\textit{I følgende afsnit vil det på baggrund af forrige HTA-spørgsmål blive undersøgt i hvilken grad der kræves oplæring af personale for korrekt anvendelse af QST-udstyr. Herunder redegøres der for omfanget af vedligeholdelsen af QST-udstyret, så det kan vurderes, om personale på afdelingen kan varetage dette, eller om det vil kræve specialiseret personale.}


\subsection{Efteruddannelse}
For at personalet på de ortopædkirurgiske afdelinger kan udføre en QST-undersøgelse med udstyret beskrevet i \chapref{TEC_chap}, er det nødvendigt at gennemgå oplæring i anvendelsen heraf. Dette er ligeledes for at sikre præcisionen af undersøgelsen som beskrevet i afsnit \chapref{EFF_chap}. Køb af trykalgometeret Algomed fra Medoc omfatter både algometeret og tilhørende software til udførelse og aflæsning af målinger \citep{AlgomedData}. Der medfølger cirka to timers undervisning i korrekt brug af udstyret og anvendelsen af den tilhørende software. Da Algomed er et håndholdt trykalgometer antages det, at undervisningen ligeledes omfatter korrekt placering og brug af dette. \\
Det antages, at oplæringstiden for trykalgometeret fra Somedic ligeledes vil være omkring to timer, da dette system indbefatter måleudstyr lignende Algomed trykalgometeret samt software \citep{SomedicSenselab2016}. Trykalgometeret fra Somedic kan dog fungere uden tilkobling til computersystem, da resultatet kan aflæses direkte på displayet på algometeret \citep{SomedicSenselab2016}. For trykalgometeret fra Somedic antages det, at fokusområderne for undervisningen vil være tilsvarende undervisningen i Algomed fra Medoc, da begge algometre er håndholdte.
For oplæring i brug af cuff-algometeret fra NociTech skal der beregnes en halv dag per person (jævnfør mailkorrespondance). Købet af udstyret fra NociTech vil omfatte manchetter med tilbehør samt software \citep{NociTech2016}. Fokusområdet for undervisningen vil således antages at være korrekt placering af manchetter og brugen af den tilhørende software.   

\subsection{Vedligeholdelse af udstyr}
Det er af Cephalon A/S oplyst, at Algomed fra Medoc kræver udskiftning af det genopladelige 9V batteri, når dette ikke længere fungerer. Derudover kræves ingen vedligeholdelse af trykalgometeret. Da trykalgometeret fra Somedic ligeledes er batteridrevet kan det antages, at vedligeholdelsen ligeledes vil omfatte udskiftning af disse. 
Det er fra NociTech oplyst, at manchetterne, der anvendes til QST-undersøgelse, kan anvendes til 200 målinger. Herefter skal disse udskiftes.

\section{Delkonklusion}
Da implementeringen af QST vil medføre et nyt trin i behandlingsforløbet for TKA-patienter, vil det lede til nye arbejdsopgaver og derved nye arbejdsgange for personalet. Det vurderes, at kirurgen, som udfører forundersøgelse og konsultation med patienter, vil få mere arbejde, da vedkommende skal samtale med en patient to gange i stedet for én. Dette sker, da de patienter, som kirurgen tidligere ville have indstillet direkte til TKA-operation nu først skal have foretaget en QST-undersøgelse, inden den endelige vurdering kan laves. Resultatet fra den ekstra undersøgelse skal vurderes på samme niveau som røntgen, og der kræves derfor en yderligere samtale med kirurgen. \\
Dette betyder, at der ved implementering af QST må forventes, at der kan konsulteres færre patienter om dagen end uden QST. Dette skal opvejes imod, at QST potentielt kan sikre at 20\% flere vil få en korrekt behandling. Det foreslåede forløb vil ligeledes kræve, at eksempelvis en sygeplejerske skal ansættes til at foretage QST-undersøgelserne. Dette vil skabe en ny kommunikationsvej mellem undersøgeren og kirurgen, men det vurderes ikke som et problem da det vil ske på niveau med kommunikationen mellem undersøger og kirurg ved røntgenundersøgelser. Det vurderes som værende hensigtsmæssigt at uddanne sygeplejersker til at foretage QST-undersøgelserne, da disse vil kunne lære det på cirka en halv arbejdsdag med undervisning fra producenter eller leverandører. Det vurderes derfor, at det ikke vil kræve ansættelse af specialuddannet personale, men i stedet en oplæring og dermed allokering af nuværende personale. Vedligeholdelse af QST-udstyret vil ligeledes ikke kræve tilsyn fra teknikere eller andre specialister, da vedligeholdelsen udelukkende omfatter udskiftning af enkelte komponenter. \\
Det konkluderes herved, at implementering af QST vil have en indvirkning på arbejdsgangen på ortopædkirurgiske afdelinger af samme omfang som eksempelvis en røntgenundersøgelse. Konsekvenserne er, at kvantiteten af gennemførte behandlingsforløb kan falde, men der vil samtidig ske en stigning i kvaliteten at udførte behandlinger. 
