\section{Omkostninger og økonomisk evaluering (ECO)}
\subsection{Formål}
I økonomidomænet undersøges det hvilke økonomiske konsekvenser der vil opstå ved implementeringen af QST. Ved implementering af QST forventes det, at der vil være nytilkomne omkostninger i forhold til indkøb, personel, drift og anlæg. Disse økonomiske omkostninger skal redegøres for, således disse kan danne grundlag for den finansielle beslutning om hvorvidt QST bør implementeres som supplement til klinikeren. \\
Herudover undersøges de økonomiske omkostninger ved QST i forhold til effekten. Udbyttet ved anvendelsen af QST sættes dermed op mod omkostningerne heraf. Denne sammenligning, bidrager til at danne grundlag for en vægtning af de økonomiske konsekvenser kontra effekt, ved implementering og brug af QST. \\ 
Ved implementering af  QST vil der ske økonomiske ændringer, og dermed opstår der en problematik omhandlende ressourceudnyttelse. Hvortil den økonomiske byrde tilhører, afhænger af hvorvidt implementeringen foregår på regionalt eller nationalt plan. For at kunne vurdere den egentlige budgetpåvirkning er det nødvendigt at kende omfanget af  omkostningerne relateret til implementeringen og brugen af QST. 
\subsection{HTA spørgsmål}
\textit{Teknologiens effekt ift omkostninger:}
\begin{itemize}
	\item Hvad indebærer omkostningerne kontra effekt ved indførelsen af QST? (Cost effectiveness) E0006 
\end{itemize}

\textit{Ressourceudnyttelse:}
\begin{itemize}
	\item Hvordan vil implementeringen af teknologien påvirke regionens budget? G0007, F0012
\end{itemize}
%\subsection{Metode}
%Vidensindsamlingen til besvarelse af det økonomiske domæne vil foregå ved at søge efter materiale igennem HTA/MTV agenturer, i relevante sundhedssystemer og sundhedsudbydere. Herunder databaser såsom Statistikbanken og Sundhedsdatastyrelsen. Hvilket HTA-spørgsmål der ønskes besvares er afhængig af, hvilke databaser der undersøges. Ved en analyse af ECO benyttes der tre typer af litteratur, review af publiceret økonomisk evidens, reviews af eksisterende økonomiske evalueringer og \textit{de Novo} økonomiske evalueringer. For at udvælge de relevante analyser til ECO må det tages højde for tre faktorer; Meningen med den økonomiske evaluering, tilgængeligheden af brugbar data samt guidelines for de enkelte ECO-analyser. For en evaluering af QST er det fundet relevant at benytte; (CEA),(CUA),(CBA) og (BIA) analyser\textbf{Dette bliver diskuteret på mødet}. Disse giver et billede af QST omkostninger målt i både monetære enheder og QALY, samt at redegøre for tilbagebetalingstiden for hele implementeringen og dennes indvirkning på hospitalets budget. Når resultater fra de forskellige analyser forelægges må det siden hen vurderes transferabiliteten mellem afdelinger/regioner med videre. Det vil ved en analyse af ECO domænet være nødvendigt at foretage antagelser og simplificeringer, såfremt der ikke foreligger præcise tal, der er nødvendige for den pågældende analyse. Antagelser lavet i forbindelse med analyserne vil blive udført på en sådan måde at de fremstår transparente for ikke at blive misvisende. 
%
%
%\fxnote{Det samlede antal af berørte patienter skal tages i betragtning, for at kunne lave et endeligt estimat af de økonomiske konsekvenser. }
