\textit{Her kommer et indledende kursiv afsnit.}
\section{Formål}
I økonomidomænet undersøges det hvilke økonomiske konsekvenser der vil opstå ved implementeringen af QST. Ved implementering af QST forventes det, at der vil være nytilkomne omkostninger i forhold til indkøb, personel, drift og anlæg. Disse økonomiske omkostninger skal redegøres for, således disse kan danne grundlag for den finansielle beslutning om hvorvidt QST bør implementeres som supplement til klinikerens beslutning. \\
Herudover undersøges de økonomiske omkostninger ved QST i forhold til effekten. Udbyttet ved anvendelsen af QST sættes dermed op mod omkostningerne heraf. Denne sammenligning, bidrager til at danne grundlag for en vægtning af de økonomiske konsekvenser kontra effekt, ved implementering og brug af QST. \\ 
Ved implementering af QST vil der ske økonomiske ændringer, og dermed opstår der en problematik omhandlende ressourceudnyttelse. Hvortil den økonomiske byrde tilhører, afhænger af hvorvidt implementeringen foregår på regionalt eller nationalt plan. For at kunne vurdere den egentlige budgetpåvirkning er det nødvendigt at kende omfanget af  omkostningerne relateret til implementeringen og brugen af QST. 
\section{HTA spørgsmål}
\textit{Teknologiens effekt i forhold til omkostninger:}
\begin{itemize}
	\item Hvad indebærer omkostningerne kontra effekt ved indførelsen af QST? %(Cost effectiveness) E0006 
\end{itemize}

\textit{Ressourceudnyttelse}
\begin{itemize}
	\item Hvordan vil implementeringen af teknologien påvirke regionens budget?% G0007, F0012
\end{itemize}

\section{Metode}
Vidensindsamlingen til besvarelse af det økonomiske domæne vil foregå ved at søge efter materiale igennem HTA/MTV agenturer, i relevante sundhedssystemer og sundhedsudbydere. Herunder databaser såsom Statistikbanken og Sundhedsdatastyrelsen. Hvilke databaser der undersøges er afhængig af, hvilke HTA-spørgsmål der ønskes besvaret. Ved en analyse af økonomiområdet benyttes tre typer af litteratur: review af publiceret økonomisk evidens, reviews af eksisterende økonomiske evalueringer og \textit{de Novo} økonomiske evalueringer. For at udvælge de relevante analyser til økonomiområdet må det tages højde for tre faktorer; Meningen med den økonomiske evaluering, tilgængeligheden af brugbar data samt guidelines for de enkelte økonomianalyser. For en evaluering af QST er det fundet relevant at benytte; Cost-Effectiveness analyse (CEA) samt Budget-Influence analyse (BIA). Disse danner grundlaf for omkostningerne af QST målt i monetære enheder. Ligeledes kan disse økonomianalyser redegøre for tilbagebetalingstiden for hele implementeringen og dennes indvirkning på hospitalets budget. Når resultater fra de forskellige analyser forelægges må det siden hen vurderes transferabiliteten mellem afdelinger/regioner. Det vil ved en analyse af økonomiområder være nødvendigt at foretage antagelser og simplificeringer, såfremt der ikke foreligger præcise tal, der er nødvendige for den pågældende analyse. Antagelser lavet i forbindelse med analyserne vil blive udført på en sådan måde at de fremstår transparente for ikke at virke misvisende. 

\section{Teknologiens effekt ift omkostninger}
\textit{Her skal der være et kursivt indledende afsnit} 

Til besvarelse af spørgsmålet omhandlende omkostningerne ved indførselen af teknologien, er en større mængde litteratur gennemgået, hvoraf den tilegnede viden ikke var tilstrækkelig. Dette kan være en indikation af den begrænsede benyttelse af QST vedrørende TKA, hvoraf der foreligger begrænsede økonomiske analyser. Der må derfor stilles yderligere uddybende spørgsmål for at danne grundlag for en estimeret udgift, ved implementering af QST. Til besvarelse heraf er producenter kontaktet, samt andet grå litteratur benyttet. Med dette forbehold må resultatet blive en estimeret omkostning. \\
Der opstilles derfor følgende uddybende spørgsmål som vil bliv stillet til producenter af QST udstyr:
\begin{itemize}  
\item Hvad er indkøbsprisen på QST-udstyr? 
\item Hvad kræver det at benytte QST-udstyr? 
\item Hvad er vedligeholdelses udgifter i forhold til QST-udstyr?
\end{itemize}
For at kunne lave et estimat for omkostningerne ved implementering af QST, er der blevet taget kontakt til 
henholdsvis Medoc, Somedic og Nocitech, der er producenter af QST-udstyr. I den forbindelse er der indhentet indkøbspriser, driftsomkostninger, samt bruger specifikationer i forhold til hvad det kræves at benytte udstyret. Med etablering- og brugsomkostninger tilegnet, er det sammenholdt med patient-flow, muligt at estimere meromkostningen forbundet med implementering af QST. 

\subsection{Omkostninger for implementering af Nocitech}\label{priser}
Ved Nocitech er der oplyst følgende priser, hvor fra der er lavet estimeringer i forhold til antallet af patienter der indstilles til TKA-operation. Priserne listet er for cuff-algometeret der måler stimulirespons ved følgende QST-parametre; PPT, PTT, TSP og CPM.

\begin{itemize}  
\item Cuff-algomerter: 125.000 DKK
\item 2 stk. cuff i  alt 1600 DKK (Levetid 200 målinger per cuff).
\item Undersøgelses tid, ca. 15-20 min.
\item Der skal påregnes en halv dag per person til oplæring.
\end{itemize}

Tal fra \citer{aarsrapport2016} viser at antallet af primære TKA-operationer i Region Nordjylland ligger i omegnen af 500-600 alloplastikker årligt. Da der anvendes to cuff's til QST-undersøgelse og hver cuff holder 200 målinger, og der tages udgangspunkt i 600 patienter årligt, vil det koste Region Nordjylland 4.800 DKK for indkøb af seks cuff's. Dette er foruden indkøbsprisen af apparaturet og ca. 58.500 DKK i lønomkostninger. Her regnes der med omkostninger til sygeplejerskelønninger på 292,21DKK pr time. \citep{DST1} \citep{DST2} Da der i Region Nordjylland laves TKA-operationer på flere afdelinger, kan implementeringsomkostningerne variere alt efter på hvor mange afdelinger QST skal implementeres på. Størstedelen af primære TKA-operationer udføres i henholdsvis Farsø og Frederikshavn, hvoraf de økonomiskeomkostninger er beregnet ud fra en implementering på disse to afdelinger. Den samlede udgift beløber sig på, 313.300DKK, hvilket inkluderer cuff-algometre, cuffs og lønninger på begge afdelinger. Dog er er der ikke påregnet uddannelse af operatører eller mertid til lægefagligvurdering af selve QST-resultatet.
%\textbf{Her kan der tilføjes yderligere beregninger såfremt der fremkommer flere tal. }

\subsection{Økonomisk påvirkning} 
Effekten af en implementering af QST som en screenings protokol skal opgøres i antallet af reducerede kroniske smerte patienter. I et præ-resultat af \citer{Blikman2016} antages det at en 10 ugers præoperativ behandling med Duloxetin vil kunne nedsætte postoperative smerter. Studier har vist at Duloxtin har haft effekt på kroniske sygdomme heriblandt artrose, hvori central sensibilisering spiller en central rolle \citep{Blikman2016}. Resultatet af en screening med QST vil være en umiddelbar indstilling til TKA, eller en indstilling til behandling med henblik på desensibilisering. Efter medicinsk behandling er antagelsen, at smertelindringen efter TKA-operation vil være signifikant forbedret end hvis patienten havde været foruden behandling med Duloxtin. \citep{Blikman2016} \\
Med QST vil der ikke forekomme en besparelse på operationen, men i stedet vil det medfører en merudgift i form af præoperativ medicinsk behandling. Såfremt desensibiliseringen viser sig effektiv, vil en økonomiske besparelse være i form af et reduceret antal kroniske postoperative smerte patienter.

\subsection{Omkostninger ved kroniske smertepatienter}
\textbf{Note til Pia: Forklaringen af hvorfor man undersøger hvad omkostningerne ved kroniske smertepatienter blive beskrevet i et andet domæne. Det virker her lidt udenfor kontekst, da forklaringen ikke er beskrevet. Grundlæggende er det fordi: Hvis man bliver diagnosticeret som disponeret for kroniske postoperative smerter, så bliver man behandlet medicinsk, for at sænke sin TS og PPT og CPM. Når disse værdier er tilpasset normalen vil man opererer og heraf vil der være en meromkostning ift medicin, men i den lange ende mister man en kronisk smertepatient, hvilket er godt da disse er meget dyre.}

Kroniske smerte patienter er en stor byrde for samfundet. Omkostninger indbefatter medicin, hospitalsydelser, yderligere operationer samt tabt arbejdsfortjeneste. I en undersøgelse fra USA estimeres patienter med reumatoid artrit til at koste mellem \$500 til \$35.400 med en gennemsnitligomkostning på 12.900 USD til 18.833 USD om året (1988-1997 USD). Operationskostningerne er ikke medregnet. \citep{Turk2002} I Danmark er det estimeret at der hvert år tabes en million arbejdsdage som et resultat af kronisk smerte. \citep{Eriksen2006} I 2003 til 2004 estimeres der i USA en økonomisk omkostning på \$7.1 milliarder, hvoraf 66\% af disse var tilskønnet 38\% af arbejderne med akut forværrelse af smerter. \citep{Phillips2009} %eksacerbationer = forværrelse af sygdom el. symptom

\subsection{Cost-effectiveness}
Da der i følge \chapref{EFF_chap} i litteraturen, ikke findes konkret evidens for sensitiviteten og specificiteten må der her laves antagelser for at kunne estimere cost-effekten af QST. For at stille et eksempel op antages to scenarier: \\
I første scenarie har QST en prædiktionsrate på $0,5$, hvilket vil svare til at kaste en mønt om man vil blive klassificeret som disponeret for kroniske postoperative smerter eller ej.
I det andet scenarie er prædiktionsraten perfekt på $1$, hvilket svare til at alle patienter bliver klassificeret korrekt, hvilket vil resulterer i den maksimale reduktion af kroniske postoperative smerte patienter. \\
Med udgangspunkt i tallene fra Region Nordjylland (RN) vil op mod 20\% af de 600 årligt TKA-opererede være kroniske smerte patienter \textbf{Referer til indledning/prævalence}. Ved første scenarie vil det resulterer i korrekt prædiktion af 60 kroniske postoperative smerte patienter, hvor 60 patienter som ville udvikle kroniske postoperative smerter ikke detekteres. I det andet senarie vil 120 patienter korrekt prædikteres som kroniske postoperative smerte patienter. \\
For at simplificere udregningen af omkostningerne, tages der i dette eksempel ikke højde for inflation og kursfluktueringer og beløbet konverteres derfor fra den oprindelige valuta til DKK med den gældende kurs den 18.11.2016.
På baggrund af tal fra USA beløber besparelsen sig for de to senarier på henholdsvis 90.537DKK og 132.177DKK per patient. I første senarie vil der samfundsmæssigt årligt kunne opnås en besparelse på mellem 5.4 og 7.9 millioner DKK. I andet senarie vil besparelsen være det dobbelte, således mellem 10.8 og 15.8 millioner DKK. Da tallene her er baseret på flere udenomsomkostninger, må der tages forbehold for at pengene ikke vil være en reel besparelse for RN direkte, men dermed en besparelse på landsplan.


%%GENTAGELSE Altså vil en investering i QST udstyr, såfremt de fremlagte forbehold er gældende, kunne betyde at for at identificere én patient som i risikogruppe for at få kroniske smerter post TKA med QST. Vil en anstået omkostning ligge imellem 2600 til 5200DKK afhængig af præcisionen. Hvilket vil fører til den samfundsmæssige besparelse på mellem 5.4 til 15.8 millioner kroner årligt.

\section{Ressourceudnyttelse}
\textit{Her kommer et indledende kursivt afsnit}

\subsection{Hvordan vil implementeringen af teknologien påvirke regionens budget?}
I RN er budgettet for sundhedssektoren 11 milliarder DKK, hvilket udgør 90\% af regionens samlede sundhedsbudget \citep{RnBudget17}. Der er ifølge \citer{RnBudget17} afsat yderligere 70 millioner DKK, som regionsrådet kan disponerer over, til nye initiativer og øvrige merudgifter. Herudover er der tilført 55 millioner DKK til nationalt at iværksætte nye initiativer. Der er således tilføjet yderligere 125 millioner DKK til i budgettet for 2017 \citep{RnBudget17}. \\
Såfremt regionen ønsker at implementerer QST på de relevante afdelinger vil den økonomiske konsekvens være afhængig af implementerings-, uddannelses- og brugsomkostningerne af det valgte udstyr, og implementeringens omfang vil påvirke regionens budget forskelligt. Overordnet vil den største udgift ikke ligge i implementeringen af udstyret, men derimod lønomkostninger til personalet. Foruden udgifter forbundet med udstyr vil der forekomme en øget udgift til den desensibiliserende medicinbehandling. Prisen på Duloxtin for behandlingsperioden på én uge med 30mg dagligt, syv uger med 60mg dagligt og nedtrapning to uger med 30mg dagligt, vil være mellem XXX og XXX DKK afhængigt af præcisionen \textbf{Indsæt priser fra Sten (og en kilde på dosseringen)}. Det vil derfor være op til RN, at bestemme om den øgede udgift vil være fordelagtig i forhold til den besparelse man på længere sigt vil kunne opnå ved en reduktion af kroniske smertepatienter.

\section{Delkonklusion}
Da tallene er baseret på data fra USA kan det diskuteres hvorvidt priserne kan overføres direkte til Danmark, da de medicinale og samfundsmæssige strukturer ikke kan overføres direkte. Der er på nuværende tidspunkt begrænset informationer, som ikke er ældre end 2004, specifikt omhandlende de økonomiske udgifter forbundet med kroniske knæsmerter for europæere. Såfremt nyt information bliver tilgængeligt bør der laves nye antagelser i forhold til udgifter og omkostninger. 
Da denne økonomiske analyse baseres på den endnu ikke beviset antagelse: at Duloxetin har en positiv virkning på et TKA behandlingsforløb, bør en endelig beslutning afvente resultaterne af studiet og være betinget af dette. Såfremt den ønskede virking opnås, vil der i regionen med en udgift per patient på 2600 til 5200DKK kunne opnås besparelser op imod 5.4 til 15.8 milioner DKK på lands basis. Det skal dog bemærkes at den virkelige besparelse formegentligt vil være mindre, da størstedelen af patientgruppen er på vej ud af arbejdsmarkedet \citep{Holmberg2015}. Derved vil den økonomiske besparelse for samfundet vil være mindre, dog stadig en mindre udgift end til smertelindrende medicin. 