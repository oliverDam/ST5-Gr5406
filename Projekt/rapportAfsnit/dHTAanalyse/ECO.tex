
\section{Formål}
I økonomidomænet undersøges det hvilke økonomiske konsekvenser der vil opstå ved implementeringen af QST. Ved implementering af QST forventes det, at der vil være nytilkomne omkostninger i forhold til indkøb, personel, drift og anlæg. Disse økonomiske omkostninger skal redegøres for, således disse kan danne grundlag for den finansielle beslutning om hvorvidt QST bør implementeres som supplement til klinikeren. \\
Herudover undersøges de økonomiske omkostninger ved QST i forhold til effekten. Udbyttet ved anvendelsen af QST sættes dermed op mod omkostningerne heraf. Denne sammenligning, bidrager til at danne grundlag for en vægtning af de økonomiske konsekvenser kontra effekt, ved implementering og brug af QST. \\ 
Ved implementering af  QST vil der ske økonomiske ændringer, og dermed opstår der en problematik omhandlende ressourceudnyttelse. Hvortil den økonomiske byrde tilhører, afhænger af hvorvidt implementeringen foregår på regionalt eller nationalt plan. For at kunne vurdere den egentlige budgetpåvirkning er det nødvendigt at kende omfanget af  omkostningerne relateret til implementeringen og brugen af QST. 
\section{HTA spørgsmål}
\textit{Teknologiens effekt ift omkostninger:}
\begin{itemize}
	\item Hvad indebærer omkostningerne kontra effekt ved indførelsen af QST? (Cost effectiveness) E0006 
\end{itemize}

\textit{Ressourceudnyttelse}
\begin{itemize}
	\item Hvordan vil implementeringen af teknologien påvirke regionens budget? G0007, F0012
\end{itemize}
\section{Metode}
Vidensindsamlingen til besvarelse af det økonomiske domæne vil foregå ved at søge efter materiale igennem HTA/MTV agenturer, i relevante sundhedssystemer og sundhedsudbydere. Herunder databaser såsom Statistikbanken og Sundhedsdatastyrelsen. Hvilket HTA-spørgsmål der ønskes besvares er afhængig af, hvilke databaser der undersøges. Ved en analyse af ECO benyttes der tre typer af litteratur, review af publiceret økonomisk evidens, reviews af eksisterende økonomiske evalueringer og \textit{de Novo} økonomiske evalueringer. For at udvælge de relevante analyser til ECO må det tages højde for tre faktorer; Meningen med den økonomiske evaluering, tilgængeligheden af brugbar data samt guidelines for de enkelte ECO-analyser. For en evaluering af QST er det fundet relevant at benytte; (CEA)og (BIA) analyser. Disse giver et billede af QST omkostninger målt i monetære enheder, samt at redegøre for tilbagebetalingstiden for hele implementeringen og dennes indvirkning på hospitalets budget. Når resultater fra de forskellige analyser forelægges må det siden hen vurderes transferabiliteten mellem afdelinger/regioner med videre. Det vil ved en analyse af ECO domænet være nødvendigt at foretage antagelser og simplificeringer, såfremt der ikke foreligger præcise tal, der er nødvendige for den pågældende analyse. Antagelser lavet i forbindelse med analyserne vil blive udført på en sådan måde at de fremstår transparente for ikke at virke misvisende. 

\section{Teknologiens effekt ift omkostninger}

\textit{Her skal der være et kursiv afsnit}

\subsection*{Hvad indebærer omkostningerne ved indførelsen af teknologien som supplement til klinikerens beslutning?}

Til besvarelse af spørgsmålet er en større mængde litteratur gennemgået, hvoraf resultatet af litteratursøgningen ikke var tilfredsstillende. Det fastslås at grundet den begrænsede benyttelse af QST foreligger der begrænsede økonomiske analyser der netop belyser den ønskede problemstilling. Der må derfor stilles yderligere uddybende spørgsmål for at danne grundlag for en estimeret udgift ved implementering af QST. Det forventes at svar på spørgsmålene må findes hos producenter, samt andet grå litteratur, med det forbehold at resultatet må blive en estimeret omkostning. 

\textit{Der opstilles derfor følgende uddybende spørgsmål som vil bliv stillet til producenter af QST udstyr. }

\subsubsection*{Uddybende spørgsmål}

\begin{itemize}  
\item Hvad er indkøbsprisen på QST udstyr? 
\item Hvad kræver det at benytte QST udstyr? 
\item Hvad er vedligeholdelses udgifter ifm QST udstyr?
\end{itemize}


For at få en idé om etableringsomkostningerne for QST udstyr er der blevet taget kontakt til hhv. \emph{Medoc, Somedic og Nocitech} der er leverandører af udstyr. I den forbindelse er der indhentet indkøbspriser, driftsomkostninger samt bruger specifikationer i forhold til hvad det kræves at benytte udstyret. Med etablering- og brugsomkostninger fastslået er det sammenholdt med patient flow muligt at estimere meromkostningen forbundet med implementeringen af QST som et supplement til klinikernes beslutning om endelig indstilling til TKA - behandling. 

\subsection{Produkt spefikke omkostninger}
\textit{I det følgende afsnit, listes og udregnes omkostninger forbundet med udgifterne for implementeringen af forskellige QST produkter. Priserne er stillet til rådighed fra producenter og leverandører af det pågældende udstyr.}
\subsubsection{Omkostninger for implementering af Nocitech}
\label{priser}

Fra Nocitech er der oplyst følgende priser, hvor fra der er lavet estimeringer i forhold til antallet af patienter der indstilles til TKA operation. Priserne listet er for Trykaglomeret der måler Stimulus-Response (Pain Detection Threshold + Pain Tolerance Threshold), TSP + CPM).

\begin{itemize}  
\item Listepris: 125.000 DKK
\item 2 stk. Cuff i  alt 1600DKK pr 200 Måling.
\item Undersøgelses tid, ca. 15-20 min.
\item Der skal påregnes en halv dag pr person til oplæring.
\end{itemize}

Tal fra Dansk Knæalloplastikregister Årsrapport 2016 viser at antallet af primære TKA operationer i region nordjylland ligger i omegnen af 500-600 alloplastikker årligt.\citep{aarsrapport2016} 

Med udgangspunkt i 600 patienter vil det koste Region Nordjylland: 4.800DKK for materialer for uden indkøbsprisen og ca. 58.500 i lønomkostninger, her regnes der med omkostninger til sygeplejerskelønninger på 292,21DKK pr time.\citep{DST1}\citep{DST2}

Da der i RN laves TKA-operationer på flere afdelinger, kan implementerings omkostningerne varierer alt efter på hvor mange afdelinger QST skal implementeres. 

Da største delen af primære TKA-operationer udføres i henholdsvis Farsø og Frederikshavn er de økonomiskeomkostninger beregnet ud fra en implementering på disse to afdelinger. Hvor den samlede udgift beløber sig på: 313.300 DKK, her er ikke påregnet uddannelse af operatører eller mertid til lægefagligvurdering.

\subsubsection{Omkostninger for implementering af Medoc}
Fra Cephalon, der er leverandør af Medoc udstyr i Danmark er oplyst priser på Algomed fra Medoc, der er et computer styret trykalgometer.

Algometeret kan benyttes både alenestående eller sammen med Medoc hovede software, som er inkluderet i prisen. Softwaren muluggøre realtids monetorering af trykket, hvorved et ensartet tryk kan opnås.

\begin{itemize}  
\item Listepris: 37.391,00 DKK ex. moms
%\item Undersøgelses tid, ca. 15 min.
%\item Der skal påregnes en halv dag pr person til oplæring.
\end{itemize}

Med udgangspunkt i samme forudsætninger som ved Cuff-algometeret fra Nocitech vil omkostningerne for undersøgelse af 600 patienter med løn og implementering på to afdelinger, Farsø og Frederikshavn, beløbe sig i ca. 133.300DKK her er ikke påregnet uddannelse af operatører eller mertid til lægefagligvurdering.

\subsection{Økonomisk påvirkning}

Effekten af en implementering af QST som en screenings protokol skal opgøres i antallet af reducerede kroniske smerte patienter. I et præ-result af \citer{Blikman2016} antages det at en 10 ugers præoperativ behandling med Duloxetin vil kunne nedsætte den post operative smerte, studier har vist at Duloxtin har haft effekt på kroniske sygdomme heriblandt artrose, hvor i central censebilicering spiller en central rolle. \citep{Blikman2016} Resultatet af en screening med QST vil være en umiddelbar indstilling til TKA, eller en indstilling til behandling med henblik på desensibilisering. Efter medicinsk behandling er antagelsen at smertelindringen efter TKA-operation vil være signifikant forbedret end hvis patienten havde været foruden behandling med Duloxtin.

Med QST værktøj vil der ikke forekomme en besparelse på operationen, men i stedet det vil medfører en merudgift i form af præoperativ medicinsk behandling. Såfremt decensebiliceringen viser sig effektiv, vil en økonomiske besparelse være i form af et reduceret antal kroniske smerte patienter.

\subsection{Kroniske Patienter}

Kroniske smerte patienter er en stor byrde for samfundet. Omkostningerne, indbefatter pharmaka, hospitalsydelser yderligere operationer samt tabt arbejdsfortjeneste. 

I en undersøgelse fra USA estimeres patienter med Reumatoid artrit til at koste mellem \$500 til \$35.400 med en gennemsnitligomkostning på \$12.900 til \$18.833  om året(1988-1997 USD), her er ikke medregnet operations omkostningerne.\citep{Turk2002}

I Danmark er det estimeret at der hvert år tabes 1 million arbejdsdage som et resultat af kronisk smerte.\citep{Eriksen2006} Mens der i 2003/4 i USA estimeres en økonomisk påvirkning på \$7.1 milliarder USD hvor af 66\% af disse var tilskønnet 38\% af arbejderne med smerte eksacerbationer. \citep{Phillips2009}

\subsection{Cost-effectiveness}

Da der \textbf{Her skal refference til effektivitets afsnittet} i litteraturen ikke findes konkrete evidens for sensitiviteten og specificiteten må der her laves antagelser for at kunne estimere cost-effekten. 

For at stille et eksempel op antages to senarier:

Første senarie har QST en prædiktionsrate på 0,5 dvs. det vil svare til at kastete en mønt om man vil blive klassificeret som disponeret for postoperative smerter eller ej.

Det andet senarie vil tage udgangspunkt i en perfekt prædiktionsrate på 1 altså her vil alle patienter blive klassificeret korrekt, hvilket vil resulterer i den maksimale  reduktion af kroniske smerte patienter.

Med udgangspunkt i tallene fra RN vil op mod 20\% af de 600 årligt TKA opererede være kroniske smerte patienter \textbf{Refferet til indledning/prævalence}.

Ved første senarie vil det resulterer i 60 smerte patienter mens det i andet senarie vil være 120 færre kroniske smerte patienter. 

For at simplificere udregningerne, tages der i dette eksempel ikke højde for inflation og kurs fluktueringer, beløbet konverteres der for fra den oprindelige valuta til DKK med veksel kursen for den 18.11.2016
I kroner og øre vil, det på baggrund af tal fra USA beløbe sig i en besparelse på henholds vis 90.537DKK til 132.177DKK pr patient. I første senarie vil der samfundsmæssigt årligt kunne opnås en besparelse på 5.4 til 7.9 milioner kroner. mens det ved andet senarie vil kunne spares det dobbelte altså mellem 10.8 og 15.8 millioner kroner. Da tallene her er baseret på flere udenoms omkostninger, må der her tages forbehold for at pengene ikke vil være en reel besparelse for RN direkte, men dermed en påvirkning på landsplan.

Altså vil en investering i QST udstyr, såfremt de fremlagte forbehold er gældende, kunne betyde at for at identificere én patient som i risikogruppe for at få kroniske smerter post TKA med QST. Vil en anstået omkostning ligge imellem 2600 til 5200DKK afhængig af præcisionen for udstyr fra Nocitech og 1110 til 2220 DKK for Medoc systemet. Hvilket vil fører til en samfundsmæssige besparelse på mellem 5.4 til 15.8 millioner kroner årligt.
\section{Ressourceudnyttelse}
\textit{Ved en implementering af en metode eller et produkt på en afdeling er det på bekostning af yderligere udgifter. I det følgende afsnit vil det vurderes hvorvidt en implementering af QST vil påvirke budgettet for sundhedsområdet i Region Nordjylland. }
\subsection{Hvordan vil implementeringen af teknologien påvirke regionens budget?}
I RN er budget for sundhedsområdet 11 miliarder kroner, hvilket udgører 90\% af regionens samlede økonomi. \citep{RnBudget17}  Der er i følge \citer{RnBudget17} afsat yderligere 70 milioner kroner som regionsrådet kan dispunerer over til nye initeritiver og øvrige merudgifter. Her ud over er der tilført 55 milioner DKK til nationalt iværksatte initeritiver. Altså i alt 125 milioner kroner som yderligere er tilføjet i budgetter 2017.

Såfremt regionen ønsker at implmenterer QST på de relevante afdelinger vil den økonomiske konsekvense være som følger. Afhænig af implimenterings-, uddandelses- og brugesomkostningerne \ref{priser} af det valgte udstyr vil det påvrike regionens budget forskelligt. Overordnet vil den største udgift ikke ligge i implementeringen af udstyret, men derimod lønomkostninger til personalet. Foruden udgifter forbundet med udstyr vild der forekomme en øget udgift til decensebilicerende behandling. Prisen på  Duloxetin for behandlings perioden på 1 uge med 30mg/daglig 7 uger med 60mg/dagligt og nedtrapning 2 uger 30mg/dagligt vil være ca. 150DKK for et kopi preberat pr 10 ugers behandling, altså 900DKK til 1800 DKK (gældende dagspris 21.11.2016)\citep{Medpriser2016} afhænigt af præsitionen.\textbf{Indsæt priser fra Sten, lægen bestemmer om der medicinen er tilskudsberettet.} Det vil derfor være op til RN at bestemme om den øgede udgift vil være fordelagtig i forhold til den besparelse man på sigt vil kunne opnå.


\section{Sammendragning}

Da tallene er baseret på data fra USA kan det diskuteres hvorvidt priserne kan overføres direkte til Danmark, da de medicinale og samfundsmæssige strukturer ikke kan overføres direkte. Der er på nuværende tidspunkt begrænset informationer som ikke er ældre end 2004 specifikt omhandlende de økonomiske udgifter forbundet med kroniskknæsmerter for Eupærer. Såfremt nyt information bliver tilgængeligt bør der laves nye antagelser i forhold til udgifter mv. 
Da denne økonomiske analyse baseres på en endnu ikke bevist antagelse, at Duloxetin har en positiv virkning på et TKA behandlingsforløb bør en endelig beslutning afvente resultaterne af studiet og være betinget af dette. Såfremt den ønskede virking opnås, vil der i region med en beskeden pr patient udgift på 2600-5200DKK(Nocitech) og 1110-2220DKK (Medoc) kunne opnå besparelser op imod 5.4 til 15.8 milioner kroner på lands basis, den virkelige besparelse vil formegentlig være mindre da \citer{Holmberg2015} er størstedelen af patientgruppen i slutningen/på vej ud af arbejdsmarkedet. Hvorved den økonomiske gevinst for samfundet vil være mindre, dog vil der stadig være en mindre udgift til smertelindrende medicin. 