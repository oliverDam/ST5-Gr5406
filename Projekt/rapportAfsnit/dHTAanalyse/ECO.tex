\section{Omkostninger og økonomisk evaluering (ECO)}
\subsection{Formål}
Formålet med ECO er at informere om værdi for pengene i relation til implementeringen af QST som et supplement til klinikerens beslutningsgrundlang til indstilling af patienter til en TKA-operation. Evidens fra SAF og EFF domænerne anvendes til at danne grundlag for analyserne. Teknologiens effekt skal vurderes i forhold til omkostningerne for at kunne danne grundlaget for, om QST skal implementeres eller ej. Hvis en implementering skal kunne finde sted, er det vigtigt at tage højde for hvor ressourcerne skal komme fra, herunder om yderligere ressourcer skal tilføjes, eller oml finansieringen skal ske gennem allokering af nuværende ressourcer. Det er vigtigt da stigende omkostninger til sundhedssektoren sætter større krav til informationer der præsenteres for beslutningstagerne.
\subsection{HTA spørgsmål}
\textit{Teknologiens effekt ift omkostninger:}
\begin{itemize}
	\item Hvad indebærer omkostningerne ved indførelsen af teknologien som supplement til klinikerens beslutning? (Cost-benefit) %E0001, E0002, E0009, 
	\item Hvad er effekten ved at implementere teknologien? (Cost–effektiveness=smerte+funktion) %E0006
\end{itemize}

\textit{Ressourceudnyttelse:}
\begin{itemize}
	\item Hvordan ændrer implementeringen af teknologien benyttelsen af anden teknologi og brugen af ressourcer?% D0023, G0006
	\item Hvordan vil implementeringen af teknologien påvirke budgettet krediteret til regionen? %G0007, F0012
\end{itemize}

\subsection{Metode \citep{HTAcore}}
Vidensindsamlingen til besvarelse af det økonomiske domæne vil foregå ved at søge efter materiale igennem HTA/MTV agenturer, i relevante sundhedssystemer og sundhedsudbydere. Herunder databaser såsom Statistikbanken og Sundhedsdatastyrelsen. Hvilket HTA-spørgsmål der ønskes besvares er afhængig af, hvilke databaser der undersøges. Ved en analyse af ECO benyttes der tre typer af litteratur, review af publiceret økonomisk evidens, reviews af eksisterende økonomiske evalueringer og \textit{de Novo} økonomiske evalueringer. For at udvælge de relevante analyser til ECO må det tages højde for tre faktorer; Meningen med den økonomiske evaluering, tilgængeligheden af brugbar data samt guidelines for de enkelte ECO-analyser. For en evaluering af QST er det fundet relevant at benytte; (CEA),(CUA),(CBA) og (BIA) analyser\textbf{Dette bliver diskuteret på mødet}. Disse giver et billede af QST omkostninger målt i både monetære enheder og QALY, samt at redegøre for tilbagebetalingstiden for hele implementeringen og dennes indvirkning på hospitalets budget. Når resultater fra de forskellige analyser forelægges må det siden hen vurderes transferabiliteten mellem afdelinger/regioner med videre. Det vil ved en analyse af ECO domænet være nødvendigt at foretage antagelser og simplificeringer, såfremt der ikke foreligger præcise tal, der er nødvendige for den pågældende analyse. Antagelser lavet i forbindelse med analyserne vil blive udført på en sådan måde at de fremstår transparente for ikke at blive misvisende. 


