\section{Formål}
I økonomidomænet undersøges det hvilke økonomiske konsekvenser der vil opstå ved implementeringen af QST. Ved implementering af QST forventes det, at der vil være omkostninger i forhold til indkøb, personel, drift og anlæg. Disse økonomiske omkostninger skal redegøres for, således disse kan danne grundlag for den finansielle beslutning om hvorvidt QST bør implementeres som supplement til klinikeren. Herudover undersøges de økonomiske omkostninger ved QST i forhold til effekten.\\ 
Ved implementering af  QST vil der ske økonomiske ændringer, og dermed opstår der en problematik omhandlende ressourceudnyttelse. For at kunne vurdere den egentlige budgetpåvirkning er det nødvendigt at kende omfanget af  omkostningerne relateret til implementeringen og drift af QST. 

\section{HTA spørgsmål}
Effekten af QST i forhold til omkostninger:
\begin{enumerate}
	\item \textit{Hvad indebærer omkostningerne kontra effekt ved implementeringen af QST?} %E0006 
\end{enumerate}
Ressourceudnyttelse:
\begin{enumerate}[resume]
	\item \textit{Hvordan vil implementeringen af QST påvirke regionens budget?} %G0007, F0012
\end{enumerate}

\section{Metode}
Ved en analyse af ECO benyttes der tre typer af litteratur, reviews af publiceret økonomisk evidens, reviews af eksisterende økonomiske evalueringer og de Novo økonomiske evalueringer. For at udvælge de relevante analyser til ECO må der tages højde for tre faktorer; Resultatet af den økonomiske evaluering, tilgængeligheden af brugbar data samt guidelines for de enkelte økonomianalyser. For en evaluering af QST er det fundet relevant at benytte cost-effectiveness analyse (CEA) og budget-influens analyse (BIA). Disse giver et billede af QST omkostninger målt i monetære enheder, samt dennes indvirkning på region Nordjyllands budget. Når resultater fra de forskellige analyser forelægges må det siden hen vurderes transferabiliteten mellem afdelinger/regioner med videre. Det vil ved en analyse af ECO-domænet være nødvendigt at foretage antagelser og simplificeringer, såfremt der ikke foreligger præcise tal, der er nødvendige for den pågældende økonomianalyse. Antagelser lavet i forbindelse med økonomianalyserne vil blive udført på en sådan måde at de fremstår transparente for ikke at virke misvisende. \\
Litteratursøgningen benyttet til at kunne besvare HTA-spørgsmål (1) tager udelukkende udgangspunkt i den generelt skitserede metode (jævnfør \secref{litteratursogning}). Der er til besvarelse af domænets emner blevet søgt i følgende databaser; Cochrane Libary, PubMed, Embase, Dansk National Research Database, Medline Google Scholar, Health Education Journal, Scandinavian Journal of Public Health, Nursing and Health Sciences og International Journal of Medical Informatics. Søgningen i disse databaser har ikke bidraget med nok viden til at kunne besvare HTA-spørgsmål (1) i ECO-domænet, hvoraf der er blevet indhentet viden hos producenter af QST-udstyr. \\
HTA-spørgmål (2) besvares på baggrund af viden omhandlende pris for implementering og brug af QST fundet ved  besvarelse af HTA-spørgsmål (1), samt litteratur omhandlende budgettet for Region Nordjylland i år 2017. 

\subsection{Tilegnelse af økonomisk viden om QST}
Til besvarelse af HTA-spørgsmål (1) er en mængde litteratur gennemgået, hvoraf resultatet af litteratursøgningen ikke var tilstrækkelig til at besvarer HTA-spørgsmålet (jævnfør \appref{Soegeprotokol}). Det fastslås at grundet den begrænsede benyttelse af QST foreligger der begrænsede økonomiske analyser der netop belyser den ønskede problemstilling. Der må derfor stilles yderligere uddybende spørgsmål for at danne grundlag for en estimeret omkostning ved implementering af QST. Det forventes at nødvendig viden kan findes hos producenter, samt andet grå litteratur, med det forbehold at resultatet må blive en estimeret omkostning.\\ 
Der opstilles følgende uddybende spørgsmål som vil blive stillet til producenter af QST-udstyr.

\begin{itemize}  
\item Hvad er indkøbsprisen på QST udstyr? 
\item Hvad kræver det at benytte QST udstyr? 
\item Hvad er vedligeholdelsesudgifter ved QST udstyr?
\end{itemize}

Der er blevet taget kontakt til henholdsvis Medoc, Somedic og Nocitech der er leverandører af QST-udstyr. I den forbindelse er der indhentet indkøbspriser, driftsomkostninger samt brugerspecifikationer i forhold til hvad det kræves at benytte udstyret. Med implementerings- og driftsomkostninger fastslået er det sammenholdt med patientflow muligt at estimere meromkostningen forbundet med implementeringen af QST som et supplement til klinikernes beslutning om endelig indstilling til en TKA-operation. 

\section{Teknologiens effekt i forhold til omkostninger} \label{priser}
\textit{I følgende afsnit undersøges teknologiens effekt i forhold til omkostningerne relateret hertil. Dette gøres for at skabe et kendskab til omkostningerne forbundet med implementering af QST, samt hvilken den økonomisk effekt der kan opnås ved implementering af QST. Der blev til følgende analyse stillet informationer tilrådighed af Nocitech og Cephalon A/S, leverandør af Medoc-udstyr i Danmark.}

\subsection{Omkostninger ved implementering af QST}
I det følgende afsnit, listes og udregnes omkostninger forbundet med udgifterne for implementeringen af forskellige QST-producenter. Priserne er stillet til rådighed af producenter og leverandører af det pågældende udstyr. Alle priser i følgende beregninger er opgivet i danske kroner (kr) med mindre andet er angivet. Der tages ikke forbehold for eventuelle kursændringer. 

Tal fra \citer{aarsrapport2016} viser at antallet af primære TKA-operationer i Region Nordjylland (RN) ligger i omegnen af 600 TKA-operationer årligt. \citep{aarsrapport2016} I RN laves TKA-operationer på afdelingerne Farsø og Frederikshavn. På baggrund af at der udføres TKA-operationer flere steder, varierer implementeringsomkostningerne alt efter hvor mange afdelinger QST skal implementeres på. \\
Den videre analyse er baseret på 600 patienter, hvilket reelt kan forventes at variere heraf. Ved implementering af QST kan det forventes at klinikeren vil henvise en patientgruppe til QST, som forinden implementeringen ikke ville have været henvist til en TKA-operation. Dette kan være patienter, som klinikeren vurderer som kroniske smertepatienter eller patienter der befinder sig på grænserne til at være egnet. 

\subsubsection{Nocitech}
Der er oplyst følgende priser for cuff-algometret der måler PPT, TSP og CPM.
\begin{itemize}  
\item Listepris af cuff-algometer: 125.000 kr. eksklusiv moms.
\item 2 manchetter: 1600 kr. per 200 måling.
\item Undersøgelsestid på cirka 15-20 minutter.
\item Der skal påregnes en halv dag per person til oplæring.
\end{itemize}

Med udgangspunkt i 600 patienter vil det koste RN 4.800 kr. for materialer. Da der i RN laves TKA-operationer på flere afdelinger, kan implementeringsomkostningerne varierer alt efter hvor mange afdelinger QST skal implementeres på. Da størstedelen af primære TKA-operationer udføres på henholdsvis hospitalerne i Farsø og Frederikshavn er de økonomiske omkostninger beregnet ud fra en implementering på disse to afdelinger. Udgifter til løn svarer til cirka 60.000 kr, her regnes der med omkostninger til sygeplejerskelønninger på $\approx 300$ kr. per time.\citep{DST1}\citep{DST2}. Den samlede udgift beløber sig således til: 314.800 kr., her er ikke påregnet uddannelse af operatører eller mertid til lægefaglig vurdering.

\subsubsection{Medoc}
Fra Cephalon, der er leverandør af Medoc-udstyr i Danmark er oplyst priser på Algomed fra Medoc, der er et computerstyret trykalgometer. Algomed kan anvendes til måling af PPT og CPM. \citep{AlgomedOnline} \citep{AlgomedData}
%Algometeret kan benyttes både alenestående eller sammen med Medoc hovedsoftware, som er inkluderet i prisen. Softwaren muliggør realtidsmonitorering af trykket, hvorved et ensartet tryk kan opnås.

\begin{itemize}  
\item Listepris: 37.391,00 kr. eksklusiv moms.
%\item Undersøgelsestid, cirka 15 minutter.
\item Der skal påregnes to timer per person til oplæring.
\item Driftsomkostninger i form af 9V batterier.
\end{itemize}

Med udgangspunkt i samme forudsætninger som ved cuff-algometeret fra Nocitech vil omkostningerne for undersøgelse af 600 patienter med løn og implementering på to afdelinger, Farsø og Frederikshavn, beløbe sig til cirka 133.300 kr. Her er ikke påregnet uddannelse af operatører eller mertid til lægefaglig vurdering. Der skal ved implementering af Algomed tages højde for at algometeret  ikke kan måle TSP, hvorved en meromkostning må forventes, i form af yderligere udstyr.

\subsection{Økonomisk effekt ved implementering af QST}
Den økonomiske påvirkning af en implementering af QST som et supplement til klinikerens vurderingsgrundlag, er en besparelse i form af antallet af TKA-operationer hvor patienten ender med at få kroniske postoperative smerter. Hvis patienten har positive QST-resultater og dermed vil være i risikogruppen for at få kroniske postoperative smerter efter en operation, vil operation af disse patienter være en fejlbehandling. Afhængigt af TKA-operationens omkostninger og QST-undersøgelsens evne til at finde patienterne i risikogruppe vil der være en besparelse i form at et mindre antal TKA-operationer. Ifølge Sundhedsstyrelsens DRG-takster for 2016 er omkostningerne forbundet med en TKA-operation i omegnen af 72.000 kr. for primære operationer. \citep{Takst2016}

\subsubsection{Cost-effectiveness}
Da der i litteraturen ikke findes konkret evidens for sensitiviteten og specificiteten må der her laves antagelser for at kunne estimere cost-effectiveness (jævnfør \chapref{EFF_chap}). På baggrund af \chapref{EFF_chap}, opstilles to scenarier for teknologiens effekt, det dårligste og den bedste. Ved det dårligste scenarie har QST en præcision på 0,5. Dette vil svare til at halvdelen af patienterne bliver korrekt klassificeret. Det bedste scenarie vil tage udgangspunkt i en perfekt præcision på 1 hvor alle patienter bliver klassificeret korrekt, hvilket vil resultere i at ingen patienter fejlklassificeres.  

Med udgangspunkt i tallene fra RN vil op mod 20\% af de 600 årligt TKA-opererede være kroniske smertepatienter (jævnfør \chapref{introduktion}). Ved det dårligste scenarie vil det resultere i 60 færre smertepatienter mens der i det bedste scenarie vil være 120 færre fejldiagnosticerede patienter. Det vil udmønte sig i en besparelse for RN på mellem 4.3 og 8.6 millioner kroner om året. Dermed vil en implementering af QST-udstyr, såfremt de fremlagte forbehold er gældende, kunne betyde at en omkostning for en QST-undersøgelse med udstyr fra Nocitech vil ligge mellem 2.600 og 5.200 kr. Med udstyr fra Medoc vil omkostningerne ligge mellem 1.110 og 2.220 kr. Dette er gældende for indkøbsåret. Efterfølgende vil udgifter udelukkende omhandle drift af udstyret. For Nocitech vil udgiften være indkøb af nye manchetter for hver 200 patienter. For Medoc består driftsudgiften i indkøb af 9V batterier. 

\section{Ressourceudnyttelse}
\textit{I følgende afsnit undersøges hvordan implementering af QST på en afdeling sker på bekostning af yderligere udgifter. I det følgende afsnit vil det vurderes hvordan en implementering af QST vil påvirke budgettet for sundhedsområdet i Region Nordjylland.}

\subsection{Påvirkning af Region Nordjyllands budget}
I RN er budgettet for sundhedsområdet 11 milliarder kr., hvilket udgører 90\% af regionens samlede økonomi. \citep{RnBudget17}  Der er ifølge \citer{RnBudget17} afsat yderligere 70 millioner kr. som regionsrådet kan disponerer over til nye initiativer og øvrige merudgifter. Herudover har RN fået 55 millioner kr. til nationalt, at iværksatte initiativer. Altså i alt 125 millioner kr. som yderligere er tilføjet i budgetter 2017.

Såfremt regionen ønsker at implementerer QST på de relevante afdelinger vil den økonomiske konsekvens være afhængig af implementerings-, uddannelses- og brugsomkostninger ved det valgte udstyr (jævnfør \secref{priser}). Overordnet vil den største udgift ikke ligge i implementeringen af udstyret, men derimod lønomkostninger til personalet som benytter QST. Det vil derfor være op til RN at bestemme om den øgede udgift vil være fordelagtig i forhold til den besparelse og forøget kvalitet, der på sigt vil kunne opnås. 

I Danmark er det estimeret at der hvert år tabes 1 million arbejdsdage som et resultat af kronisk smerte. \citep{Eriksen2006} Ved at implementering af QST med en høj præcision kan det forventes at nogle patienter vil genvinde deres mobilitet og heraf kunne genoptage deres arbejde. Patientgruppen som stadig befinder sig på arbejdsmarkedet er i den samlede patientgruppe en minoritet (jævnfør \chapref{introduktion}). Ved at denne andel kan genoprette deres arbejde vil det kunne bidrage til regionens budget i form af skatter.

\subsection{Transferabilitet}
Der vil ved implementering af QST i andre regioner skulle tages udgangspunkt i antallet af afdelinger hvor primære TKA-operationer udføres, samt antallet af patienter. Der vil i den forbindelse skulle laves nye udregninger for at estimere udgiften for den pågældende afdeling. Udgiften til den enkelte patient afhænger af antallet af patienter og mængden af det nødvendige QST-udstyr. Da en undersøgelse tager 15-20 minutter bør der ligeledes overvejes om en QST-enhed per afdeling er tilstrækkeligt. QST-udstyr vil, såfremt budgettet tillader det i den pågældende region, godt kunne implementeres på et nationalt plan.


\section{Delkonklusion}
Ved en implementering af QST vil dette bidrage til en meromkostning ved alle undersøgelser relateret til beslutningen vedrørende indstilling til en TKA-operation. Dette er tilfældet da der i første købsår vil komme en udgift per patient og per QST-undersøgelse på henholdsvis 2.600 til 5.200 ved anvendelse af Nocitech udstyr, og 1.110 til 2.220 ved anvendelse af Medoc udstyr. Disse omkostninger varierer som et resultat af at præcisionen af QST, endnu ikke er fastsat, heraf er omkostninger et estimat og vil sandsynligvis befinde sig i dette område. Afhængig af præcisionen for QST vil den potentielle besparelserne, foruden føromtalte brugsomkostninger, udelukkende bestå af operationsomkostninger. Dette vil bidrage henholdsvis for det dårligste og bedste scenarie, til en besparelse på 4.3 til 8.6 millioner kr. Gevinsten i form af returnering til arbejdsmarkedet, såfremt patienten har været sygemeldt, vil formentlig være af mindre karakter, da størstedelen af patientgruppen enten er på vej ud af arbejdsmarkedet eller pensioneret. Det er således på denne baggrund ikke muligt at vurdere den økonomisk besparelse for implementeringen af QST-undersøgelserne, som følge af genvundet arbejdskraft, men en besparelse vil være at finde i et reduceret antal primære TKA-operationer. Implementeringen af QST vil konsekvent betyde en meromkostning i den præoperative fase, men skabe en besparelse på antallet af udførte TKA-operationer. Den økonomiske konsekvens ved implementering af QST, vil umiddelbart heraf bestå af en besparelse af et beløb i omegnen af 4,3 til 8,6 millioner kr. En direkte overføring af omkostninger til andre regioner er ikke muligt, da antagelserne tager udgangspunkt i lokale forhold, antal behandlings steder og patient antal.

Ved implementering af QST vil budgettet i RN påvirkes ved en meromkostning bestående af, implementerings-, uddannelses- og brugsomkostninger. Den største budgetmæssige påvirkning vil ikke befinde sig på selve erhvervelse af QST-udstyret, men de vedvarende driftsomkostningerne. Meromkostningen til vurderingsprocessen vil være i omegnen af 1.110 til 5.200 kr. pr patient, afhængig af udstyr. Heraf skal RN beslutte hvorvidt besparelsen og den forøgede undersøgelseskvalitet, der på sigt vil kunne opnås ved implementering af QST, er stor nok kontra omkostningerne. 