\subsection{Undersøgelse af PPT, TS og CPM}

Opbygning af afsnit:
Det er fundet at central sensibilisering har betydning for udviklingen af kroniske postoperative smerter – dette kan testes ved de tre parametre PPT, TS og CPM. Hver af de tre parametre har forbindelse til central sensibilisering på følgende måde. 
Vigtig pointe at der er ingen evidens er for at parametrene virker uafhængigt af hinanden – de skal testes samtidig. (Alle tre/begge parametre skal være til stede) 
Hver af parametrene kan testes ved anvendelse af forskellig slags stimuli, men mekanisk stimuli er den mest almindelig. Tests af parametrene med mekanisk stimuli kan f.eks. ske ved NociTech maskinen hvorved disse udføres således. 
   
Studier antyder, at central sensibilisering har betydning for udviklingen af kroniske smerter efter en TKA-operation \citep{Suokas2012}. Central sensibilisering opstår som følge af de degenererende forandringer, der sker i et led på grund af knæartose. Ved knæartrose kan der, i området omkring det påvirkede led, opleves hyperalgesi, som er en forstærket reaktion på smerte. Dette betegnes også som perifer sensibilisering. \citep{Arendt-Nielsen2015} Efter længere tids udsætning for smertefuld stimuli kan denne hyperalgesi, på grund af øget aktivitet i nervefibrene i centralnervesystemet, sprede sig til andre dele af kroppen, hvormed central sensibilisering opstår. Hermed opleves hyperalgesi på steder, der ikke er forbundet med det påvirkede led, eksempelvis på armen eller det modsatte ben. Det er af flere studier blevet antydet, at en del af patienterne med knæartrose ligeledes har central sensibilisering. Dette kan være med til at forklare den ringe sammenhæng mellem smerter og radiologiske fund, som ses ved nogle knæartrosepatienter. \citep{Leary2016} Efter en succesfuld og smertefri TKA operation forbedres den centrale sensibilisering i nogle tilfælde. Ved patienter med kroniske postoperative smerter forbedres den centrale sensibilisering ikke i samme grad som for andre patienter. \citep{Arendt-Nielsen2015} \\
Central sensibilisering kan undersøges ud fra forskellige QST-parametre. Parametrene er tryk-smerte tærsklen (PPT), temporal summation (TS) og betinget smertemodulation (CPM). Disse parametre er blevet vist til at kunne skelne mellem rakse personer og personer med knæartrose. \citep{Arendt-Nielsen2015} Flere studier har ligeledes undersøgt om en eller flere af parametrene kan være med til at fænotype-bestemme knæartrosepatienter, således de patienter med central sensibilisering kan identificeres. \citep{Leary2016} \citep{Wylde2015b} Kun mindre studier har fundet en signifikant sammenhæng mellem en af de tre preoperative QST-scores og kronisk postoperative smerter, mens større studier ingen sammenhæng har fundet. \citep{Leary2016} \citer{Petersen2016} har undersøgt de tre parametre, og fandt ligesom tidligere studier ingen signifikant sammenhæng mellem patienter, som havde anormal score på én af de undersøgte parametre og dårligere postoperativ smertelindring. Herimod fandt \citer{Petersen2016}, at patienter, som havde en anormal score for både TS og CPM, havde mindre smertelindring end andre patienter. Dette antyder QST-parametrene kan anvendes til identificering af patienter med forhøjet risiko for udvikling af kroniske postoperative smerter. 

Hver af de tre QST-parametre kan testes ved forkellige typer stimuli eksempelvis mekanisk, kemisk eller termisk. Oftest anvendes mekanisk stimuli i form af tryk. \citep{Suokas2012} Til tilførelsen af tryk kan anvendes udstyr fra eksempelvis Somedic eller NociTech. \citep{Wylde2015} \citep{Petersen2016} \\
PPT kan testes ved tilførelse af tryk på et område omkring det påvirkede knæ. Det påførte tryk stiger indtil patienten begynder at opfatte trykket som smertefuldt, og angiver dette ved eksempelvis tryk på en knap. PPT defineres som det påførte tryk da patienten angav trykket blev smertefuldt, og angives dermed i kPa. Oftest gentages målingen tre gange, hvorefter gennemsnittet af de tre målinger anvendes som patientens videres PPT-værdi. \citep{Petersen2015b} \citep{Wylde2015}      

      
    