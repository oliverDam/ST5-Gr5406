\section{Formål}
I dette domæne undersøges og analyseres det hvilke konsekvenser en implementering af QST-protokollen vil have for patienten. Hvis QST-protokollen implementeres som et supplement til klinikeren i forbindelse med vurdering af, hvorvidt der skal foretages en TKA-operation, er det væsentligt at vide, om alle patienter kan undersøges med QST-protokollen, eller om der findes faktorer, som vil nødvendiggøre eksklusion af patienter fra undersøgelsesmetoden. Dette er relevant at undersøge, da dette potentielt vil medføre, at nogle patienter ikke kan vurderes på samme grundlag som andre. Derudover er det relevant at undersøge, hvordan patienternes behandlingsforløb vil påvirkes ved implementering af QST-protokollen. Patientens videre behandlingsforløb kan potentielt være anderledes end det behandlingsforløb der ville igangsættes uden QST-undersøgelsen. Derfor er det væsentligt at undersøge, om det vil have konsekvenser for patienten, at implementere QST-protokollen.
 
\section{Analyse-spørgsmål}
Påvirkning af QST på patienten: 
\begin{enumerate}
\item \textit{Er der faktorer vedrørende QST-protokollen, som kan ekskludere en gruppe patienter fra at benytte denne?} %(Fedme~cuff/hjerte/pacemaker?) H0012
\item \textit{Hvilken betydning vil implementering af QST-protokollen have for patienternes behandlingsforløb?} %H0006
\end{enumerate}

\section{Metode}
Til besvarelse af dette domæne tager litteratursøgningen udgangspunkt i den generelle metode (jævnfør \secref{litteratursogning}). I litteratursøgningen er der fokuseret på anvendelse af litteratur omhandlende QST-protokollens påvirkning på knæartrosepatienter. Det er i søgningen valgt ikke at ekskludere andre sygdomme, da det antages, at mængden af litteratur, der behandler det område, der ønskes undersøgt, kan være begrænset. Dette antages, da protokollen stadig hovedsageligt anvendes på forskningsniveau i forhold til knæartrose. Søgeprotokollen for SOC-domænet kan findes i \appref{SOC_sog}. \\
Der er gennem litteratursøgningen ikke fundet studier, der konkret undersøger analyse-spørgsmålene. Derfor anvendes informationer fra TEC-, SAF- og ORG-domænerne, da det vurderes, at disse domæner vil bidrage med information om hvordan QST-undersøgelsen vil have indflydelse på, hvilke patienter der vil kunne gennemgå undersøgelsen, og hvilken påvirkning undersøgelsen vil have på disse. I TEC- og SAF-domænerne undersøges henholdsvis teknologiens funktion og hvilke sikkerhedsmæssige aspekter, der skal tages højde for. Dette kan være medvirkende til at præcisere hvilke patienter, der kan gennemgå undersøgelsen eller om der forekommer faktorer, der ekskluderer visse grupper, hvilket bidrager til besvarelse af analyse-spørgsmål (1). I ORG-domænet undersøges det, hvordan arbejdsgange og det generelle forløb for udredning af knæartrose ændres. Det antages, at dette domæne kan medvirke til at undersøge, hvordan forløbet ligeledes vil ændres fra et patientmæssigt perspektiv, hvilket bidrager til besvarelse af analyse-spørgsmål (2). 
 
\section{Påvirkning af QST på patienten}
\textit{I dette afsnit undersøges det, hvorvidt der ved anvendelse af QST-protokollen er faktorer, som kan bidrage til, at patienter ekskluderes. Ligeledes undersøges det, hvilke konsekvenser det vil have for de patienter, der kan undersøges med QST-protokollen, at QST implementeres som en del af den nuværende undersøgelsesproces forud for TKA-operationer.}

\subsection{Eksklusion af patienter}
Det er i TEC-domænet fundet, at QST-undersøgelser med trykstimuli eksempelvis kan udføres med et håndholdt trykalgometer fra Medoc eller et cuff-algometer fra NociTech. 
De sikkerhedsmæssige problemstillinger, der kan opstå ved påførelse af mekanisk tryk, ved anvendelse af ovennævnte udstyr, ikke væsentlige, da disse undersøgelser ligger indenfor grænserne for skadespåførelse både i forhold til styrke og varighed jævnfør SAF-domænet i \chapref{SAF_chap}. Hermed kan det antages, at der ikke forekommer sikkerhedsmæssige faktorer, der gør det nødvendigt at ekskludere patienter fra QST-protokollen, hvormed alle patienter bør kunne undersøges med teknologien. Det kan imidlertid overvejes om der forekommer begrænsninger i forhold til hvilket QST-udstyr, der benyttes; manchetten fra NociTech kan have begrænsninger i forhold til patienter med arme eller ben, hvis omkreds ligger over eller under intervallet, som manchettens omkreds kan indstilles til. Denne problematik kan være relevant hvis patienten enten er over- eller undervægtig. En anden faktor som kan have betydning for patienters mulighed for at blive undersøgt med QST-protokollen er patienters mentale tilstand. Ved QST-protokollen tages der ikke højde for patienternes intelligens, mentale tilstedeværelse eller bias i retning mod et bestemt testresultat \citep{Dyck1998}. 
Hvis patienten ikke er i stand til at følge instruktionerne, der kræves for udførelsen af QST-undersøgelsen, antages det, at grundlaget for at vurdere patientens egnethed til en TKA-operation svækkes, da resultatet kan være misvisende for patientens egentlige tilstand. Hermed vil det hyppigste eksklusionsgrundlag antageligvis være faktorer forbundet med patienten.

\subsection{Betydning af QST-protokollen for patienter}
Ved en implementering af QST-protokollen som supplement til klinikerens beslutning vil der være to mulige udfald; patienten kan være eller ikke være disponeret for udviklingen af kroniske postoperative smerter. Dette er gældende, hvis det antages, at QST-protokollen har en høj præcision. I et behandlingsforløb vil begge resultater være gavnlige for patienten. Ved at anvende QST-protokollen vil det være muligt at vælge den korrekte behandlingsstrategi. Hvis patienten, på baggrund af klinikerens samlede vurdering, er disponeret for at få kroniske smerter efter en TKA-operation fravælges operationen. Dette er ensbetydende med, at patienten, foruden generelle risici forbundet med operationen, også vil undgå en behandling, som kan medføre, at denne udvikler kroniske postoperative smerter. Her vil patientens behandlingsforløb ifølge \citer{brostrom2012} skulle genvurderes, hvor et alternativ er smertelindrende behandling med henblik på senere konsultation. I \appref{Borgerforloeb} ses en figur der illustrerer forløbet yderligere. Hvis ingen af disse løsninger er en mulighed, må patienten accepterer situationen. Dette sætter et fokus på vigtigheden af videre behandlingsalternativer, hvilket vil blive belyst i perspektiveringen jævnfør \secref{perspektivering}. Hvis resultatet af QST-undersøgelsen derimod indikerer, at patienten ikke vil være disponeret for at få kroniske postoperative smerter kan patienten, såfremt klinikerens øvrige undersøgelser understøtter det, modtage en TKA-operation. \\
Implementeringen af QST vil både for klinikeren og patienten forbedre beslutningsgrundlaget i forhold til valg af det videre behandlingsforløb. Således vil både sand negative og sand positive QST-resultater være hensigtsmæssige.


\section{Delkonklusion}
På baggrund af, at der ikke foreligger sikkerhedsmæssige problematikker ved benyttelse af QST-protokollen kan det antages, at patientens mentale nedsatte kapacitet kan forhindre visse patienter i at få QST-undersøgelsen. Der kan ligeledes være faktorer, som betyder, at nogle patienter ikke kan undersøges med udstyret der benyttes til en QST-undersøgelse. Disse udfordringer afhænger af hvilket udstyr der anvendes til at udføre undersøgelserne.\\
De patientmæssige konsekvenser ved implementering af QST-protokollen som beslutningsværktøj til klinikeren vil således være, at protokollen er med til at sikre et bedre behandlingsforløb for den enkelte patient. Dette gøres idet de patienter, der vurderes til at være i risiko for udvikling af kroniske smerter tilbydes en anden behandling end en TKA-operation. Heraf vil en implementering af QST-protokollen, hvis denne fungerer efter hensigten, bidrage til korrekt behandling af knæartrosepatienter. Dette vil gælde for både patienterne, som er disponeret for at få kroniske postoperative smerter og dem, der ikke er.


