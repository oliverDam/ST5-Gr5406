\section{Formål}
I dette domæne undersøges og analyseres det hvilke konsekvenser en implementering af QST som supplement til klinikerens vurdering vil have på patienten. Hvis QST implementeres som et supplement til klinikeren i forbindelse med vurdering af, hvorvidt der skal foretages en TKA-operation, er det væsentligt at vide, om alle patienter kan undersøges med QST, eller om der findes faktorer, som vil bidrage til eksklusion af patienter fra undersøgelsesmetoden. Dette er relevant at undersøge, da dette potentielt vil medføre, at nogle patienter ikke kan vurderes på samme grundlag som andre, grundet den manglende undersøgelse. Derudover er det relevant at undersøge, hvordan patienten vil påvirkes ved implementering af QST-undersøgelserne som supplement til klinikerens vurderingsproces. Patientens videre behandlingsforløb kan potentielt være anderledes end det behandlingsforløb der ville igangsættes uden QST-undersøgelsen. Derfor er det væsentligt at undersøge, om det vil have positive eller negative konsekvenser for patienten, at implementere QST.
 
\section{HTA spørgsmål}
Teknologiens påvirkning på patienten:
\begin{enumerate}
\item \textit{Er der eksterne faktorer vedrørende teknologien til stede, som kan  ekskludere en gruppe patienter fra at benytte teknologien?} %(Fedme~cuff/hjerte/pacemaker?) H0012
\item \textit{Hvilken betydning vil implementering af teknologien have for patienten?} %H0006
\end{enumerate}


\section{Metode}
Til analyse af SOC-domænet er der foretaget en litteratursøgning med fokus på QST i forhold til patienten med udgangspunkt i den generelt beskrevne metode for litteratursøgning (jævnfør \secref{litteratursogning}. Der er derud over søgt i Mendeley, som er databasen, hvor fundet litteratur fra tidligere søgninger er gemt og udgør dermed en specialiseret database for dette projekt.  Det er i søgningen valgt ikke at ekskludere andre sygdomme, da det antages, at mængden af litteratur, der behandler det område, der ønskes undersøgt, kan være begrænset. Dette gøres, da teknologien stadig anvendes på forsøgsbasis i forhold til knæartrose. Der er ekskluderet QST metoder der benytter varme-stimuli, samt studier med præoperativ medicinering. Der er gennem litteratursøgningen ikke fundet studier, der konkret undersøger HTA-spørgsmålene. Derfor anvendes informationer fra TEC-, SAF- og ORG-domænet, da det vurderes, at disse domæner vil have indflydelse på, hvilke patienter der vil kunne gennemgå undersøgelsen, og hvilken påvirkning undersøgelsen vil have på disse. I TEC- og SAF-domænet undersøges henholdsvis teknologiens udformning og hvilke sikkerhedsmæssige aspekter, der skal tages højde for, hvilket kan være medvirkende til at præcisere hvilke patienter, der kan gennemgå undersøgelsen eller om der i forhold til disse domæner forekommer faktorer, der ekskluderer visse grupper. I ORG-domænet undersøges, hvordan arbejdsgange og det generelle forløb for udredning af knæartrose ændres. Det antages, at dette domæne kan medvirke til at undersøge, hvordan forløbet ligeledes vil ændres fra et patientmæssigt perspektiv.   
 
\section{Teknologiens påvirkning på patienten}
\textit{I dette afsnit undersøges, hvorvidt der ved anvendelse af QST-undersøgelserne, er eksterne faktorer til stede, som kan bidrage til at patienter ekskluderes. Ligeledes undersøges det, hvilke konsekvenser det vil have for de patienter, der kan undersøges med QST, at metoden implementeres som del af den nuværende undersøgelsesproces forud for TKA-operationer.}
\subsection{Eksklusion af patienter}
Det er i TEC-domænet fundet, at QST-undersøgelser med trykstimuli kan udføres med et håndholdt trykalgometer, fra Somedic eller et cuff-algometer, fra Nocitech. 
Ifølge resultaterne fra SAF-domænet er de sikkerhedsmæssige problemstillinger, der kan opstå ved påførelse af mekanisk tryk på en patient, ikke væsentlige ved anvendelse af ovennævnte udstyr, da disse undersøgelser ligger indenfor grænserne for skadespåførelse både i forhold til styrke og varighed (jævnfør \chapref{SAF_chap}). Såfremt de identificerede sikkerhedsforanstaltninger, i form af personaleuddannelse og krav til undersøgelseslokaler overholdes, kan det antages, at der ikke forekommer faktorer, der gør det nødvendigt at ekskludere patienter fra QST-undersøgelse, hvormed alle patienter bør kunne undersøges med teknologien. Det kan imidlertid overvejes om der forekommer begrænsninger i forhold til hvilket QST-udstyr, der benyttes; cuffen fra Nocitech kan have begrænsninger i forhold til patienter med arme eller ben hvis omkreds ligger over eller under intervallet som cuffens omkreds kan indstilles til. Dette kan være problematisk hvis patienten enten er overvægtig eller undervægtig i forhold til tilpasning af manchetten. Ved QST-undersøgelser tages der i algoritmerne ikke højde for patienternes intelligens, psykologiske tilstedeværelse eller bias i retning af et bestemt testresultat \citep{Dyck1998}.  
Hvis patienten ikke er i stand til at følge instruktionerne, der kræves af undersøgelsen, antages det, at grundlaget for vurdere patientens egnethed til TKA-operation svækkes, da resultatet kan være misvisende for patientens tilstand.


\subsection{Betydning af QST for patienten}
På nuværende tidspunkt udvikler omkring 20\% af de patienter som får en TKA-operation kroniske postoperative smerter. Ved en implementering af QST som supplement til kirurgens beslutning vil der være to mulige udfald; patienten kan være eller ikke være disponeret for udviklingen af kroniske postoperative smerter. Dette er gældende, hvis det antages, at QST-undersøgelserne har en høj præcision. I et behandlingsforløb vil begge resultater være gavnlige for både patient og den ortopædkirurgiske afdeling. Ved at udføre QST-undersøgelsen vil det være muligt at vælge den korrekte behandlingsstrategi, såfremt QST fungerer efter hensigten. Hvis patienten, på baggrund af klinikerens samlede vurdering, er disponeret for kroniske smerter efter en TKA-operation fravælges operationen. Dette er ensbetydende med, at patienten, foruden generelle risici forbundet med operationen, også vil undgå en behandling som kan medføre at de udvikler kroniske postoperative smerter. Hvis resultatet af QST-undersøgelsen derimod indikerer, at patienten ikke vil være disponeret for kroniske postoperative smerter kan patienten, såfremt klinikerens øvrige undersøgelser understøtter det, modtage en TKA-operation. \\
Implementeringen af QST vil både for klinikeren og patienten forbedre beslutningsgrundlaget i forhold til valg af det videre behandlingsforløb. Således vil både sandt negative og sandt positive QST-resultater være hensigtsmæssige for alle parter.


Ifølge de nationale kliniske retningslinjer for knæartrose vil patientforløbet for en artrosepatient, der har gennemført alle tænkelige alternativer, være en TKA. \citep{brostrom2012} Ved implementering af QST vil patienten undergå undersøgelsen. Hvis QST resultatet er positivt, hvorved patienten klassificeres som i risiko for kroniske postoperative smerter, vil patienten såfremt den kliniske vurdering er enig, ikke indstilles til en TKA-operation. Her vil patientens behandlingsforløb ifølge \citer{brostrom2012} skulle genvurderes, hvor et alternativ er smertelindrendebehandling med henblik på senere konsultation. I \appref{Borgerforloeb} ses en figur der illustrerer forløbet yderligere. Hvis ingen af disse løsninger er en mulighed må patienten accepterer situationen. Dette sætter et fokus på vigtigheden af videre behandlingsalternativer, hvilket vil blive belyst i perspektiveringen (se \secref{perspektivering}). 
\section{Delkonklusion}
På baggrund af at der ikke foreligger sikkerhedsmæssige problematikker ved benyttelse af QST-undersøgelser kan det antages, at kun patientens mentale kapacitet kan nødvendiggøre eksklusion af visse patientgrupper fra undersøgelsen. Der kan også opstå udfordringer i forhold til udstyret der skal benyttes i forhold til en QST-undersøgelse, som kan bidrage til eksklusion af patienter. Undersøgelsen vil være medvirkende til at give kirurgen et bedre beslutningsgrundlag for patientens videre forløb og sikre en korrekt og mere hensigtsmæssig behandling. \\
De patientmæssige konsekvenser ved implementering af QST som beslutningsværktøj til klinikeren vil således være positive, da QST er med til at sikre det bedst mulige behandlingsforløb for den enkelte patient, idet de patienter, der vurderes til at være i risiko for udvikling af kroniske smerter tilbydes en anden behandling end en TKA-operation. Heraf vil konsekvensen ved en implementering af QST-undersøgelserne, hvis disse fungerer efter hensigten, bidrage til en korrekt behandling af en patientgruppe, hvori en andel er disponeret for udviklingen af kroniske postoperative smerter.


