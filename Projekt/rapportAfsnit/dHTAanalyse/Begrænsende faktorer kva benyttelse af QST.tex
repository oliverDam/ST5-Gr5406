\section{Begrænsende faktorer ved benyttelse af QST}
Ved benyttelsen af QST som supplement til klinkerens beslutning, bør teknologiens begræsende faktorer vurderes. Som nævnt er den største konsensus omkring diagnostisk relevante QST-parametre PPT, TS og CPM. Ved benyttelsen af PPT, bør metodens udførsel undersøges. Det kan forestilles, at hvis de tre målinger til at danne PPT-værdien, udføres med for kort et interval, kan der opstå komplikationer. Hvis målingerne tages med for kort interval, kan det tænkes at en patient med faciliteret TS, vil resulterer i en falsk positiv PPT-værdi. Dette er ensbetydende med at patientens faciliterede TS akkumulerer PPT målingen til en størrelsen, som reelt ikke er tilstede. \\
For at kunne benytte QST som et led i den diagnostiske process, kræves det at klinikeren har adgang til et sæt normativ data til klassificering af abnormale tilstande. Det normative datasæt skal bestå af normale tærskler og tolerancer, samt abnormale tærskler og tolerancer, førend klinikeren kan adskille patientgrupper fra hinanden. Udvikling af sådanne normative datasæt er nødvendigt førend en mulig implementering, eller, hvis tilgængelig, benytte studieproducerede sæt. Ved fremadrettet benyttelse af normative data kræves det at den nøjagtige metode fra produktionen heraf benyttes. Det er heraf yderst nødvendigt at et normativt datasæt produceres efter en procedure som med garanti kan benyttes andet steds. Hvis ikke kræves det en reproduktion af normative datasæt ved enhver ny implementering af QST. Hvis ikke den nøjagtige metode benyttes kan det forestilles at resultaterne vil afvige fra det normative datasæt, og heraf skabe falsk negativ/positiv(e) resultater. Heraf vil genbrugelige normativ datasæt  hjælpe til at skabe pålidelig resultater, og dermed højne teknologiens sensitivitet og specificitet.  \citep{Yarnitsky1997} Problematikken vedrørende benyttelsen af normativ data ses ligeledes i studiet af \citer{Petersen2016}, hvor forsøgspopulationen inddelles i grupperinger vurderet på baggrund af arbitrære valg. Det tydeliggøres tilmed i studiet, at en normative inddeling af patienter er nødvendig, og bør optimeres og gøres generaliserbar, førend implementering af QST. Optimeringen skal bidrage til at øge sensitiviteten og dermed øge muligheden for at kunne forudsige patienters reaktion på TKA. \\
En anden metodisk problematik, på baggrund af metodens subjektive opbygning er, at algoritmerne ikke tager højde for patienternes intelligens, psykologiske tilstedeværelse eller at være biased mod et bestemt testresultat. Der kendes ikke til en metode som kan påvise om patienterne bevidst, som ubevidst, fejlrapporterer. \citep{Dyck1998}  \\ 
Patientgruppen tilhørende knæartrose, er som beskrevet, i nogle tilfælde relateret til svær overvægt. Det kan tilmed forestilles, at nogle patienter tyndere end normalen, da patientmålgruppen befinder sig en den ældre del af befolkningen. Dette kan antages at medføre teknologiske begrænsninger ved benyttelse af mekanisk QST. Det kan forestilles, at det er svært et tilpasse en standardiseret cuff, til patienter som befinder sig i ydre punkterne for hver tilstand. Det kræves at teknologien kan tilpasses alle patientgrupper, og producere et korrekt resultat. 