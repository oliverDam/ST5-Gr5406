\section{Formål}
I dette domæne analyseres de etiske problemstillinger relateret til implementeringen af QST-protokollen som supplement til klinikerens vurdering af en patients indstilling til en TKA-operation. Indførelse af QST på de ortopædkirurgiske afdelinger kan medføre, at der opstår nye etiske problematikker, der bør undersøges. Ved en implementering af QST-protokollen vil etiske problemstillinger, der især knytter sig til de patientmæssige aspekter, være relevante. Da klinikeren, ved hjælp af protokollen, antageligvis skal undersøge en stor gruppe patienter, der ikke får kroniske smerter, er der flere faktorer, der skal tages i betragtning i forhold til de konsekvenser, der opstår ved en implementering af QST-protokollen. Heraf vil en undersøgelse af konsekvenserne for fejlresultater, både i form af falsk negative og falsk positive resultater være nødvendig for at vurdere teknologiens egnethed som et supplement til klinikerens vurdering.

\section{Analyse-spørgsmål}
Patientetik:
\begin{enumerate}
\item \textit{Hvad er betydningen af falsk positive eller falsk negative testresultater, for patienten?} %F0003
\end{enumerate}

\section{Metode}
Til besvarelse af dette domæne tager litteratursøgningen udgangspunkt i humanistisk sundhedslitteratur. Dette betyder, at databaser, som PsychInfo og Journal of medical ethics, inddrages sammen med videnskabelige databaser, som beskrevet i den generelle metode (jævnfør \secref{litteratursogning}). Søgeprotokollen for ETH-domænet kan findes i\appref{ETH_sog}.
Til besvarelse af analyse-spørgsmål (1) undersøges de etiske dilemmaer, der kan opstå ved at give patienter falsk positive og falsk negative testresultater som led i et behandlingsforløb. Der overvejes herunder, hvordan dette kan påvirke patienten og vedkommendes videre behandling, samt hvilke etiske principper der kan anvendes til bestemmelse af hvilken type fejl, der er mest hensigtsmæssig.

\section{Patientetik}
\textit{I dette afsnit analyseres de etiske problemstillinger, der knyttes til anvendelse af QST og testresultaters påvirkning af patienten. Konsekvenserne i forhold til falsk positive og falsk negative testresultater vil blive analyseret ud fra etiske principper.}

\subsection{Betydningen af falske QST-resultater}
I forbindelse med undersøgelser, der giver et binært resultat i forhold til tilstedeværelsen af en sygdom eller ej, er der en risiko for falsk positive og falsk negative resultater, hvis ikke undersøgelsens præcision er på 100~\%. Som det beskrives i EFF-domænet, jævnfør \chapref{EFF_chap}, er det for de fleste medicinske undersøgelser nødvendigt for brugerne af undersøgelsen at overveje, hvilken type fejl de helst vil undgå. For QST-protokollen vil et falsk positivt resultat kunne betyde, at en patient som ville have gavn af en TKA-operation, ikke får denne. Ligeledes vil det for et falsk negativt resultat betyde, at en patient i risikogruppen for at få kroniske postoperative smerter efter en TKA-operation, får operationen og dermed har større risiko for at få smerter. Den etiske problematik består hermed i at vurdere hvilken af de to typer fejl, det er mest hensigtsmæssigt at undgå \citep{Kraemer2011}. \\
Denne problematik kan vurderes ud fra forskellige etiske synsvinkler, hvormed det ud fra et etisk perspektiv kan være mest hensigtsmæssigt at undgå falsk positive resultater, mens det set fra et andet perspektiv kan være mest hensigtsmæssig at undgå falsk negative resultater \citep{Kraemer2011}. Eksempelvis vil det ud fra forsigtigshedsprincippet være mest hensigtsmæssig at undgå falsk negative resultater. Dette princip bygger på tre epistemiske principper, som hver især er en del af forsigtighedsprincippet. \citep{Peterson2007} Det første princip er præferencen for falsk positive resultater. Denne præference er en generel konsensus i klinisk praksis, hvor der hellere vil fejlidentificere mange der ikke er syge end at overse én patient som er syg. \citep{Peterson2007}. Dette afhænger i høj grad af hvilke konsekvenser der er ved henholdsvis et falsk positiv og et falsk negativ resultat. I tilfældet med QST-protokollen vil det hermed skulle vurderes, om konsekvenserne ved falsk negative resultater er så betydelige, at disse overskygger konsekvenserne ved ikke at identificere alle syge patienter. \citep{Peterson2007} \\
Det andet epistemiske princip angiver, at alle ekspertvurderinger skal medtages som legitime, således ikke kun meningen af den mest influentielle ekspert medtages \citep{Peterson2007}. I forhold til beslutningen om hvorvidt en patient skal indstilles til en TKA-operation eller ej, betyder dette, at klinikeren og patienten, som skal tage beslutningen, skal inddrage alle parametre også selvom disse modsiger hinanden. Disse parametre er eksempelvis klinikerens objektive vurderinger, røntgenbilleder og resultaterne fra QST-undersøgelserne. Idet der i dette projekt lægges op til at QST skal benyttes som supplement, og ikke som enestående diagnoseparameter, understøttes benyttelsen som supplement af det andet epistemiske princip. \\
Det tredje epistemiske princip angiver, at vurderingsgrundlaget ikke nødvendigvis højnes ved tilføjelse af ny viden. Dette princip vil gælde i tilfælde, hvor ny viden kan bremse muligheden for at en behandling der kan gavne mange, ikke bliver anvendt fordi den kan skade få, hvis det ikke er muligt at identificere de få. \citep{Peterson2007} I forhold til implementeringen af QST-protokollen kan protokollen ses som en måde at efterkomme dette princip, idet implementeringen af QST-protokollen vil gøre det muligt at identificere de få, som TKA-behandlingen ikke gavner. \\
Ud fra de tre ovennævnte epistemiske principper vil det, ved anvendelse af forsigtighedsprincippet, være mest hensigtsmæssigt, at undgå falsk negative resultater hvis QST-protokollen anvendes som en del af et samlet vurderingsgrundlag, hvor alle dele af grundlaget tages i betragtning.

Et andet etisk perspektiv der kan anvendes til vurdering af hvilken type fejl, der er mest hensigtsmæssig, bygger på klinikeres vurdering af, hvor farlig sygdommen er for patienten. Hvis en uopdaget sygdom vil have alvorlige konsekvenser for en patients helbred, vil falsk negative resultater være de mest hensigtsmæssige at undgå. Omvendt vil det være mest hensigtsmæssigt at undgå falsk positive resultater, hvis konsekvenserne af dette er at en rask patient bliver sygeliggjort eller får mindsket livskvalitet. \citep{Kraemer2011} Hermed vurderes det ud fra et konsekventialistisk perspektiv, hvilken type fejl det er mest hensigtsmæssigt at undgå. Ud fra det konsekventialistiske perspektiv bedømmes en handling ud fra de konsekvenser, det kan få. Ses disse konsekvenser ud fra hvad der vil være mest hensigtsmæssigt for det størst mulige antal personer vurderes de falsk positive og falsk negative resultater ud fra et utilitaristisk perspektiv. \citep{Kraemer2011} Ud fra dette perspektiv kan det antages, at falsk positive resultater fra QST-protokollen vil være mest hensigtsmæssige at undgå, idet resultater fra QST-protokollen kun antyder højere risiko for udviklingen af kroniske postoperative smerter. Hermed vil ikke alle patienter med abnormale QST-resultater udvikle kroniske smerter efter en TKA-operation. Dette vil betyde, at patienten har fået det bedst mulige resultat, samtidig med at det kan antages, at sundhedssystemet vil have sparet penge i forhold til, hvis patienten havde fået et positivt QST-resultat. Denne antagelse gøres på baggrund af SOC-domænet, hvor det blev fundet, at patienter som ikke får en TKA-operation gennemgår udredningsforløbet på senere tidspunkter, hvormed disse patienter stadig vil være en udgift for sundhedssystemet, jævnfør \chapref{SOC_chap}. Disse sparede penge kan således anvendes til behandling af andre patienter, hvormed flere mennesker vil have gavn af et falsk negativt resultat end et falsk positivt resultat. \\
Heraf kan det ud fra forskellige etiske perspektiver argumenteres for hver af de to fejltyper. Vurdering af hvilken af de to fejltyper, der hermed er mest hensigtsmæssig, afhænger af hvilke etiske principper, der anvendes. Dette skal overvejes ved bestemmelsen af et normativt datasæt, idet grænseværdierne bestemmer andelen af falsk positive og falsk negative resultater hvormed det bestemmes hvilken af de to fejltyper, der vil forekomme flest af som det ses i EFF-domænet jævnfør \chapref{EFF_chap}.  

\section{Delkonklusion}
Beslutningsgrundlaget for indstilling af en patient til en TKA-operation kan påvirkes af både falsk positive og falsk negative QST-resultater. Et givent QST-resultat kan påvirke klinikerens og patientens planlægning af det videre behandlingsforløb. Et falsk negativt resultat vil betyde at patienten, som reelt er i risiko for udvikling af kroniske postoperative smerter, får operationen. Modsat vil falsk positive resultater betyde at patienter som ville have gavn af operationen bliver klassificeret som uegnet. Ud fra hvilke etiske perspektiver der anvendes kan både falsk positive og falsk negative resultater være de mest hensigtsmæssige. Ud fra et forsigtighedsprincip vil falsk positive resultater være mest hensigtsmæssige, mens det ud fra et utilitaristisk perspektiv være falsk negative resultater, der er mest hensigtsmæssige. Hvilket perspektiv QST-protokollen anvendes ud fra bestemmes i fastsættelsen af grænseværdier for det normative datasæt, idet disse grænseværdier påvirker andelen af falsk positive og falsk negative resultater. 

%I forbindelse med undersøgelser, der giver et binært resultat i forhold til tilstedeværelsen af en sygdom eller ej, er der en risiko for falsk positive og falsk negative resultater, hvis ikke undersøgelsens præcision er på 100\%. Som det beskrives i EFF-domænet, jævnfør \chapref{EFF_chap}, er det for de fleste medicinske undersøgelser nødvendigt for udvkilerne af undersøgelsen at overveje hvilken type fejl de helst vil undgå. For QST-protokollen vil et falsk positivt resultat kunne betyde at en patient som ville have gavn af en TKA-operation, ikke får denne. Ligeledes vil det for et falsk negativt resultat betyde at en patient i riskiogruppen for at få kroniske postoperative smerter efter en TKA-operation, får operationen og dermed har større risiko for at få smerter efter denne. Den etiske problematik består hermed i at vurdere hvilken af de to typer fejl, det er bedst at undgå \citep{Kraemer2011}. \\
%Denne problematik kan vurderes ud fra forskellige etiske principper og synsvinkler, hvormed det ud fra et etisk princip er mest hensigtsmæssigt at undgå falsk positive resultater, mens det set fra et andet perspektiv er mest hensigtsmæssig at undgå falsk negative resultater \citep{Kraemer2011}.      
%
%Ved indførelse af QST-protokollen som supplement til klinikerens vurdering, vil alle patienter, der indstilles til en TKA, få målt fysiologiske parametre relateret til smerteopfattelse for at opnå viden til vurdering af risikoen for udviklingen af kroniske postoperative smerter. 
%Det klarlægges i ORG-domænet, jævnfør \chapref{ORG_chap}, at implementering af QST-protokollen vil medføre en obligatorisk undersøgelse ved protokollen, inden endelig indstilling til TKA-operation. Ved en sådan undersøgelse er der, som det fremgår af EFF-domænet \chapref{EFF_chap}, fire mulige testresultater; sand positiv/negativ og falsk positiv/negativ. Jævnfør \chapref{EFF_chap} er hverken sensitiviteten eller specificiteten ved QST 100\%, og det forventes derfor, at teknologien i nogle tilfælde vil give falske resultater. Dette resulterer i en etisk problemstilling, da disse forkerte resultater kan have indvirkning på patientens behandlingsforløb. 
%Ved forundersøgelse, inden indstilling til TKA-operation, inddrager klinikeren røntgenbilleder, og ved implementering af QST-protokollen, vil testresultater herfra ligeledes medtages i overvejelserne (jævnfør \chapref{ORG_chap}). \\
%Den samlede vurdering tager således udgangspunkt både i røntgenbilleder, QST-resultater og klinikerens samtale med patienten. Problematikken opstår, når klinikeren modtager et falsk svar fra QST-protokollen og anerkender resultatet som sandt. Den samlede vurdering kommer således til at bygge på et forkert grundlag, hvilket i retrospekt kan lede til valg af den forkerte behandling. 
%For falsk positive svar kan dette medføre, at vurderingen vil indikere, at patienten ikke indstilles til TKA-operation, selvom denne skulle have modtaget TKA.
%For falsk negative svar kan dette betyde, at vurderingen vil indikere, at patienten skal indstilles til TKA-operation, selvom vedkommende ikke skulle have modtaget TKA.
%I begge tilfælde fejldiagnosticeres patienten og vil sandsynligvis fortsat opleve smerter i det afficerede knæled.

%Delkonklusion Herudfra er der både fordele og ulemper ved en implementering af QST-protokollen. En fordel er muligheden for at tilføje et redskab, der kan skåne særligt udsatte patienter for kroniske smerter, samt tilføjelsen af en kvantitativ måling til en ellers kvalitativ undersøgelse. En ulempe ved implementeringen af QST-protokollen vil være risikoen for falske testresultater, der kan medføre et forkert beslutningsgrundlag, hvilket kan føre til valget af et forkert behandlingsforløb for den pågældende patient. Behandlingsforløbet har således ikke hjulpet patienten, og implementeringen af QST-protokollen har haft ingen eller negativ effekt på patientens behandlingsforløb. Det er derfor væsentligt, at klinikeren inddrager alle fakta fra udredningsprocessen og ikke udelukkende baserer den endelige vurdering på baggrund af QST-resultater.