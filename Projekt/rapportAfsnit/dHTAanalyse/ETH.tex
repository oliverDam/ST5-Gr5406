\section{Formål}
I dette domæne analyseres de etiske problemstillinger relateret til implementeringen af QST som supplement til klinikerens vurdering af en patients henvisning til en TKA-operation. Indførelse af ny teknologi i sundhedsvæsenet kan medføre, at der opstår nye etiske problematikker, der bør undersøges. Dette kan således også være tilfældet ved en implementering af QST på ortopædkirurgiske afdelinger. Ved en implementering af QST vil etiske problemstillinger, der især knytter sig til de patientmæssige aspekter være relevante. Da klinikeren, ved hjælp af QST, antageligvis skal undersøge en stor gruppe patienter, der ikke får kroniske smerter, er der flere faktorer, der skal tages i betragtning i forhold til de konsekvenser, der opstår ved en implementering af QST. Heraf vil en undersøgelse af konsekvenserne for fejlresultater, både i form af falsk negative og falsk positive resultater være nødvendig for at vurdere teknologiens egnethed som et supplement til klinikerens vurdering.



\section{HTA spørgsmål}
Patientetik:
\begin{enumerate}
\item \textit{Hvordan påvirkes patienten af henholdsvis falsk positive og falsk negative testresultater?} %F0003
\end{enumerate}


\section{Metode}
Der vil til litteratursøgningen til ETH-domænet tages udgangspunkt i humanistisk sundhedslitteratur. Dette betyder, at databaser, som PsychInfo, inddrages sammen med videnskabelige databaser, som beskrevet i den generelle metode. 
Til besvarelse af HTA-spørgsmål (1) undersøges de etiske dilemmaer, der kan opstå ved at give patienter falsk positive og falsk negative testresultater som led i et behandlingsforløb. Der overvejes herunder, hvordan dette kan påvirke patienten og dennes videre behandling.


\section{Patientetik}
\textit{I dette afsnit analyseres de etiske problemstillinger, der knyttes til anvendelse af QST og testresultaters påvirkning af patienten. Konsekvenserne i forhold til falsk positive og falsk negative testresultater vil blive undersøgt og analyseret.}


Ved indførelse af QST-undersøgelse som supplement til klinikerens vurdering, vil alle patienter, der indstilles til en TKA, få målt fysiologiske parametre relateret til smerteopfattelse for at opnå viden til vurdering af risikoen for udviklingen af kroniske postoperative smerter. 
Det klarlægges i ORG-domænet, jævnfør \chapref{ORG_chap}, at implementering af QST vil medføre en obligatorisk undersøgelse ved QST, inden endelig indstilling til TKA-operation. Ved en sådan undersøgelse er der, som det fremgår af EFF-domænet \chapref{EFF_chap}, fire mulige testresultater; sand positiv/negativ og falsk positiv/negativ. Jævnfør \chapref{EFF_chap} er hverken sensitiviteten eller specificiteten ved QST 100%, og det forventes derfor, at teknologien i nogle tilfælde vil give falske resultater. Dette resulterer i en etisk problemstilling, da disse forkerte resultater kan have indvirkning på patientens behandlingsforløb. 
Ved forundersøgelse, inden indstilling til TKA-operation, inddrager klinikeren røntgenbilleder, og ved implementering af QST, vil testresultater herfra ligeledes medtages i overvejelserne (jævnfør \chapref{ORG_chap}). \\
Den samlede vurdering tager således udgangspunkt både i røntgenbilleder, QST-resultater og klinikerens samtale med patienten. Problematikken opstår, når klinikeren modtager et falsk svar fra QST-undersøgelsen og anerkender resultatet som sandt. Den samlede vurdering kommer således til at bygge på et forkert grundlag, hvilket i retrospekt kan lede til valg af den forkerte behandling. 
For falsk positive svar kan dette medføre, at vurderingen vil indikere, at patienten ikke indstilles til TKA-operation, selvom denne skulle have modtaget TKA.
For falsk negative svar kan dette betyde, at vurderingen vil indikere, at patienten skal indstilles til TKA-operation, selvom vedkommende ikke skulle have modtaget TKA.
I begge tilfælde fejldiagnosticeres patienten og vil sandsynligvis fortsat opleve smerter i det afficerede knæled.


\section{Delkonklusion}
På baggrund af ETH-analysen vurderes det således, at beslutningsgrundlaget kan påvirkes på forskellige måder ved henholdsvis falsk positive og falsk negative testresultater. Et givet testresultat kan påvirke klinikerens og patientens planlægning af det videre behandlingsforløb. 
Fra klinikerens perspektiv er der både fordele og ulemper ved en implementering af QST. En fordel er muligheden for at arbejde med et redskab, der kan skåne særligt udsatte patienter for kroniske smerter, og tilføjelse af en kvantitativ måling i en ellers kvalitativ undersøgelse. En ulempe er risikoen for falske testresultater, der kan medføre et forkert beslutningsgrundlag, hvilket kan føre til valget af et forkert behandlingsforløb for den pågældende patient. Behandlingsforløbet har således ikke hjulpet patienten, og implementeringen af QST har haft negativ effekt på patienten. Det er derfor væsentligt, at klinikeren inddrager alle fakta fra udredningsprocessen og ikke udelukkende baserer den endelige vurdering på baggrund af QST-resultater.