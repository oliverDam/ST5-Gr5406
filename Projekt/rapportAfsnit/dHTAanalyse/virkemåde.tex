\subsection{Undersøgelse af PPT, TS og CPM}
   
De tre QST-parametre PPT, TS og CPM kan testes ved forkellige typer stimuli eksempelvis mekanisk, kemisk eller termisk. Oftest anvendes mekanisk stimuli i form af tryk. \citep{Suokas2012} Til tilførelsen af tryk kan udstyr fra eksempelvis Somedic eller NociTech anvendes. \citep{Wylde2015} \citep{Petersen2016} \\
Ved test af PPT undersøges det, om patienten har en forstærket reaktion på tryk. Dette kan gøres både på områder i umiddelbar nærhed af det påvirkede knæ og på områder, som er længere væk fra knæet. En lav PPT, og dermed højere sensitivitet for stimuli, i området omkring det påvirkede knæ antyder perifær sensibilisering, mens en lav PPT i områder væk fra det påvirkede knæ antyder central sensibilisering. \citep{Suokas2012} PPT kan testes ved tilførelse af tryk på et område væk fra det påvirkede knæ. Dette område kan eksempelvis være på armen. Det påførte tryk stiger indtil patienten begynder at opfatte trykket som smertefuldt, og angiver dette ved eksempelvis tryk på en knap. PPT defineres som det påførte tryk, da patienten angav, at trykket blev smertefuldt, og angives dermed i kPa. Oftest gentages målingen tre gange, hvorefter gennemsnittet af de tre målinger anvendes som patientens PPT-værdi. \citep{Petersen2015b} \citep{Wylde2015} 

En forhøjet TS kan antyde central sensibilisering, da reguleringen af den temporale summation i neuroner er formindsket ved central sensibilisering. Hermed reagerer personer med central sensibilisering stærkere på gentagende stimuli end personer som ikke har central sensibilisering. \citep{Arendt-Nielsen2015} For at undersøge en patients TS påføres patienten gentagende tryk med samme intensitet. Eksempelvis i studiet af \citep{Petersen2016} blev patienten tilført tryk på et sekunds varighed efterfulgt af en pause på et sekund. Patienten blev i alt tilført 10 tryk. For hvert tryk angav patienten smerten på ud fra VAS, og TS blev udregnes som gennemsnittet af VAS for de første fire tryk minus gennesmnittet af VAS for de sidste tre tryk. \citep{Petersen2016} TS kan ligeledes udregnes ved at trække VAS-scoren for det sidste stimuli fra VAS-scoren for det første stimuli. \citep{Petersen2015b} 

Nedadgående smertekontrol er en betydende faktor for udviklingen af central sensibilisering. Den nedadgående smertekontrol regulerer neuronernes reaktion på stimuli, og består af en balance mellem inhiberende og fremmende signaler. For personer med normal nedadgående smertekontrol er denne hovedsageligt inhiberende. Ved central sensibilisering forskubbes balancen i den nedadgående smertekontrol, således neuronernes reaktion på stimuli ikke inhibieres på samme niveau som tidligere. Denne forskubning i balancen kan ske enten ved, at færre inhiberende signaler sendes eller, at flere fremmende signaler sendes til neuronerne. \citep{Arendt-Nielsen2015} Den nedadgående smertekontrol undersøges ved CPM. Ved test af CPM udsættes patienten for smertefuld stimuli et sted på kroppen, mens PPT måles et andet sted på kroppen, som benævnes teststedet. Før den smertefulde stimuli tilføres patienten er PPT på teststedet målt. \citep{Petersen2016} Den smertefulde stimuli der tilføres patienten kan være eksempelvis termisk eller mekanisk. CPM defineres som forskellen i PPT på teststedet før og efter den smertefulde stimuli er tilført et andet sted på kroppen. \citep{Petersen2015b} 
     
    