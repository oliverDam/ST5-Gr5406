\section{Diskussion}
\textit{I følgende afsnit vil projektets vurderingselementer blive sammendraget og diskuteret. Der vil hertil blive fremlagt overvejelser omkring vurderingselementerne og deres betydning for den samlede konklusion på projektets problemformulering. Afsluttende metoden for udarbejdelsen af projektet blive diskuteret i forhold til hvilken betydning udførelsen af en HTA på nuværende tidspunkt kan have for implementeringen af QST-protokollen.}

\subsection{Præcision af QST-protokollen} \label{Praecision} %TEC, EFFkvantita
En implementering af QST-protokollen skal bidrage som en kvantitativ undersøgelsesmetode i vurderingen af knæartrosepatienter. Dette er hensigtsmæssigt, da det nuværende vurderingsgrundlag hovedsageligt understøttes af en kvalitativ vurdering af patienten på baggrund af undersøgelser og samtaler. Selvom QST-protokollen er en kvantitativ metode er det væsentligt at pointere, at den ikke kan karakteriseres som objektiv, da resultaterne afhænger af patienternes egne subjektive opfattelser og vurderinger af sensation og smerte.\\ 
Da QST-undersøgelser af knæartrosepatienter forløber på forskningsniveau, er mængden af litteratur, der behandler området i forhold til PPT, TSP og CPM begrænset. Det kan dermed overvejes, om studierne udført på nuværende tidspunkt er tilstrækkelige til at dokumentere potentialet for PPT, TSP og CPM i forhold til undersøgelse af knæartrosepatienter, eller om der bør foretages yderligere studier heraf. Der bør i denne forbindelse foretages studier, der undersøger korrelationen af datasættene mellem PPT, TSP og CPM som én samlet undersøgelse og udviklingen af kroniske postoperative smerter. De studier, der på nuværende tidspunkt er udført for de enkelte parametre og udviklingen af kroniske postoperative smerter, finder en P-værdi, der indikerer en signifikant forskel, mens R-værdien viser lav eller ingen korrelation. Dermed kan der stilles spørgsmålstegn ved hvor god en prædiktor QST er for tilstanden der undersøges. Dokumentation for at QST-protokollen reelt kan prædiktere risikoen for udvikling af kroniske postoperative smerter, skal dermed udarbejdes, før teknologien kan implementeres. Da smerte er multifaktorielt, er det muligt, at QST-protokollen kan indikere en risiko for udviklingen af kroniske postoperative smerter, men det vil være relevant at undersøge, om der kan inddrages andre parametre, der kan tage højde for flere relevante faktorer.\\
Selvom sensitivitet og specificitet for QST-protokollen ikke på nuværende tidspunkt er undersøgt, kan det antages, at yderligere forbedring og undersøgelse af de tre parametre, og eventuelt inddragelse af flere, vil være medvirkende til at gøre QST-protokollen mere præcis. Dette vil danne grundlag for udarbejdelse af en optimal QST-protokol tilpasset knæartrose. I denne forbindelse er det ligeledes væsentligt at undersøge, om der bør inddrages andre kvantitative undersøgelsesmetoder til at supplere QST-undersøgelserne for at sikre en højere præcision af undersøgelsesforløbet.

For at QST kan implementeres som en del af klinisk praksis, bør det sikres, at de parametre, der skal indgå i den endelige QST-protokol for knæartrose, har en høj reproducerbarhed. Det er i EFF-domænet fundet, at reproducerbarheden generelt er højest for PPT-målinger, imens studier viser middel til god reproducerbarhed for TSP-målinger og generelt meget varierende resultater i forhold til reproducerbarheden af CPM-målinger. Da resultaterne for CPM-målinger er varierende, bør det undersøges, hvorvidt der er mulighed for at forbedre reproducerbarheden heraf, da resultaterne ellers kan være upålidelige. Det er nødvendigt med en højere pålidelighed for målingen af denne parameter for at det kan indgå som en del af den QST-protokol, der implementeres i klinisk praksis. Det kan med fordel også forsøges at forbedre reproducerbarheden for TSP, da denne befinder sig i intervallet middel til god.\\ 
Generelt for anvendelse af QST-udstyr, vil det kræve, at der er en kliniker til stede i rummet under hele undersøgelsen. Det bør i den forbindelse overvejes, om dette kan påvirke resultaterne fra undersøgelsen. Det kan antages, at de mest korrekte resultater opnås, når patienten ikke påvirkes eller stræber efter at præstere på et bestemt niveau under målingerne.

\subsection{Normative datasæt} \label{Normativ_data} % TEC, EFF
Det er et generelt problem ved QST-protokollen, at den grundlæggende er baseret på patientens subjektive vurdering af sensoriske indtryk og at der ikke er udarbejdet normative datasæt for de enkelte parametre. Klinikeren har således ingen fastsat reference at vurdere en patients QST-resultater ud fra, og der kan i princippet ikke skelnes mellem normale og abnormale resultater. For i praksis at kunne anvende QST-protokollen som et diagnostisk supplement, skal der forelægges normative data for QST-parametrene. Resultater kan således sammenlignes med definerede normalværdier for parametrene, så patienter kan vurderes ud fra disse. Dette vil kræve, at der udvikles en standardiseret metode til udførelse af QST-undersøgelser, og ud fra denne metode udvikles et datasæt, som fastsætter standarder og normalværdier for de tre QST-parametre. Ud fra dette datasæt vil en kliniker objektivt kunne bedømme, om en patients QST-resultater er normale eller ej, og det kan vurderes, hvorvidt patienten er i risikogruppen for at udvikle kroniske postoperative smerter. Ved udvikling af dette datasæt skal QST-protokollens normalværdier defineres. Det kan ligeledes være relevant at undersøge, om der kan udvikles algoritmer, som kan tage højde for faktorer, der kan have indvirkning på QST-resultaterne, såsom komorbiditet, etnicitet, køn og alder. Faktorer som disse skal klarlægges i udviklingen af et normativt datasæt og det skal vurderes, om disse faktorer kan påvirke QST-resultaterne i en sådan grad, at der skal tages højde for disse i de normative datasæt. Efter udviklingen af normative datasæt skal det ligeledes foretages yderligere studier af nøjagtigheden for QST. Herunder skal det undersøges hvor præcist QST kan identificere patienter, så de kan behandles korrekt.
Det vil herudover være relevant at undersøge, hvorvidt der kan udvikles metoder, der kan tage højde for patienters mentale tilstand, så patienter, der ikke forstår instruktioner omkring protokollens udførelse, ligeledes kan undersøges med denne. \\
Det fremstår som et nuværende problem ved QST, at der ikke findes en standardiseret metode til udførelsen af QST-undersøgelser. Når forskellige metoder anvendes, forekommer der variation i resultaterne, hvorved det er besværligt at sammenligne resultater og konklusioner studier imellem. Derved opstår uenighed omkring præcisionen af QST-parametrene, pålideligheden af resultaterne og dermed det reelle udbytte af QST-protokollen. Udvikling af en standardiseret metode til udførelse af QST-undersøgelser vil kunne bidrage til en mere præcis målemetode, hvilket vil øge pålideligheden.

\subsection{Patientaspekter} \label{Patient_aspekt} %SAF, ETH
Da QST stadig anvendes på forskningsniveau for knæartrosepatienter, er en endelig metode til udførelse endnu ikke fastlagt. Dermed kræves yderligere undersøgelser af sikkerhed. Der er opstillet retningslinjer for udførelse af QST af DFNS, der skal sikre korrekt undersøgelsesteknik og overholdelse af patientsikkerheden, men ved tilføjelse af cuff-algometer til udførelse af QST-protokollen, skal der ligeledes opstilles retningslinjer for denne. De nuværende retningslinjer dækker QST-undersøgelser ved trykpåvirkning og termisk påvirkning. Trykpåvirkning udføres på nuværende tidspunkt med et håndholdt trykalgometer, men som det er belyst i dette projekt, kan cuff-algometre ligeledes benyttes til undersøgelse af PPT, TSP og CPM. Der bør således foretages en konkret undersøgelse af de sikkerhedsmæssige risici ved anvendelse af cuff-algometre. En fordel ved QST er, at metoden er non-invasiv. Dette øger generelt patientsikkerheden, da patienter blandt andet ikke udsættes for infektionsfare. \\
Det fremlægges i dette projekt, at cuff-algometret overholder grænsen for trykpåvirkning og ligger under brudstyrken for væv, og derfor umiddelbart ikke vil kunne påføre skade på patienter. Til forskel fra trykalgometret, påvirker cuff-algometret ikke et specifikt punkt, men vil påføre et tryk hele vejen omkring en patients arm eller ben. Det bør således istedet undersøges, om kompression kan have sikkerhedsmæssige risici for patienten. Her vil det være mindre relevant at undersøge hudens brudstyrke, men nærmere hvordan væv under huden, som muskler og blodkar, påvirkes af tryk. \\
Nuværende QST-undersøgelser, som anvender cuff-algometer, har en umiddelbar begrænsning i forhold til udstyr, da manchetten ikke kan tilpasses alle typer patienter. Det kan antages, at en undervægtig eller overvægtig patient vil være problematisk, at udføre en korrekt QST-undersøgelse på, da manchetten muligvis ikke ville kunne tilpasses korrekt. 
Dette problem vil umiddelbart kunne løses ved udvikling af manchetter i andre størrelser.


\subsection{Økonomi og organisation} \label {ECO_ORG} %ECO, ORG
I ORG-domænet er der opstillet et forslag til, hvordan QST-protokollen kan integreres i den nuværende undersøgelsesproces således, at den nuværende proces forstyrres mindst muligt. På baggrund af dette forslag antages det, at klinikerens vurdering af QST-resultater tidsmæssigt vil være i samme omfang som vurderingen af et røntgenbillede. Det er imidlertid væsentligt at gennemføre et prøveforløb på de ortopædkirurgiske afdelinger, hvor implementeringsforslaget testes for at vurdere, om processen i praksis er hensigtsmæssig eller kan optimeres. Selvom det antages, at klinikerens vurdering af et QST-resultat ikke vil være af afgørende tidsmæssig betydning, er det essentielt at undersøge, om den tidsmæssige antagelse for vurdering af QST-resultater er realistisk. Da algoritmerne bag QST-undersøgelserne på nuværende tidspunkt ikke tager højde for individuelle forhold som alder og intelligens, kan det antages, at klinikeren ved nogle patienter skal anvende længere tid til vurdering af QST-resultaterne. Den endelige undersøgelsesproces for knæartrose vil muligvis indeholde andre faktorer end QST-protokollen.\\ 
Ifølge oversigten over mulige forløb for patienter med knæartrose kan en patient efter konsultation og undersøgelse på ortopædkirurgisk afdeling henvises tilbage til sin praktiserende læge i primærsektoren (jævnfør \appref{Borgerforloeb}). Dette kan gøres, hvis klinikeren vurderer, at der forekommer faktorer, der gør det uhensigtsmæssigt at operere patienten på det pågældende tidspunkt. Hvis den praktiserende læge på et senere tidspunkt igen henviser den pågældende patient til ortopædkirurgisk afdeling, skal patienten igen gennemgå undersøgelses- og vurderingsprocessen forud for en TKA-operation. Det er således væsentligt at tage højde for, at nogle patienter skal gennemgå en QST-undersøgelse flere gange i deres sygdomsforløb. Det kan antages, at de gentagne undersøgelsesforløb, der er forlænget som følge af implementeringen af QST-protokollen, kan være medvirkende til at skabe et tidsmæssigt pres på de ortopædkirurgiske afdelinger. Flere af patienterne med knæartrose gennemgår imidlertid ligeledes det nuværende undersøgelsesforløb flere gange, hvorfor det kan antages, at det tidsmæssige pres fra QST-protokollen ikke vil være betydeligt (jævnfør \appref{Borgerforloeb}). \\ 

Det er i ECO-analysen konkluderet, at der som følge af implementeringen af QST vil kunne opnås en besparelse i intervallet 4.3 til 8.6 millioner kr. årligt i RN, som følge af en reduktion i antallet af TKA-operationer. I praksis vil der imidlertid ikke forekomme en reel besparelse, da reduktionen i antallet af operationer sandsynligvis vil medføre, at operationerne i stedet kan tilbydes de patienter, der på baggrund af forundersøgelsen er erklæret egnet til at modtage TKA-operationen. Det reelle udbytte kan dermed antages at være en reduktion af ventetiden på en TKA-operation, hvilket er hensigtmæssigt for både patienter og de ortopædkirurgiske afdelinger.\\
Beregningerne foretaget i ECO-analysen i forhold til implementering af det beskrevne QST-udstyr er gældende for det første år efter køb. Efter det første år vil udgifterne, som følge af afskrivning, falde løbende. Det er desuden væsentligt at pointere, at den estimerede merudgift ved implementering af Medoc fremfor NociTech ikke er reel, da der skal anvendes supplerende udstyr til at undersøge TSP og CPM. Der skal således påregnes en ekstra udgift ved køb af udstyret fra Medoc. \\
Den primære fordel ved implementering af QST-protokollen er, at teknologien kan være medvirkende til at sikre, at de enkelte patienter får den korrekte behandling, hvormed flere patienter vil opnå bedre funktionalitet og smertereduktion efter deres udrednings- og behandlingsforløb. Dermed er det sandsynligt, at flere patienter vil have mulighed for at komme tilbage på arbejdsmarkedet, hvilket kan reducere det samlede antal af tabte arbejdsdage på nationalt plan. Dette vil ligeledes bidrage til en økonomisk gevinst for de enkelte regioner. Det er imidlertid væsentligt at tage højde for den relativt høje gennemsnitsalder i patientgruppen, der indikerer, at en større del af patienterne er i pensionsalderen. \\
Det vil være muligt at implementere QST-protokollen på nationalt plan, såfremt budgettet tillader det i de enkelte regioner. I denne forbindelse skal der tages højde for befolkningstætheden i de enkelte regioner og dermed hvor mange knæartrosepatienter, der skal undersøges med QST-protokollen. Dette vil have betydning for hvor meget QST-udstyr, der skal indkøbes, og hvor meget personale der skal oplæres og udføre undersøgelserne. Der vil således kunne forekomme væsentlige variationer i de økonomiske udgifter til implementeringen af QST-protokollen i de enkelte regioner, men udgifterne skal opvejes med det udbytte, der kan opnås i de enkelte regioner i forhold til reduktion af ventetiden på TKA-operationer.

\subsection{Metode} \label{Metode_diskussion} % vores metode 
Projektet er udarbejdet med udgangspunkt i en kombination af AAU-modellen og HTA-core modellen. Ved at anvende HTA-core, kan det antages, at der er opnået en tydelig adskillelse af de enkelte aspekter, QST har indflydelse på.
Antallet af analyseperspektiver i HTA-modellen ses som en styrke ved modellen. Som det fremgår af metoden i \chapref{metode} er LEG-domænet ikke medtaget i den endelige analyse, da det er vurderet at dette ikke bidrager til besvarelse af problemformuleringen. Det er imidlertid muligt, at det vil blive relevant at undersøge dette domæne på et senere tidspunkt i forhold til QST-protokollen. Gennem analyserne er det fundet, at forskningen og udviklingen af QST-protokollen stadig befinder sig i et relativt tidligt stadie, hvor det for flere af domænerne viste sig besværligt at tilkomme litteratur, der kunne bidrage til besvarelse af analyse-spørgsmålene. Det kan dermed overvejes om tidspunktet, hvorpå det er valgt at lave en HTA af QST-protokollen, har været hensigtsmæssigt. Det skal overvejes, hvorvidt denne HTA påvirker fremtiden for QST i forbindelse med knæartrose, da analysen kan stille teknologien i et godt eller dårligt lys, på baggrund af resultaterne. Som det tidligere diskuteres, kræver det yderligere studier omhandlende sensitivitet, specificitet og reproducerbarhed af QST-protokollen, før det kan vurderes i præcis hvilket omfang, protokollen kan anvendes som supplement til en klinikers vurderingsgrundlag. Det er muligt, at en teknologivurdering foretaget for tidligt i udviklingsprocessen kan have uhensigtsmæssig indflydelse på eventuelle beslutningstageres indtryk af den pågældende teknologi, grundet de begrænsede mængder studier.

Som det nævnes, har udbyttet af litteratursøgningen for nogle af analyse-spørgsmålene været mangelfuld. Det kan overvejes, om en yderligere systematisering af litteratursøgningen havde været fordelagtig. Det er muligt, at en revurdering af inklusions- og eksklusionskriterier samt yderligere synonymer for konkrete søgeord kunne have bidraget til et bedre søgeresultat for nogle analyse-spørgsmål. Der skal imidlertid tages højde for, at QST-undersøgelse af knæartrosepatienter er et relativt nyt tiltag, hvormed mængden af litteratur til visse analyseperspektiver må antages at være begrænset, eksempelvis i forhold til det patientmæssige perspektiv, da teknologien ikke er implementeret som en del af klinisk praksis. Det kunne således have været en fordel at gentage litteratursøgningen for flere af analyse-spørgsmålene som ikke gav tilstrækkelige resultater, for at søge efter nye publikationer. Dette er dog ikke gjort på grund tidsmangel.\\
De diskuterede problemstillinger er vurderet som værende relevante på nuværende tidspunkt. Såfremt en HTA af QST til undersøgelse af knæartrosepatienter udarbejdes på et senere tidspunkt er det muligt, at andre problemstillinger vil være relevante at inddrage. Det antages, at der efter denne projektperiode vil udarbejdes flere studier, der er relevante for de forskellige domæner.
