\section{Perspektivering}\label{perspektivering}
\textit{I dette afsnit vil der blive perspektiveret til videre behandlingsforløb som resultat af implementeringen af QST, samt problemstillinger, der kan opstå i forbindelse med videreudviklingen af undersøgelserne.}

\subsection{Kroniske smertepatienter}
Det fremlægges i projektet, at effekten af QST-undersøgelser af knæartrosepatienter  vil være en reduktion i antallet af operationer. Denne reduktion vil enten kunne tillade operation af flere patienter, der er vurderet som værende egnede til  en TKA-operation, hvormed ventelisterne kan forkortes, ellers vil reduktionen kunne medvirke til at nedbringe eventuelt overarbejde på en ortopædkirurgisk afdeling. Ses implementeringen af QST-undersøgelserne i et større perspektiv vil det endelige resultat være en reduktion i antallet af kroniske smertepatienter. \\
Kroniske smertepatienter er en økonomisk byrde for samfundet, og når faktorer som farmaka, hospitalsydelser og tabt arbejdsfortjeneste medregnes, vil disse patienter medføre betydelige udgifter for samfundet. SmerteDanmark, der er en tværsektoriel interesseorganisation indenfor smerteområdet i Danmark lægger fokus på at udbrede informationer om smerte til patienter, behandlere og beslutningstagere. Omkring 850.000 danskere over 18 år lider af kroniske smerter. Dette er tre til fire gange så mange patienter som der er på henholdsvis diabetes- eller kræftområdet. \citep{SmerteDanmark} De mest udbredte kroniske smerter findes i hofte og knæ, ryggen samt skulder og nakke \citep{SmerteDanmark1}. Af de kronisk smerteramte i Danmark har 17~\% haft sygefravær inden for de seneste 14 dage, mens 28~\% af de smerteramte har været nødsaget til at stoppe med at arbejde. Kronisk smerte er skyld i en million tabte arbejdsdage årligt i Danmark, hvor omkostningerne for samfundet er mere end 40 milliarder kr. om året. Heraf bruges 2,2 milliarder kr. til kontakten til sundhedsvæsenet, to milliarder kr. til dækning af sygefravær og 35 milliarder kr. til førtidspensionering. Hvis QST-undersøgelser kan benyttes til at identificere knæartrosepatienter, der er i risiko for at udvikle kroniske postoperative smerter, vil dette kunne spare samfundet for en andel af kroniske smertepatienter, hvilket både vil have patientmæssige og økonomiske gevinster. \citep{SmerteDanmark}

\subsection{Forløb for patienter med positiv QST-undersøgelse}
Ved et positivt QST-resultat vil patienten klassificeres som værende i risiko for at udvikle kroniske postoperative smerter. Hvis patienten på baggrund af klinikerens samlede undersøgelse vurderes til ikke at ville få gavn af en TKA-operation, vil patienten indgå i forløbet beskrevet i SOC- og ORG-domænet, jævnfør henholdsvis \chapref{SOC_chap} og \chapref{ORG_chap}. Mulighederne for videre behandling er, såfremt alle alternativer er forsøgt, optrapning af den medicinske behandling, henvisning til smerteklinikker og til sidst accept af situationen. Det sidste stadie er ikke et ønsket udfald hverken for patienten, der skal leve med smerterne, eller samfundet, som skal afholde udgifter forbundet med patienten.
Studier antyder, at behandling med desensibiliserende medicin, såsom pregabalin og duloxetin der på nuværende tidspunkt benyttes som psykofarmaka, kan have en gavnlig effekt som præoperativ behandling med henblik på at opnå færre kroniske smerter efter TKA-operationerne. I et forsøg af \citer{Buvanendran2010} blev 120 patienter behandlet med pregabalin 14 dage forinden en TKA-operation. Resultatet efter TKA-operationerne var et færre antal patienter med kroniske neuropatiske smerter, for patienter der modtog pregabalin, i forhold til placebogruppen. Behandling med pregabalin var imidlertid forbundet med en højere risiko for, at patienterne tidligt efter operationen var konfuse og i en sedativ tilstand. \citep{Buvanendran2010} I et præ-resultat af \citer{Blikman2016} antages det, at en 10 ugers præoperativ behandling med duloxetin vil kunne nedsætte de kroniske postoperative smerter fra en TKA-operation. Studier har vist, at duloxtin har haft en effekt på kroniske sygdomme, heriblandt artrose, hvori central sensibilisering spiller en central rolle. \citep{Blikman2016} En potentiel behandlingsmulighed, på baggrund af resultater fra  \citer{Blikman2016} forventet i 2017, kan være behandling med psykofarmaka forud for en TKA-operation. Det vil imidlertid være nødvendigt at lave flere studier på området før effekten af behandlingen kan fastlægges. Såfremt disse behandlinger har et positivt resultat, vil QST kunne komplementere den desensibiliserende behandling, ved at bidrage til bedre klassifikation af patienter i forhold til risikoen for at udvikle kroniske postoperative smerter. Herefter vil patienten kunne modtage den korrekte behandling. Med en medicineringsplan forud for operation, vil andelen af patienter med nedsatte eller ingen smerter forøges. En kombination af QST udvælgelse,  præmedicinering samt kirurgisk behandling af knæartrosepatienter med central sensibilisering vil således for disse i fremtiden være et muligt behandlingsforløb.

\subsection{Videre forskning}
For at kunne implementere QST-undersøgelserne effektivt er det essentielt at få udviklet normative datasæt for undersøgelserne.  Ved QST-undersøgelserne kan patienternes resultater, grundet den subjektive indflydelse jævnfør \secref{Praecision} omhandlende, påvirkes af pårørende og klinikeren der foretager undersøgelsen på patienten. Hermed vil en større grad af automatisering af undersøgelsesproceduren være fordelagtigt. Denne automatisering vil være en fordel, da dette kan mindske det subjektive aspekt ved undersøgelsen. Dette er tilfældet, da teknologien i højere grad udfører undersøgelsen uden klinikerens tilstedeværelse. \\
Som nævnt i diskussionen er der flere faktorer, som kan være prædiktorer for udviklingen af kroniske postoperative smerter. Eksempler på disse risikofaktorer kan være katastrofetænkning og komorbiditeter. Det kan derfor være fordelagtigt at undersøge sammenhængen mellem risikofaktorer og QST-protokollen.\\
Kroniske postopertive smerter er ikke kun et problem ved TKA-operationer, men også ved andre operationer såsom hoftealloplastikker. \citep{Suokas2012} Ligeledes kan det antages, at QST vil kunne anvendes til undersøgelse af patienter med ryg- og lændesmerter. Ifølge \citer{Wilder-Smith2010} vil QST-undersøgelserne kunne benyttes til at prædiktere kronisk smerte efter abdominalkirurgi. Dette indikerer et potentiale for at benytte QST til at forudsige kroniske smerter efter andre operationer.