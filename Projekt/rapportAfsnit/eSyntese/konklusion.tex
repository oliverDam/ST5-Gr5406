\section{Konklusion}
\textit{Følgende afsnit konkluderer på baggrund af analyse-spørgsmålene for dette projekt. Dette samles til en besvarelse af problemformuleringen.}

Igennem udarbejdelsen af dette projekt vurderes konsekvenserne vedrørende implementering og brug af QST som supplement til klinikerens beslutning om indstillingen til en TKA-operation. QST kan benyttes som værktøj med en prædiktiv information om risikoen for udviklingen af kroniske postoperative smerter. QST-parametrene, der undersøges, er PPT, TSP og CPM.\\
På baggrund af analyserne af de forskellige domæner og ovenstående diskussion kan det dermed konkluderes, at QST skal videreudvikles og optimeres for at agere som supplement til klinikerens beslutningsgrundlag. Hvis QST bliver implementeret efter optimeringen, vil denne kunne bidrage til, at klinikeren korrekt kan kategorisere alle knæartrosepatienter, som henvises til et TKA-behandlingsforløb. Hermed højnes kvaliteten af behandlingsforløbet for knæartrosepatienter i RN. Såfremt optimeringen udføres, ses der udfra analyserne, at implementering af QST vil være en økonomisk hensigtsmæssig beslutning. Hvis QST bliver optimeret i en grad, hvor normative datasæt udarbejdes og præcisionen kan vurderes som tilstrækkelig høj, vurderes det, at QST hensigtsmæssigt vil kunne implementeres på de ortopædkirurgiske afdelinger i RN og bruges som supplement til klinikerens vurdering af en patients egnethed til en TKA-operation.

%Igennem analyse af TEC- og EFF-domænerne er der blevet påvist en sammenhæng, ved benyttelse af QST, imellem præoperative og postoperative QST-resultater vedrørende udviklingen af kroniske smerter efter TKA-operation. På baggrund af studierne, som har påvist ovenstående, skal der tages forbehold for dårlig korrelation. Der forekommer imidlertid signifikante forskelle , hvormed QST har potentialet til at kunne benyttes i den beskrevne sammenhæng. Tages potentialet i betragtning og på antagelsen af at dette opfyldes, vil klinikerens vurderingsgrundlag hæves og beslutningerne vil kunne træffes på et hævet grundlag. På baggrund af analysen af SAF-domænet er det blevet påvist, at der ikke er nogle sikkerhedsmæssige risici ved benyttelse af QST, og derfor skal der ikke foretages nogle sikkerhedsforanstaltninger forud for en eventuel implementering og brug af QST. Der bør imidlertid foretages yderligere undersøgelser af sikkerheden af de enkelte QST-parametre, med det udvalgte QST-udstyr.  Analysen af ORG-domænet påviste, at implementeringen af QST, såfremt den opstillede metode for implementering af QST i den nuværende vurderingsproces er realistisk, ikke vil medføre store organisatoriske ændringer. Arbejdsgangen og den interne kommunikation på ortopædkirurgisk afdeling påvirkes umiddelbart ikke betydeligt, da selve QST-undersøgelsen ikke tager længere tid end cirka 20 minutter at udføre. Dog skal det tages forbehold for yderligere undersøgelser inden for den organisatoriske påvirkning. Ved analyse af ECO-domænet er det, med forbehold for antagelser, påvist, at implementeringen og benyttelsen af QST ikke ville medføre store økonomiske påvirkninger. De økonomiske udregninger tager udgangspunkt i en afskrivningsperiode på et år. Den efterfølgende årrække vil være billigere, og antageligvis udelukkende bestå af driftsomkostninger. Ved en reduktion i antallet af patienter der indstilles til TKA-operation, kan det antages, at ventelisterne vil forkortes, da der således kan tages flere patienter ind, der på baggrund af klinikerens vurdering, er erklæret egnede til at modtage operationen.
%
%Det er igennem analyse af ETH-domænet blevet diskuteret, hvorledes falsk negative og falsk positive resultater påvirker patienten. Med den nuværende præcision af QST, vil disse fejltyper antageligvis være til stede i relativt høj grad. Dette scenarie er til stede da der på nuværende tidspunkt ikke er udarbejdet normative datasæt til klassificering af patienters risiko for udvikling af kroniske postoperative smerter. Dette er ligeledes uhensigtsmæssigt, at præcisionen af QST ikke kendes på nuværende tidspunkt. Dette medfører at virkningsgraden af teknologien endnu ikke konkret er kendt.