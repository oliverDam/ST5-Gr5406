På \figref{fig:AAUmodel} ses AAU-modellen. 

\begin{figure}[H] 
	\begin{center}
		\includegraphics[width=0.5\textwidth]{figures/cMetode/AAUmodel}
	\end{center}
	\caption{AAU-modellen, der anvendes i projektet. Det ses, hvordan det initierende problem afgrænser det første, brede område, der undersøges i problemanalysen, hvorefter der igen foretages en afgrænsning efter opstilling af problemformuleringen.} 
	\label{fig:AAUmodel} 
\end{figure}


\section{Projektets metode}
I dette projekt anvendes HTA-core modellen, hvilket afspejles i projektets metode. Målet med HTA-core modellen er at analysere og diskutere de essentielle problemstillinger ved QST, samt at præsentere resultaterne fundet ved analysen og diskussionen \citep{HTAcore}. Ydermere bidrager benyttelsen af HTA-core modellen til en systematisk og bred vurdering af QST. Dette giver et evidensbaseret grundlag til beslutningstagning i forhold til eventuel implementation af QST. \citep{mtvhaandbog2007} \citep{HTAcore} HTA-core modellen anvendes til besvarelse af problemformuleringen, da denne lægger op til en analyse af væsentlige faktorer ved QST, set ud fra et sundhedsvidenskabeligt perspektiv. Ved anvendelse af HTA-core inddeles analysen af teknologien i ni forskellige områder \citep{HTAcore}:

\begin{enumerate}
\item Sundhedsmæssigt problem og nuværende brug af teknologien (CUR)
\item Beskrivelse og tekniske karakteristika for teknologien (TEC)
\item Sikkerhed (SAF)
\item Klinisk effektivitet (EFF)
\item Omkostninger og økonomisk evaluering (ECO)
\item Etisk analyse (ETH)
\item Organisatoriske aspekter (ORG)
\item Patient- og sociale aspekter (SOC)
\item Juridiske aspekter (LEG)
\end{enumerate}

Det første analyseområde, CUR, er besvaret ved benyttelse af en AAU inspireret tilgang. Først indledes projektet med et initierende problem, der lægger op til en bred, alsidig undersøgelse af relevante aspekter for projektet, hvilke undersøges i problemanalysen for at finde et konkret problem, som ønskes besvaret. Igennem analysen bliver det initierende problem undersøgt og afslutningsvis afgrænset, hvormed der kan opstilles en problemformulering. Problemformuleringen definerer det konkrete problem, og besvares igennem de resterende vurderingelementer i HTA-core modellen. De første to kapitler (\chapref{introduktion} og \chapref{problemanalyse}), udgør dermed i dette projekt CUR analysen. De resterende vurderingselementers metode vil blive beskrevet i deres respektive analyser. \\	
Vurderingselementer er en betegnelse for det samlede produkt af analyse af et af de ni områder. Hvert område bliver inddelt i emner der hver skal klarlægge et bestemt problem indenfor området. Hvert emne specificeres yderligere ved konkrete spørgsmål som ønskes besvaret. Når et eller flere spørgsmål under et emne er besvaret, kan besvarelsen af emnet samles med svar fra andre emner i området, i en delkonklusion. Delkonklusionen er besvarelsen af et område, og svaret indgår som et vurderingselement til den endelige besvarelse af problemformuleringen.  \citep{HTAcore}. \figref{fig:vurderingselement} viser den beskrevne opbygning af vurderingselementet. 

%I hver analyse inddeles området i forskellige emner. Ud fra emnerne opstilles konkrete spørgsmål, der ønskes besvaret. Hvert områdes tilhørende emner og spørgsmål skal virke til besvarelse af problemformuleringen, hvorved de fungere som et af vurderingselementerne i den samlede teknologivurdering. \citep{HTAcore}. \figref{fig:vurderingselement} viser den beskrevne opbygning af vurderingselementet. 

\begin{figure}[H] 
\begin{center}
\includegraphics[width=0.5\textwidth]{figures/cMetode/vurderingselement}
\end{center}
\caption{Opbygningen af et vurderingselement. Figuren viser, hvordan vurderingselementet samlet er udgjort af de tre elementer; område, emne og problematik.}
\label{fig:vurderingselement} 
\end{figure}

I dette projekt opstilles analyseområderne med inspiration fra retningslinjerne i metodehåndbogen for HTA-core modellen. Disse retningslinjer omfatter blandt andet hvilke emner, der bør afdækkes indenfor de enkelte områder samt generelle forslag til spørgsmål. Det fremgår desuden, hvilke andre vurderingselementer, området relaterer til. Det fremgår ligeledes, hvilken metode og hvilken type litteratur, der kan være medvirkende til at besvare et givet område. \citep{HTAcore} 
Der er ikke altid grundlag for at foretage analyse inden for alle ni områder. Ved eksklusion af et analyseområde argumenteres der hvorfor dette område ikke er relevant i den pågældende HTA. \citep{HTAcore} Eksempelvis vil området “Sundhedsmæssigt problem og nuværende brug af teknologien” ikke blive inkluderet i dette projekt, da emnerne og problematikkerne indenfor dette vurderingselement besvares af problemanalysen, som blev lavet forud for problemformuleringen. 
Til besvarelse af den opstillede problemformulering, er hver af de tilbageværende otte områder uddybet med emner og spørgsmål. \textbf{Dette kan eksempelvis ses i afsnit \ref{X}.}
Før besvarelse af hvert af disse områder, argumenteres der for det enkelte områdes relevans for besvarelsen af problemformuleringen. Dette skal sikre at kun de vurderingselementer der bidrager til besvarelse af problemformuleringen besvares i projektet. %Efter analyse af hvert område opsummeres vigtige pointer og resultater fra analysen i en delkonklusion. (det er nævnt tidliger)
Afslutningsvis vil de enkelte vurderingselementer og problemformuleringen, sammenfattes i en syntese indeholdende en diskussion af resultater og en konklusion på problemformuleringen. 

\subsection{Litteratursøgning}
Litteratursøgning og -vurdering i projektet tager udgangspunkt i retningslinjerne opstillet i Metodehåndbogen for Medicinsk Teknologivurdering udarbejdet af Sundhedsstyrelsen \citep{metodehaandbogen}. Da projektet udarbejdes på baggrund af videnskabelig litteratur, er det væsentligt, at litteraturen findes og vurderes ved en systematisk fremgangsmåde, således at problemformuleringen besvares på et veldokumenteret grundlag. Litteratursøgningen vil derfor være den samme for alle områder af rapporten. For hvert områdesmetode vil litteratursøgning kun beskrives i så fald søgningen for det specifikke område afviger fra den generelle litteratursøgningsmetode som beskrevet her. \\
Generelt for litteratursøgning vil der blive søgt på sundhedsvidenskabelige databaser som: PubMed, MEDLINE, EMBASE og Cochrane Library. Aalborg Universitetsbiblioteks søgemaskine \textit{Primo} vil ligeledes tages i brug og anvendes som generel søgeværktøj, da denne dækker mange databaser. For ligeledes at sikre ensartethed gennem rapporten vil udvælgelsen og vurderingen af litteratur forløbe efter samme søgestrategi for alle områder. Søgeprotokollen har ligeleds til formål at gøre det muligt for interessenter at forstå, hvordan litteratursøgningen er forløbet. \citep{metodehaandbogen}
Til at udarbejde søgeprotokollen i dette projekt, er der opstillet en skabelon, der systematiserer litteratursøgningen. Et uddrag af skabelonen ses i \figref{fig:soegeprotokol}.

\begin{figure}[H]
\begin{center}
\includegraphics[width=0.5\textwidth]{figures/cMetode/soegeprotokol}
\end{center}
\caption{Skabelon for projektets søgeprotokol. Det fremgår af skabelonen, at der for hver søgning skal opstilles ét eller flere fokuserede spørgsmål samt inklusions- og eksklusionskriterier. Det skal ligeledes dokumenteres, hvilke databaser, der er anvendt til søgningen. For hver database skal de specifikt anvendte søgeord opstilles, og antallet hits skal efterfølgende fremgå.}
\label{fig:soegeprotokol} 
\end{figure}

De fokuserede spørgsmål i søgeprotokollen har til formål at bidrage til besvarelsen af problemformuleringen indenfor de otte analyseområder. Spørgsmålene skal være helt afgrænsede og entydige, således at det er muligt at finde konkret litteratur, og besvare dem præcist. \citep{metodehaandbogen}  
For at afgrænse og sikre relevansen af søgeresultaterne, opstilles inklusions- og eksklusionskriterier for søgningerne. Søgningerne kan eksempelvis afgrænses til kun at indeholde bestemte typer studier, en eller få specifikke sygdomme, en afgrænset aldersgruppe med mere. \citep{metodehaandbogen} \\
%Dokumentationen af valgte databaser samt tilhørende specifikke søgeord er en væsentlig del af søgeprotokollen, da databaserne har forskellige sprog til litteratursøgning, hvormed det er vigtigt, at søgningen i en given database er foretaget med anvendelse af søgetermerne specifikke for den valgte database. \citep{metodehaandbogen}
Efter litteratursøgningen vurderes den fundne litteratur med henblik på at udvælge de kilder, der bedst kan besvare de fokuserede spørgsmål og dermed den overordnede problemstilling for projektet. Det er således essentielt at kontrollere, om det, der undersøges i udvalgt litteratur, har relevans for de fokuserede spørgsmål eksempelvis ved gennemlæsning af abstract, metode samt resultater. Ligeledes vurderes studiers evidensniveau, for at sikre kvaliteten af besvarelserne. \citep{metodehaandbogen} 
Litteraturen i dette projekt inddeles ud fra evidenshierakiet i \cite{metodehaandbogen}. Evidenshierakiet omfatter følgende syv punkter, hvor kilder med det højeste evidensniveau er placeret øverst i listen:

\begin{enumerate}
\item Metaanalyser og systematiske undersøgelser 
\item Randomiserede kontrollerede undersøgelser (RCT’s)
\item Ikke-randomiserede kontrollerede undersøgelser
\item Kohorte undersøgelser
\item Case-kontrol undersøgelser
\item Deskriptive undersøgelser, mindre serier
\item Konsensusrapporter, ikke-systematiske oversigtsartikler, ledere, ekspertudtalelser
\end{enumerate}

%Metaanalyser og systematiske undersøgelser er sekundær litteratur og har det højeste evidensniveau. Denne type litteratur er statiske sammenfatninger af primær litteratur med samme afgrænsede problemstilling. \citep{denstoredanske2009} \\
%Randomiserede kontrollerede undersøgelser (RCT) er primær litteratur, hvor der foretages en sammenligning af to forsøgsgrupper. Den ene gruppe udsættes for en påvirkning, mens den anden gruppe fungerer som kontrolgruppe. Udvælgelsen af forsøgspersoner foregår tilfældigt. \citep{denstoredanske2009a} \\
%Ikke-randomiserede kontrollerede undersøgelser er ligesom RCT’s primær litteratur, hvor to forsøgsgrupper sammenlignes. Ved disse undersøgelser sker udvælgelsen af forsøgspersoner ikke tilfældigt, hvormed evidensniveauet falder da der ikke på samme måde som ved RCT tages højde for bias i forsøgsgrupperne. \citep{denstoredanske2009a} \\
%Ved kohorte undersøgelser følges flere forsøgsgrupper over en periode for at undersøge, hvorvidt bestemte eksponeringer har indflydelse på udviklingen af helbredsfænomener, herunder sygdom og død. \citep{metodehaandbogen} \\
%I case-kontrol undersøgelser, forsøges det at undersøge forskellige faktorers indflydelse på udvikling af bestemte sygdomme. Dette gøres ved en sammenligning mellem en forsøgsgruppe med den pågældende sygdom og en forsøgsgruppe bestående af raske personer. I modsætning til eksempelvis kohortestudiet følges forsøgsgrupperne ikke over tid, hvormed der ikke kan udføres en opfølgende undersøgelse. Det er således ikke muligt at estimere betydningen af  risikofaktorerne. \citep{denstoredanske2012} \\
%Deskriptive studier er studier hvor der foretages analyser til beskrivelse af et fænomen. I modsætning til de andre typer studier påvirkes forsøgsgrupperne ikke. I stedet undersøges nuværende tendenser, eksempelvis med henblik på senere udførelse af et forsøg. \citep{} \\
%Fælles for gruppen af konsensusrapporter, ikke-systematiske oversigtsartikler samt ledere og ekspertudtalelser er, at materialet oftest er udtryk for subjektive holdninger, der ikke er underbygget af tilstrækkelige mængder supplerende litteratur, der undersøger området. \citep{metodehaandbogen}  