Artrose er den mest udbredte gigtsygdom. Derudover er den en af de mest udbredte kroniske sygdomme i Danmark (\textbf{Tilføj:Hvad er kronisk sygdom}) \citep{sygdom}. Artrose en kronisk ledsygdom der kan ramme alle ledstrukturer, men rammer primært ledfladernes bruskdele \citep{schroder}. \\
Typiske symptomer ved artrose er smerter, herunder belastningssmerter, og i nogle tilfælde hvilesmerter, ømhed og ledstivhed . Sekundært påvirkes senerne og muskulaturene ved det afficerede led, og disse forandringer kan føre til et funktionstab. Ved artrose vil håndfunktionen og gangfunktionen være de hyppigst påvirkede funktioner \cite{schroder}. \\
Ifølge spørgeskemabaserede data fra den nationale sundhedsprofil 2013, er prævalensen på 800.000 personer i alt i Danmark. Derudover medfører sygdommen  over 20.000 indlæggelser i det danske sundhedsvæsen på årlig basis \citep{sygdom}. Forekomsten af artrose stiger med alderen, hvor det især er de 55-84 årige der er overrepræsenterede. Desuden er der en højere forekomst blandt kvinder end mænd \citep{sygdom}. 
Personer der har et uddannelsesniveau der svarer til en kort eller ingen uddannelse, vil også have flere indlæggelser end personer der har en mellemlang- til lang uddannelse. Overvægt, tidligere skader, muskelsvaghed, arvelige anlæg mm. spiller også en væsentlig rolle i udviklingen af artrose. Det antages at forekomsten af artrose fortsat vil være stigende, på baggrund af at den gennemsnitlige levealder sammenholdt med en øget forekomst af overvægt er stigende \citep{sygdom}. 
En af de hyppigst forekommende artroseformer er knæartrose. Denne gren af artrose er den førende årsag til funktionsnedsættelse i de nedre ekstremiteter \cite{bezwick2012}. Kigger man på gruppen af de 60-70 årige har 40\% af kvinderne og 25\% af mændene artrose i knæleddet schroder.\\
Knæartrose medfører at ledbrusken nedbrydes, samtidig med at der forløber en række reaktioner i knoglen under brusken, samt i synovialmembranen \citep{brostrom2012}.
Som følge af den tiltagende bruskmangel kan der opstå ledskurren og fejlsstilling, hvilket kan medføre belastningssmerter og i sidste ende funktionstab \citep{ugeskrift2011}.  
Der er stor variation i hvordan personer der lever med denne sygdom påvirkes, og nogle vil derfor kunne leve relativt upåvirkede med sygdommen, mens andre vil opleve at sygdommen svækker både arbejdsevne og livskvalitet \citep{sygdom}. Derudover forekommer der kun i ringe grad sammenhæng mellem røntgenfund og patientens oplevede symptomer, og derfor vil behandlingen ofte beror på en vurdering af begge dele [10]. Der findes en række behandlingsmuligheder der har til hensigt at forbedre funktionen af det afficerede led. Hvilken behandlingsstrategi der vælges afhænger af flere faktorer eksempelvis patientens alder, aktivitetsniveau, samt graden af artrosen. Inden et eventuel operativ indgreb foretages, bør patienten have prøvet non-operativ behandling, og i disse tilfælde vil fokus være på træning, livsstilsomlægning og patientuddannelse \citep{schroder}.\\
Ved svære artrosetilfælde, hvor artrosen er radiologisk eller artroskopisk påvist, kan knæalloplastik være en mulighed. Her vil det dreje sig om en individuel vurdering af patientens gener kontra de ricisi der er forbundet med et operativ indgreb \citep{schroder}. Knæalloplastik, som er en fællesbetegnelse over unikompartmental knæalloplastik (UKA) og total knæalloplastik (TKA), er operative indgreb, hvor patienten får udskiftet knæleddet enten helt eller delvist. Der er sket en stigning af disse operationer fra 2500 i år 2000, til over 9000 i år 2015 \citep{aarsrapport2016}. En tilsvarende tendens ses i flere lande. Det ses generelt, baseret på målinger udført med American Knee Society's knæscore, at grænsen for hvor fremskreden artrosen skal være før operationen udføres, er flyttet i perioden 1997-2010. Dermed udføres TKA-operationerne generelt tidligere i sygdomsforløbet. Det menes, at en generel stigende efterspørgsel for operationen har forårsaget denne ændring. \citep{aarsrapport2011} Dette kan være en medvirkende faktor til stigningen i antallet af TKA-operationer, da flere patienter således opfylder kriterierne for at blive indstillet til operation. I næsten 90\% af alle tilfælde vil alloplastiken være en TKA \citep{aarsrapport2016}. En væsentlig problemstilling der er forbundet med TKA-operationer, er at det er påvist, at for patienter der har gennemgået en ellers succesfuld operation, vil omtrent 20\% opleve kroniske smerter et år efter operationen. For tilsvarende operation af hofteleddet er dette tal cirka 9\% \citep{bezwick2012}. Der er således fortsat grundlag for at søge optimering af undersøgelses- og behandlingsprocessen for de enkelte patienter med knæartrose.  
%Det er dermed relevant fortsat at søge forbedring af henholdsvis behandling og forståelse for patienternes egne forudsætninger for et vellykket behandlingsforløb i forbindelse med knæartrose.\\\\

\textbf{Initierende problemstilling}
\begin{center}
	\textit{Hvordan hånderes patienter med svær knæartroser, og hvilke problematikker forekommer i denne proces?}
\end{center}

%
%
%%Forslag 1)\\
%%Hvilke sundhedsmæssige problemer er forbundet med at mange(Pia: Hvad er mange ?)  knæartrosepatienter  lever med kroniske smerter efter de er blevet opereret?
%%
%%
%%
%%Forslag 2)\\
%%Hvordan kan det postoperative forløb forbedres for patienter der har gennemgået en total knæalloplastikoperation?
%%
%%
%%
%%Forslag 3)\\
%%Hvilke tiltag benytter Aalborg universitetshospital til at forbedre behandlingen for TKA patienter?
%%
%%
%%
%%
%%Forslag 4)\\
%%Hvilke diagnosticringsværktøjer kan benyttes til at screene patienter for  kroniske postoperative smerter efter TKA for dermed at kunne tilbyde patienten et bedre behadnlingsforløb?\\\\
%
%
%
%
%Artrose, eller slidgigt i daglig tale, er den mest udbredte gigtsygdom der findes. Derudover er den en af de mest udbredte kroniske sygdomme i Danmark \citep{sygdom}. Denne sygdom er en kronisk ledsygdom der kan ramme alle ledstrukturer, dog rammer den primært ledfladernes bruskdele \citep{schroder}. 
%Typiske kendetegn ved denne sygdom er smerter, herunder belastningssmerter, og i nogle tilfælde hvilesmerter, ømhed og ledstivhed \citep{schroder}. Sekundært påvirkes senerne og muskulaturene ved det afficerede led, og disse forandringer kan føre til et funktionstab. Ved artrose vil håndfunktionen og gangfunktionen være de hyppigst påvirkede funktioner \citep{schroder}. 
%Ifølge spørgeskemabaserede data fra den nationale sundhedsprofil 2013, er prævalensen på 800.000 personer i alt i Danmark. Derudover medfører sygdommen  over 20.000 indlæggelser i det danske sundhedsvæsen på årlig basis \cite{sygdom}. Forekomsten af artrose stiger med alderen, hvor det især er de 55-84 årige der er overrepræsenterede. Desuden er der en højere forekomst blandt kvinder end blandt mænd \cite{sygdom}. 
%Personer der har et uddannelsesniveau der svarer til grundskole/kort uddannelse, vil også have flere indlæggelser end personer der har en mellemlang/lang uddannelse. Overvægt, tidligere skader, muskelsvaghed, arvelige anlæg mm. spiller også en væsentlig rolle i udviklingen af artrose. Den øgede gennemsnitlige levealder sammenholdt med en øget forekomst af overvægt, vil sandsynligvis medføre at  forekomsten af artrose vil stige i fremtiden. Derudover kan det nævnes at artrose er årsag til 3,2\% af alle tilkendelser af førtidspensioner i Danmark \cite{sygdom}.
%En af de hyppigst forekommende artroseformer er knæartrose. Denne afgrening af artrose er den førende årsag til funktionsnedsættelse i de nedre ekstremiteter \citep{johnson2014}. Kigger man på gruppen af de 60-70 årige har 40\% af kvinderne og 25\% af mændene artrose i knæleddet \citep{schroder}.
%Sygdommen medfører at ledbrusken nedbrydes, samtidig med at der forløber en række reaktioner i knoglen under brusken, samt i synovialmembranen \citep{brostrom2012}.
% Som følge af den tiltagende bruskmangel kan der opstå ledskurren og fejlsstilling, hvilket kan medføre belastningssmerter og i sidste ende funktionstab \cite{ugeskrift2011}.  
%Der er stor variation i hvordan personer der lever med denne sygdom påvirkes, og nogle vil derfor kunne leve relativt upåvirkede med sygdommen, mens andre vil opleve at sygdommen svækker både arbejdsevne og livskvalitet \cite{sygdom}. Derudover forekommer der kun i ringe grad sammenhæng mellem røntgenfund og patientens oplevede symptomer, og derfor vil behandlingen ofte bero på en vurdering af begge dele \citep{ugeskrift2011}. Der findes en række behandlingsmuligheder der har til hensigt at forbedre funktionen af det afficerede led. Hvilken behandlingstrategi der vælges afhænger af flere faktorer, bla.  alder, patientens situation, samt graden af artrosen. Inden et eventuel operativ indgreb foretages, bør patienten have prøvet non-operativ behandling, og i disse tilfælde vil fokus være på træning, livsstilsomlægning og patientuddannelse \citep{schroder}.
%Ved svære artrosetilfælde, hvor artrosen er radiologisk eller artroskopisk påvist, kan knæalloplastik være en mulighed. Her vil det dreje sig om en individuel vurdering af patientens gener kontra de ricisi der er forbundet med et operativ indgreb \citep{schroder}. Knæalloplastik, som er en samlebetegnelse over total knæalloplastik (TKA) og unikompartmental knæalloplastik (UKA), er indgreb hvor patienten får udskiftet knæleddet enten helt eller delvist. Der er sket en stigning af disse operationer fra 2.500 i år 2000, til over 9000 i år 2015 \citep{aarsrapport2016}.  I næsten 90\% af alle tilfælde vil alloplastiken være en TKA \citep{ugeskrift2011} \citep{aarsrapport2016}. En væsentlig problemstilling der er forbundet med TKA-operationer, er at for patienter der har gennemgået en ellers succesfuld operation, vil omtrent 20\% opleve kroniske smerter et år efter operationen \citep{bezwick2012}. Fra et patientperspektiv kan operationen derfor betragtes som en risikabel satsning, hvorfor det er relevant fortsat at søge at forbedre henholdsvis behandling og forståelse for patienternes egne forudsætninger for et vellykket behandlingsforløb.\\\\
%\textbf{Initierende problemstilling}
%
%Forslag 1)\\
%Hvilke sundhedsmæssige problemer er forbundet med at mange(Pia: Hvad er mange ?)  knæartrosepatienter  lever med kroniske smerter efter de er blevet opereret?
%
%
%
%Forslag 2)\\
%Hvordan kan det postoperative forløb forbedres for patienter der har gennemgået en total knæalloplastikoperation?
%
%
%
%Forslag 3)\\
%Hvilke tiltag benytter Aalborg universitetshospital til at forbedre behandlingen for TKA patienter?
%
%
%
%
%Forslag 4)\\
%Hvilke diagnosticringsværktøjer kan benyttes til at screene patienter for  kroniske postoperative smerter efter TKA for dermed at kunne tilbyde patienten et bedre behadnlingsforløb?\\\\
%
%%[1] Knæartrose - nationale kliniske retningslinjer og faglige visitationsretningslinjer - Sundhedsstyrelsen 2012 \\
%%[2] Basisbog i medicin og kirugi, Torben V. Schroeder, 5. Udgave, ISBN: 9788762810068\\
%%[3] Sygdomsbyrden i Danmark (Sygdomme) - Sundhedsstyrelsen og Statens insitut for folkesundhed, Syddansk universitet, 2015\\
%%[4]Abnormal Quantitative Sensory Testing is Associated With Persistent Pain One Year After TKA, Vikki Wylde,Clin Orthop Relat Res. 2015 Jan; 473(1): 255–257.  doi:  10.1007/s11999-014-4023-x \\
%%[5] The epidemiology of osteoarthritis, Johnson, Victoria L.
%%Hunter, David J., Best Practice and Research: Clinical Rheumatology, 2014, vol 28 \\
%%[10] Graduering af knæartrose, Simonsen et al,  ugeskrift for læger, 2011\\
%%[11] What proportion of patients report long-term pain after total hip or knee replacement for osteoarthritis? A systematic review of prospective studies in unselected patients, Bezwick et al., BMJ Open. 2012 Feb 22;2(1):e000435. doi: 10.1136/bmjopen-2011-000435 \\
%%[12] Dansk Knæalloplastikregister Årsrapport Kvalitetsindikatorer, H. Odgaard et al, 2016
