Artrose, eller slidgigt i daglig tale, er den mest udbredte gigtsygdom der findes. Derudover er den en af de mest udbredte kroniske sygdomme i Danmark [3]. Denne sygdom er en kronisk ledsygdom der kan ramme alle ledstrukturer, dog rammer den primært ledfladernes bruskdele [2].
Typiske kendetegn ved denne sygdom er smerter, herunder belastningssmerter, og i nogle tilfælde hvilesmerter, ømhed og ledstivhed. Sekundært påvirkes senerne og muskulaturene ved det afficerede led, og disse forandringer kan føre til et funktionstab. Ved artrose vil håndfunktionen og gangfunktionen være de hyppigst påvirkede funktioner [2]. 
Ifølge spørgeskemabaserede data fra den nationale sundhedsprofil 2013, er prævalensen på 800.000 personer i alt i Danmark. Derudover medfører sygdommen  over 20.000 indlæggelser i det danske sundhedsvæsen på årlig basis [3]. Forekomsten af artrose stiger med alderen, hvor det især er de 55-84 årige der er overrepræsenterede. Desuden er der en højere forekomst blandt kvinder end blandt mænd [3]. 
Personer der har et uddannelsesniveau der svarer til grundskole/kort uddannelse, vil også have flere indlæggelser end personer der har en mellemlang/lang uddannelse. Overvægt, tidligere skader, muskelsvaghed, arvelige anlæg mm. spiller også en væsentlig rolle i udviklingen af artrose [3]. Den øgede gennemsnitlige levealder sammenholdt med en øget forekomst af overvægt, vil sandsynligvis medføre at  forekomsten af artrose vil stige i fremtiden. Derudover kan det nævnes at artrose er årsag til 3,2\% af alle tilkendelser af førtidspensioner i Danmark[3].
En af de hyppigst forekommende artroseformer er knæartrose. Denne afgrening af artrose er den førende årsag til funktionsnedsættelse i de nedre ekstremiteter [5]. Kigger man på gruppen af de 60-70 årige har 40\% af kvinderne og 25\% af mændene artrose i knæleddet [2].
Hele knæleddet rammes og er patofysiologisk karakteriseret ved gradvist fremadskridende destruktion af ledbrusken, ledsaget af reaktioner i knoglen under brusken og synovialmembranen [1]. Som følge af den tiltagende bruskmangel kan der opstå ledskurren og fejlsstilling, hvilket kan medføre belastningssmerter og i sidste ende funktionstab [10].  
Der er stor variation i hvordan personer der lever med denne sygdom påvirkes, og nogle vil derfor kunne leve relativt upåvirkede med sygdommen, mens andre vil opleve at sygdommen svækker både arbejdsevne og livskvalitet [3]. Derudover forekommer der kun i ringe grad sammenhæng mellem røntgenfund og patientens oplevede symptomer, og derfor vil behandlingen ofte bero på en vurdering af begge dele [2].
Artrose kan ikke helbredes, men medicinsk behandling vil søge at dæmpe smerter der er forbundet med sygdommen, samt at forbedre funktionen af det afficerede led. Behandlingen af knæartrose afhænger af sværhedsgraden af artrosen, samt patientens alder og situation. I lette og moderate tilfælde er behandlingen altovervejende non-operativ, og i disse tilfælde vil fokus være på træning, livsstilsomlægning og patientuddannelse[2].
Ved svære artrosetilfælde, hvor artrosen er radiologisk eller artroskopisk påvist, kan knæalloplastik være en mulighed. Her vil det dreje sig om en individuel vurdering af patientens gener kontra de ricisi der er forbundet med et operativ indgreb [2]. Dog bør behandling af knæartrose altid indledes med non-operativ behandling.
Knæalloplastik, som er en samlebetegnelse over total knæalloplastik (TKA) og unikompartmental knæalloplastik (UKA), er indgreb hvor patienten får udskiftet knæleddet enten helt eller delvist. Der er sket en stigning af disse operationer fra 2.500 i år 2000, til over 9000 i år 2015 [12].  I næsten 90\% af alle tilfælde vil alloplastiken være en TKA [10] [12].
For patienter der har gennemgået en ellers succesfuld operation, vil omtrent 20\% opleve kroniske smerter et år efter operationen [11]. Fra et patientperspektiv kan operationen derfor betragtes som en risikabel satsning, hvorfor det er relevant fortsat at søge at forbedre henholdsvis behandling og forståelse for patienternes egne forudsætninger for et vellykket behandlingsforløb. 
\\
\textbf{Initierende problemstilling}

Forslag 1)\\
Hvilke sundhedsmæssige problemer er forbundet med at mange(Pia: Hvad er mange ?)  knæartrosepatienter  lever med kroniske smerter efter de er blevet opereret?



Forslag 2)\\
Hvordan kan det postoperative forløb forbedres for patienter der har gennemgået en total knæalloplastikoperation?



Forslag 3)\\
Hvilke tiltag benytter Aalborg universitetshospital til at forbedre behandlingen for TKA patienter?




Forslag 4)\\
Hvilke diagnosticringsværktøjer kan benyttes til at screene patienter for  kroniske postoperative smerter efter TKA for dermed at kunne tilbyde patienten et bedre behadnlingsforløb?\\\\

%[1] Knæartrose - nationale kliniske retningslinjer og faglige visitationsretningslinjer - Sundhedsstyrelsen 2012 \\
%[2] Basisbog i medicin og kirugi, Torben V. Schroeder, 5. Udgave, ISBN: 9788762810068\\
%[3] Sygdomsbyrden i Danmark (Sygdomme) - Sundhedsstyrelsen og Statens insitut for folkesundhed, Syddansk universitet, 2015\\
%[4]Abnormal Quantitative Sensory Testing is Associated With Persistent Pain One Year After TKA, Vikki Wylde,Clin Orthop Relat Res. 2015 Jan; 473(1): 255–257.  doi:  10.1007/s11999-014-4023-x \\
%[5] The epidemiology of osteoarthritis, Johnson, Victoria L.
%Hunter, David J., Best Practice and Research: Clinical Rheumatology, 2014, vol 28 \\
%[10] Graduering af knæartrose, Simonsen et al,  ugeskrift for læger, 2011\\
%[11] What proportion of patients report long-term pain after total hip or knee replacement for osteoarthritis? A systematic review of prospective studies in unselected patients, Bezwick et al., BMJ Open. 2012 Feb 22;2(1):e000435. doi: 10.1136/bmjopen-2011-000435 \\
%[12] Dansk Knæalloplastikregister Årsrapport Kvalitetsindikatorer, H. Odgaard et al, 2016
