Knæartrose er en af de hyppigst forekommende artroseformer og er den førende årsag til funktionsnedsættelse i de nedre ekstremiteter \citep{Beswick2012}. I 2011 henvendte cirka 60.000 patienter sig hos egen læge med symptomer på knæartrose \citep{brostrom2012}. Prævalensen for knæartrose i Danmark forventes at stige. Dette antages da både andelen af ældre og overvægtige er stigene, og begge disse er væsentlige faktorer for udviklingen af knæartrose. \citep{sygdom} Hermed kan det ligeledes forventes, at flere personer vil blive behandlet for knæartrose. \\    
Knæartrose er en kronisk ledsygdom, der primært rammer ledfladens bruskdele, men alle ledstrukturer kan påvirkes af sygdommen. \citep{schroder} Typsike symptomer på knæartrose er funktionsnedsættelse, smerter ved daglige aktiviteter og ledstivhed. Der er variation i graden af symptomerne på knæartrose blandt patienter, hvormed nogle patienter kan leve relativt upåvirkede med sygdommen, mens andre oplever, at sygdommen svækker både arbejdsevne og livskvalitet. \citep{sygdom} Herudover er der kun i ringe grad en sammenhæng mellem den radiologiske artrose og patientens gener. Hermed bygger vurderingen om behandling af patienten af en sammenfatning af røntgenfund og en klinikers objektive fund. \citep{ugeskrift2011} Idet knæartorse er en kronisk sygdom findes der ingen kur og behandling af sygdommen har heraf fokus på at mindske eller fjerne symptomerne. Behandling af knæartrose kan være både invasiv og non-invasiv, hvor non-invasiv behandling eksempelvis er vægtreduktion og medicinering mens invasiv behandling eksempelvis er knæalloplastikker. \\
Selv med non-invasiv behandling vil de fleste knæartrosepatienter nå til et stadige hvor artrosen er så fremskreden, at kun invasiv behandling vil kunne lindre deres symptomer. \citep{brostrom2012} Denne invasive behandling består i en, enten hel eller delvis, udskriftning af det påvirkede knæled. Denne type behandling kaldes en knæalloplastik og kan opdeles i en total knæalloplastik (TKA) eller unikompartmental knæalloplastik (UKA). Den hyppigste form for knæalloplastik er en TKA, hvor der i 2015 blev udført cirka 8.800 af disse i Danmark. \citep{aarsrapport2016} TKA-operationer anvendes som den sidste behandlingsmetode for knæartrosepatienter, hvormed resultatet af operationen har stor betydning for patienterne \citep{brostrom2012}. Herudfra er det problematisk, at cirka 20~\% af knæartrosepatienterne, som undergår en TKA-operation, udvikler kroniske postoperative smerter. Disse patienter har dermed ikke fået reduceret deres smerter ved TKA-operationen. For en tilsvarende operation af hofteleddet er dette tal cirka 9~\%. \citep{Beswick2012} Herudfra antydes det at der er mulighed for at formindske andelen af knæartrosepatienter som udvikler kroniske smerter efter en TKA-operation.     

\section{Initierende problemstilling}
På baggrund af ovenstående introduktion udledes følgende overordnede problemstilling: 
\begin{center}
	\textit{Hvorfor får nogle knæartrosepatienter kroniske postoperative smerter efter en total knæalloplastik?}
\end{center}

%Nedbrydningen af ledbrusk fører til påvirkning af sener og muskulatur i leddet, hvormed disse forandres. Disse forandringer kan føre til funktionstab. Typiske symptomer på knæartrose er smerter, herunder belastningssmerter, ømhed og ledstivhed og mange patienter har i de sene stadier også hvile- og natlige smerter. 

%Af disse blev 12,5~\% opereret for sygdommen. \citep{brostrom2012} \\
%Andelen af personer med knæartrose stiger med alderen; for 60 til 70 årige har 40~\% kvinder og 25~\% mænd artrose i knæleddet \citep{schroder}. 

%Væsentlige faktorer for udvikling af knæartrose er overvægt, tidligere skader, muskelsvaghed og arvelige anlæg. Det antages, at forekomsten af artrose fortsat vil være stigende, da den gennemsnitlige levealder og antallet af overvægtige er stigende. \citep{sygdom}\\

%Der findes en række behandlingsmetoder der har til formål at forbedre funktionaliteten af det afficerede led. Valget af behandlingsmetode afhænger af flere faktorer, herunder patientens alder, aktivitetsniveau samt graden af artrose. Behandlingsmetoderne kan både være non-invasive og invasive. Inden et eventuelt invasivt indgreb foretages, bør patienten have prøvet non-invasive behandlingsmetoder, hvor fokus eksempelvis vil være på træning og livsstilsomlægning. \citep{schroder}
%Ved svære artrosetilfælde kan knæalloplastik være en mulighed. Her vil der foretages en individuel vurdering af patientens gener og de risici, der er forbundet med et invasivt indgreb \citep{schroder}. Knæalloplastik, som omfatter total knæalloplastik (TKA) og unikompartmental knæalloplastik (UKA), er et indgreb, hvor patienten får udskiftet knæleddet enten helt eller delvist. Der er sket en stigning i antallet disse operationer fra 2.500 i år 2000, til 9.800 i år 2015 \citep{aarsrapport2016}.\\
%Næsten 90~\% af alle udførte knæalloplastikker er TKA-operationer \citep{aarsrapport2016}. En væsentlig problemstilling forbundet med TKA-operationer er, at af patienter, der har gennemgået en operation, vil cirka 20~\% opleve kroniske smerter et år efter operationen.  Der er således fortsat grundlag for at søge forbedring af resultatet af en udført TKA-operation.  
%Det er dermed relevant fortsat at søge forbedring af henholdsvis behandling og forståelse for patienternes egne forudsætninger for et vellykket behandlingsforløb i forbindelse med knæartrose.\\\\

%Artrose er den mest udbredte gigtsygdom. Derudover er den en af de mest udbredte kroniske sygdomme i Danmark (\textbf{Note: tilføj hvad en kronisk sygdom er}) \citep{sygdom}. Artrose er en kronisk ledsygdom der kan ramme alle ledstrukturer, men primært rammes ledfladernes bruskdele \citep{schroder}. \\
%Typiske symptomer på artrose er smerter, herunder belastningssmerter,  ømhed og ledstivhed og mange patienter har i de sene stadier af artrosoe hvile- natteligesmerter. Sekundært påvirkes senerne og muskulaturene ved det afficerede led, og disse forandringer kan føre til et funktionstab. Ved artrose vil håndfunktionen og gangfunktionen være de hyppigst påvirkede funktioner \cite{schroder}. \\
%Ifølge spørgeskemabaserede data fra den nationale sundhedsprofil 2013 med data fra 2010 til 2012, er prævalensen for artrose på 800.000 personer i Danmark. Andelen af henvendelser i almen praksis med artrose som primære grund er henholdsvis 6,8\%~ og 8,6\%~for mænd og kvinder. Artrose medfører over 20.000 indlæggelser på årligt basis i det danske sundhedsvæsen, hvoraf 12.000 er kvinder.\citep{sygdom} Forekomsten af artrose stiger med alderen, hvor det især er de 55+ årige der er repræsenterede. Desuden er der en højere forekomst blandt kvinder end mænd især ved 55+ år. \citep{sygdom} Personer der har et uddannelsesniveau der svarer til ingen eller kort uddannelse, har ligeledes flere indlæggelser end personer der har en mellemlang- til lang uddannelse. Overvægt, tidligere skader, muskelsvaghed, arvelige anlæg med mere spiller også en væsentlig rolle i udviklingen af artrose. Det antages at forekomsten af artrose fortsat vil være stigende, da den gennemsnitlige levealder og forekomsten af overvægtige er stigende \citep{sygdom}. 
%
%
%%Forslag 1)\\
%%Hvilke sundhedsmæssige problemer er forbundet med at mange(Pia: Hvad er mange ?)  knæartrosepatienter  lever med kroniske smerter efter de er blevet opereret?
%%
%%
%%
%%Forslag 2)\\
%%Hvordan kan det postoperative forløb forbedres for patienter der har gennemgået en total knæalloplastikoperation?
%%
%%
%%
%%Forslag 3)\\
%%Hvilke tiltag benytter Aalborg universitetshospital til at forbedre behandlingen for TKA patienter?
%%
%%
%%
%%
%%Forslag 4)\\
%%Hvilke diagnosticringsværktøjer kan benyttes til at screene patienter for  kroniske postoperative smerter efter TKA for dermed at kunne tilbyde patienten et bedre behadnlingsforløb?\\\\
%
%
%
%
%Artrose, eller slidgigt i daglig tale, er den mest udbredte gigtsygdom der findes. Derudover er den en af de mest udbredte kroniske sygdomme i Danmark \citep{sygdom}. Denne sygdom er en kronisk ledsygdom der kan ramme alle ledstrukturer, dog rammer den primært ledfladernes bruskdele \citep{schroder}. 
%Typiske kendetegn ved denne sygdom er smerter, herunder belastningssmerter, og i nogle tilfælde hvilesmerter, ømhed og ledstivhed \citep{schroder}. Sekundært påvirkes senerne og muskulaturene ved det afficerede led, og disse forandringer kan føre til et funktionstab. Ved artrose vil håndfunktionen og gangfunktionen være de hyppigst påvirkede funktioner \citep{schroder}. 
%Ifølge spørgeskemabaserede data fra den nationale sundhedsprofil 2013, er prævalensen på 800.000 personer i alt i Danmark. Derudover medfører sygdommen  over 20.000 indlæggelser i det danske sundhedsvæsen på årlig basis \cite{sygdom}. Forekomsten af artrose stiger med alderen, hvor det især er de 55-84 årige der er overrepræsenterede. Desuden er der en højere forekomst blandt kvinder end blandt mænd \cite{sygdom}. 
%Personer der har et uddannelsesniveau der svarer til grundskole/kort uddannelse, vil også have flere indlæggelser end personer der har en mellemlang/lang uddannelse. Overvægt, tidligere skader, muskelsvaghed, arvelige anlæg mm. spiller også en væsentlig rolle i udviklingen af artrose. Den øgede gennemsnitlige levealder sammenholdt med en øget forekomst af overvægt, vil sandsynligvis medføre at  forekomsten af artrose vil stige i fremtiden. Derudover kan det nævnes at artrose er årsag til 3,2\% af alle tilkendelser af førtidspensioner i Danmark \cite{sygdom}.
%En af de hyppigst forekommende artroseformer er knæartrose. Denne afgrening af artrose er den førende årsag til funktionsnedsættelse i de nedre ekstremiteter \citep{johnson2014}. Kigger man på gruppen af de 60-70 årige har 40\% af kvinderne og 25\% af mændene artrose i knæleddet \citep{schroder}.
%Sygdommen medfører at ledbrusken nedbrydes, samtidig med at der forløber en række reaktioner i knoglen under brusken, samt i synovialmembranen \citep{brostrom2012}.
% Som følge af den tiltagende bruskmangel kan der opstå ledskurren og fejlsstilling, hvilket kan medføre belastningssmerter og i sidste ende funktionstab \cite{ugeskrift2011}.  
%Der er stor variation i hvordan personer der lever med denne sygdom påvirkes, og nogle vil derfor kunne leve relativt upåvirkede med sygdommen, mens andre vil opleve at sygdommen svækker både arbejdsevne og livskvalitet \cite{sygdom}. Derudover forekommer der kun i ringe grad sammenhæng mellem røntgenfund og patientens oplevede symptomer, og derfor vil behandlingen ofte bero på en vurdering af begge dele \citep{ugeskrift2011}. Der findes en række behandlingsmuligheder der har til hensigt at forbedre funktionen af det afficerede led. Hvilken behandlingstrategi der vælges afhænger af flere faktorer, bla.  alder, patientens situation, samt graden af artrosen. Inden et eventuel operativ indgreb foretages, bør patienten have prøvet non-operativ behandling, og i disse tilfælde vil fokus være på træning, livsstilsomlægning og patientuddannelse \citep{schroder}.
%Ved svære artrosetilfælde, hvor artrosen er radiologisk eller artroskopisk påvist, kan knæalloplastik være en mulighed. Her vil det dreje sig om en individuel vurdering af patientens gener kontra de ricisi der er forbundet med et operativ indgreb \citep{schroder}. Knæalloplastik, som er en samlebetegnelse over total knæalloplastik (TKA) og unikompartmental knæalloplastik (UKA), er indgreb hvor patienten får udskiftet knæleddet enten helt eller delvist. Der er sket en stigning af disse operationer fra 2.500 i år 2000, til over 9000 i år 2015 \citep{aarsrapport2016}.  I næsten 90\% af alle tilfælde vil alloplastiken være en TKA \citep{ugeskrift2011} \citep{aarsrapport2016}. En væsentlig problemstilling der er forbundet med TKA-operationer, er at for patienter der har gennemgået en ellers succesfuld operation, vil omtrent 20\% opleve kroniske smerter et år efter operationen \citep{Beswick2012}. Fra et patientperspektiv kan operationen derfor betragtes som en risikabel satsning, hvorfor det er relevant fortsat at søge at forbedre henholdsvis behandling og forståelse for patienternes egne forudsætninger for et vellykket behandlingsforløb.\\\\
%\textbf{Initierende problemstilling}
%
%Forslag 1)\\
%Hvilke sundhedsmæssige problemer er forbundet med at mange(Pia: Hvad er mange ?)  knæartrosepatienter  lever med kroniske smerter efter de er blevet opereret?
%
%
%
%Forslag 2)\\
%Hvordan kan det postoperative forløb forbedres for patienter der har gennemgået en total knæalloplastikoperation?
%
%
%
%Forslag 3)\\
%Hvilke tiltag benytter Aalborg universitetshospital til at forbedre behandlingen for TKA patienter?
%
%
%
%
%Forslag 4)\\
%Hvilke diagnosticringsværktøjer kan benyttes til at screene patienter for  kroniske postoperative smerter efter TKA for dermed at kunne tilbyde patienten et bedre behadnlingsforløb?\\\\
%
%%[1] Knæartrose - nationale kliniske retningslinjer og faglige visitationsretningslinjer - Sundhedsstyrelsen 2012 \\
%%[2] Basisbog i medicin og kirugi, Torben V. Schroeder, 5. Udgave, ISBN: 9788762810068\\
%%[3] Sygdomsbyrden i Danmark (Sygdomme) - Sundhedsstyrelsen og Statens insitut for folkesundhed, Syddansk universitet, 2015\\
%%[4]Abnormal Quantitative Sensory Testing is Associated With Persistent Pain One Year After TKA, Vikki Wylde,Clin Orthop Relat Res. 2015 Jan; 473(1): 255–257.  doi:  10.1007/s11999-014-4023-x \\
%%[5] The epidemiology of osteoarthritis, Johnson, Victoria L.
%%Hunter, David J., Best Practice and Research: Clinical Rheumatology, 2014, vol 28 \\
%%[10] Graduering af knæartrose, Simonsen et al,  ugeskrift for læger, 2011\\
%%[11] What proportion of patients report long-term pain after total hip or knee replacement for osteoarthritis? A systematic review of prospective studies in unselected patients, Bezwick et al., BMJ Open. 2012 Feb 22;2(1):e000435. doi: 10.1136/bmjopen-2011-000435 \\
%%[12] Dansk Knæalloplastikregister Årsrapport Kvalitetsindikatorer, H. Odgaard et al, 2016
